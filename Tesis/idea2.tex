\documentclass[12pt]{article}
\usepackage{setspace}
\setstretch{1.5}
\usepackage[utf8]{inputenc}
\usepackage[T1]{fontenc}
\usepackage{url}
\usepackage[spanish]{babel}
\usepackage{apacite}
\usepackage{fullpage}

\begin{document}

\section{La verdad Importa}

\subsection{Enunciando una tesis} \label{enunc}

\noindent Pensemos en un diálogo. Nuestros personajes son Andrés y Berto. Andrés defiende la existencia de los \textit{nomi}, Mientras que Berto señala que los \textit{nomi} no existen. Andrés cita como evidencia el hecho de que ha leído acerca de las huellas que dejan estos organismos, que sólo rondan en la noche. Además, ha visto sus huellas. Berto señala que Andrés nunca ha visto uno \textit{nomi} de primera mano y que de acuerdo a nuestra mejor teoría de la evolución por selección natural, dichos organismos no pueden existir. Qué oración es verdadera ``existen los \textit{nomi}'' o ``no existen los \textit{nomi}'', ¿cómo decidimos cuál enunciado es verdadero?

Comencemos por señalar que a los seres humanos nos interesa la verdad. Nos sentimos, por ejemplo, molestos cuando alguien incumple una promesa. Esta importancia que le damos a la verdad guía también nuestras empresas epistémicas\footnote{utilizo el concepto de ``empresas epistémicas'' como algo muy general y no bien definido. El concepto puede ser usado para describir tanto a la investigación científica, como a la curiosidad de un niño que está aprendiendo a sumar.}. Entonces nos importa la verdad, en particular su vínculo con el conocimiento. Platón en su diálogo con Menón \cite[\P\P 97a-98b]{platonmeno} señala que hay un factor de seguridad ligado al conocimiento, que no se encuentra en la mera opinión verdadera\footnote{Por supuesto esto está más relacionado con la justificación que con la verdad propiamente. Sin embargo, cabe señalar que la verdad juega un papel importante en la explicación de lo que implica saber algo.}. La caracterización ofrecida por Platón como una ``creencia verdadera justificada'' fue puesta en duda por Gettier en un famoso artículo \citeyear{Gettier}. La conclusión de Gettier es que la caracterización de Platón no ofrece condiciones suficientes para decir que S sabe que `p'.

Pero el análisis de Gettier sólo aplica a uno de los lados del condicional material. Si bien las condiciones no son suficientes, esto no señala un problema con que dichas condiciones sean necesarias. Aceptamos, por ejemplo, que la verdad es una condición necesaria para saber algo, porque el conocimiento es fáctico. Si sabes que hay un gato sobre la alfombra, entonces hay un gato sobre la alfombra. Hablamos de creencias falsas, pero no de conocimiento falso. El ``saber que p'' tiene una estrecha conexión con la verdad y por lo general admitimos que conocimiento implica verdad.

Siguiendo esta línea de razonamiento, muchos epistemólogos han defendido que la verdad es una parte fundamental del conocimiento. Pritchard, por ejemplo, señala  

\begin{quote}
It wasn’t all that long ago that the idea that truth is the fundamental epistemic good was orthodoxy in epistemology. Indeed, this was the kind of claim that was so commonplace that it was almost not worth stating, as to do so would be somewhat superfulous. \cite[p. 1]{Pritchard2021}
\end{quote}

Pritchard nos dice que esta conexión que tiene el conocimiento con la verdad se debe a que valoramos \textbf{epistémicamente} a la verdad. De alguna manera tenemos que ser capaces de corregir creencias y quizás la forma más intuitiva de señalar esto es que el mundo impone ciertos contreñimientos sobre el conocimiento humano. Gila Sher acuñó el término ``fricción epistémica'' para describir esta relación \citeyear{Sher2016}. Una tarea similar es la que lleva a cabo Blåsjö en \citeyear{Blaasjoe2022} donde señala que los geómetras griegos estaban comprometidos con un programa operacionalista: las construcciones geométricas son cosntrucciones físicas concretas, que además representan cómo es el mundo. Según el autor, esto además servía para evitar contradicciones. Por el momento asumamos esta tesis. No como una definición de verdad, sino del papel que juega la verdad en nuestras empresas epistémicas.

A pesar del peso intuitivo que tiene la tesis del valor epistémico de la verdad, varios filósofos han presentado problemas en contra de esta tesis. Entre estos problemas se encuentra el \textit{problema de la absorción} [swamping problem] y el problema de las \textit{verdades irrelevantes}. 

Pritchard explica ambos problemas de la siguiente manera. El problema de la absorción indica que la verdad es valiosa porque es epistémicamente útil. Si esto es verdad, entonces que valoremos la verdad es parasitario a la utilidad que esta nos brinda y dicha utilidad no hace diferencia entre conocimiento y creencia verdadera. El problema de las verdades irrelevantes establece que si la verdad es el valor fundamental, entonces ante dos verdades cualesquiera, no podríamos elegir cuál deberíamos creer -- por ejemplo hay una respuesta verdadera sobre el número total de granos de arena en Cancún y una respuesta verdadera sobre si la luna gira alrededor de la tierra, pero no queda claro que queramos saber la respuesta a la primer pregunta. Dado que queremos distinguir entre las respuestas que son importantes de las que no, entonces la verdad no es lo único importante para el conocimiento.

Otros filósofos señalan que en nuestras empresas epistémicas no están dirigidas a la verdad, sino a otros valores epistémicos, por ejemplo, al entendimiento de un fenómeno \cite{Elgin2004}; otros como Papineau \cite{Papineau2021} defienden que deberíamos olvidarnos del conocimiento, y decantarnos por la pura creencia verdadera. Pritchard considera que ambas estrategias son demasiado apresuradas.

Hasta ahora he descrito una tesis filosófica: que el conocimiento implica verdad. Señalé un par de razones de por qué la tesis es interesante, además de su poder intuitivo. Indiqué un par de argumentos que se han dirigido en contra de esta tesis. A continuación presentaré los argumentos de Pritchard para defender la tesis de los contraargumentos que se han esgrimido en contra de que la verdad es el valor epistémico fundamental.

\subsection{Contra ``la verdad no es el valor fundamental''}

Para dar una respuesta a estos problemas, Pritchard nos dice que cada uno de estos problemas depende de asumir que el objeto de investigación epistémica son las \textit{proposiciones}. Si negamos esta suposición, ¿qué forma tendría que tomar una teoría del conocimiento que tenga como valor fundamental a la verdad?

Antes de continuar con la exposición de Pritchard quisiera hacer un par de aclaraciones. La condición de 'verdad' involucrada en el conocimiento es de un tipo diferente a las de creencia y justificación. Tanto creencia como justificación son condiciones epistémicas; la verdad, por otra parte, es una condición de otro tipo. Al menos creemos comúnmente que la verdad no es una condición epistémica, si lo fuera, nuestras creencias podrían modificar cómo es el mundo [Habrá una parte dedicada a teorías de la verdad]. El que yo crea que mi cortina es verde\footnote{De hecho es azul.}, no la hace verde. Adecuamos nuestras creencias para que encajen con el fenómeno que queremos comprender\footnote{En \cite{sep-knowledge-analysis} los autores señalan que esta es una condición metafísica. Trato de ser neutro con respecto a si es una condición metafísica, porque si asumimos una teoría pragmática de la verdad (al estilo de William James o Richard Rorty), entonces la verdad es una condición epistémica. Como esto es apenas una presentación, evito el compromiso. Por ahora seguiré trabajando con la intuición señalada en \ref{enunc}}.  

Continuemos con la exposición de Pritchard \cite{Pritchard2021}. Pritchard nos dice que al involucrar a la teoría de la virtud epistémica, negar que el objeto epistémico por antonomasia es la proposicón y asumir a la verdad como el valor epistémico fundamental, entonces somos capaces de resolver los problemas antes mencionados. Pritchard además usa un marco de la teoría de las virtudes epistémicas. A grandes rasgos, las virtudes epistémicas, se entienden a la manera en como Aristóteles entendía la virtud: son deliberaciones que hacemos para llegar a un fin, estas por supuesto son decisiones voluntarias que se practican de acuerdo con el fin que queremos lograr\footnote{``Con relación a las mismas cosas son, pues, el cobarde, el temerario y el valiente, pero se conducen diferentemente a su respecto. Aquéllos pecan por exceso y por defecto, en tanto que éste guarda el medio y el deber.'' \cite{aristotelesnico} Especialmente los capítulos 2-8 del libro III.}. 

Uno de los pioneros de la teoría de la virtud epistémica es Ernst Sosa. Sosa \citeyear{Sosa2017-SOSE} articula una teoría del conocimiento que caracteriza a las virtudes epistémicas como el ejercicio de ciertas habilidades de los agentes tales que al ejercitarlas, constituyen el concepto de conocimiento. Sosa también señala que si bien el fin de las empresas epistémicas es la verdad, esto no es lo que valoramos como agentes: lo que valoramos es el \textbf{concocimiento}. Sosa \citeyear[pp. 11-113]{Sosa2017-SOSE} propone que el conocimiento, a diferencia de la creencia verdadera tiene las siguientes características: no sólo valoramos que el conocimiento sea certero (característica que comparte con la creencia verdadera), sino que valoramos que sea certero porque depende de las habilidades del agente; por último, valoramos que el conocimiento sea certero porque el agente llegó al objetivo debido a sus habilidades. Estas tres condiciones son las que constituyen la propiedad de que el conocimiento es apto. Asumiendo esta caracterización del conocimiento, Sosa argumenta que el problema de la absorción y de las verdades irrelevantes deja de tener sentido ya que si bien la verdad juega un papel importante en el conocimiento, el valor de la verdad esta dado por el conocimiento asociado a dicha verdad.

Si bien Sosa señala que la verdad no es en lo que recae el valor del conocimiento, no señala que la verdad no sea una condición necesaria del conocimiento, al menos el concepto de certeza involucra una relación entre creencias y mundo, que a primera vista no es diferente de la versión intuitiva que estamos usando. Esto hasta ahora no contradice a Pritchard en que la verdad es fundamental en nuestras empresas epistémicas. Los dos están de acuerdo en que el valor del conocimiento no depende del sólo hecho de que algo sea verdad, sino de que el proceso para obtenerla descanse en las virtudes del agente. Pritchard también señala que cambiar el valor fundamental del conocimiento por algún otro como la comprensión es un cambio demasiado apresurado. Esto se debe a que la persona que sepa ciertas verdades, tendrá más verdades en su vecindario de creencias que aquel que sólo tiene una creencia verdadera.

Con este marco explicado, Pritchard resuelve los dos problemas que se le achacan al veritismo. Con el problema de la absorción nos presenta una analogía con un chef (p. 6 ). La analogía es la siguiente: que un chef haga comida deliciosa y luego la pruebe para saber si lo es, no significa que su interés era probarla y no hacerla deliciosa. El probarla sólo es una manera de asegurarse que es deliciosa. Si es verdad que el valor fundamental de cualquier empresa epistémica es la verdad, eso explicaría por qué nos interesa obtener verdades: si bien esto tiene implicaciones prácticas, cualquier consecuencia depende de que de hecho nuestro conocimiento apunte a la verdad y la consiga. Más aún, las virtudes intelectuales nos ofrecen un marco para solventar el problema de las verdades irrelevantes: queremos no sólo que nuestras creencias sean verdaderas, sino que además el proceso para llegar a la verdad sea fiable y que esté guiado por características de un agente. Un agente virtuoso puede sopesar entre dos verdades: una irrelevante, la otra de más peso.

Si podemos resolver estos problemas para el veritismo, entonces no es claro que debamos abandonar a la verdad como una condición necesaria para el conocimiento, menos aún como el valor fundamental del conocimiento.

%Estoy de acuerdo con que la verdad es una condición necesaria del conocimiento.

\subsection{Veritismo y dos casos históricos.}

\noindent Hasta este punto desarrollé los argumentos de Pritchard. Lo anterior es evidencia a favor de que la verdad es una condición necesaria para el conocimiento. Si no lo fuera, muchas de nuestras empresas epistémicas no tendrían sentido, por que no podríamos corregir nuestras creencias con base en nueva evidencia, porque la verdad es externa a los agentes. Incluso si tomamos a la creencia verdadera como básica, la verdad sigue inmiscuida: la verdad es una propiedad que nos asegura que vamos por el camino correcto. Las explicaciones y demás consecuencias del conocimiento también dependen de que la verdad sea una condición necesaria. Más aún, las virtudes intelectuales nos ofrecen un marco para solventar problemas como el de las verdades irrelevantes: queremos no sólo que nuestras creencias sean verdaderas, sino que además el proceso para llegar a la verdad sea fiable y que esté guiado por características loables de un agente. Un agente virtuoso puede sopesar entre dos verdades: una irrelevante, la otra de más peso.

Estoy de acuerdo con lo que señala Pritchard de que la verdad es una condición necesaria del conocimiento (dejemos por el momento si la verdad es una motivación), tanto así que es valorada por los investigadores. Este marco, incluso encaja bien con cómo podemos evaluar la investigación científica a la luz de la teoría presentada, dadas las características de los agentes y dado que sus resultados son aptos. 

Sin embargo, defender la tesis descrita hasta ahora se vuelve complicado cuando tratamos de empatarla con la práctica científica. La investigación científica es probablemente la manera más sistematizada que tenemos los seres humanos para producir conocimiento. Muchas de nuestras explicaciones dependen de conocer algunos hechos acerca de las diferentes disciplinas científicas. Sabemos, por ejemplo, que para que haya una combustión se necesita combustible y oxígeno. Si cualquiera de estas variables está en 0, entonces no hay combustión. Sabemos, por tanto, que sin combustible u oxígeno, no hay combustión. Esto es una explicación, sabemos que es una buena explicación porque es verdad. Una explicación que utilice información falsa no es una buena explicación. Considérese, por ejemplo, que tiro mi taza de café al piso. Y que explico esto en base a señalar (falsamente) que justo antes de que mi brazo golpeara la taza, el viento la empujo y fue esta la razón por la que cayó al piso. Esto por supuesto es una mala explicación. De manera sucinta, las teorías y explicaciones tienen que representar adecuadamente el mundo.

Supongamos que las explicaciones verdaderas son las únicas explicaciones que nos interesan. 

Podemos incluso citar el uso contemporáneo de teorías. Por ejemplo, la mecánica Newtoniana y la teoría de la selección natural de Darwin, son teorías que usamos cotidianamente para explicar fenómenos. La teoría de la selección natural nos ayuda a explicar fenómenos biológicos como la adaptación y la especiación. Con base en esta teoría, podemos explicar, por ejemplo, por qué un grupo dentro de una población tiene más descendencia que otro grupo dentro de la misma población; podemos también saber, por ejemplo, cuáles son los ancestros comunes de especies contemporáneas, \textit{e. g.}, las aves de los dinosaurios. Por su parte, utilizamos la teoría newtoniana para explicar el movimiento de los astros y hacer predicciones de qué posición tomarán en un momento dado, podemos explicar la fuerza que se imprime en una superficie cuando es golpeada por una masa con cierta aceleración y nos sirve para explicar el movimiento de objetos apelando a la inercia.

Esto por supuesto son grandes ventajas de las teorías: nos permiten explicar fenómenos. A pesar de que estas teorías nos permiten explicar una amplia variedad de fenómenos, no es claro que sean verdaderas. En lo siguiente trataré de presentar la teoría newtoniana y exponer algunos de los problemas que varios investigadores han detectado en la teoría.

A pesar de que ambas teorías nos permiten explicar un amplio rango de fenómenos, las dos tienen problemas. Sabemos que la distinción newtoniana entre ``tiempo'' y ``espacio'' es falsa. Para poder explicar el movimiento, Newton desarrolló una teoría que tiene como entidades el \textbf{tiempo absoluto} y el \textbf{espacio absoluto}. Un cuerpo se mueve o permanece en reposo con respecto al espacio y tiempo absolutos. Todos los puntos en el espacio absoluto permanecen constantes durante diferentes intervalos temporales. Para explicar la llamada ``primera ley''  de Newton (inercia), es necesario definir qué significa que un cuerpo esté en reposo o en movimiento. Según Newton sabemos que un cuerpo está en movimiento, porque ocupa distintos puntos del espacio en diferentes intervalos de tiempo. Si el movimiento entre los puntos se da en intervalos iguales de tiempo, entonces el movimiento es uniforme. Como señala Freudenthal `` The distinction between 'rest' and 'uniform motion' implies, however, an absolutely resting frame of reference, and this can only be absolute space'' \citeyear{Freudenthal1986}

El movimiento en el espacio absoluto, sin embargo, no puede ser percibido, mientras que las posiciones relativas de los cuerpos sí. Esto, por supuesto no significa que haya que echar a la basura ambas entidades, ya que el espacio y tiempo absolutos juegan un papel en la explicación del movimiento newtoniana y dicha explicación recupera muchos de los fenómenos que buscamos explicar.

Ambas entidades son parte fundamental de la teoría newtoniana, ya que permiten explicar los movimientos relativos \cite{Maudlin2014Filosofia:7985}. Señalé que la teoría newtoniana sugiere la existencia de una entidad de la que no es claro que tengamos certeza de que existe. Algunos teóricos como Leibniz sugirieron que algo extraño estaba sucediendo en la física newtoniana: si sólo tenemos evidencia de los movimientos relativos y un cuerpo en el espacio absoluto no se mueve con respecto a nada, entonces no tenemos evidencia de que exista el espacio absoluto. Newton en la \textit{principia} presenta dos experimentos mentales para señalar la existencia del espacio absoluto. Uno de ellos es en el cual señala que si atamos dos globos con un cordón y los hacemos rotar sobre su eje en direcciones contrarias, la cuerda se tensa. Si aceptamos que es verdad que la cuerda se tensa incluso en el espacio absoluto, entonces tenemos que aceptar que sí hay movimiento en el espacio absoluto, los globos se mueven con respecto a dicho marco, aunque estén en reposo uno respecto del otro \cite[pp. 6-12]{book:360820}.

Leibniz no fue el único en sospechar de la teoría newtoniana. Físicos como Ernst Mach (véase especialmente el capítulo 2) \citeyear{Mach2013} señala que debido a que diferentes marcos de referencia inerciales tienen las mismas consecuencias empíricas que hablar de espacio absoluto, entonces no es necesario apelar a dicha entidad, algo que sabemos gracias a la relatividad galileana\footnote{Por supuesto, Newton no hablaba de un marco privilegiado, sino que se refería al espacio absoluto como una entidad física. Pero lo que es importante notar es que podemos tener las mismas consecuencias sin apelar a dicha entidad.}.

Más aún, la relatividad señala que donde Newton distinguía dos entidades, realmente sólo hay una: espacio-tiempo. Incluso en la teoría de la relatividad, no es necesario apelar al tiempo absoluto para dar explicaciones. En la relatividad no hay un marco privilegiado y como consecuencia de la estructura del espacio-tiempo tenemos la dilatación temporal.

A partir de lo dicho aquí podemos obtener dos conclusiones. La primera de ellas es que la teoría de Newton es muy útil: nos permite describir una variedad de fenómenos desde la aceleración en caída libre de cuerpos en la tierra, hasta el movimiento de los cuerpos celestes. Sin embargo teorías físicas más recientes niegan que haya \textbf{espacio absoluto}, incluso señalan que la distinción entre dos entidades: espacio y tiempo, no es la más adecuada para describir la estructura del mundo. Por lo cuál podemos concluir que la teoría de Newton es falsa. La segunda conclusión es que la verdad es una condición necesaria del conocimiento. Si no tuviéramos explicaciones verdaderas, simplemente no tenemos explicación. 

Aquí hay algo extraño: ¡ambas conclusiones no pueden ser verdaderas! Por lo que será necesario o bien dar un par de razones a favor de cómo ambas conclusiones pueden ser compatibles, o bien negar alguna de las conclusiones a las que hemos llegado. Si tomamos la segunda estrategia, dos opciones se abren ante nosotros: o bien la teoría newtoniana sí es verdadera después de todo, o bien la verdad no es una condición necesaria del conocimiento.

Tomemos el segundo disyunto: que la verdad no es una condición necesaria para el conocimiento. El argumento que ofrece Pritchard, y que expusimos líneas arriba, señala que la verdad es una condición necesaria porque de otra manera no podríamos explicar muchas de nuestras metas cognitivas, su marco tiene la ventaja de resolver los problemas con los que comúnmente se embiste esta tesis. El primer problema que encuentro con este marco es la afirmación de que ``las virtudes epistémicas son aquellas que son constitutivas del conocimiento'', algo con lo que tanto Pritchard como Sosa están de acuerdo\footnote{Por supuesto la tesis que señalan ambos no es exactamente la misma. Sosa afirma que el objetivo de nuestras empresas epistémicas es el conocimiento, mientras que Pritchard parece sugerir que la finalidad es la verdad adquirida por un agente virtuoso. Sin embargo, recordemos que ambos afirman que la verdad es una condición necesaria del conocimiento, que es lo que está en cuestión aquí. } es con que la verdad es una condición necesaria para el conocimiento.

Lo anterior son razones para señalar que la verdad sí es una condición necesaria del conocimiento. Sin embargo, el argumento de las virtudes es problemático porque no  hay duda de que Newton es un agente virtuoso, sin embargo, su uso de las virtudes no lo llevó a la verdad (que es una de las condiciones necesarias para el conocimiento). Si esto es verdad, entonces habría que negar que las virtudes epistémicas son constitutivas del conocimiento.

%Si bien podemos negar que las virtudes epistémicas son constitutivas del conocimiento, el marco de Sosa no es el único marco de epistemología de la virtud.

Por supuesto, esto sólo es un problema para la teoría de las virtudes, y no para la tesis de que la verdad es una condición necesaria del conocimiento. Supongamos que la verdad es una condición necesaria del conocimiento y que el conocimiento es una condición necesaria para la ciencia. Si esto es verdad, entonces habría que señalar cómo a pesar de los problemas detectados, la teoría de Newton es verdadera a pesar de todo. Con la finalidad de no descartar que la verdad es una condición necesaria del conocimiento. Pero esto también es problemático para cualquier teoría epistémica que acepte la tesis de que la verdad es una condición necesaria del conocimiento. La teoría de Newton funciona, aunque no es claro que debamos asumir que es verdadera. A pesar de ello explica muy bien, lo que parece indicar que necesitamos esclarecer cómo una teoría falsa puede dar explicaciones verdaderas.

Lo dicho anteriormente, por supuesto, se relaciona con el problema del realismo científico. De manera muy sucinta y poco precisa, los realistas científicos defienden la tesis de que las teorías científicas son literalmente verdaderas. Esto implica que las entidades que aparecen en la teoría de hecho existen y que las relaciones entre objetos que señala la teoría reflejan cómo de hecho es el mundo.

Esta tesis tiene muchas aristas. En primer lugar hay un aspecto epistémico involucrado. Si saber implica verdad, entonces el hecho de que alguien señalé que Newton sabía que el espacio absoluto existe, entonces el espacio absoluto de hecho existe. Otro aspecto del realismo científico es semántico. Saber si efectivamente los términos individuales que son parte de las oraciones de una teoría, de hecho refieren a un objeto. Por último, está el aspecto ontológico. Este último aspecto del realismo consiste en determinar si los objetos de la teoría de hecho existen.

Si bien todos estos problemas están relacionados (por ejemplo determinar que el espacio absoluto de hecho existe, haría que las oraciones donde aparece dicho término individual refiriera y que alguien que sepa la teoría sabe que el tiempo absoluto existe), vale la pena tener estos aspectos separados. 

Lo que he señalado hasta ahora está relacionado sólo con el problema epistémico. Nuestro problema es que no es claro que la teoría de Newton sea verdadera, ya que nuevas teorías han mostrado ser mejores descripciones del mundo que la teoría newtoniana, por tanto, Newton no sabía que el espacio absoluto existiese. Esto es sólo una manera de exponer algo que Laudan ya había señalado: la historia de la ciencia ha mostrado que los términos individuales de teorías exitosas no siempre refieren, por lo que dichas teorías son falsas \cite{Laudan1981}. Creo que la moraleja que nos da Laudan es que debemos ser cautelosos al formular una tesis realista sobre la ciencia. 

Pero hemos llegado de nuevo al punto inicial. Si la hipótesis del espacio absoluto es falsa, y la teoría newtoniana depende de dicha entidad y la estructura es de derivación, entonces los teoremas extraídos de dichas hipótesis son falsas. Por lo que nadie sabría la teoría newtoniana. Más aún, si suponemos que las buenas explicaciones son explicaciones verdaderas, entonces no podríamos explicar nada con la teoría newtoniana. Hay que desecharla por alguna teoría física más moderna.

Pero lo anterior claramente es falso, la teoría de Newton es explicativa. Tal como señala Laudan, hay teorías explicativas que postulan entidades falsas: aún cuando la teoría depende de dichas entidades, sigue siendo explicativa. El problema ahora está a nivel ontológico y relacionado con el argumento a la mejor explicación: del hecho de que una teoría sea explicativa, no podemos pasar a que las entidades postuladas de hecho existan. Recordemos que uno de los argumentos más usados para defender el realismo el así llamado ``argumento del no-milagro'' depende de la inferencia a la mejor explicación\footnote{No todos los argumentos a favor del realismo dependen de dicho tipo de inferencia. Si bien el argumento del ``no-milagro'' depende de aceptar que esta inferencia es válida, no quiere decir que este sea el único argumento a favor del realismo científico. Por otro lado, no es necesario negar que hay que desechar este tipo de inferencia. Cabe destacar que este tipo de inferencia no asegura que de premisas verdaderas pasemos a una conclusión verdadera, por lo que una manera de defender el realismo científico es hacer que las inferencias de este tipo sean más ``robustas'', es decir, que podamos asegurar la existencia de los objetos de acuerdo a una inferencia no-deductiva. Por supuesto, aún cuando podamos hacer esto todavía queda margen de error ya que estas inferencias son falibles. Para una exposición más detallada véase \cite{Saatsi2010-SAAFVC-2}.}. Este argumento no es deductivamente válido, por lo que siempre hay lugar para el error. Si esto es verdad, entonces el realista que dependa del argumento de los no-milagros está en problemas para justificar su tesis.

Parece entonces que el problema no está en cómo justificamos creencias, sino en el hecho de que esperamos demasiado de la justificación que podemos ofrecer y de cómo se conecta con la verdad. Incluso un argumento deductivo perfectamente válido puede no tener conclusión verdadera debido a que una de las premisas es falsa. Difícilmente en ciencias empíricas podemos ofrecer un grado de certeza del 100\%. Los métodos más utilizados, entre ellos las herramientas estadísticas, no nos dan ese grado de certeza. Aún cuando deseemos que toda la información que usemos en una inferencia sea verdadera y, por tanto, nuestras conclusiones sean verdaderas (utilizando métodos deductivos) si nuestras premisas son conclusiones de un argumento no-deductivo, entonces no hay manera de asegurar tal certeza.

%Alguien podría señalar que si bien la verdad no es algo que podamos obtener (en ciencias empíricas), sin duda es una motivación para los investigadores. Pero esto también es falso, véase por ejemplo el artículo de Boris Hessen (al menos creo que es una forma de obtener la negación de la hipótesis).

Pero entonces qué teoría de la verdad podríamos adoptar tal que no incluyamoLa teoría correspondentista de la verdad tiene una carga intuitiva muy fuerte, pero difícilmente da lugar a los errores que hay en la historia de la ciencia. Por otro lado, no es claro cómo podemos incluir proposiciones que sólo son plausibles en los razonamientos (como aquellas conclusiones de argumentos no-deductivos). Entonces, ¿deberíamos deshacernos de la verdad como una condición del conocimiento?

Si defendemos algo como lo qeu sugiere esta última pregunta

%La verdad juega un papel mediador. Si la verdad es un concepto que no involucra propiedades epistémicas, entonces tiene sentido que podamos corregir nuestras creencias con base en nuevos descubrimientos. Si tiene propiedades epistémicas, entonces la verdad cambia  en tanto nuestras creencias cambian. Esto último no sucede, de ser así, podríamos modificar a conveniencia el color, tamaño y forma de los objetos. Entonces la verdad no involucra propiedades epistémicas. 

% Habrá qué decir a este respecto de cómo la verdad por correspondencia es sumamente intuitiva cuando hablamos de conocimiento cotidiano, mientras que en ámbitos donde tenemos que evaluar entidades no-observables es más difuso cómo juega el mismo papel. PAra esto leer el artículo "A note on truth and reference" de Penelope Maddy. Aunque MAddy señala que podemos decir que no hay un concepto homogéneo de verdad, sino que en contextos cotidianos una teoría correspondentista de la verdad podría ser útil, mientras que en un contexto de investigación no parece muy útil, no me queda claro cómo defender esto. No me queda claro porque muchas de nuestras explicaciones cotidianas dependen de nuestras "empresas epistémicas". Por ejemplo explicar por qué el cielo es azul, o por qué mi cultivo murió ¿es conocimiento cotidiano o de investigación? ¿Cómo hacemos ese corte? Los ejemplos pueden ser ilustrativos "Consider, for example, Hartry Field’s story of the  ancient Greek whose directivities for ‘Zeus is throwing thunderbolts’ correlate neatly with the worldly support of lightning; his utetrance would serve  as a warning to others." MAddy dice que para decir que estos casos son claramente falsos debemos hablar de qué correlaciones vamos a tomar como genuinamente referenciales.  Aunque también menciona que "Here I’m not  out to define ‘truth’ or to provide any theoretical account at all; the whole  point of classifying ‘true’ and ‘refers’ with ‘excuses’ and ‘know’ rather than  with ‘time’ and ‘hardness’ is to deny that they’re things to be theorized about  rather than words whose complex usage is to be studied." Ahora creo que lo qeu defiende MAddy no es un pluralismo alético, sino una nueva manera de investigar el uso de la palabra verdad, que no es homogénea en todo contexto, ya que otras palabras pueden describir mejor la situación que sólo ser verdaderas. 



Otra postura que evita los problemas relacionados con el conocimiento (y con ello la verdad) involucrados en la investigación científica, es el antirrealismo. El argumento de Laudan me parece contundente en el debate entre realistas y antirrealistas. Dado lo expuesto hasta ahora, vale la pena tomar esta postura en serio.

Hasta ahora he descrito una gran variedad de tesis. 










\bibliographystyle{apacite}
\bibliography{tesisdocto.bib}




\end{document}
