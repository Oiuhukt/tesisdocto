\documentclass[12pt]{article}
\usepackage{setspace}
\setstretch{1.5}
\usepackage[utf8]{inputenc}
\usepackage[T1]{fontenc}
\usepackage{url}
\usepackage[spanish]{babel}
\usepackage{apacite}
\usepackage{fullpage}

\begin{document}

\section{Idea 2}

\noindent A los seres humanos nos interesa la verdad. Nos sentimos, por ejemplo, molestos cuando alguien incumple una promesa. Esta importancia que le damos a la verdad guía también nuestras empresas epistémicas\footnote{utilizo este concepto de empresas epistémicas como algo muy general y no bien definido. El concepto puede ser usado para describir tanto a la investigación científica, como a la curiosidad de un niño que está aprendiendo a sumar.}. Platón en su diálogo con Menón \cite[\P\P 97a-98b]{platonmeno} señala que hay un factor de seguridad ligado al conocimiento, que no se encuentra en la mera opinión verdadera\footnote{Por supuesto esto está más relacionado con la justificación que con la verdad propiamente. Sin embargo, cabe señalar que la verdad juega un papel importante en la explicación de lo que implica saber algo.}. Esta caracterización de conocimiento fue puesta en duda por Gettier en un famoso artículo \citeyear{Gettier}. La conclusión de Gettier es que la caracterización del conocimiento como ``creencia verdadera justificada'' no ofrece condiciones suficientes para decir que S sabe que `p'.

Pero el análisis de Gettier sólo aplica a uno de los lados del condicional material. Si bien las condiciones no son suficientes, esto no señala un problema con que dichas condiciones sean necesarias. Aceptamos, por ejemplo, que la verdad es una condición necesaria para saber algo. Si algo es falso, entonces no lo sabes. Hablamos de creencias falsas, pero no de conocimiento falso. El ``saber que p'' tiene una estrecha conexión con la verdad y por lo general admitimos que conocimiento implica verdad.

Siguiendo esta línea de razonamiento, muchos epistemólogos han defendido que la verdad es una parte fundamental del conocimiento. Pritchard, por ejemplo, señala  

\begin{quote}
It wasn’t all that long ago that the idea that truth is the fundamental epistemic good was orthodoxy in epistemology. Indeed, this was the kind of claim that was so commonplace that it was almost not worth stating, as to do so would be somewhat superfulous. \cite[p. 1]{Pritchard2021}
\end{quote}

Pritchard señala que esta conexión que tiene el conocimiento con la verdad se debe a que valoramos epistémicamente a la verdad. Varios problemas se han presentado en contra de la tesis de que el valor epistémico fundamental es el de la verdad. Entre estos problemas se encuentra el problema de la absorción \textit{swamping problem} y el problema de las \textit{verdades irrelevantes}. El problema de la absorción señala que la verdad es valiosa porque es epistémicamente útil. Si esto es verdad, entonces que valoremos la verdad es parasitario a la utilidad que esta nos brinda. El problema de las verdades irrelevantes señala que si la verdad es el valor fundamental, entonces ante dos verdades que tienen el mismo peso, no podríamos elegir con cuál deberíamos permanecer. Otro problema, que no depende de las consecuencias de asumir que la verdad es el valor epistémico fundamental, señala que en nuestras empresas epistémicas no están dirigidas a la verdad, sino a otros valores epistémicos, por ejemplo, al entendimiento de un fenómeno; otros como Papineau \cite{Papineau2021} defienden que deberíamos olvidarnos del conocimiento, y decantarnos por la pura creencia verdadera.

Para dar una respuesta a estos problemas, Pritchard señala que cada uno de estos problemas depende de asumir que el objeto de investigación epistémica son las \textit{proposiciónes}. Si negamos esta suposición, ¿qué forma tendría que tomar una teoría del conocimiento que tenga como valor fundamental a la verdad?

Antes de continuar con la exposición de Pritchard quisiera hacer un par de aclaraciones. La condición de 'verdad' involucrada en el conocimiento es de un tipo diferente a las de creencia y justificación. Tanto creencia como justificación son condiciones epistémicas; la verdad, por otra parte, es una condición de otro tipo. Al menos creemos comúnmente que la verdad no es una condición epistémica, si lo fuera, nuestras creencias podrían modificar cómo es el mundo. El que yo crea que mi cortina es verde\footnote{De hecho es azul.}, no la hace verde. Adecuamos nuestras creencias y justificaciones para que encajen con el fenómeno que queremos comprender\footnote{En \cite{sep-knowledge-analysis} los autores señalan que esta es una condición metafísica. Trato de ser neutro con respecto a si es una condición metafísica, porque si asumimos una teoría pragmática de la verdad (al estilo de William James ), entonces la verdad es una condición epistémica. Como esto es apenas una presentación, evito el compromiso.}.  

Ahora bien, continuemos con la exposición de Pritchard \cite{Pritchard2021}. Pritchard nos dice que al involucrar a la teoría de la virtud epistémica, negar que el objeto epistémico por antonomasia es la proposicón y asumir a la verdad como el valor epistémico fundamental, entonces somos capaces de resolver los problemas antes mencionados. Estas hipótesis nos permiten generan una teoría del conocimiento centrada en virtudes. A grandes rasgos, las virtudes epistémicas, se entienden a la manera en como Aristóteles entendía la virtud: las virtudes son deliberaciones que hacemos para llegar a un fin, estas por supuesto son decisiones voluntarias que se practican de acuerdo con el fin que queremos lograr\footnote{``Con relación a las mismas cosas son, pues, el cobarde, el temerario y el valiente, pero se conducen diferentemente a su respecto. Aquéllos pecan por exceso y por defecto, en tanto que éste guarda el medio y el deber.'' \cite{aristotelesnico} Especialmente los capítulos 2-8 del libro III.}. Sosa \citeyear{Sosa2017-SOSE} articula una teoría del conocimiento que caracteriza a las virtudes epistémicas  como el ejercicio de ciertas habilidades de los agentes tales que al ejercitarlas, constituyen el concepto de conocimiento. 

Sosa también señala que si bien el fin de las empresas epistémicas es la verdad, esto no es lo que valoramos como agentes: lo que valoramos es el \textbf{concocimiento}. Sosa \citeyear[pp. 11-113]{Sosa2017-SOSE} propone que el conocimiento, a diferencia de la creencia verdadera tiene estas características: no sólo valoramos que el conocimiento sea certero (característica que comparte con la creencia verdadera), sino que valoramos que sea certero porque depende de las habilidades del agente; por último, valoramos que el conocimiento sea certero porque el agente llegó al objetivo debido a sus habilidades: el conocimiento es apto. Asumiendo esta caracterización del conocimiento, Sosa argumenta que el problema de la absorción y de las verdades irrelevantes deja de tener sentido ya que si bien la verdad juega un papel importante en el conocimiento, el valor de la verdad esta dado por el conocimiento asociado a dicha verdad.

Sosa concuerda con Pritchard en que la verdad es fundamental en nuestras empresas epistémicas. Los dos están de acuerdo en que el valor del conocimiento no depende del sólo hecho de que algo sea verdad, sino de que el proceso para obtenerla descansa en las virtudes del agente. Pritchard también señala que cambiar el valor fundamental del conocimiento por algún otro como la comprensión es un cambio demasiado apresurado. Esto se debe a que la persona que sepa ciertas verdades, tendrá más verdades en su vecindario de creencias que aquel que sólo tiene una creencia verdadera.

Para el problema de la absorción, Pritchard hace una analogía con un chef (p. 6 ). La analogía es la siguiente:  que un chef haga comida deliciosa y luego la pruebe para saber si lo es, no significa que su interés era probarla y no hacerla deliciosa. El probarla sólo es una manera de asegurarse que es deliciosa. Si es verdad que el valor fundamental de cualquier empresa epistémica es la verdad, eso explicaría por qué nos interesa obtener verdades: si bien esto tiene implicaciones prácticas, cualquier consecuencia depende de que de hecho nuestro conocimiento apunte a la verdad y la consiga. 

Lo presentado anteriormente es un marco teórico presentado por Pritchard, tomando elementos de la teoría de la virtud epistémica de Sosa. Estoy de acuerdo con que el valor epistémico fundamental es la verdad. Si no lo fuera, muchas de nuestras empresas epistémicas no tendrían sentido: incluso si tomamos a la creencia verdadera como básica, la verdad sigue inmiscuida. Las explicaciones y demás consecuencias del conocimiento, también dependen de que la verdad sea o condición necesaria. Además, si la verdad es algo que es externo a los agentes, funciona como un objeto con el que podemos corregir nuestras creencias. Más aún, las virtudes intelectuales nos ofrecen un marco para solventar problemas como el de las verdades irrelevantes: queremos no sólo que nuestras creencias sean verdaderas, sino que además el proceso para llegar a la verdad sea fiable y que esté guiado por características loables de un agente. Un agente virtuoso puede sopesar entre dos verdades: una irrelevante, la otra de más peso.

Estoy de acuerdo con lo que señala Pritchard de que la verdad es un fin (dejemos por el momento si la verdad es una motivación) que valoran los investigadores. Este marco, incluso encaja bien con cómo podemos evaluar la investigación científica a la luz de la teoría presentada. Sin embargo, la noción de ``verdad'' es más complicada cuando tratamos de empatarla con la práctica científica. La investigación científica es probablemente la manera más sistematizada que tenemos los seres humanos para producir conocimiento. Muchas de nuestras explicaciones dependen de conocer algunos hechos acerca de las diferentes disciplinas científicas. Sabemos, por ejemplo, que para que haya una combustión se necesita combustible y oxígeno. Si cualquiera de estas variables está en 0, entonces no hay combustión. Sabemos, por tanto, que sin combustible u oxígeno, no hay combustión. Esto es una explicación, sabemos que es una buena explicación porque es verdad. Una explicación que utilice información falsa no es una buena explicación. Considérese, por ejemplo, que tiro mi taza de café al piso. Y que explico esto en base a señalar que justo antes de que mi brazo golpeara la taza, el viento la empujo y fue esta la razón por la que cayó al piso. Esto por supuesto es una mala explicación.

Podemos incluso citar el uso contemporáneo de teorías, por ejemplo, la mecánica Newtoniana y la teoría de la selección natural de Darwin, son teorías que usamos cotidianamente para explicar fenómenos. La teoría de la selección natural nos ayuda a explicar fenómenos biológicos como la adaptación y la especiación. Con base en esta teoría, podemos explicar, por ejemplo, por qué un grupo dentro de una población, tiene más descendencia que otro grupo dentro de la misma población; podemos, también saber, por ejemplo, cuáles son los ancestros comunes de especies contemporáneas, \textit{e. g.}, las aves de los dinosaurios. Por su parte, utilizamos la teoría newtoniana para explicar el movimiento de los astros y hacer predicciones de qué posición tomarán en un momento dado, podemos explicar la fuerza que se imprime en una superficie cuando es golpeada por una masa con cierta aceleración y nos sirve para explicar el movimiento de objetos apelando a la inercia.

A pesar de que ambas teorías nos permiten explicar un amplio rango de fenómenos, las dos tienen problemas. Sabemos que la distinción newtoniana entre ``tiempo'' y ``espacio'' es falsa. Para poder explicar el movimiento, Newton desarrolló una teoría que tiene como entidades el \textbf{tiempo absoluto} y el \textbf{espacio absoluto}. Un cuerpo se mueve o permanece en reposo con respecto al espacio y tiempo absolutos. Todos los puntos en el espacio absoluto permanecen constantes durante diferentes intervalos temporales. Para explicar la llamada ``primera ley''  de Newton (inercia), es necesario definir qué significa que un cuerpo esté en reposo o en movimiento. Según Newton sabemos que un cuerpo está en movimiento, porque ocupa distintos puntos del espacio en diferentes intervalos de tiempo. Si el movimiento entre los puntos se da en intervalos iguales de tiempo, entonces el movimiento es uniforme. Como señala Freudenthal `` The distinction between 'rest' and 'uniform motion' implies, however, an absolutely resting frame of reference, and this can only be absolute space'' \citeyear{Freudenthal1986}

El movimiento en el espacio absoluto, sin embargo, no puede ser percibido, mientras que las posiciones relativas de los cuerpos sí. Esto, por supuesto no significa que haya que echar a la basura ambas entidades, ya que el espacio y tiempo absolutos juegan un papel en la explicación del movimiento newtoniana y dicha explicación recupera muchos de los fenómenos que buscamos explicar.

Señalé que ambas entidades son parte fundamental de la teoría newtoniana, ya que permiten explicar los movimientos relativos \cite{Maudlin2014Filosofia:7985}. Señalé también que la teoría newtoniana sugiere la existencia de una entidad de la que no es claro que tengamos certeza de que existe. Algunos teóricos como Leibniz sugirieron que algo extraño estaba sucediendo en la física newtoniana: si sólo tenemos evidencia de los movimientos relativos y un cuerpo en el espacio absoluto no se mueve con respecto a nada, entonces no tenemos evidencia de que exista el espacio absoluto. Newton en la \textit{principia} presenta dos experimentos mentales para señalar la existencia del espacio absoluto. Uno de ellos es en el cual señala que si atamos dos globos con un cordón y los hacemos rotar sobre su eje en direcciones contrarias, la cuerda se tensa. Si aceptamos que es verdad que la cuerda se tensa incluso en el espacio absoluto, entonces tenemos que aceptar que sí hay movimiento en el espacio absoluto, los globos se mueven con respecto a dicho marco, aunque estén en reposo uno respecto del otro \cite[pp. 6-12]{book:360820}.

Otros físicos como Ernst Mach (véase especialmente el capítulo 2) \citeyear{Mach2013} señala que debido a que diferentes marcos de referencia inerciales tienen las mismas consecuencias empíricas que hablar de espacio absoluto, entonces no es necesario apelar a dicha entidad, algo que sabemos gracias a la relatividad galileana\footnote{Por supuesto, Newton no hablaba de un marco privilegiado, sino que se refería al espacio absoluto como una entidad física. Pero lo que es importante notar es que podemos tener las mismas consecuencias sin apelar a dicha entidad.}.

Más aún, la relatividad señala que donde Newton distinguía dos entidades, realmente sólo hay una: espacio-tiempo. Incluso en la teoría de la relatividad, no es necesario apelar al tiempo absoluto para dar explicaciones. En la relatividad no hay un marco privilegiado y como consecuencia de la estructura del espacio-tiempo tenemos la dilatación temporal.

A partir de lo dicho aquí podemos obtener dos conclusiones. La primera de ellas es que la teoría de Newton es muy útil: nos permite describir una variedad de fenómenos desde la aceleración en caída libre de cuerpos en la tierra, hasta el movimiento de los cuerpos celestes. Sin embargo teorías físicas más recientes niegan que haya \textbf{espacio absoluto}, incluso señalan que la distinción entre dos entidades: espacio y tiempo, no es la más adecuada para describir la estructura del mundo. Por lo cuál podemos concluir que la teoría de Newton es falsa. La segunda conclusión es que la verdad es una condición necesaria del conocimiento. Si no tuviéramos explicaciones verdaderas, simplemente no tenemos explicación. Más aún no es posible saber algo falso. 

Aquí hay algo extraño: ¡ambas conclusiones no pueden ser verdaderas! La conclusión sería entonces negar alguna de las conclusiones a las que hemos llegado. O bien la teoría newtoniana sí es verdadera después de todo, o bien la verdad no es una condición necesaria del conocimiento. 






\bibliographystyle{apacite}
\bibliography{tesisdocto.bib}




\end{document}
