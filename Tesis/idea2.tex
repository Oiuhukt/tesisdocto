\documentclass{article}
\usepackage{setspace}
%\setstretch{1.5}
\usepackage[utf8]{inputenc}
\usepackage[T1]{fontenc}
\usepackage{url}
\usepackage[spanish]{babel}
\usepackage[textsize=tiny]{todonotes}
\usepackage{apacite}
%\usepackage{fullpage}


%Sección de comentarios generales:
%¿debería integrar la discusión sobre el objetivo de la indagación? Aquí sólo estoy asumiendo de manera intuitiva que nuestras empresas epistémicas, especialmente la práctica científica, tienen como objetivo llegar a la verdad. Pero esto es debatible.
%¿Modificar el término empresas epistémicas por prácticas como sugiere Jane Friedman?

\begin{document}

\listoftodos

\chapter{La verdad Importa}

\noindent En este capítulo argumento que la verdad es el valor fundamental del \textit{entendimiento}. En primer lugar, expongo la tesis veritista, \textit{i.e}, que la verdad es una condición necesaria del conocimiento y un valor fundamental. Después presento un par de problemas que se le han presentado a dicha tesis: el ``problema del drenado'' y el problema de las ``verdades irrelevantes''. Presento la solución de Pritchard y un par de casos históricos que parece señalar que en el contexto de la investigación no es claro empatar la tesis veritista. Para explicar los fallos y señalar que los casos históricos no son problemáticos para la tesis veritista hay que hacer un cambio en qué tomamos como objeto de conocimiento. Por lo que un cambio hacia el entendimiento es necesario para defender la tesis veritisa.

\section{Enunciando una tesis} \label{enunc}

\noindent Pensemos en un diálogo. Nuestros personajes son Andrés y Berto. Andrés defiende la existencia de los \textit{nomi}, Mientras que Berto señala que los \textit{nomi} no existen. Andrés cita como evidencia el hecho de que ha leído acerca de las huellas que dejan estos organismos, que sólo rondan en la noche. Además, ha visto sus huellas. Berto señala que Andrés nunca ha visto uno \textit{nomi} de primera mano y que de acuerdo a nuestra mejor teoría de la evolución por selección natural, dichos organismos no pueden existir. Qué oración es verdadera ``existen los \textit{nomi}'' o ``no existen los \textit{nomi}'', ¿cómo decidimos cuál enunciado es verdadero?

Comencemos por señalar que a los seres humanos nos interesa la verdad. Nos sentimos, por ejemplo, molestos cuando alguien incumple una promesa. Esta importancia que le damos a la verdad guía también nuestras empresas epistémicas\footnote{Utilizo el concepto de ``empresas epistémicas'' como algo muy general y no bien definido. El concepto puede ser usado para describir tanto a la investigación científica, como a la curiosidad de un niño que está aprendiendo a sumar.}. Entonces nos importa la verdad, en particular su vínculo con el conocimiento. Platón en su diálogo con Menón \citeyear[\P\P 97a-98b]{platonmeno} señala que hay un factor de seguridad ligado al conocimiento, que no se encuentra en la mera opinión verdadera\footnote{Por supuesto esto está más relacionado con la justificación que con la verdad propiamente. Sin embargo, cabe señalar que la verdad juega un papel importante en la explicación de lo que implica saber algo.}. La caracterización ofrecida por Platón como una ``creencia verdadera justificada'' fue puesta en duda por Gettier en un famoso artículo \citeyear{Gettier}. La conclusión de Gettier es que la caracterización de Platón no ofrece condiciones suficientes para decir que S sabe que `p'.

Pero el análisis de Gettier sólo aplica a uno de los lados del condicional material. Si bien las condiciones no son suficientes, esto no señala un problema con que dichas condiciones sean necesarias. Aceptamos, por ejemplo, que la verdad es una condición necesaria para saber algo, porque el conocimiento es fáctico. Si sabes que hay un gato sobre la alfombra, entonces hay un gato sobre la alfombra. Hablamos de creencias falsas, pero no de conocimiento falso. El ``saber que p'' tiene una estrecha conexión con la verdad y por lo general admitimos que conocimiento implica verdad.

Siguiendo esta línea de razonamiento, muchos epistemólogos han defendido que la verdad es una parte fundamental del conocimiento. Pritchard, por ejemplo, señala  

\begin{quote}
It wasn’t all that long ago that the idea that truth is the fundamental epistemic good was orthodoxy in epistemology. Indeed, this was the kind of claim that was so commonplace that it was almost not worth stating, as to do so would be somewhat superfulous. \citeyear[p. 1]{Pritchard2021}
\end{quote}

Pritchard nos dice que esta conexión que tiene el conocimiento con la verdad se debe a que valoramos \textbf{epistémicamente} a la verdad. De alguna manera tenemos que ser capaces de corregir creencias y quizás la forma más intuitiva de señalar esto es que el mundo impone ciertos constreñimientos sobre el conocimiento humano. Gila Sher acuñó el término ``fricción epistémica'' para describir esta relación \citeyear{Sher2016}. Una tarea similar es la que lleva a cabo Blåsjö en \citeyear{Blaasjoe2022} donde señala que los geómetras griegos estaban comprometidos con un programa operacionalista: las construcciones geométricas son construcciones físicas concretas, que además representan cómo es el mundo. Según el autor, esto además servía para evitar contradicciones. Por el momento asumamos esta tesis. No como una definición de verdad, sino del papel que juega la verdad en nuestras empresas epistémicas.

A pesar del peso intuitivo que tiene la tesis del valor epistémico de la verdad, varios filósofos han presentado problemas en contra de esta tesis. Entre estos problemas se encuentra el \textit{problema de la absorción} [swamping problem] y el problema de las \textit{verdades irrelevantes}. 

Pritchard explica ambos problemas de la siguiente manera. El problema de la absorción indica que la verdad es valiosa porque es epistémicamente útil. Si esto es verdad, entonces que valoremos la verdad es parasitario a la utilidad que esta nos brinda y dicha utilidad no hace diferencia entre conocimiento y creencia verdadera. El problema de las verdades irrelevantes establece que si la verdad es el valor fundamental, entonces ante dos verdades cualesquiera, no podríamos elegir cuál deberíamos creer -- por ejemplo hay una respuesta verdadera sobre el número total de granos de arena en Cancún y una respuesta verdadera sobre si la luna gira alrededor de la tierra, pero no queda claro que queramos saber la respuesta a la primera pregunta. Dado que queremos distinguir entre las respuestas que son importantes de las que no, entonces la verdad no es lo único importante para el conocimiento.

Otros filósofos señalan que en nuestras empresas epistémicas no están dirigidas a la verdad, sino a otros valores epistémicos, por ejemplo, al entendimiento de un fenómeno \cite{Elgin2004}; otros como Papineau \citeyear{Papineau2021} defienden que deberíamos olvidarnos del conocimiento, y decantarnos por la pura creencia verdadera. Pritchard considera que ambas estrategias son demasiado apresuradas.

Hasta ahora he descrito una tesis filosófica: que el conocimiento implica verdad. Señalé un par de razones de por qué la tesis es interesante, además de su poder intuitivo. Indiqué un par de argumentos que se han dirigido en contra de esta tesis. A continuación presentaré los argumentos de Pritchard para defender la tesis de los contraargumentos que se han esgrimido en contra de que la verdad es el valor epistémico fundamental.

\section{Contra ``la verdad no es el valor fundamental''}

\noindent Para dar una respuesta a estos problemas, Pritchard nos dice que cada uno de estos problemas depende de asumir que el objeto de investigación epistémica son las \textit{proposiciones}. Si negamos esta suposición, ¿qué forma tendría que tomar una teoría del conocimiento que tenga como valor fundamental a la verdad?

Antes de continuar con la exposición de Pritchard quisiera hacer un par de aclaraciones. La condición de 'verdad' involucrada en el conocimiento es de un tipo diferente a las de creencia y justificación. Tanto creencia como justificación son condiciones epistémicas; la verdad, por otra parte, es una condición de otro tipo. Al menos creemos comúnmente que la verdad no es una condición epistémica, si lo fuera, nuestras creencias podrían modificar cómo es el mundo [Habrá una parte dedicada a teorías de la verdad]. El que yo crea que mi cortina es verde\footnote{De hecho es azul.}, no la hace verde. Adecuamos nuestras creencias para que encajen con el fenómeno que queremos comprender\footnote{En \cite{sep-knowledge-analysis} los autores señalan que esta es una condición metafísica. Trato de ser neutro con respecto a si es una condición metafísica, porque si asumimos una teoría pragmática de la verdad (al estilo de William James o Richard Rorty), entonces la verdad es una condición epistémica. Como esto es apenas una presentación, evito el compromiso. Por ahora seguiré trabajando con la intuición señalada en \ref{enunc}}.  

Continuemos con la exposición de Pritchard \cite{Pritchard2021}. Pritchard nos dice que al involucrar a la teoría de la virtud epistémica, negar que el objeto epistémico por antonomasia es la proposicón y asumir a la verdad como el valor epistémico fundamental, entonces somos capaces de resolver los problemas antes mencionados. A grandes rasgos, las virtudes epistémicas, se entienden a la manera en como Aristóteles entendía la virtud: son deliberaciones que hacemos para llegar a un fin, estas por supuesto son decisiones voluntarias que se practican de acuerdo con el fin que queremos lograr\footnote{``Con relación a las mismas cosas son, pues, el cobarde, el temerario y el valiente, pero se conducen diferentemente a su respecto. Aquéllos pecan por exceso y por defecto, en tanto que éste guarda el medio y el deber.'' \cite{aristotelesnico} Especialmente los capítulos 2-8 del libro III.}. 

Uno de los pioneros de la teoría de la virtud epistémica es Ernst Sosa. Sosa \citeyear{Sosa2017-SOSE} articula una teoría del conocimiento que caracteriza a las virtudes epistémicas como el ejercicio de ciertas habilidades de los agentes tales que al ejercitarlas, constituyen el concepto de conocimiento. Sosa también señala que si bien el fin de las empresas epistémicas es la verdad, esto no es lo que valoramos como agentes: lo que valoramos es el \textbf{concocimiento}. Sosa \citeyear[pp. 11-113]{Sosa2017-SOSE} propone que el conocimiento, a diferencia de la creencia verdadera tiene las siguientes características: no sólo valoramos que el conocimiento sea certero (característica que comparte con la creencia verdadera), sino que valoramos que sea certero porque depende de las habilidades del agente; por último, valoramos que el conocimiento sea certero porque el agente llegó al objetivo debido a sus habilidades. Estas tres condiciones son las que constituyen la propiedad de que el conocimiento es apto. Asumiendo esta caracterización del conocimiento, Sosa argumenta que el problema de la absorción y de las verdades irrelevantes deja de tener sentido, ya que si bien la verdad juega un papel importante en el conocimiento, el valor de la verdad está dado por el conocimiento asociado a dicha verdad.

Si bien Sosa señala que la verdad no es en lo que recae el valor del conocimiento, no señala que la verdad no sea una condición necesaria del conocimiento, al menos el concepto de certeza involucra una relación entre creencias y mundo, que a primera vista no es diferente de la versión intuitiva que estamos usando. Esto hasta ahora no contradice a Pritchard en que la verdad es fundamental en nuestras empresas epistémicas. Los dos están de acuerdo en que el valor del conocimiento no depende del sólo hecho de que algo sea verdad, sino de que el proceso para obtenerla descanse en las virtudes del agente. Pritchard también señala que cambiar el valor fundamental del conocimiento (que es la verdad) por algún otro como la comprensión es un cambio demasiado apresurado. Esto se debe a que la persona que sepa ciertas verdades, tendrá más verdades en su vecindario de creencias que aquel que sólo tiene una creencia verdadera o que sólo comprende.

Con este marco explicado, Pritchard resuelve los dos problemas que se le achacan al veritismo. Con el problema de la absorción nos presenta una analogía con un chef (p. 6 ). La analogía es la siguiente: que un chef haga comida deliciosa y luego la pruebe para saber si lo es, no significa que su interés era probarla y no hacerla deliciosa. El probarla sólo es una manera de asegurarse que es deliciosa. Si es verdad que el valor fundamental de cualquier empresa epistémica es la verdad, eso explicaría por qué nos interesa obtener verdades: si bien esto tiene implicaciones prácticas, cualquier consecuencia depende de que de hecho nuestro conocimiento apunte a la verdad y la consiga. Más aún, las virtudes intelectuales nos ofrecen un marco para solventar el problema de las verdades irrelevantes: queremos no sólo que nuestras creencias sean verdaderas, sino que además el proceso para llegar a la verdad sea fiable y que esté guiado por características de un agente. Un agente virtuoso puede sopesar entre dos verdades: una irrelevante, la otra de más peso.

Si podemos resolver estos problemas para el veritismo, entonces no es claro que debamos abandonar a la verdad como una condición necesaria para el conocimiento, menos aún como el valor fundamental del conocimiento. Sin embargo, cuando empatamos el veritismo con la práctica científica, entramos en problemas\todo{Creo con convicción que el objetivo de la indagación, en contextos de investigación, es obtener conocimiento. Esto implica obtener verdades. ¿habrá que decir algo respecto a esto?}.

\section{Veritismo y dos casos históricos.}

\noindent Hasta este punto desarrollé los argumentos de Pritchard. Lo anterior es evidencia a favor de que la verdad es una condición necesaria para el conocimiento. Si no lo fuera, muchas de nuestras empresas epistémicas no tendrían sentido, porque no podríamos corregir nuestras creencias con base en nueva evidencia, porque la verdad es externa a los agentes. Incluso si tomamos a la creencia verdadera como básica, la verdad sigue inmiscuida: la verdad es una propiedad que nos asegura que vamos por el camino correcto. Las explicaciones y demás consecuencias del conocimiento también dependen de que la verdad sea una condición necesaria. Más aún, las virtudes intelectuales nos ofrecen un marco para solventar problemas como el de las verdades irrelevantes: queremos no sólo que nuestras creencias sean verdaderas, sino que además el proceso para llegar a la verdad sea fiable y que esté guiado por características loables de un agente. Un agente virtuoso puede sopesar entre dos verdades: una irrelevante, la otra de más peso.

Estoy de acuerdo con lo que señala Pritchard de que la verdad es una condición necesaria del conocimiento (dejemos por el momento si la verdad es una motivación), tanto así que es valorada por los investigadores. Este marco, incluso encaja bien con cómo podemos evaluar la investigación científica a la luz de la teoría presentada, dadas las características de los agentes y dado que sus resultados son aptos. 

Sin embargo, defender la tesis descrita hasta ahora se vuelve complicado cuando tratamos de empatarla con la práctica científica. La investigación científica es probablemente la manera más sistematizada que tenemos los seres humanos para producir conocimiento. Muchas de nuestras explicaciones dependen de conocer algunos hechos acerca de las diferentes disciplinas científicas. Sabemos, por ejemplo, que para que haya una combustión se necesita combustible y oxígeno. Si cualquiera de estas variables está en 0, entonces no hay combustión. Sabemos, por tanto, que sin combustible u oxígeno, no hay combustión. Esto es una explicación, sabemos que es una buena explicación porque es verdad. Una explicación que utilice información falsa no es una buena explicación. Considérese, por ejemplo, que tiro mi taza de café al piso. Y que explico esto con base en señalar (falsamente) que justo antes de que mi brazo golpeara la taza, el viento la empujo y fue esta la razón por la que cayó al piso. Esto por supuesto es una mala explicación. De manera sucinta, las teorías y explicaciones tienen que representar adecuadamente el mundo.

Supongamos que las explicaciones verdaderas son las únicas explicaciones que nos interesan (porque sólo las buenas explicaciones son las explicaciones verdaderas.). La suposición anterior es claramente falsa. Para señalar esto, podemos incluso citar el uso contemporáneo de teorías. Por ejemplo, la mecánica Newtoniana y la teoría de la selección natural de Darwin, son teorías que usamos cotidianamente para explicar fenómenos. La teoría de la selección natural nos ayuda a explicar fenómenos biológicos como la adaptación y la especiación. Con base en esta teoría, podemos explicar, por ejemplo, por qué un grupo dentro de una población tiene más descendencia que otro grupo dentro de la misma población; podemos también saber, por ejemplo, cuáles son los ancestros comunes de especies contemporáneas, \textit{e. g.}, las aves de los dinosaurios. Por su parte, utilizamos la teoría newtoniana para explicar el movimiento de los astros y hacer predicciones de qué posición tomarán en un momento dado, podemos explicar la fuerza que se imprime en una superficie cuando es golpeada por una masa con cierta aceleración y nos sirve para explicar el movimiento de objetos apelando a la inercia.

Esto por supuesto son grandes ventajas de las teorías: nos permiten explicar fenómenos. A pesar de que estas teorías nos permiten explicar una amplia variedad de fenómenos, no es claro que sean [\textit{literalmente}] verdaderas. En lo siguiente presentaré la teoría newtoniana y expondré algunos de los problemas que varios investigadores han detectado en ella.

A pesar de que ambas teorías nos permiten explicar un amplio rango de fenómenos, las dos tienen problemas. Sabemos que la distinción newtoniana entre ``tiempo'' y ``espacio'' es falsa. Para poder explicar el movimiento, Newton desarrolló una teoría que tiene como entidades el \textbf{tiempo absoluto} y el \textbf{espacio absoluto}. Un cuerpo se mueve o permanece en reposo con respecto al espacio y tiempo absolutos. Todos los puntos en el espacio absoluto permanecen constantes durante diferentes intervalos temporales. Para explicar la llamada ``primera ley''  de Newton (inercia), es necesario definir qué significa que un cuerpo esté en reposo o en movimiento. Según Newton sabemos que un cuerpo está en movimiento, porque ocupa distintos puntos del espacio en diferentes intervalos de tiempo. Si el movimiento entre los puntos se da en intervalos iguales de tiempo, entonces el movimiento es uniforme. Como señala Freudenthal `` The distinction between 'rest' and 'uniform motion' implies, however, an absolutely resting frame of reference, and this can only be absolute space'' \citeyear{Freudenthal1986}

El movimiento en el espacio absoluto, sin embargo, no puede ser percibido, mientras que las posiciones relativas de los cuerpos sí. Esto, por supuesto no significa que haya que echar a la basura ambas entidades, ya que el espacio y tiempo absolutos juegan un papel en la explicación del movimiento newtoniana y dicha explicación recupera muchos de los fenómenos que buscamos explicar\todo{Ambas entidades son parte fundamental de la teoría newtoniana, ya que permiten explicar los movimientos relativos. Recuerdo que Maudlin señala esto, pero ahora me suena extraño, Revisar la sección.} 

Señalé que la teoría newtoniana sugiere la existencia de una entidad de la que no es claro que tengamos certeza de que existe. Algunos teóricos como Leibniz sugirieron que algo extraño estaba sucediendo en la física newtoniana: si sólo tenemos evidencia de los movimientos relativos y un cuerpo en el espacio absoluto no se mueve con respecto a nada, entonces no tenemos evidencia de que exista el espacio absoluto. Newton en la \textit{principia} presenta dos experimentos mentales para señalar la existencia del espacio absoluto. Uno de ellos es en el cual señala que si atamos dos globos con un cordón y los hacemos rotar sobre su eje en direcciones contrarias, la cuerda se tensa. Si aceptamos que es verdad que la cuerda se tensa incluso en el espacio absoluto, entonces tenemos que aceptar que sí hay movimiento en el espacio absoluto, los globos se mueven con respecto a dicho marco, aunque estén en reposo uno respecto del otro \cite[pp. 6-12]{book:360820}.

Leibniz no fue el único en sospechar de la teoría newtoniana. Físicos como Ernst Mach (véase especialmente el capítulo 2) \citeyear{Mach2013} señala que debido a que diferentes marcos de referencia inerciales tienen las mismas consecuencias empíricas que hablar de espacio absoluto, entonces no es necesario apelar a dicha entidad, algo que sabemos gracias a la relatividad galileana\footnote{Por supuesto, Newton no hablaba de un marco privilegiado, sino que se refería al espacio absoluto como una entidad física. Pero lo que es importante notar es que podemos tener las mismas consecuencias sin apelar a dicha entidad.}.

Más aún, la relatividad señala que donde Newton distinguía dos entidades, realmente sólo hay una: espacio-tiempo. Incluso en la teoría de la relatividad, no es necesario apelar al tiempo absoluto para dar explicaciones. En la relatividad no hay un marco privilegiado y como consecuencia de la estructura del espacio-tiempo tenemos la dilatación temporal.

A partir de lo dicho aquí podemos obtener dos conclusiones. La primera de ellas es que la teoría de Newton es muy útil: nos permite describir una variedad de fenómenos desde la aceleración en caída libre de cuerpos en la tierra, hasta el movimiento de los cuerpos celestes (¡nos permitió incluso llegar a la luna!). Sin embargo, teorías físicas más recientes niegan que haya \textbf{espacio absoluto}, incluso señalan que la distinción entre dos entidades: espacio y tiempo, no es la más adecuada para describir la estructura del mundo. Por lo cual podemos concluir que la teoría de Newton es falsa. La segunda conclusión es que la verdad es una condición necesaria del conocimiento. Si no tuviéramos explicaciones verdaderas, simplemente no tenemos explicación. 

Aquí hay algo extraño: ¡ambas conclusiones no pueden ser verdaderas! Por lo que será necesario o bien dar un par de razones a favor de cómo ambas conclusiones pueden ser compatibles, o bien negar alguna de las conclusiones a las que hemos llegado. Si tomamos la segunda estrategia, dos opciones se abren ante nosotros: o bien la teoría newtoniana sí es verdadera después de todo, o bien la verdad no es una condición necesaria del conocimiento.

Tomemos el segundo disyunto: que la verdad no es una condición necesaria para el conocimiento. El argumento que ofrece Pritchard, y que expusimos líneas arriba, señala que la verdad es una condición necesaria porque de otra manera no podríamos explicar muchas de nuestras metas cognitivas, su marco tiene la ventaja de resolver los problemas con los que comúnmente se embiste esta tesis. El primer problema que encuentro con este marco es la afirmación de que ``las virtudes epistémicas son aquellas que son constitutivas del conocimiento''. Pero tanto Pritchard como Sosa están de acuerdo\footnote{Por supuesto la tesis que señalan ambos no es exactamente la misma. Sosa afirma que el objetivo de nuestras empresas epistémicas es el conocimiento, mientras que Pritchard parece sugerir que la finalidad es la verdad adquirida por un agente virtuoso. Sin embargo, recordemos que ambos afirman que la verdad es una condición necesaria del conocimiento, que es lo que está en cuestión aquí.} es con que la verdad es una condición necesaria para el conocimiento. Al menos eso parece indicar la condición ``accurate'' de Sosa, suponiendo que el objetivo de nuestras ``empresas epistémicas'' es la verdad'\footnote{Por el momento estamos asumiendo que el objetivo de nuestras empresas epistémicas es la verdad. Creo que lo dicho hasta este punto me permite asumir esta tesis sin problemas. Sin embargo, creo que la investigación tiene un par de detalles que hay que tomar en cuenta, en especial cuando usamos modelos para describir fenómenos. Esto es algo que discutiré en los siguientes capítulos.}. 

Lo anterior son razones para señalar que la verdad sí es una condición necesaria del conocimiento. Más aún, señalamos que el mundo impone cierta fricción en nuestras creencias, por lo que el conocimiento es fáctico. Además, no creo que haya duda de que Newton es un agente virtuoso, sin embargo, su uso de las virtudes no lo llevó a la verdad (que es una de las condiciones necesarias para el conocimiento). Si esto es verdad, entonces habría que negar que las virtudes epistémicas son constitutivas del conocimiento. Afortunadamente, el marco de virtudes de Sosa no es el único que tenemos disponible\todo{Si bien podemos negar que las virtudes epistémicas son constitutivas del conocimiento, el marco de Sosa no es el único marco de epistemología de la virtud. El trabajo de Zagzebski puede ayudar acá, Tengo que poner una breve descripción de esto.}

Por supuesto, esto sólo es un problema para la teoría de las virtudes, y no para la tesis de que la verdad es una condición necesaria del conocimiento. Supongamos que la verdad es una condición necesaria del conocimiento y que el conocimiento es una condición necesaria para la ciencia. Entonces habría que señalar cómo a pesar de los problemas detectados, la teoría de Newton es verdadera a pesar de todo. Es decir, tomamos el otro disyunto. Pero esto también es problemático para cualquier teoría en filosofía de la ciencia que se tome en serio que somos agentes falibles. El argumento de Laudan \citeyear{Laudan1981} nos enseñó que las buenas explicaciones no están necesariamente conectadas a la verdad. La teoría de Newton funciona, aunque no es claro que debamos asumir que es verdadera.

Ninguna de las conclusiones a la que nos llevan los disyuntos rescata las intuiciones con las que iniciamos. La otra salida es hacer que ambas tesis sean compatibles. Creo que puedo hacer eso. La siguiente sección estará dedicada a hacer compatibles que la verdad es una condición necesaria y que podemos estar equivocados, aún teniendo buenas explicaciones.

\section{Compatibilidad de las tesis: un caso a favor del entendimiento}

\noindent Recordemos que Pritchard además de optar por un marco de virtudes epistémicas, señala que hay que modificar el objeto epistémico. En lugar de que el objeto de conocimiento sean proposiciones, modifica la teoría para que el nuevo objeto sea diferente. Pritchard no señala exactamente cuál debería ser el nuevo objeto, pero podemos inferirlo a partir de que habla de ´´entendimiento''. 

Hace algunos años, los epistemólogos han tratado de modificar el objeto de estudio de la epistemología. En lugar de tratar de explicar el valor del conocimiento proposicional, han sugerido un cambio hacia el entendimiento. Este cambio tiene algunas motivaciones: entre ellas está la falla del programa del análisis del conocimiento y la falta de una respuesta contundente para el escéptico. 

El cambio que proponen aquellas filósofas que sugieren el cambio de conocimiento a entendimiento depende de que no juzgamos las proposiciones, sino que tenemos más objetos que somos capaces de juzgar: relaciones entre diferentes fenómenos, estructuras, pedazos de información, relaciones de dependencia, etc. Los epistemólogos que han transitado de un enfoque proposicional a uno que toma otro tipo de objetos, señalan también que tenemos más formas de juzgar creencias además de ser verdaderas o falsas. Entre estas diferentes maneras de juzgar creencias tenemos: creencias justificadas, racionales, fiablemente formadas, virtuosamente formadas, etc. Al mismo tiempo este enfoque no renuncia a la verdad como uno de los objetivos de nuestras empresas epistémicas \cite{Grimm2012-GRITVO-4}.

Si bien las motivaciones para realizar este cambio responden a problemas puramente epistémicos, este cambio de enfoque nos permite incorporar el hecho de que las proposiciones como tal no son lo único a lo que los investigadores deberían prestar atención. Somos agentes falibles que pueden tener entendimiento de un fenómeno, aun cuando no obtenemos el objetivo que es ``la verdad''.

Esto indica que si bien nuestro objetivo en la investigación es la verdad, no es algo que podamos obtener tan fácilmente. No parece que esto sea una afirmación problemática: somos agentes falibles y las herramientas que tenemos disponibles para justificar hipótesis no son perfectas. Una de las herramientas más utilizadas en investigación empírica es la estadística, y Sabemos que un resultado estadístico no justifica al 100\% una hipótesis. Como señala Deborah Mayo ``How do humans learn about the world despite threats of error due to incomplete and variable data?'' \cite[p.~ xi]{Mayo2018} 

Esto parecería ir en contra de nuestra intuición inicial: que el objetivo de la investigación científica es la verdad. Pero como he señalado, la verdad es difícil de obtener: nuestras capacidades cognitivas y herramientas no son perfectas. Aun así, nuestro problema es cómo obtener la verdad a pesar de nuestra falibilidad. La tesis de que la verdad es una condición necesaria para el conocimiento y que nuestros métodos sean falibles, no implica que no haya manera de obtenerla.\todo{Probablemente, sea bueno discutir aquí a Boris Hessen. Esquema de argumento. P1 La verdad es una condición necesaria para el conocimiento. Esto lo señalé a partir de que es la única manera que tenemos para cambiar creencias con base en nueva evidencia. Además de presentar una solución a dos problemas clásicos contra la tesis veritista (gracias Pritchard). P2. Nuestros métodos son falibles. Si le creemos a los constructivistas, podríamos extraer una conclusión ontológica a partir de premisas epistémicas. Creo que es incorrecto porque asumiendo una teoría correspondentista de la verdad, esto no se sigue (para eso servirá discutir teorías de la verdad, por lo que esta premisa todavía es débil: hay que señalar por qué es una buena teoría). Entonces no se sigue que debido a nuestras limitaciones epistémicas, la verdad no existe o es inasequible. Al menos sabemos que si sabemos que hay un gato sobre el tapete, es porque hay un gato sobre el tapete. Tratar de dar razones a favor de una teoría correspondentista que no dependan de nuestra habilidad para modificar creencias: eso sería circular.}

Si bien es verdad que somos agentes falibles, de esto no se sigue que debamos abandonar por completo nuestros compromisos veritistas. En el libro citado anteriormente de Deborah Mayo, hay una frase que encaja muy bien con la teoría falibilista de Peter Klein que quiero presentar a continuación. Mayo dice ``We set sail with a simple tool: If little or nothing has been done to rule out flaws in inferring a claim, then it has not passed a severe test.'' \cite[p.~xii]{Mayo2018}.

Lo que señala Mayo es una intuición que sólo se puede descartar con razones de peso. Como agentes en investigación, sabemos que somos falibles. Esto no es nuevo. La pregunta importante es cómo a pesar de nuestra falibilidad podemos obtener conocimiento, antes de apresurarnos a descartar la posibilidad de obtener conocimiento en absoluto\todo{Esta frase suena extraña. Lo que quiero decir es que podemos tener conocimiento a pesar de nuestra falibilidad.}.

Por suerte, Peter Klein \citeyear{Klein2019} ha desarrollado una teoría del conocimiento que encaja muy bien con lo que he señalado hasta ahora. De acuerdo con lo que Klein llama ``infinitismo derrotable'' [defeasible infinitism], los seres humanos somos agentes falibles. Pero valoramos el conocimiento: no sólo la creencia verdadera, sino el ``conocimiento de verdad''. Esto quiere decir: conocimiento para el que tenemos suficientes razones.

El punto de Klein es que, como agentes, \textit{justificamos} creencias. Esto es algo que hacemos. Si tenemos a nuestra disposición razones para sostener una creencia cualquiera $x$, y no tenemos a nuestra disposición una creencia $y$ que disminuya\footnote{Hay varias maneras en las cuales nueva información puede disminuir nuestra justificación (estoy usando deliberadamente información, a diferencia de Klein, cuya discusión está enmarcada en términos de proposiciones). Nueva información $z$ haga que el enlace entre una creencia $x$, sobre la que justificamos una creencia $y$, nos haga retractarnos de la fuerza del enlace entre $x$ y $y$. Por ejemplo, supongamos que leemos un estudio que señala una fuerte correlación entre el omeparazol y problemas cardiacos. Luego descubrimos que el estudio está sesgado. En este caso dos opciones se abren ante nosotros: o bien retractar nuestra creencia, o bien ofrecer razones para señalar por qué no hay sesgo después de todo.} nuestra justificación de alguna manera, entonces tenemos conocimiento certero.\todo{Klein señala que la única manera de hacer sentido a lo anterior es el \textit{infinitismo.}}

Para que esta tesis sea plausible, Klein defiende su teoría en contra de lo que él llama ``el riesgo de desconfirmación empírica''. Este problema, señala Klein, embiste a las teorías epistémicas que señalan que una creencia debe tener una cadena causal adecuada (entre ellas la teoría de Sosa) para que dicha creencia pueda constituir conocimiento. El problema consiste en lo siguiente: supongamos que tenemos una creencia falsa causada por evidencia empírica. Al momento de corregir nuestra creencia con base en nueva evidencia empírica, ¿cómo afecta causalmente nuestra creencia original?

Una sugerencia es señalar que lo que hice fue corregir mi creencia. Pero la creencia formada por nueva evidencia tiene una historia causal completamente diferente a la anterior. Entonces no queda claro si ahora tengo dos creencias causalmente formadas, o una creencia causalmente revisada. No es claro cuál es la cadena causal de cada una de nuestras creencias. Si es difícil saber si hay una relación causal entre eventos (es necesario investigar empíricamente), aún más difícil saber si nuestras creencias fueron causadas por diferentes eventos 

\begin{quote}
I take that as a good prima facie reason for thinking that the diffrence between real knowledge and less paradigmatic forms of knowledge or ignorance depend on the quality of reasons for the belief, not the etiology of the belief. \cite[p~.403]{Klein2019}
\end{quote}

¿Qué podemos decir de lo señalado anteriormente con respecto a los ejemplos? Cambiar el objeto de entendimiento de proposiciones por otros como : modelos, pedazos de información, etc. Hace sentido del hecho de que tanto Newton como Darwin \textit{entendían} el mundo, aun si sólo de forma local. Esto es, en un rango de aplicación limitado a objetos macrofísicos. Lo cual indica que hay valor en las investigaciones que realizaron, aun cuando nueva información nos dice que estaban equivocados (al menos para velocidades cercanas a la luz y objetos subatómicos en la teoría newtoniana; el gradualismo que implica la teoría de la selección natural de Darwin). 

Más aún, sabemos que los humanos somos agentes falibles. Hay derroteros para muchas de nuestras razones para sostener creencias y tener conocimiento. Nueva información hace que nos retractemos de nuestras creencias a menos que podamos derrotar dichos derroteros. En la época de Newton hubo investigadores que dudaron de la existencia del espacio absoluto, Mach también dio razones en contra de que dicha entidad existiese. Ahora sabemos que podemos hacer mucho trabajo en física sin la necesidad de postular dicha entidad. Pero Newton tenía buenas razones para sostener sus creencias, además la investigación racional opera de esa manera: desarrollamos teorías y nueva información señala que nuestras creencias son erróneas. A menos que tengamos razones para desechar esa nueva información, cambiar de creencias es lo que hará que tengamos conocimiento. Lo que importa es continuar con la investigación.


\section{Conclusiones}

En este caspítulo defendí la tesis veritista: que el objetivo de la investigación es la verdad. Presenté dos problemas que algunas filósofas señalan en contra de la tesis. A estos problemas, expuse las soluciones que ofrece Pritchard a favor del veritismo. Sin embargo, no quedaba claro que, si el objetivo de la iunvestigación es la verdad, cómo dicha tesis es compatible con el falibilismo. Para esto presenté dos casos históricos que, si bien son casos de éxito en investigación, no son hipótesis verdaderas. Para hacer compatible la tesis veritista con el falibilismo, presenté dos teorías epistémicas que ayudan a rescatar nuestras intuiciones originales: el cambio de objeto epistémico y una teoría de la certeza basada en justificación falible.\todo{Señalar algo sobre agentes virtuosos en el marco de Sosa, o bien el de Zagzebski}

Quisiera terminar señalando dos cosas que restan por hacer. Mi objetivo final es decir cómo los modelos causales sirven en la investigación para concluir que hay una relación causal entre dos fenómenos. Creo que el marco presentado en este capítulo es útil para poder mostrar que los compromisos que tienen los investigadores cuando utilizan modelos es menos problemático de lo que pudiera parecer a primera vista.

\bibliographystyle{apacite}
\bibliography{tesisdocto.bib}


\end{document}
