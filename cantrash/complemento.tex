\section{Problems with the ``artifactualist'' and ``de-idealizer'' stances}

In the previous section I made a brief characterization of the three stances under discussion.
At the end of the characterization of the ``causal relevant'' stance, I made clear the core differences between this stance, and the ``artifactualist'' \& ``de-idealizer'' stances.
The order of exposition is biased towards the causal relevantist, I think that this stance has a more addecuate picture of what actual researchers do when they idealize phenomena.

In this section I want to argue that the de-idealizer and the artifactualist have the things wrong.
The artuifactualists claim that her advantage is to not rely on the relation between idealization and the world.
The de-idealizer argues that new tools with more computational power will render idealizations obsolete.

Both claims are misguided with respect to the actual research methods.
As researchers we want to give explanations; we also want our explanations to be good.
If we just turn our heads against the world, we will make a theory about an idealization, not about actual phenomena.









They are right about the second affirmation, but I have restrictions about the first.



Artifactualist say that their stance Highlights the restance highlights the relation between the HH model, and other models and analogies already at dispossal.








This debate is related to a larger problem, that is, if \emph{truth} plays a crucial role in our theories: either in theory building or in theory justification.
There are at least two positions in the debate: one side argues that \emph{truth} is the epistemic goal of inquiry, the other side denies this.






\section{Changing the object of study}

During the past century, philosophy of science became a fluorishing disciplie.
The members of the Vienna circle disscussed different epistemic and metaphysical aspects of our scientific theories.

%buscar como dibujar una línea en esta parte del documento%

\section{On models and targets}

In the philosophy of science, there is an ongoing debate related to modelling ostartegies used in the sciences.

Recently, there have been a lot of interrelated questions between the aforementioned categories.


Mathematics seems to be a kind of abstract endeavour. And I agree that we need an ontological commitment towards \emph{abstracta}.
But starting with


\section{going crazy}

Let me see. In the inferential conception defended by Suárez, there is no need for truth. Just to represent, with some kind of intentios is enough. He also says that the means to distinguish between mere arbitrary denotation is that we can draw informative inferences about the target from the source. I do not think this is quite right.

There must be some kind of relation between the source and the target, such that, serves us as a guide to draw inferences. This can be seen just as an addendum to Suarez's inferential account.

Take, for instance, the non-euclidean geometry. There was noo intended model, nor a clear application in science. But, when we learnt about relativity theory (no sé si esto es correcto, hay que leer algunas notas históricas), we had a model of the formal theory.

Probablemente haya que leer el libro de Schurz "optimality justification".


"s to identify those ways of forming beliefs that are both feasible for creatures like us and optimal among those that are feasible as ways of achieving the goal in question."

"Schurz wishes to produce an internalist epistemology, and so in order to justify basic beliefs or methods of inference that preserve justification, it is not enough to observe that the basic beliefs are true nor that the rules of inference are conditionally reliable, as an externalist veritist like Goldman "


"Schurz agrees, but he argues that in order to justify using a method of inference to form our judgements, we needn’t establish that it is reliable. We need only show that it is in some sense optimally reliable among the various methods of inference available to us—it’s the best we can do"


After all, rationality requires no more than this: to be rational is to do the best you can within the constraints imposed on you.

Natalia Carrillo y Tarja Knuutila están en lo correcto por las razones equivocadas.
Están en lo correcto en señalar que  "Accordingly, idealization tends to be holistic in that it is not often easily attributable to just some specific parts of the model" [Carrillo2021-CARAAP-12, pp. 3].

The last quote affirms two things, first that "idealizations tends to be holistic" and that "idealization is not often attributable to just some specific parts of the model".
They are right about the second affirmation, but I have restrictions about the first.

What does it mean for something to be holistic? and, why it tends to be that way and not the other?\footnote{Between the deficiency theory and the epistemic benefit theory.}.
They also say that the deficiency account, the epistemic benefit account, and the artefactual one, share a common assumption: that idealizations deliberatively misrepresent their target.

We indeed have \emph{ad nauseam} examples of models that are far from accurate, and they certainly misrepresents.
And they defend the artefactualist position based on the claim that it deals better with this, because it doesn't take the relation between source and target as a point of departure.

This, i will argue, is wrong. The first thing i want to do in this paper is criticize the notion of holism used in this definition.
This is sort of a more analytic task.
The second thing i want to say is that the artifactual stance fails because it doesn't explain the role that models play in the actual scientifica research.
This is more of a philosophy of science task.
I will present some examples of how models are used in mathematics.
Mathematical research relies heavily in making abstractions, so that is why i think mathematics can serve as an iluminating case in this debate.
The relation between source and target is the central task of a good theory of modeling in the sciences.

There is no clear reference about what they mean by 'holism'.

"Briefly,  the hypothesis is that the action potential depends on a rapid sequence of changes  in the permeability to the sodium and potassium ions." Hodkin 1951


\section{Creo que ya tengo suficiente información para decir que su ejemplo de HH es incorrecto}

El modelo de HH está descrito señalando que la parte importante del modelo es la relación con la pila galvánica.
Sin embargo, este uso de ecuaciones de los resultados obtenidos por Galvani, ya estaba en uso cuando HH publicaron sus resultaados.
Lo mencionan en su paper de 1951.
Además, ya había resultados previos de experimentos con calamares gigantes.
Primero, para hacer sus estudios tenían que saber que los axones de esta especie tienen 0.5cm. de diámetro.
Pero además, ya se habían obtenido resultados del comportamiento de los axones en la espeecie.
Lo que estaba en debate era qué tan permeables son los extremos de la célula.
Esto estaba en debate porque había resultados encontrados: un bando sostenía que la región exterior de la célula era completamente permeable, mientras que otros señalaban que era completamente impermeable.


La aportación interesante que hicieron H\&H fue concluir que era semipermeabeable y modelaros esta semipermeabilidad como puertas que abren y cierran de forma estocástica.



\section{notas para agregar o algo así}

\footnote
{Some philosophers have argued that truth isn't the epistemic goal of scientific inquiry \cite{elgin2017true, Potochnik2017-POTIAT-3}, therefore, we can have good explanations that quote false \emph{explanantia}.
	And there are philosophers who believe that truth is what gives value to our knowledge \cite{pritchard2021, pritchard2021a}, and as such, it should be the goal of scientific inquiry.
	This is a larger problem, and I will not discusse it further.
},


\section{Contra Knuutila y CArrillo}
The artifactual stance, they argue, doesn't suppose that the target is being somewhat accurately represented.
So we do not need to address the relation between the model with idealizations and the world, and there is no need to use words like "misrepresents" and the like.

So science takes place with full knowledge of what is wrong with the relation between the model and the world, and as such, this must be a point of departure for the philosophy of modeling.



%Like the Price equation in evolutionary theory, or the perfect reasoner in economics.  %poner ejemplo: podrá ser el ejemplo clásico del \emph{homo economicus}]% As examples of models in evolutionary biology

The artefeactualist stance, they say, gets right the relation between analogies.
The Hedgwig and huxley model of the membrane as a semipermeable barrier get the analogies already available and shows a path between different models.

The argument resides in the emphasis that other "modeling stances" make about the relations between the model and the world.
Let us say that the epistemic benefits stance and the distortion of reality stance, take too much care on the model and how accurate is representing the target.
The artefactual stance doesn't put too much care at clarifying if the model accurately represents or not.
Instead, the artefactualist stance highlights the relation between the HH model, and other models and analogies already at dispossal.

In this section I want to deal with two things.
First, and less important than the next step, is to recognize that the artefactualist didn't resolve the issue of the accuracy of representation.
They just put the problem on the side, and argue that the accuracy of the model isn't an important aspect of modelling in the sciences.
But, I can't see why the other stances cannot do just that, that being, to highlight the relations between models, or analogies.

It seems to me that there is no evidence in favour of this conclusion.
To clarify why I'm Saying this, suppose that analogies behave like this: If we make an analogy\footnote{A clear criterion of what an analogy is it's a problem by it's own right. }, then every other analogy that is a natural extension of the set of analogies in the first set, then we say that both are of the same family.
To give an example, take the analogy of Galvani --the tissue of the organism reacting to electricity--, and what \cite{Carrillo2021-CARAAP-12} says that the Huxley Hodgkin model consist in: an analogy between the nerve cells and electrical circuits.

%Como ciomentario a la nota del párrafo anterior, dirigir al lector a "And we need certain metaphors in order to access the worlds they enable us to access".Del autor cuyo nombre todavía no me aprendo.


Here, we have a relation between the "analogy" given by Galvani, and the electrical circuit "analogy" given by Huxley \& Hodgkin.
Since we have already an analogy with electrical current, it seems plausible to say that the circuit analogy became available, they affirm that

\begin{quote}
	As these measuring artifacts were developed and the nerve cell was rendered in electrical  terms, various theoretical and representational tools traditionally associated to electrical  engineering and electromagnetism  became available for theorizing about nerve impulse generation and transmission. \cite{Carrillo2021-CARAAP-12}, p. 12.
\end{quote}





" While the deficiency accounts aim for de-idealization, the epistemic benefit accounts offer reasons for why scientists might be justified in not de idealizing their models." Cita de Carillo Y Knuuthila. [Carrillo2021-CARAAP-12, pp. 2]

"idealization  makes a positive epistemic contribution in identifying the contributions of these causal  factors/ difference-makers."
\cite{Carrillo2021-CARAAP-12}, p. 4.

idealizations introduce distortion into models with respect to our
knowledge of worldly target systems. In other words, idealizations deliberately misrepresent. Distortions of these kinds are not difficult to find: the classic examples concern limiting concepts, e.g. when assuming that  a thermodynamic system has an infinite number of particles, or treating populations of discrete individuals as continuous. In these kinds of cases, the model world undoubtedly involves features that are known not to hold in worldly target systems. \cite{Carrillo2021-CARAAP-12}, p. 2

"At the bottom there is  the idea of idealization as a distortion that already suggests a model-world comparison, i.e.  representing the worldly systems differently from how they actually are, and ascribing to  them properties they do not have" \cite{Carrillo2021-CARAAP-12}

"The focus is on genuine possibilities and not just some counterfactual scenarios  within the range of known behavior of some actual systems." \cite{Carrillo2021-CARAAP-12}, p. 9.

"With this equipment, new empirical discoveries were made. After performing  experiments on squid giant axons, Hodgkin and Huxley realized that Bernstein’s account was  not entirely correct, since the membrane did not “collapse” but briefly became first  permeable to potassium and later permeable to sodium" p. 12


on the distinction abstraction/idealization, and if we can distinguish between both
