in applications of induction, are already known not to be identical in all respects. That is to say the assumption, made in Carnap's type of inference, that the evidence ascribes to the individuals only the same property P1 in both cases, and that there are not initially knownto be any differencesbetweenthem, is at best an idealizationof the real situation. It will generallybe the case that, if the total evidenceis takeninto account,superficially similar instances will be found to be different in some respects. p. 320


It follows from this theorem that neither of the analogical arguments (a) and (b) are justifiable within Carnap's A-system. This means that a large number of similarities between instances as in (a), and a large number of occurrences of the same correlation in otherwise different instances as in (b), both fail to give increasing confirmation as these numbers respectively increase.

Porque si tomamos absolutamente toda la evidencia, obviamente los analogandos van a ser muy diferentes entre sí, dados todos sus aspectos. Además, es plausible pensar que uno de los analogandos en realidad tenga una relación más cercana con C que con B. Si tomamos toda la evidencia, estas inferencias no están justificadas en el cálculo de Carnap
