\section{Apuntes historia de la ciencia}

Esto es del artículo de Corina Yturbe \cite{Yturbe1995}

one of the philosophical theses in favor of the contraposition between the internalist approach and the externalist approach is the so-called doctrine ofthe two contexts, developed by positivism.

became the only aspect relevant to the history of science. The consideration of non-rational, external factors, according to this position, not only fails to increase our understanding of scientific development, but even obscures the fundamental question of the rational validation of scientific arguments. p.75

The search for new explanatory programs is characterized above all by the attempt to reconcile the internalist and externalist approaches. But, in view of the fact that both approaches are committed to theses concerning the nature of science that are not only incompatible, but are unsustainable, their union without important changes in their philosophical presuppositions cannot result in a viable program. p. 79

The general tendency in the historiography of the sciences is to consider the social function of science, as weIl as so me aspects of the matrix from wh ich the problematic is formed, as external elements, while the conceptual apparatus and the problem field are treated as internal. But we should not think of science as having two independent histories.

External factors are not only found in the context of discovery, they are present also in the development of the concepts, problems, methods, problem fields, etc. of scientific discourses: that is, external factors are present in the context of validation itself. There are scientific discourses in which ideological conceptions pass on to form part of the body of the science itself, functioning as principles which define its field of study or guide its research; thus, external factors can become internal. p. 85

This conception constitutes the dominant framework in which the philosophy of science has been developed, and consists in drawing a radlcal distinction between the context of discovery and the context of validation. p. 75


