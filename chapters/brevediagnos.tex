% !TEX root = ../main.tex
% LTeX: language=es, en

\chapter{Un rápido diagnóstico}

\section{Un breve repaso de dos posturas}

\noindent El siguiente será un breve y burdo repaso de las
\emph{metodologías}\sidenote{Esto que describo es igual a lo que Suárez
llama \say{posturas}} \emph{externalistas} e \emph{internalistas}. 
Las llamo \say{metodologías} porque difieren en términos sustanciales
--por ejemplo, el de \emph{verdad}-- y es a partir de estas diferencias
que ambas metodologías ofrecen una interpretación distinta de la historia
de la ciencia. Al usar el término \emph{metodología}, no quiero
comprometerme con si son teorías en contradicción, o teorías
complemetarias; esto es algo que no puedo discutir aquí y el nombre
\say{metodologías}, me parece, es lo suficientemente neutro\sidenote{ Lo
que llamo aquí \emph{metodologías}, corresponde a lo que Suárez llama
\emph{postura} \cite{suárez emergence} }.

%La sección \ref{sbc:yturbe} la dedico a analizar una supuesta alternativa para el debate anterior.
%Además, esta alternativa, es más cercana a la filosofía de la ciencia.

\subsection{Dos metodologías: \emph{externalismo} e \emph{internalismo}}

\noindent Digamos que en historia de la ciencia se suelen distinguir dos
corrientes: los \emph{internalistas} y los \emph{externalistas}. De manera
procaz, los internalistas argumentan que la historia de la ciencia debería
interpretarse como un progreso de ideas; cada uno de los nuevos
descubrimientos en la ciencia están motivados por intenciones internas
a los investigadores: la búsqueda de la verdad, curiosidad natural, el
gusto por explicar fenómenos, etc.

Con mejores métodos de investigación, los investigadores ofrecerán mejores respuestas a sus preguntas.
Tendrán, por decirlo de alguna manera, respuestas más detalladas del fenómeno que les interesa y descubrirán nuevos fenómenos que no contemplaban originalmente.
Todo este proceso guiado siempre por sus intereses.
Al ser guiados por sus intereses, y sólo por sus intereses, los investigadores son inmunes a los fenómenos culturales que les rodean.
Dicho de otra manera: que la ciencia es inmune a la cultura. Creo que
podemos identificar a Weinberg\sidenote{Esto es tramposo porque no sé si
los historiadores de la ciencia estiman a Weinberg como uno de sus
colegas. Aparece en este escrito para fines puramente explicativos.} como
un internalista, cuando dice que:

	\begin{quote}
	
	The word \say{discovery} in the subtitle is also problematic.
	I had thought of using The Invention of Modern Science as
	a subtitle. After all, science could hardly exist without human
	beings to practice it. I chose \say{Discovery} instead of
	\say{Invention} to suggest that science is the way it is not so
	much because of various adventitious historic acts of invention,
	but \emph{because of the way nature is}. \parencite[Prefacio,
	énfasis agregado]{Weinberg2015}

	\end{quote}

Quiero destacar el énfasis que hice en la cita anterior. Creo que no hay
mucho peligro en asumir que la investigación es la mejor manera que
tenemos de producir conocimiento; y en particular, creo que parte de este
conocimiento depende de \say{la forma en la que la naturaleza es.} Creo,
además, que la investigación tiene como meta ofrecer explicaciones
\emph{verdaderas} de los fenómenos; puedo decir entonces que me alíneo con
Weinberg en un sentido restringido de \say{descubrimiento}\sidenote{ El
sentido restringido del que hablo es algo que quiero discutir a lo largo
de este capítulo introductorio. Al menos es parte de lo que voy a defender
en este documento. El lector tendrá que leer el trabajo completo
y regresar a esto cuando termine.}. Para regresar a la caracterización del
\emph{internalismo}: el internalista estima el papel que la \emph{verdad}
juega en la ciencia; la ciencia \emph{descubre} fenómenos, no los
\emph{inventa}.

Por supuesto, decirlo de esta manera esconde varios problemas, por ejemplo, esconde que \say{ciencia} no es un término que esté libre de debate; debate que una vez que entramos en los detalles se torna bastante complejo.
Hay por lo menos dos maneras en las que el término es problemático.

En primer lugar, en filosofía de la ciencia, los filósofos debaten si es posible distinguir entre ciencia $/$ no-ciencia.\sidenote{El debate es más sofisticado y no podría explicarlo con justicia en este documento.
	El lector interesado puede revisar la entrada \parencite{sep-pseudo-science}.}.
En segundo lugar, los historiadores de la ciencia debaten si es posible definir en qué periodo comienza la \emph{ciencia} tal como la conocemos en tiempos contemporáneos.

Y si adoptamos una metodología externalista, estas discusiones se vuelven cada vez más claras; y para lograr esta claridad debemos resaltar los \say{actos históricos fortuitos} de los que habla Weinberg; todo ello con la finalidad de elucidar cómo proceden las personas cuando investigan, qué influencias externas afectan el proceso de investigación; y que la historia debe jugar un papel central, si es que queremos responder estas preguntas.

Un externalista explicará los procesos de investigación
científ<t_€ý>hlua require"cmp.utils.feedkeys".run(306) kuica
a partir de las condiciones históricas en las que se desarrollaron;
periodos de tiempo durante los cuales se desenvolvieron los investigadores
y sus investigaciones. El externalista resalta que la investigación
científica no es una actividad que pueda separarse de su contexto
\say{cultural}.

Quiero hacer notar que introduje deliberadamente el término \say{verdad.}
La naturaleza de la verdad es una discusión que requiere más detalle; pero este término es central para la discusión que seguiré a lo largo del documento.
Por ello, quiero rápidamente señalar cómo la \say{verdad} difiere en las dos metoddologías que he descrito.
Sobre la \say{verdad,} Shapin se expresa claramente cuando afirma que \say{All claims have to win credibility, and credibility is the outcome of contingent social and cultural practice.} \parencite[Capítulo 2]{shapin2010never}

Adoptando, digamos, una metodología externalista, Shapin se ha dedicado a estudiar detalladamente como los fenómenos sociales irrigan conceptos tan centrales en la investigación científica como es el de \emph{verdad}; y para aclarar el punto, la siguiente cita es ilustrativa

\begin{quote}
	The notion of truthfulness was thus central to the description of gentle qualities.
	Through the Renaissance and into the eighteenth century an honorable man and an honest man were interchangeable designations: \say{honesty} included the notion of truthtelling but was understood far more broadly to include concepts of probity, uprightness, fairdealing, and respectability. \parencite[pp. 70-71]{Shapin1995}
\end{quote}

Es decir que la verdad estaba asociada no sólo a ser honesto, sino con la figura del hombre honorable; la nobleza y la riqueza suelen estar asociadas al hombre honorable, por tanto su \say{palabra} valía más que las de otros.

Si lo que he descrito hasta ahora es ligeramente correcto y asumimos que Weinberg es un \emph{internalista} y que Shapin es un \emph{externalista}, entonces los párrafos de arriba son suficientes para saber cómo la \emph{verdad} juega un papel en ambas metodologías.
Para el internalista el objetivo de la ciencia es la verdad.
Esta verdad, por supuesto, debería estar acorde con la forma en la que la naturaleza es.
Por otro lado, el externalista nos dice que la verdad siempre está
atravesada por un contexto histórico y que no siempre depende del modo en
que la naturaleza es, sino también por consideraciones sociales como la
honorabilidad. <t_k€>ýrequire"cmp.utils.feedkeys".run(283) uParece que
esta breve caracterización me ha desviado del tema. Pero el papel que
juega la verdad en ambas metodologías es algo que los filósofos han
discutido ampliamente. Además, esta exposición es importante porque
quiero convencer al lector de que hay que considerar seriamente la
distinción entre el \emph{contexto de descubrimiento} y el \emph{contexto
de justificación.} ¿Por qué lo anterior está relacionado con el
\emph{contexto de justificación} y el \emph{contexto de descubrimiento}?
Porque hay una autora que señala que la distinción entre factores
\emph{internos} y \emph{externos}, surge a partir de la distinción de los
dos contextos.

Para lograr mi propósito, quiero discutir un artículo que presenta un caso más cercano a la filosofía de la ciencia.
Un artículo que relaciona por un lado a las metodologías que he discutido hasta aquí, y por otro cómo éstas juegan un papel en la filosofía de la ciencia.
Me dedico a este caso a continuación.

\subsection{Un caso más cercano a la filosofía de la ciencia}\label{sbc:yturbe}

\noindent En el artículo \citetitle{Yturbe1995}, la autora argumenta que es un falso dilema tener que elegir entre las metodologías \emph{internalista} y \emph{externalista.}
Yturbe dice, por ejemplo que

\begin{quote}
	The general tendency in the historiography of the sciences is to consider the social function of science, as weIl as some aspects of the matrix from which the problematic is formed, as external elements, while the conceptual apparatus and the problem field are treated as internal. \emph{But we should not think of science as having two independent histories.} \parencite[p.85, Énfasis agregado]{Yturbe1995}
\end{quote}

La oración en itálicas implica que no deberíamos hacer una distinción entre factores internos y factores externos.
Además, Corina defiende que una teoría a la vez internalista como externalista no es una teoría plausible.
Al menos esto parece sugerir cuando dice que

\begin{quote}
	The search for new explanatory programs is characterized above all by the attempt to reconcile the internalist and externalist approaches.
	But, in view of the fact that both approaches are committed to theses concerning the nature of science that are not only incompatible, but are unsustainable, their union without important changes in their philosophical presuppositions cannot result in a viable program. \parencite[p. 79]{Yturbe1995}
\end{quote}

Su conclusión se basa en que no es posible marginar ambos factores; además, la autora sugiere que esta supuesta distinción tiene sus orígenes en la distinción entre el \emph{contexto de descubrimiento} y el \emph{contexto de justificación} que trazaron los positivistas.
La autora señala, por ejemplo que

\begin{quote}
	External factors are not only found in the context of discovery, they are present also in the development of the concepts, problems, methods, problem fields, etc. of scientific discourses: that is, external factors are present in the context of validation itself.
	There are scientific discourses in which ideological conceptions pass on to form part of the body of the science itself, functioning as principles which define its field of study or guide its research; thus, external factors can become internal. \parencite[p. 85]{Yturbe1995}
\end{quote}

La cita anterior sugiere que el \emph{contexto de justificación} involucra sólo \emph{factores internos,} mientras que el \emph{contexto de descubrimiento} involucra sólo \emph{factores externos.}
La autora nos dice que los factores externos contagian al contexto de justificación, cuando dichos factores se convierten en parte de las teorías --generando conceptos, principios centrales, etc.--
La autora dice que esta ceguera a considerar que los \emph{factores externos} afectan al \emph{contexto de justificación,} surge de la \say{doctrina de los dos contextos.}
Corina nos dice que \say{One of the philosophical theses in favor of the contraposition between the internalist approach and the externalist approach is the so-called doctrine of the two contexts, developed by positivism.} \parencite[p. 75]{Yturbe1995}

Debido a que quiero convencer al lector de que deberíamos adoptar la distinción entre los dos contextos, debo señalar que la autora se equivoca.
Que es perfectamente plausible hacer la distinción entre el \emph{contexto de descubrimiento} y el \emph{contexto de justificación} sin que esto implique la distinción enctre factores \emph{internos} y \emph{externos.}

Para lograr esto, voy a señalar que la premisa, en la que la autora nos dice que el contexto de justificación$/$descubrimiento es equivalente a los factores internos$/$externos, es falsa y que la equivalencia descansa en una confusión.
Esta confusión se debe, en parte, a la mala caracterización de las tesis del \say{positivismo}; la otra parte, me parece, se debe a una interpretación poco caritativa de la historia de la ciencia y los propósitos de los investigadores.

El argumento de la mala caracterización del positivismo lo doy en este mismo capítulo.
El argumento sobre la interpretación y los propósitos de los investigadores será dado a lo largo del capítulo 2.
Pero, este argumento comienza en este capítulo, en especial la sección \ref{ssc:aprender}.

Recordemos que el argumento que voy a ofrecer es para concluir que la autora ofrece una mala caracterización del positivismo.
Esta conclusión depende de que el \emph{contexto de descubrimiento} y el \emph{contexto de justificación} no son equivalentes a los \emph{factores externos} y \emph{factores internos} respectivamente; lo que Yturbe llama \say{la doctrina de los dos contextos}.
Para este propósito quiero repasar brevemente en qué consisten el \emph{contexto de descubrimiento} y el \emph{contexto de justificación.}



\paragraph{La distinción} entre el \emph{contexto de descubrimiento} y el \emph{contexto de justificación}, generalmente está asociada a Reichenbach\sidenote{
	Stillwell comenta que la distinción puede trazarse hasta Arquímedes.
	Al menos eso entiendo cuando afirma que \say{Archimedes was probably the first mathematician candid enough to explain that there is a difference between the way theorems are discovered and the way they are proved.} \parencite[p. 56]{stillwell1989mathematics}
	}.
En su libro \citetitle{reichenbach1938experience}, el autor usa esta distinción para \say{excluir} los así llamados factores externos (sean sociales, políticos o económicos.)\sidenote{
	El pasaje no dice exactamente esto, sino algo más cercano a que tenemos claro cómo hacer un análisis filosófico$/$formal para la justificación de teorías, mientras que no tenemos manera de hacer ese análisis en el contexto de descubrimiento.
	Este lado de la investigación científica, digamos, es muy heterogéneo como para que sea posible ofrecer un análisis formal.
	}.
Esta distinción, nos dice Reichenbach, sirve para ilustrar el hecho de que no tenemos a la mano un \say{método} para analizar el fenómeno del descubrimiento científico; lo que algunos filósofos \parencite{reichenbach1938experience, Seo2015} llaman el momento \emph{Eureka}.
Pero no tener una herramienta de análisis formal, no implica que lo que sucede en el contexto de descubrimiento sea absolutamente deleznable.

Que los factores \emph{externos} pueden afectar el \emph{contexto de justificación}, fue un tema que se discutió ampliamente durante los años del círculo de Viena\sidenote{
	O positivismo lógico o empirismo lógico, cada quien elige su etiqueta favorita.
} y no es una distinción que, digamos,  permaneció fija a lo largo de la historia del positivismo lógico. 
O aquello a lo que Yturbe se refiere como \say{positivismo}.

Como me parece que la distinción juega un papel importante en (i) cómo interpretamos filosóficamente la historia de la ciencia y que juega un papel clave en (ii) cómo los investigadores realizan investigación, necesito convencer al lector de que esta distinción es útil; 
y me pareceque ofrecer una justificación para (i) y (ii), es suficiente para convencer al lector de que vale la pena adoptar la distinción.
Llamaré a (i) el \emph{objetivo metodológico} y a (ii) el \emph{objetivo epistémico}
Para lograr estos objetivos, quiero repasar brevemente cómo algunos de los miembros del círculo de Viena lidiaron con la distinción que nos ocupa.


\subsection{El círculo de Viena}

\noindent Quiero comenzar señalando que los positivistas\sidenote{
	Me voy a referir con este término a los filósofos que aparecen en la publicación de la \emph{International Encyclopedia of Unified Science} \parencite{Carnap1938-CARFOL-10}.
	Tanto del comité organizador, como el comité asesor.
	}, 
estaban al tanto de cómo los \emph{factores externos} influyen en el \emph{contexto de justificación}.
Dicho de otra manera, que los aspectos sociales, políticos y económicos afectan las prácticas de las personas dedicadas a hacer investigación.

La afirmación de Reichernbach sobre la carencia de un análisis formal del \emph{contexto de justificación} es una carencia que los miembros del círculo buscaron solventar; y lograron dicho análisis con mayor o menor éxito.

\paragraph{Neurath}, nos recuerda Joseph Bentley, sostenía que

\begin{quote}
	Despite his advocacy for scientific methods, Neurath never takes this method to be set in stone, nor does he attempt to portray science as an enterprise of purely objective methods, completely divorced from social, historical, and material contexts or the personalities of scientific practitioners. \parencite[p.~41]{Bentley2023}
\end{quote}

Neurath sabía que los fenómenos sociales y políticos afectan las prácticas científicas; sin embargo, sostenía también, que estudiar esta relación debe llevarse a cabo con otro tipo de herramientas.
Bentley lo expresa mucho mejor cuando señala que \say{As in the case of theory-choice, decision-making is central. 
But to make metatheoretical decisions, metatheoretical information is needed.} \parencite[p.~62]{Bentley2023}

Neurath llama \say{beahaviouristics of scholars} al conjunto de herramientas y métodos para reunir esta información metateórica; 
esta disciplina sería aquella con las herramientas para analizar aspectos históricos, políticos, económicos, etc.
Lo que aquí estamos llamando factores externos.

Y al hablar específicamente de la justificación de creencias, Bentley señala que \say{[science], Neurath maintains, it is still the best we have} \parencite[p.~41]{Bentley2023}.
Y si somos capaces de reconocer cómo las prácticas, digamos los factores externos,  juegan un papel, entonces podemos evaluar y mejorar nuestras prácticas de investigación.
Utilizando la terminologá que sugiero, Neurath ofrece una respuesta al
problema metodológico. Neurath nos dice que durante el \emph{proceso de
justificación} de una teoría, los factores externos están involucrados en
la justificación. Es sólo cuando incluimos dichos factores que podemos
mejorar el proceso de justificación

No es tarea fácil mejorar las herramientas que usamos en investigación, porque modificar una herramienta tiene como consecuencia analizar detalladamente no sólo la herramienta, sino que hay que revisar los resultados que dependen de ella.
Para mejorar nuestras prácticas, no debemos olvidar que Neurath fue un \say{holista} confirmacional; esto significa que Neurath creía que no podemos confirmar o falsificar una proposición aislada, sino que todo el conocimiento es juzgado a la vez; 
cada vez que modifiquemos una hipótesis o alguna de nuestras herramientas, hay que hacer cambios en otras hipótesis y herramientas.\sidenote{
	Tanto fue su interés por sistematizar la comunicación entre diferentes comunidades científicas que diseñó el proyecto de la Enciclopedia de la  Ciencia Unificada.
	De esta manera el trabajo se distribuye entre distintas comunidades y se coteja con el trabajo de otras comunidades de investigación.
	}

Sólo podemos hacer mejores nuestras herramientas y minimizar los errores cuando somos capaces de reconocer el papel que juegan los factores externos, Bentley expresa mejor el objetivo de Neurath diciendo \say{if we recognize the $\ldots$ creation of the norms, methods and values of science, it can be made better.} \parencite[p.~41]{Bentley2023}

Quiero enfatizar que Neurath no fue el único miembro de los positivistas lógicos que reconoció el papel que juegan estos factores en la investigación;
las personas que realizan esta actividad tienen ciertos sesgos políticos y sociales, son personas que tienen que tomar decisiones y que tratan de justificar sus hipótesis con las mejores herramientas a la mano.
Philipp Frank expresa mucho mejor la relación entre los factores \emph{externos} y el \emph{contexto de justificación.}



\paragraph{Philipp Frank sostenía que} los aspectos \say{externos} eran
una parte crucial del análisis que los filósofos pueden ofrecer de las
ciencias. En su artículo \citetitle{Frank1956}, Frank argumenta que los
aspectos sociales jamás han estado excluidos del proceso de
justificación de teorías. Esto es más o menos claro cuando Frank nos dice
que \say{The special mechanism by which social powers bring about
a tendency to accept or reject a certain theory depends upon the structure
of the society within which the scientist operate.}
\parencite[p.~143]{Frank1954}

Para su argumento, Frank comienza enfatizando que a lo largo de la historia, la elección de teorías nunca ha sido arbitraria;
supongamos, por ejemplo, que tenemos dos teorías en competencia $a$ y $b$.
Los investigadores no simplemente deciden entre $a$ o $b$ sopesando cuál de las dos tiene más consecuencias empíricas.
Frank nos recuerda que tomar una decisión entre $a$ y $b$ involucra diferentes aspectos políticos y sociales.
Estos aspectos son parte del proceso de justificación de una teoría en donde no sólo revisamos las consecuencias empíricas de la teoría, sino además qué tan coherente es con otros dominios (digamos biología, química, economía, etc.)
Estos aspectos tanto teóricos como sociales están involucrados en la justificación de teorías porque la investigación es un producto realizado por seres humanos;
como seres humanos, tenemos capacidades cognitivas limitadas \parencite{Potochnik2017-POTIAT-3}, además tenemos una gran cantidad de sesgos implícitos \parencite{nordell2021end} y estar al tanto de dichos fenómenos ayuda a mejorar nuestras herramientas.
Si para el proceso de elección de teorías sólo evaluamos a la teoría al medir qué tanto está \emph{de acuerdo con los hechos}, nunca podríamos decir si una teoría es \emph{mejor que} la otra.

Hay que aclarar que Frank expresa que una teoría \emph{es mejor que} otra en términos de la \emph{utilidad} de la teoría.
Diha utilidad, depende de los propósitos que queremos lograr con una teoría particular;
y si la teoría va a ser \emph{usada}, entonces debe ser \emph{simple.}
Esto lo expresa diciendo que \say{The final theory has to be in fair
agreement with observations and also has to be sufficiently simple to be
usable.} \parencite[p.~14]{Frank1956} Siguiendo esta línea de
razonamiento, Frank afirma que una teoría no se juzga sólo a partir de las
consecuencias empíricas, sino también de acuerdo a qué uso queremos darle:
cuando es usada para fines \emph{prácticos.}

Frank lo expresa mucho mejor al decir que \say{[h]owever, the situation becomes much more complex, if we mean by simplicity not only simplicity of the mathematical scheme but also simplicity of the whole discourse by which the theory is formulated.} \parencite[p.~4]{Frank1956}
Hay que saber, además, que el discurso bajo el que la teoría está formulada, en algunas ocasiones, sirve para fines propagandísticos.
Tanto Neurath como Frank condenaban dichos fines propagandísticos, pero la única manera de saber cuándo los fines son propagandísticos es conociendo cuáles son los supuestos de la teoría y la finalidad que se le está dando.
Si involucramos la utilidad de la teoría y el discurso bajo el cual está formulada, como filósofos de la ciencia podemos ofrecer un análisis más completo.

Las afirmaciones anteriores, que hay que considerar el discurso bajo el cuál la teoría está formulada y que el valor de la teoría depende también de los fines para los cuales va a ser usada, son afirmaciones inscritas en el problema de la \emph{elección de teorías} y en el problema de la \emph{subdeterminación empírica de las teorías.}
Supongamos que queremos decidir si una teoría $a$ es mejor que $b$. 
Según lo que nos dicen Neurath y Frank debemos involucrar los factores externos y es en este sentido en el que los factores externos son parte del contexto de justificación.
Aceptar o rechazar una teoría es un proceso que involucra \emph{justificar} las teorías y decidir cuál es \emph{mejor} para los propósitos que deseamos.
Es durante la fase de aceptación de la teoría, cuando los factores externos toman un papel central, porque de no ser por la finalidad práctica de una teoría, no tendríamos manera de decidir qué teoría es más adecuada.
Don Howard señala esto al decir que 

\begin{quote}
	On Neurath’s view, it is a contingent fact, well supported by historical evidence, that we do choose among empirically equivalent theories on the basis of our estimation of the likelihood of their serving our favored social and political ends, this especially in sciences like economics and sociology. \parencite[p.~5]{Howard2006}
\end{quote}

Muchas de estas afirmaciones se deben a que los miembros del círculo de Viena estaban interesados en saber si $a$ es equivalente a $b$;
y algunos de los miembros del círculo argumentaban que la única manera en que $a$ y $b$ sean equivalentes es si hacen las mismas predicciones, es decir, si tienen las mismas consecuencias empíricas.
O como también dice Don Howard

\begin{quote}
	The place of values in science is secured by the fact that, on Neurath’s view, logic and experience underdetermine theory choice.
	But turn that argument around and it implies that values come into play only within what I like to call the domain of underdetermination. 
	That is, logic and experience are first allowed to do all of the work they can do. 
	Only then do we ask which of several empirically equivalent theories is most conducive to the achievement of our social and political ends. \parencite[p.~10]{Howard2006}
\end{quote}

Señalo lo anterior porque no es completamente obvio por qué las consecuencias empíricas no son suficientes para resolver el problema de la elección de teorías;
No es obvio porque es completamente plausible que $a$ y $b$ tengan \emph{distintas} consecuencias empíricas;
además, suponiendo que de hecho hay subdeterminación empírica, es completamente plausible que $a$ y $b$ sean contradictorias entre sí y que tengan las \emph{mismas} consecuencias empíricas.

Para aclarar por qué no es suficiente, hay que saber que esta conclusión está motivada por el fenómeno de la \emph{subdeterminación empírica de las teorías.}
El fenómeno de la subdeterminación empírica sucede cuando en un punto del tiempo la evidencia no es suficiente para determinar qué teoría deberíamos aceptar.
Dos teorías contradictorias entre sí pueden hacer las mismas predicciones, en cuyo caso, la evidencia no es suficiente para decidir entre una teoría u otra.
Ahora, supongamos que en un momento dado sólamente existe una teoría.
Supongamos además que esta teoría ha hecho predicciones exactas hasta ahora y que dada la teoría predecimos que sucederá un fenómeno $f$.
Llega la fecha de la predicción y $f$ no sucede.

Si la descripción anterior le parece plausible al lector, entonces hay que investigar en qué consistió el error.
Sabemos además que una teoría depende de un conjunto de supuestos y que si $f$ no sucede, entonces hay que revisar cuál de los supuestos de la teoría falló.
Si el valor de la teoría depende sólo de las predicciones que podemos obtener al hacer uso de ella, entonces la teoría ha perdido su valor.

Pero no es obvio que las teorías que hacen predicciones erróneas pierden completamente su valor, hay al menos un ejemplo en el que modificar alguno de los supuestos ha llevado a descripciones exitosas: el caso del desccubrimiento de NEptuno  
Si hay casos reales de subdeterminación empírica en la historia de la ciencia no es algo que pueda discutir aquí.
Pero saber que puede suceder un caso como este es suficiente para concluir que si la evidencia es lo único que importa, entonces puede haber casos de subdeterminación empírica 



haciendo énfasis en que estos factores sociales jamás han desaparecido del \emph{contexto de justificación}.
En particular, ante dos teeorías en competencia, la decisión no puede tomarse sin este tipo de consideraciones prácticas.
Consideraciones prácticas que dependen del contexto histórico y sociocultural\sidenote{En algunas ocasiones la utilidad de la teoría es funcionar como propaganda, algo que señalan los autores de \parencite{Lewontin2017}  y también Frank en \parencite{}.}.

Hasta este punto, me parece que la evidencia textual muestra suficiente información para concluir que hay una confusión en la caracterización de la llamada \emph{doctrina de los dos contextos} \parencite{Yturbe1995}.
Se suele caracterizar a los miembros del círculo de Viena defendiendo tesis exageradamente estrictas, Philipp Frank incluso lo menciona en \say{cita de Frank} y más recientemente, Bentley también cuando dice que \say{citga de bentley}.
Y esta no es la única mención a esto \say{cita del número especial}

Suele mencionarse también que los miembros del círculo no prestaron atención a ciertos temas por estar demasiado ocupados con la física [citar al de Sahorta] o que el problema de la reducción de teorías incluía el proyecto de reducción de las ciencias especiales a la física [citar a Raphael van Riel].
Afortunadamente, existen intentos recientes por ofrecer una caracterización justa de las afirmaciones de los miémbros del \emph{círculo de Viena.}

Quiero señalar que esto puede ser una preocupación más filosófica que histórica, y por lo tanto, parecer que como Yturbe \cite{Yturbe1995} estoy confundiendo una tesis filosófica y una tesis sobre historiografía.
No soy historiador, por lo que no puedo más que referirme a dos autores en particular que, me parece, vinculan las preocupaciones de los miembros del círculo con la historiografía de la ciencia.
A esto me deddico en la siguiente sección.


\subsection{Justificar una teoría es el objetivo principal de la investigación}

Llegado a este punto, he señalado que la caracterización que hace la autora de la \emph{doctrina de los dos contextos} es incorrecta; 
más aún, prometí que valía la pena recuperar esta distinción porque me parece que el objetivo principal de la investigación 
científica es el \emph{proceso} de \emph{justificar} teorías.
Quiero llegar a esta conclusión a partir de lo que dije anteriormente: como la caracterización de Yturbe es errónea, tenemos una alternativa; 
nuestra alternativa es que hay una manera de separar el \emph{contexto de justificación} del \emph{contexto de descubrimiento}.




\paragraph{Justificar} una teoría, nos dicen ambos filósofos, involucra necesariamente aspectos pragmáticos.
La práctica científica no está separada de su contexto cultural.
Pero cuando lidiamos con procesos de justificación, lo mejor que tenemos son las herramientas que usamos en investigación.
Herramientas que pueden cambiar con el tiempo y que necesitarán una justificación más robusta.
Todo esto con el objetivo de señalar si una oración es verdadera o falsa.
Reconociendo que la verdad y la falsedad de las oraciones no está separada de otras oraciones.

Estos comentarios sobre el \say{materialismo dialéctico} y el \say{pragmatismo americano},
% Mencionar la mención al operacionalismo 
, pueden sonar fuera de lugar.
Pero es importante mencionar esto porque las fuentes históricas que voy a citar son, una de ellas soviéticas, usando la llamada metodología del \say{materialismo histórico}, mientras que la otra fuente ofrece una interpretación operacionalista de la matemática griega.

El pragmatismo americano se vincula al operacionalismo

Pero

Cómo juega el verificacionismo un papel?, esto es, por qué la manera de verificar si una oración es verdadera o falsa depende de cómo se verifica?
Lo único que nos piden los positivistas es que tengamos cuidado al juzgar si la oración es verdadera.
Al juzgar una oración, debemos tener en cuenta qué operaciones serían necesarias para saber si la oración es verdadera.
A los términos singulares de la oración, se les da una definición \emph{operacional}, donde por operacional, los positivistas se refieren a qué experiencias físicas serían necesarias para decir que una oración es verdadera.

Some has to do with it's sustainability.
Sólo para hacer la distinción entre el contexto de descubrimiento y el contexto de justificación.
Pero no hay que confundir a la operación con la cosa.
Hay que distinguir claramente entre la decisión de cómo representar un fenómeno y entre el fenómeno mismo.

Sobre la representación

pues introduciendo
Digamos que el operacion

Frank traza también un puente entre el pragmatismo, el materialismo histórico soviético y la filosofía del positivismo.


%\section{citas}



%Value considerations are not intended to trump considerations of logic and experience; they are intended to respect them. Don Howard p.10

%The place of values in science is secured by the fact that, on Neurath’s view, logic and experience underdetermine theory choice. But turn that argument around and it implies that values come into play only within what I like to call the domain of underdetermination. That is, logic and experience are first allowed to do all of the work they can do. Only then do we ask which of several empirically equivalent theories is most conducive to the achievement of our social and political ends. Don Howard p. 10

The freedom of choice is not a freedom simply to deny the manifest evidence of the senses. One has to interpret, one has to tell a coherent story, and one has to tell a story that works. p. 14


This means there is no possibilityt of isolating a class of priviledged sentences, to act as a fixed foundation Naturalism and the Vienna Circle

Every term  introduced into the theory must be accompanied by  a description of the physical operations by which may  be tested the degree to which the property expressed  by this term may be attributed to a given physical  system. The description of these operations, the  "operational definition" of this term, may be more or  less direct; perhaps only a combination of terms will  correspond to a certain operation. Professor Bridgman's views have frequently been labelled "operationalism," although he himself is not pleased by this name.

But the outlook of the so-called "Vienna Circle"  has been only one particularly coherent doctrine  among the many intellectual fruits which have  emerged from the soil of Central-European positivism.

In particular, Charles  W . Morris of Chicago recognized the connection  with American pragmatism and publicized the idea  of cooperation between the two groups. It was decided, for the purpose of this cooperation, to call a  special congress, for which the name "Congress for  the Unity of Science" was coined by Otto Neurath.

The conception of the relative worthlessness of the theory in  comparison to the phenomenon gives to the theorizing  of such an investigator something especially free and  imaginative.

The known connections among  phenomena form a network; the theory seeks to pass  a continuous surface through the knots and threads  of the net. Naturally, the smaller the meshes, the  more closely is the surface fixed by the net. Hence,  as our experience progresses the surface is permitted  less and less play, without ever being unequivocally  determined by the net.

nce this possibility has been substantiated,  the whole of analysis can proceed to develop as usual.  But now when a theorem about derivatives is set up  and somebody begins to subtilize about it, asking  whether this theorem is really in agreement with the  "nature" of the differential and going into profound  and skeptical deliberations concerning this "nature,"  he can be told quite simply: "I could express this  theorem, if I took enough time, as a theorem about  integers; the nature of this theorem is hence no more  and no less mysterious than that of the natural  numbers."

The  atoms are auxiliary conceptions just like others which  can be employed advantageously in a limited domain.  They are not suitable for an epistemological foundation.

Más aún, Frank veía en la práctica científica una manera de resolver problemas como el de la representación y el de la elección, digamos \say{temporal}, de una teoría.
Esto queda bastante claro cuando Frank afirma

	\begin{quote}
	
	If we look for an answer to the question of wether a certain theory, say
	the copernican system or the theory of relativity, is preferred or true,
	we have to ask the preliminary question: what purpose is the theory to
	serve? \parencite[p. 15]{Frank1954} 

	\end{quote}

Ambos filósofos estaban al tanto de que los factores externos afectan el contexto de justificación.
Ambos fueron miembros del círculo de Viena y si ambos sostenían que hay una distinción entre el \emph{contexto de descubrimiento} y \emph{el contexto de justificación}; entonces no son equivalentes el \emph{contexto de descubrimiento} y los \emph{factores externos}; y  tampoco son equivalentes el \emph{contexto de justificación} y \emph{los factores internos}.

Me parece que con señalar que los miembros del círculo de Viena que he discutido hasta ahora, no se alineaban con las tesis descritas por \parencite{Yturbe1995}, es suficiente para decir que hay una confusión en la premisa que usa la autora.
Sin embargo, quiero decir un poco más sobre las tesis a las que se adherían Neurath y Frank; además quiero señalar la relación que ve Frank entre la postura del \say{empirismo lógico} y la \say{filosofía de la Unión Soviética}, que es algo que quiero discutir en la siguiente sección.

\paragraph{Tanto Neurath como Frank} sostienen una forma de verificacionismo: que el significado de de una oración depende del método de comprobación.
Pero, ambos están de acuerdo en que nunca tenemos una imagen perfecta, con la que podamos saber con completa seguridad, que una oración es verdadera o falsa.



%\section{Citas Amsterdamski}

%The sociology of science studies the evolution of science as a  social institution, the styles of scientific thinking, the reception of scientific  ideas and their social determination. Finally, logic and the methodology  of science (often called the philosophy of science), as a branch of epistemology, study the structure of scientific theories, their development, the  rules of concept and theory formation, the criteria of accepting and  refuting the claims of science, and the relations between theory and  empirical data.

%Anyone who has even a slight interest in the history of science must have  found himself faced by many bewildering problems. How was it possible,  for instance, that the most remarkable minds of their times held opinions  that could today be disproved by any high school student, often on the  basis of empirical facts known even then? It is just this last circumstance  which appears to be most striking, for it is one thing to accept theories  which later may be proved imprecise or simply false, but are not contradicted by known facts, and quite a different matter to maintain opinions regardless of known facts which clearly contradict them. If the first  instance seems quite natural to us, as we believe that it is the discovery of  new facts which usually compel us to revise old theories, then the second  instance appears to be perplexing; in fact, it seems to contradict the very  nature of scientific knowledge.

%the paintings of the Aztecs were more valuable than those of the European  Renaissance, the philosopher or historian of science sees nothing wrong  with comparing the theories of Ptolemy and Copernicus, Newton and  Einstein, or Darwin and Mendel. He is convinced that these theories had  to provide a coherent explanation of the same domains of phenomena (the  movement of celestial bodies, mechanics, the mechanism of heredity),  and that in attempting to formulate explanations they utilized the same  criteria for evaluating the results of their work.

%Although we are prone to evaluate works of art in a relativistic manner,  taking into account the aesthetic criteria and historical circumstances  which are particular to a given period, we tend to evaluate the achievements of scientists (just their achievements, not their merits) on the grounds  of some supra-historical criteria. More precisely, we tend to evaluate  scientific results on the grounds of those criteria which are accepted by  contemporary science and which we take (for better or for worse - we  will come back to this problem) as supra-historical.

%It was not only the Copernican theory that for years had to deal with a  mass of anomalies. The same thing happened with Newton's theory, with  quantum mechanics, and in general with any new theory which required a  radical reconstruction of large areas of acquired knowledge. This is the  fate of any new theory which attempts to introduce a new order into the  domain of phenomena which it describes and which were ordered in a  different way by its predecessors. Since every such theory must take into  account experiments and observations previously carried out, it must  therefore reinterpret them, make them fit into the new theoretical framework. Sometimes this reinterpretation is extremely complex and hardly  discernible to anyone but the specialist, especially in cases when the meaning of those basic concepts which became a part of everyday language  undergo a radical change. Thus after the rise of the theory of relativity, the  meaning of such notions as 'mass', 'velocity' or 'simultaneity' had  changed, and this fact should be taken in account when we investigate its  relation to Newtonian mechanics.

%such as, for instance, the contemporary state of mathematical knowledge,  the theoretical situation in other disciplines, the state of the technical  apparatus available or the accepted epistemological and ontological  beliefs, etc. To quote F. Jacob, there is" ... a domain which thought strives  to explore, where it seeks to establish order and attempts to construct a  world of abstract relationships in harmony not only with observations and  techniques but also with current practices, values and interpretations."15

%It is worth remembering that the empirical confirmation of the Copernican theory, which at the same time disproved the Ptolemaic system,  was only provided many years after the death of Copernicus. Copernicus  himself could offer no decisive fact which would confirm his own theory.  "No fundamental astronomical discovery, no new sort of astronomical  observation, persuaded Copernicus of ancient astronomy's inadequacy or  of the necessity for change."!7 The prognoses of celestial phenomena  provided by either theory proved to be essentially the same.

%The content of Newtonian mechanics not only stepped outside the  boundaries of empirical data and, therefore, could not be deduced from  them, but also was incompatible with observational data which were  known at the time when it was formulated - such as the impossibility of  distinguishing between absolute and relative motion.

%In any case, neither statements of scientists nor  reasonings of philosophers provide sufficient confirmation of the thesis,  that in posing experimental questions about nature, and reading the  answers by means of measuring instruments, scientists were free from  hypotheses and presuppositions and that in declaring the glory of radical  empiricism they were not subject to any philosophy or metaphysics. We  would indeed be quite naive if we took their words in this matter.


%Secondly, what is the true relationship between the empirical basis of  science (i.e., facts), and theory, and what changes of this basis can provide  a satisfactory foundation for the understanding of the process of the  evolution of scientific cognition?



%Los factores sociales y políticos son medios para sleccionar una teoría.
%De acuerdo a los objetivos que buscamos en la investigación.
%Y estos objetivos son influencias que imbuyen el contexto de justificación.
%
%Incluso Carnap, que es al que más le llueven vergazos y del que se sotiene tenía las ideas máws radicales deel círculo, llegó a aceptar que las teorías tienen sin duda un aspecto pragmático. 
%Neurath y Philipp Frank exponen este tipo de preocupaciones.


%Neurath fue falibilista, coherentista (sobre la justificación) y holista.
%Por supuesto que sostenía una forma de \emph{verificacionismo}, Bentley lo expresa más claramente \say{Theories must be sensitive to, grounded in and testible by experience} \parencite[p. 38][]{Bentley2023}.
%Este criterio es bastante vago, pero Bentley lo describe no como una tesis, sino como una postura; una postura que es difícil capturar con una sola afirmación \parencite[p.38][Cfr.]{Bentley2023} 


%\subsection{Citas Phillip Frank}

%The misinterpretation of scientific principles, as will be shown, can be avoided if, in every statement found in books on physics or chemistry, one is careful to distinguish an experimentally testable assertion  about observable facts from a proposal to represent the facts in a certain way by word or diagram.
%If this distinction is sharply drawn, there will no longer be any room for the interpretation of physics in favor of a spiritualistic or a materialistic metaphysics. [p. 4-5]

%The views represented in the present book are closely associated with the movement now generally  called "logical empiricism" or "logical positivism."
%I must confess that I do not like these words either.
%But a long life among views and theories has shown me that if we want a view to be regarded as a respectable tree in the garden of opinions it must have a label just as much as the elms and oaks in our public gardens. [p. 5]

%* recordar que Philipp Frank fue fundador del grupo original, que 20 años después se convirtió en el círculo de Vienna.

%* Cualquier reconstrucción se le va a quedar corta a este libro.

%But the outlook of the so-called "Vienna Circle"  has been only one particularly coherent doctrine  among the many intellectual fruits which have  emerged from the soil of Central-European positivism. Mathematicians such as R. von Mises, Κ. Menger, and Κ. Godei, physicists such as E. Schroedinger,  economists such as J. Schumpeter, lawyers such as  H. Kelsen, and sociologists such as E. Zilsel had their  roots in this environment. The whole intellectual  background of this general movement can be understood best through R. von Mises' textbook of positivism. [p. 9]

%Several young American philosophers  traveled to Vienna and Prague in order to come into  scientific contact with Schlick and Carnap. Among  them were W . V . Quine (now at Harvard) and  E. Nagel (now at Columbia). In particular, Charles  W . Morris of Chicago recognized the connection  with American pragmatism and publicized the idea  of cooperation between the two groups. []

%But what does it mean when we say that a question  is insoluble? Let us suppose, for example, that someone has asserted that the problem of a regular airplane  route to the planet Neptune is insoluble, or that the  production of a living organism from lifeless matter  is insoluble. 
%Despite this assertion, the person making  it can describe quite accurately the concrete experience we should have if the problem were solved.

\section{¿Qué podemos aprender de la historia de la ciencia?}\label{sub:aprender}


\noindent Quiero comenzar esta sección confesando lo siguiente: estoy de acuerdo con la autora, en particular con el carácter de la conclusión anterior, esto es, que es complicado--si no es que imposible--, separar entre los procesos externos y los procesos internos que influyen en la investigación científica.
Pero aceptar esto --eso quiero argumentar-- no implica que debamos deshacernos de la distinción entre el \emph{contexto de investigación} y el \emph{contexto de descubrimiento}.
En lo que resta de la sección quiero hacer un breve repaso de los argumentos de la autora.

Yturbe nos dice que el contexto de descubrimiento, se dedica a analizar factores externos que fueron influyendo en los factores internos; mientras que el \emph{contexto de justificación} sólo se dedica a analizar los factores internos al desarrollo teórico.
Lo que ella llama \say{the doctrine of the two contexts}\sidenote{
	\say{one of the philosophical theses in favor of the contraposition between the internalist approach and the externalist approach is the so-called doctrine of the two contexts, developed by positivism.} \cite[p. 75]{Yturbe1995}
}
La autora nos dice que la doctrina de los dos contextos hace una distinción que no se puede hacer.
El contexto de justificación es siempre influenciado por el contexto de descubrimiento porque la ideología, la economía, las relaciones de poder, etc. influyen en la toma de decisiones\sidenote{Por usar una caricatura: a quién se le asigna presupuesto}, por lo tanto la doctrina es incorrecta.

Como dije al principio de esta sección, estoy de acuerdo con que es casi imposible hacer la distinción entre \emph{historia interna} e \emph{historia externa}, pero aceptar esto, no implica deshacernos de la distinsión \emph{contexto de descubrimiento} y \emph{contexto de justificación}.
Para sustentar esto, dependo de la caracterización que hace la autora sobre \say{la doctrina de los dos contextos}, la caracterización de la doctrina está ligeramente sesgada, si no es que completamente inadecuada.
En la siguiente sección quiero ofrecer mis razones para esta conclusión.

\subsection{La doctrina}

\noindent En el texto de \textcite[p.75]{Yturbe1995}, la autora ofrece una descripción de la doctrina, ella señala que los filósofos doctrinarios afirman que \say{according to this position, [factores externos], not only fails to increase our understanding of scientific development, but even obscures the fundamental question of the rational validation of scientific arguments.}
Además que \say{This conception constitutes the dominant framework in which the philosophy of science has been developed, and consists in drawing a radlcal distinction between the context of discovery and the context of validation.}


Su argumento para decir que hay un colapso entre factores externos (contexto de descubrimiento) y factores internos (contexto de justificación) se encuentra en la página 85

\begin{quote}

	External factors are not only found in the context of discovery, they are present also in the development of the concepts, problems, methods, problem fields, etc. of scientific discourses: that is, external factors are present in the context of validation itself.
	There are scientific discourses in which ideological conceptions pass on to form part of the body of the science itself, functioning as principles which define its field of study or guide its research; thus, external factors can become internal.

\end{quote}

La doctrina de los dos contextos implica que podemos claramente separar entre el \emph{contexto de descubrimiento} y el \emph{contexto de justificación}.
Pero los factores externos están presentes en el contexto de justificación, tanto así que pueden convertirse en factores internos.
Esto implica que los contextos no son claramente separables, y que, por tanto, la doctrina es incorrecta.

Contrario a lo que dice la autora

\begin{quote}
	The search for new explanatory programs is characterized above all by the attempt to reconcile the internalist and externalist approaches. But, in view of the fact that both approaches are committed to theses concerning the nature of science that are not only incompatible, but are unsustainable, their union without important changes in their philosophical presuppositions cannot result in a viable program. \parencite[p. 79]{Yturbe1995}
\end{quote}

Creo que es verdad que no podemos distinguir entre factores externos y factores internos en la ciencia.
Sin embargo, me parece que lo que expresa la cita anterior es claramente erróneo: no se siguie la conclusión \say{que no pueden resultar en un programa viable}, porque depende de señalar que la distinción entre  \emph{contexto de descubrimiento} $|$ \emph{contexto de justificación} es equivalente a la distinción \emph{factores externos} $|$ \emph{factores internos} y estas distinciones no son equivalentes.



\subsection{En defensa de la doctrina}

\noindent En años recientes ha habido un creciente interés en discutir las publicaciones del Círculo de Vienna \parencite{Bentley2023, Richardson2023, Suarez2024, Riel2014}.
Si hago una tosca generalización, diría que literatura reciente se ha dedicado a señalar que los miembros del Círculo de Vienna, no eran tan herméticos como se pensaba.
Además, la cantidad de temas que discutieron es más vasta de lo que se creía.
Particularmente pensando en que cada uno de los miembros difería de los otros en tesis centrales.

\textcite{Suarez2024}, por ejemplo, discute que una preocupación genuina sobre los modelos en la ciencia, precede, data y prosigue a los años del Círculo de Vienna, el autor nos dice \say{I focus particularly on the nineteenth-century modelers and summarize their insights and contributions, which, I claim, remain essentially unsurpassed.} (p. 20)

Además, los miembros del Círculo de Vienna tuvieron una preocupación genuina por las prácticas científicas--Neurath, por ejemplo--
Tenían presente que ciertos factores externos pueden influenciar a la investigación científica.
\textcite[p. 24]{Bentley2023} lo expresa mejor: \say{Frank's historical work, which frequently anticipates Kuhnian or post-Kuhnian themes, consistently emphasizes the significance of non-scientific, external factors on the decision making of scientists.}
Los miembros del Círculo de Vienna tenían claro que los factores externos influyen en el proceso de la investigación científica, presentes incluso en el \emph{contexto de justificación}.

Hablando en particular del trabajo de Neurath y las tesis que sostenía, Otto defendía una teoría coherentista de la justificación.
Neurath argumentaba que nuestras teorías científicas están subdeterminadas empíricamente.
En cualquier momento del tiempo hay hipótesis en pugna.
Pero si las teorías están empíricamente subdeterminadas, es posible que dos teorías internamente coherentes y externamente inconsistentes entre sí, convivan al mismo tiempo.

Pero si el escenario anterior es plausible, entonces es patente que aquél que defienda una teoría coherentista de la confirmación, no puede resolver el problema de la elección racional de teorías.
Debido a que todo el tiempo hay teorías en pugna, como filósofos, deberíamos poder explicar por qué es racional elegir entre una teoría y sus rivales. Para poder resolver este problema, Neurath defendía que los factores externos son información clave para explicar la racionalidad de la elección de los investigadores.
Factores externos como los eventos históricos fortuitos.

\textcite{Bentley2023} nos recuerda que Neurath fue falibilista sobre el conocimiento, es decir,  sostenía que cada una de nuestras creencias puede estar equivocada.
Sumado a esto, sostenía que la confirmación es holista: no son oraciones particulares, sino grupos de oraciones, los que contrastamos con el mundo.
Bentley lo expresa mejor \say{$\ldots$ it is always possible for a system of beliefs to be altered to accommodate a statement or for the statement to be rejected $\ldots$} (p. 21).



Esto último inició como un abstract que mandé a un taller y se fue volviendo más largo mientras leía.

Hablando de modelos en la filosofía de la ciencia durante el periodo del Círculo de Vienna, una figura importante que mencionar es Mary Hesse.
Mary Hesse trabajó con especial atención el uso de modelos en la ciencia.

``This historical chapter introduces the emergence in the nineteenth century of  what I call the modeling attitude.
This is a stance toward scientific work and  discovery, and it continues to this day." (p. 43)





