%!TEX root = ../main.tex




   
\chapter{La verdad importa}
\label{ch:truthmatter}

\section{Introducción}
 
Los seres humanos valoramos la sinceridad.
Confiamos en los expertos y tratamos de intercambiar información con nuestros pares.
También valoramos que la información sea correcta.
Damos valor a las personas que saben algo que nosotros no.
Por supuesto aún cuando tengamos confianza en la información que nos proporcionan otros, queremos que la información sea acertada y dudamos de la información que sinceramente nos ofrecen cuando es errónea.
Confiamos en lo que \emph{saben} y confiamos que no nos están \emph{mintiendo} cuando nos dan información.
Por usar términos de Williams \parencite{williams2002} al darle valor a la información dada por otros, esperamos que sea \emph{Precisa} y \emph{Sincera}.
Dadas estas condiciones, es sensato esperar que las personas haciendo investigación sean sinceras y precisas.

Sin embargo, a pesar de lo intuitivo de estos puntos, es bastante más complicado evaluar nuestras teorías científicas con base en la precisión y la sinceridad.
La historia de la ciencia nos ha mostrado que los debates teóricos no están cerrados de una vez y para siempre: descubrimos nuevos fenómenos y modificamos nuestras treorías con base en la evidencia.
Aún así, la sinceridad y la precisión es algo esperable.
¿Cómo podemos reconciliar ambos fenómenos?
Por un lado los cambios en nuestras teorías y por otro la sinceridad y la precisión.
Partamos de estos puntos que creo que son bastante intuitivos y nada controvertidos para explicar la parte central de este capítulo.

En este capítulo, me propongo a exponer la tesis \textit{veritista} como la presenta Duncan Pritchard \parencite{pritchardEpistemicValueCognitive2021}.
La tesis veritista afirma que la verdad es el valor epistémico fundamental.
Es decir que valoramos los estados epistémicos de los agentes debido a que son verdaderos.
Esto es una forma de monismo sobre el valor: el único valor de cualquier estado epistémico depende de que sea verdadero.
En particular, filósofos veritistas que afirman que el \emph{conocimiento} es el estado epistémico por antonomasia, señalan que valoramos el conocimiento porque es verdadero y no porque sirva a otros fines.

Hay al menos dos maneras en las que se ha señalado que el conocimiento no es valioso. 
La primera línea de argumentos, por ejemplo, discute si la \textit{creencia verdadera} es distinta del conocimiento.
Al final, ambos estados epistémicos tienen los mismo fines intrumentales.
Que yo tenga hambre y que yo sepa que hay comida en el refrigerador, no hace diferencia con que crea verdaderamente que hay comida en el refrigerador: al final, saciaré mi hambre.
En este caso, lo valioso es saciar mi hambre, que es un valor de naturaleza instrumental.

La segunda línea de argumentos, descansa en el hecho de que \textit{la verdad} puede restringirnos de otros fines cognitivos valiosos \parencite{elgin2017true}.
Elgin sugiere que, cuando analizamos investigaciones científicas, debemos relajar nuestros compromisos con la verdad.
En ciencia encontramos idealizaciones y modelos que difieren de una representación precisa de la realidad, por lo que deberíamos abandonar el compromiso con la verdad.

Para este trabajo, me ocuparé de la segunda línea de argumentos. 
Para señalar esto, quiero exponer los argumentos de Pritchard contra Elgin, que consisten en mostrar que la tesis veritista no está comprometida con evaluar el número de proposiciones verdaderas.
Veritistas como Pritchard nos dicen que las verdades tienen que tener una conexión profunda con la realidad.
Después de esto, quiero señalar que aún en los casos que menciona Elgin, la verdad es un valor que no podemos rechazar.
Para esto, presento los argumentos de Klein que permiten lidiar con este tipo de casos. 

\subsection{Tensiones en la evaluación de teorías}

De Paul \parencite{depaul2001} ofrece una divertida historia sobre el problema del valor en epistemología.
De Paul señala que hay dos niveles con respecto al problema del valor.
Uno de ellos es señalar que el conocimiento no es el único estado epistémico que es valioso.
Valoramos la sabiduría, por ejemplo, y la sabiduría no es algo que esté relacionado con la verdad.
Si bien, parece haber acuerdo en que el conocimiento no es el único estado epistémico valioso, los filósofos que afirman la tesis veritista defienden que el valor que damos a los diferentes estados epistémicos depende de que sean verdaderos. 
Pero esto resulta polémic para estados epistémicos como la comprensión y la sabiduría.

Por mi parte, estoy de acuerdo con el veritismo: creo que el valor de cualquier estado epistémico depende de que sean verdaderos.
Por usar una metáfora de Sher (\citeyear{sher2016}), la verdad impone fricción en nuestras creencias, así es como sabemos que hicimos las cosas bien, es decir, que nuestras creencias son precisas.
En este capítulo quiero exponer la tesis de Pritchard que ha desarrollado en (\citeyear{pritchard2021, pritchard2021a}) defendiendo el veritismo.
No obstante, mis creencias personales no justifican esta tesis, falta por señalar qué papel juega la verdad y de qué manera conceptualizarla para explicar por qué es valiosa.
Por ejemplo, sabemos que hay errores en la historia de la ciencia.
Teorías "erróneas" que, sin embargo, explican fenómenos, hacen las cosas bien; son precisas\footnote{Dado el papel central que juega la verdad, el segundo capítulo de este trabajo estará dedicado a las teorías de la verdad.}.
Debido a esto, parece haber un conflicto entre la tesis veritista y el valor que damos a las teorías científicas.
Algunos filósofos señalan que la verdad no juega un papel prominente y que deberíamos evaluar a las teorías de acuerdo a otros valores.
Por ejemplo, evaluamos a las teorías científicas si salvan los fenómenos, representan bien su objeto de estudio e incluso la estética \parencite{ivanova2020}.
Mi pretensión es señalar que la verdad también es uno de los valores a tener en cuenta en esta evaluación y que el conflicto con el veritismo es sólo aparente.
Si tomamos en cuenta la justificación y las virtudes de los agentes que hacen investigación, el conflicto desaparece.

\subsection{El plan}

Para este capítulo, el plan es el siguiente: comienzo exponiendo algunas motivaciones, luego el problema del Menón y la tesis \textit{veritista} (recordemos que la tesis señala que el conocimiento es valioso porque es verdadero).
Después presento dos argumentos que se han usado en la literatura para tratar de mostrar que el veritismo es falso; a saber, el argumento de las \emph{verdades irrelevantes} [VI] y el \emph{problema del drenado}\footnote{En inglés "swamping problem". Decidí traducirlo como "problema del drenado", por la siguiente razón: los valores instrumentales consumen cualquier valor que asociemos a la verdad en el conocimiento. Esto en analogía con el artículo de Ned Block "¿se drenan hasta desaparecer los poderes causales?" de (\citeyear{block2013})} [PD].
Aunque pretendo defender la tesis veritista exponiendo la respuesta de Pritchard a VI y AD, los casos históricos parecen debilitar la plausibilidad de la tesis.\footnote{Por supuesto, el argumento más general para esto es la llamada meta-inducción pesimista \parencite{laudan1981}.}
Por lo que hay una tensión entre lo que nos ha enseñado la historia de la ciencia y el veritismo.
Menciono estos ejemplos porque mi interés general es la evaluación de teorías.\footnote{En términos muy generales, los realistas científicos señalan que hay que entender literalmente las oraciones afirmadas por las teorías científicas. Esta tesis se divide en varios compromisos: epistémicos, semánticos y metafísicos. Por ahora sólo voy a formular la tesis epistémica, que es lo que me concierne en este capítulo: un realista científico está comprometido con que nuestras mejores teorías científicas son descripciones exactas del mundo.Pero sabemos, gracias a muchos casos históricos, que las mejores teorías del pasado han sido revisadas, corregidas y —en algunos casos— descartadas. Es clara la tensión que hay entre el realismo científico, el compromiso del realista con la verdad y los múltiples casos históricos.} 

Para conciliar este conflicto quiero presentar un marco, desarrollado por Klein (\citeyear{klein2019})que me permite señalar que los casos históricos expuestos no presentan problema a la tesis veritista.
Esto porque lo que señalamos en los casos históricos depende no sólo de que las teorías sean verdaderas a secas, sino también de cómo justificamos hipótesis y qué papel juegan las virtudes epistémicas en la investigación científica.

\subsection{El problema del valor}

Quiero señalar que a los seres humanos nos interesa la verdad.
Esta importancia que le damos a la verdad guía también nuestras empresas epistémicas\footnote{Utilizo el concepto de ‘empresas epistémicas’ como algo muy general y no bien definido. El concepto puede ser usado para describir tanto a la investigación científica, como a la curiosidad de un niño que está aprendiendo a sumar.}.
Platón en su diálogo con Menón \parencite[][¶¶ 97a-98b]{platon2008}, señala que hay un factor de seguridad ligado al conocimiento, que no se encuentra en la mera opinión verdadera.
Sócrates pide a Menón señalar cuál es la diferencia entre creencia verdadera y conocimiento.

\begin{quote}
    Sóc. — Por lo tanto, la opinión verdadera, en relación con la rectitud del obrar, no será peor guía que el discernimiento; y es esto, precisamente lo que antes omitíamos al investigar acerca de cómo era la virtud, cuando afirmábamos que solamente el discernimiento guiaba correctamente al obrar.
\end{quote}
   
En efecto, también puede hacerlo una opinión que es verdadera.
Lo que señala Sócrates a Menón es que si sólo tomamos en cuenta el valor instrumental del conocimiento, no hay ninguna diferencia entre saber y creer verdaderamente.
En términos prácticos, no hay diferencia entre mi creencia verdadera de llegar a la Ciudad de México desde Aguascalientes, que sea diferente a mi conocimiento del trayecto.

Sin embargo, nos parece que el conocimiento tiene más valor que la mera creencia verdadera.
Platón hace una analogía con las estatuas de dédalo: así como el conocimiento y la creencia verdadera, el conocimiento es como las estatuas de dédalo que están sujetas.
La creencia verdadera, como las estatuas sin base, comienzan a moverse.\footnote{No es claro exactamente cuál es la sugerencia platónica. Williamson (\citeyear{williamson2002}) sugiere que el conocimiento es más estable que la creencia verdadera frente a nueva evidencia. Por utilizar un tecnicismo, los derrotadores para la creencia forman un conjunto de cardinalidad mayor que los derrotadores para el conocimiento.}
Distinguir entre por qué es más valioso el conocimiento que la creencia verdadera es lo que detona el problema del valor en epistemología.

Para fines narrativos, llamaré \emph{veritistas} a aquellos que defienden que el valor del conocimiento radica en que sea verdadero.
Hay al menos dos maneras de negar la tesis veritista.
Una de ellas es señalar que el conocimiento no es el único estado epistémico valioso.
Valoramos otros estados epistémicos como la comprensión, la sabiduría, la racionalidad, etc.
Estos estados son epistémicamente valiosos sin que su valor dependa de que sean verdaderos \parencite{kvanvig2003}, aun si valoramos el conocimiento porque es verdadero.

Otra manera de negar la tesis es señalar que la verdad no es el valor fundamental de ningún estado epistémico, por tanto, tampoco la característica por la cual el conocimiento tiene valor.
Catherine Elgin (\citeyear{elgin2004})ha desarrollado una epistemología que toma en serio las falsedades que encontramos comúnmente en las ciencias, afirmando que cuando realizamos investigación, la verdad es irrelevante.

Kvanvig, por ejemplo, señala que la comprensión es un estado epistémico que no tiene los problemas del conocimiento proposicional.

Si bien Kvanvig asume que el problema de la naturaleza del conocimiento involucra necesariamente a la \emph{verdad} como un componente, esto no implica que la comprensión involucre dicho componente.

Supongamos, por ejemplo, que tomo un libro que contiene una cantidad $n$ de proposiciones verdaderas, de manera tal que es una lista de proposiciones sin conexión alguna entre ellas: la suma de "$2+2 = 2²$", "Yo estoy aquí ahora", "Borges escribió 'El Aleph'", etc.
Luego me doy a la tarea de memorizar todas esas proposiciones.
Si bien tengo un conjunto de proposiciones verdaderas, no es el caso que comprendo lo que dice el libro.

Kvanvig señala que casos como el anterior fallan en ser casos de comprensión porque carezco de las relaciones estructurales relevantes entre las diferentes proposiciones verdaderas.

\begin{quote}
     For when understanding comes to mind, the central elements in focus are ones concerned with structural relationships between various pieces of information grasped by the possessor of understanding, unlike the central element of non-accidentality in focus when one is reflecting on the concept of knowledge. \parencite{kvanvig2009}
\end{quote}
   
Grimm en (\citeyear{grimm2012})hace un recuento de las tesis a favor de la comprensión, señalando que la comprensión puede suplantar al conocimiento proposicional.
Las motivaciones detrás de esto es que la comprensión puede evitar casos Gettier, que la comprensión es un estado epistémico mas transparente y que es claramente un logro cognitivo.

Estos comentarios parecen señalar que la comprensión es un estado claramente distinto al conocimiento proposicional.
Un estado que puede solucionar muchos de los problemas de la epistemología clásica y que ofrece una mejor imagen de los logros cognitivos de los agentes.

Que un nuevo concepto resuelva tantos problemas al mismo tiempo suena sospechoso. 
No es obvio que una teoría del conocimiento más robusta no pueda lidiar con algunos de estos problemas.
Pienso en particular en las relaciones estructurales de las que habla Kvanvig ¿El conocimiento proposicional excluye dichas relaciones?

Esto no es obvio. El debate entre coherentistas y fundacionalistas en teorías de la justificación nos ha enseñado que hay relaciones entre nuestras diferentes creencias.
Ambas teorías buscan explicar cuándo estamos justificados en afirmar que una proposición es verdadera.
Ambas teorías están de acuerdo con que justificamos nuestras creencias con base en otras creencias.
En lo que difieren es en la naturaleza de la justificación.
Los fundacionalistas nos dicen que cada creencia está justificada y que la justificación de que $p$ es verdadera depende de la justificación de que $q$ es verdadera.
La justificación está dada por cada una de las creencias que sostenemos .

Por su lado, los coherentistas señalan que la justificación no está dada por cada una de las proposiciones en nuestro sistema de creencias, sino que el sistema completo de creencias está justificado cuando el sistema es coherente. %<!--- Faltan las citas del libro verde --->

En epistemología clásica el conocimiento está separado en componentes: justificación, creencia y verdad.
Estas teorías de la justificación claramente abogan por relaciones estructurales en el conocimiento.
No es obvio que la comprensión esté dividida. Sólo parece agrupar estos componentes en uno y le pusieron un nuevo nombre.

Además, esto no resuelve los problemas que regularmente asociamos al concepto de "conocimiento".
Supongamos que la \emph{comprensión} efectivamente es distinta al \emph{conocimiento}, entonces el conocimiento sigue teniendo los mismos problemas.
Sólo hemos pasado el problema a otro lado.
Nuestra noción de conocimiento todavía tiene que ser analizada.

Entonces no es claro que el conocimiento no pueda involucrar dichas relaciones estructurales.
Más apremiante que los problemas anteriores, es la motivación para evitar casos Gettier que tiene la comprensión.
Me parece que podemos generar casos Gettier para el entendimiento.
Por lo que la comprensión no excluye casos accidentales.

Una actriz mexicana alguna vez dijo que sentía mucho pésame por las víctimas del "surimi" de Singapur.
Entendemos que la palabra que quiso usar era "Tsunami".
La actriz puede incluso comprender \emph{lo que quiso decir}, pero accidentalmente usó la palabra incorrecta.\footnote{Me encantaría decir que este ejemplo es original, pero está inspirado en la discusión de Davidson en "A nice derangement of epitaphs" (\citeyear{davidson1986}). Por supuesto, la discusión en este artículo es sobre fallos en composicionalidad lingüística. Pero me parece que podemos extraer una moraleja para el tema de la comprensión epistémica.}
Pero la actriz puede tardar en comprender \emph{lo que dijo}, incluso jamás comprenderlo, nunca volver a prestar atención a la afirmación que hizo.

El ejemplo anterior pretende señalar lo siguiente: en un sentido importante, la actriz comprende \emph{lo que quiso decir}.
Pero podemos estar de acuerdo en que la actriz no comprende \emph{lo que dijo}.
Si evaluamos la expresión, la afirmación es claramente incorrecta.
Pero somos capaces de comprender qué fue lo que quiso decir con su expresión.
Creo que este ejemplo sirve para dudar que la comprensión sea invulnerable a casos Gettier.


\subsection{La receta Gettier}

En esta sección quiero repasar brevemente el problema del análisis del conocimiento.
Este problema no es uno que quiera discutir a profundidad, sin embargo, esclarece los componentes que solemos esperar que tenga un agente que sabe que $p$
Además, sirve para contextualizar el ejemplo de la sección anterior.
Quisiera recordarle al lector el problema Gettier y la "receta" de Zagzebski para generar casos Gettier.

A partir de lo discutido con Menón y Teeteto, Platón sugirió que el conocimiento tiene tres componentes: el conocimiento es una “creencia, verdadera y justificada”.
De manera ingeniosa, esta caracterización del conocimiento fue puesta en duda por Gettier (\citeyear{gettier1963}).
El argumento de Gettier depende de presentar dos contraejemplos, lo que hace Gettier con estos contraejemplos es presentar dos casos que cumplen las tres condiciones establecidas y en los cuáles no diríamos que los agentes en cuestión "saben".
Incluso tenemos una receta para generar contraejemplos tipo Gettier \parencite{zagzebski1994}, sólo hay que agregar un evento fortuito que haga nuestras creencias verdaderas.

Por ejemplo, supongamos que voy viajando en el autobús.
Voy viendo la ventana, comienzo a sentirme somnoliento, volteo hacia el frente de la unidad y creo ver a una amiga (digamos Atziri, para que sea más sencilla la narración).
Sin embargo, la persona que vi no es Atziri, sino otra persona muy parecida a ella.
Me quedo dormido durante 10 minutos.
Durante el intervalo en el que me quedé dormido, la persona que vi se baja del autobús y se sube Atziri, quien casualmente lleva un atuendo idéntico a la persona que se bajó.
Despierto de mi sueño y al bajar de la unidad paso al lado de Atziri, la saludo y la invito a tomar un café (siempre es un gusto hablar con ella).

Este ejemplo es uno generado por la receta de Zagzebski.
La conclusión de Gettier es que la caracterización de Platón no ofrece condiciones suficientes para decir que S sabe que $p$.

Hay que recordar que el análisis de Gettier sólo aplica a uno de los lados del condicional material.
Gettier comienza señalando que el análisis de Platón pretende ofrecer condiciones necesarias y suficientes que señalan qué es el conocimiento\footnote{Caracterizar al conocimiento de manera tal que recuperemos todos y sólo los casos en los cuales podamos aplicar el término se conoce como el problema del análisis del conocimiento \parencite{ichikawa2018}.}.
Es decir, que captura todos y sólo los casos que constituyen la aplicación del término "saber".

Gettier utiliza para sus contraejemplos dos casos de inferencia defectuosa.
Pero incluso podemos generar casos para el conocimiento no inferencial, como mi falta de atención en el autobús.

La literatura posterior a Gettier trató de agregar condiciones que excluyeran  estos casos \parencite{zagzebski1994}.
Y si bien las condiciones no son conjuntamente suficientes, esto no señala un problema con que dichas condiciones sean necesarias.

En particular quiero señalar que, en principio, aceptamos que la verdad es una condición necesaria para decir que sabemos algo, digamos que una proposición es verdadera.
Si sabes que hay un gato sobre la alfombra, entonces hay un gato sobre la alfombra.
El conocimiento tiene una estrecha conexión con la verdad y la justificación es el adhesivo.

Si bien, la verdad es una condición necesaria del conocimiento (el conocimiento es fáctico), esto no implica que nuestro conocimiento sea valioso porque es verdadero.
Como prometí al inició del capítulo, quiero presentar los argumentos de Pritchard contra los problemas que han discutido algunos filósofos concluyendo que la verdad no es el componente valioso del conocimiento.
En particularm, hay dos argumentos para rechazar a la verdad como valor: el argumento de las \emph{verdades irrelevantes} [VI] y el \emph{problema del drenado} [PD].

Hasta este momento del capítulo, he señalado mis pretensiones, he dicho que encuentro un problema con la tesis veritista y casos de historia de la ciencia;he señalado, también, que pretendo ofrecer razones para decir que la tensión es aparente.
Además, he ofrecido razones en contra de remplazar al conocimiento por la comprensión.

En el siguiente apartado hago un breve repaso de por qué tenemos la intuición de que el conocimiento es valioso porque es verdadero.
Luego presentaré los argumentos de Pritchard contra VI y PD.
Al final diré por qué los argumentos de Pritchard me parecen relevantes para la evaluación de teorías.

\subsection{Problemas con el valor de la verdad}

Que la verdad es un componente necesario del conocimiento ha hecho que muchos epistemólogos centren sus esfuerzos para explicar por qué el conocimiento es valioso porque es verdadero.

Siguiendo esta línea de razonamiento, filósofos como Pritchard \parencite{pritchard2021a} han defendido que la verdad es el valor epistémico fundamental del conocimiento, \textit{i. e.}, un valor final, no instrumental.
Pritchard, por ejemplo, señala que:

\begin{quote}
   It wasn’t all that long ago that the idea that truth is the fundamental epistemic good was orthodoxy in epistemology. Indeed, this was the kind of claim that was so commonplace that it was almost not worth stating, as to do so would be somewhat superfulous. \parencite{pritchard2021a}
\end{quote}

Esta sugerencia es altamente intuitiva.
Pero como también podemos ver en la cita anterior, esto ha ido cambiando a lo largo de la historia de la filosofía.

La verdad como componente del conocimiento no hace una diferencia con las creencias verdaderas.
En ambos casos la verdad estñá involucrada.
Una sugerencia es desechar el concepto de conocimiento y quedarnos con la pura creencia verdadera \parencite{papineau2021}.
Sin embargo, dicha sugerencia es apresurada.

Con esta sugerencia, me parece, estamos olvidando un componente importante de cómo obtenemos verdades, en particular en la justificación de creencias.
La justificación de nuestras creencias no es algo que podamos obtener fácilmente.
Requerimos trabajo para afirmar que las creencias que tenemos son verdaderas.
Recabamos evidencia, escuchamos atentamente, comparamos información y buscamos nueva información para fijar o rechazar nuestras creencias.
Todos estos fenómenos ese encuentran entre los esfurerzos que usamos para justificar lo que creemos.

La justificación juega un papel importante en el conocimiento.
Sugiero que la \emph{justificación} es un componente en el que recae el esfuerzo de nosotros los agentes, y que el pago que obtenemos por dicho esfuerzo es a tener verdades.
Justificar creencias no es tarea fácil y trataré de motivar esta sugerencia en la siguiente sección.
Quiero mencionar que esta tesis marca una diferencia entre la \emph{creencia verdadera} y el \emph{conocimiento}, al mismo tiempo que es compatible con el \emph{veritismo}: porque el valor del conocimiento depende de que sea verdadero.
Esto entra en clara contradicción con la propuesta de Elgin expuesta al principio de este capítulo.

Dada la sugerencia anterior y los componentes que tomo de William: \emph{sinceridad} y \emph{certeza}, puedo señalar por qué maximizar la verdad es uno de los puntos centrales del veritismo.

Pero esta sugerencia rápidamente nos pone a merced de VI.
Si el veritismo depende de maximizar el número de proposiciones verdaderas, entonces es muy sencillo obtener verdades.
Tomemos un número natural cualquiera $n$ y sumemos 1.
Sumemos 1 consecutivamente a cada $n+1$ y tenemos un número potencialmente infinito de creencias verdaderas, que además están justificadas.
Pero claramente hay una diferencia entre este proceso trivial y, digamos, los axiomas de Peano.
Claramente es más valioso el trabajo realizado por Peano en su investigación para axiomatizar la aritmética.
Por ello, maximizar el número de proposiciones verdaderas es una empresa fútil.
Llamemos a esto \emph{la tesis de la maximización}

Sin embargo, un veritista no tiene por qué aceptar la tesis de la maximización. Pritchard nos dice que un veritista no tiene por qué maximizar el \emph{número} de creencias verdaderas.
Esto no es un componente necesario de la tesis veritista.
Pritchard dice que el veritista está interesado en verdades que "tengan un contacto cognitivo con la realidad" \parencite{pritchardEpistemicValueCognitive2021}.
Nos importa la verdad, pero no cualquier verdad. 

En esta sección me dedicaré a presentar los argumentos de Pritchard contra PD y VI. Antes de continuar con dicha exposición quiero motivar por qué es deseable centrarse en la verdad como valor fundamental del conocimiento.

Primero que nada, de alguna manera tenemos que ser capaces de corregir creencias y quizás la forma más intuitiva de hacer esto es que el mundo impone ciertos constreñimientos sobre el conocimiento humano.
Gila Sher usa el término 'fricción epistémica' para describir esta relación.
La preocupación de Sher reside en que el conocimiento debe tener un fundamento.
En particular, Sher sugiere que este fundamento lo encontramos en el mundo "Groundedness in the world is veridicality, i.e., compliance with substantial standards of truth, evidence, and justification." \parencite[][p. 9]{sher2016}.

Una sugerencia similar es la que hace Blåsjö \parencite{blasjo2022}.
El autor señala que los geómetras griegos estaban comprometidos con un programa operacionalista: las construcciones geométricas son construcciones físicas concretas, que buscan representan cómo es el mundo.
Incluso Descartesa en su libro de geometría genera instrumentos físicos para trazar curvas algebráicas \parencite{descartes2018}. 
Resumiendo mucho de su tesis, Blåsjö dice:

\begin{quote}
    Yet operationalism celebrates concrete constructions and embraces their physicality and real-worldness. This is a point that invites confusion, and indeed I shall argue that previous literature has fallen into misinterpretations for this reason. From a modern point of view, it is natural to take for granted that the foundations of mathematics is a matter of pure theory, while constructions with physical tools can only be of practical relevance. This is completely wrong, according to the operationalist perspective. To understand the philosophy of Greek geometry, we must abandon the dogma that to make mathematics rigorous it “should” be separated from any links to physical reality and turned into purely formal and abstract theory. Operationalism, in contrast to this modern dogma,anchors mathematical rigour in the physical realm. Technical mathematical sources detailing constructions with various curve-tracing devices have often been misinterpreted as quasi-practical, whereas the operationalist perspective suggests that they should instead be read as epistemologically motivated foundational investigations. (p. 590)
\end{quote}

Según el autor, esto además servía para evitar contradicciones en los métodos geométricos de la época.
Aún cuando fallamos en representar de forma precisa cómo es el mundo, buscamos que nuestras creencias estén lo suficientemente justificadas. 
Por el momento asumamos que la verdad juega un papel importante en nuestras 'empresas epistémicas' tanto mundanas como teóricas.
Continúo la siguiente sección exponiendo los argumentos de Pritchard.


\subsection{Pritchard contra VI y PD}

Como mencioné antes, a pesar del peso intuitivo que tiene la tesis del valor epistémico de la verdad, varios filósofos han presentado problemas en contra de esta tesis.
Entre los problemas más apremiantes están  PD y VI.

Pritchard explica ambos problemas de la siguiente manera.
PD nos indica que aún cuando el conocimiento es verdadero, lo valioso de saber algo depende de otros factores distintos a la verdad, por ejemplo, que es útil.

Esta sugerencia no distingue entre la mera creencia verdadera y el conocimiento.
Además, esto motiva la sugerencia de Elgin, si el valor del conocimiento depende de otros factores, no hace una diferencia tener creencias falsas, siempre y cuando sean epistémicamente útiles para otros propósitos.

En este sentido, la verdad no es un valor fundamental, sino sólo una manera de obtener otros bienes valiosos: modificar el mundo de alguna manera, obtener beneficios de algún tipo, etc.
Pero si esto es verdad, entonces podemos llevar a cabo todas nuestras actividades con sólo creencias verdaderas, incluso creencias falsas. Si esto es el caso, entonces que valoremos la verdad es parasitario a la utilidad que esta nos brinda.

Por otro lado, VI establece que si la verdad es el valor fundamental, entonces ante dos verdades cualesquiera, no podríamos elegir cuál deberíamos creer.
Si maximizar el número de creencias verdaderas es el objetivo del conocimiento, obtener dicho objetivo es bastante sencillo: podemos simplemente memorizar el contenido de un libro de "fun facts"

Más aún, hay verdades que no son interesantes en absoluto.
Por ejemplo, hay una respuesta verdadera sobre el número total de granos de arena en Cancún y una respuesta verdadera sobre si la luna gira alrededor de la tierra.
Dado que queremos distinguir entre las respuestas que son importantes de las que no, entonces la verdad no puede ser lo único que da valor al conocimiento.

Para dar una respuesta a estos problemas, Pritchard \parencite{pritchard2021, pritchard2021a} nos dice que estos problemas surgen al asumir que el objeto de evaluación epistémica es el \emph{número de proposiciones} verdaderas.
Pritchard señala que si además involucramos la teoría de la virtud epistémica en el veritismo, entonces somos capaces de resolver los problemas antes mencionados, Pritchard apunta que:

\begin{quote}
  A true statement of fundamental science may be expressed as a single proposition, but it ofers us a great deal by way of cognitive contact with reality. In contrast a long list of trivial empirical claims might offer us hardly any cognitive contact with reality at all. In the sense that matters to us, there is more truth in the former than in the latter, even if the latter involves more true propositions. \parencite[][pp. 1353-1354]{pritchard2021}
\end{quote}

Y el paso a las virtudes intelectuales, está en el siguiente párrafo del mismo artículo "In particular, we should understand how to achieve the epistemic good of truth via appeal to what an intellectually virtuous inquiry would involve." (p. 1354).

Antes de continuar, quisiera exponer de qué trata la teoría de las virtudes epistémicas, que es una forma de fiabilismo sobre la justificación \parencite{klein2019}.

\paragraph{Desfase: teoría de las virtudes epistémicas}

A grandes rasgos, los teóricos de las virtudes epistémicas, se dividen en dos grandes grupos: \emph{fiabilistas} y \emph{responsabilistas}.
Ambas facciones describen a las virtudes a la manera en como Aristóteles\footnote{"Con relación a las mismas cosas son, pues, el cobarde, el temerario y el valiente, pero se conducen diferentemente a su respecto. Aquéllos pecan por exceso y por defecto, en tanto que éste guarda el medio y el deber." \parencite{aristoteles2012}, especialmente los capítulos 2-8 del libro III.}
entendía la virtud: acciones deliberadas llegar a un fin.
Por supuesto, esto no incluye cualquier acción posible.
Supongamos, por ejemplo que mi meta es amasar una fortuna.
La forma más complicada para hacerlo es trabajando (e idealmente ganando un sueldo justo) para generar ingresos, diseñando estrategias de inversión y ahorrando; otra manera de hacerlo es explotando trabajadores y robando tanto como pueda; otra manera de lograr este fin es simplemente heredar una gran fortuna.

De las tres estrategias mencionadas anteriormente, en términos evaluativos, la primera estrategia es más virtuosa que la segunda.
Mientras que la última no constituye un ejercicio de mi parte para ganar ingresos.
Es por ello que el ejercicio de estas virtudes, decisiones voluntarias que practicamos para llegar al fin que queremos lograr.\footnote{Es controvertido señalar exactamente cuál es el \emph{telos} de nuestras empresas epistémicas. En particular el fin de la investigación o de la indagación. Varios candidatos entran en esta canasta: la verdad, la comprensión, etc. Por ahora dejaré de lado este problema, pero el trabajo de Friedman puede ofrecer una respuesta para esto.}

Uno de los pioneros de la teoría de la virtud epistémica es Ernest Sosa.
Sosa \parencite{sosa2017} articula una teoría del conocimiento que caracteriza a las virtudes epistémicas como el ejercicio de ciertas habilidades de los agentes, tales que al ejercitarlas, son constitutivas del conocimiento.
Podemos describir este proceso diciendo que dichas habilidades virtuosas aseguran que tengamos más creencias verdaderas que falsas, de ahí la etiqueta \emph{fiabilista}.

Sin embargo la versión de Sosa, no es la única caracterización de las virtudes epistémicas. 
Zagzebski \parencite{zagzebski1996} distingue su teoría de la de Sosa, señalando que la teoría de Sosa está más relacionada con el consecuencialismo que con la teoría de virtudes aritotélicas.
Zagzebski señala que si el objetivo de la teoría de Sosa es obtener más creencias verdaderas que falsas, entonces la teoría ética relevante es la consecuencialista: mientras más creencias verdaderas tengamos, mucho mejor.
Pero una teoría de las virtudes más comprometida con la teoría de las virtudes aristotélicas, como la que propone Zagzebski, puede explicar cómo le damos valor a los productos generados por los agentes epistémicos, aún cuando no produzcan creencias verdaderas.

La literatura ha llamado \emph{responsabilista} a la teoría de Zagzebski, porque no se centra sólo en los productos epistémicos \textit{per se} (habitualmente creencias verdaderas), sino que toma en cuenta otras características que no necesariamente están relacionadas con obtener la verdad: el deseo de obtener creencias verdaderas no constituye conocimiento, aún cuando es una motivación importante para obtener conocimiento.

Hay un debate sobre si ambas aproximaciones son realmente excluyentes.
A primera vista parece que es sólo una cuestión acerca de a qué decidimos llamar 'virtud epistémica'.
Esto incluso se vuelve más complicado debido a que tanto Sosa como Zagzebski, nos dan un conjunto de virtudes y extensionalmente no es claro que sean excluyentes.
No es claro porque cuando tomamos en cuenta qué habilidades son necesarias para obtener conocimiento, ambas teorías (representadas extensionalmente) se traslapan.
Supongamos, por ejemplo, que un investigador virtuoso está recolectando evidencia para cierta hipótesis $h$.
Debido a que es un investigador virtuoso, es capaz de observar con atención la evidencia y seguir buenos patrones de inferencia (virtudes fiabilistas).

Sin embargo, sin darse cuenta, hay un error en los datos recabados, digamos que definió una función en R que devolvía valores muy bajos que apoyan la hipótesis. Uno de sus asistentes de investigación se da cuenta de el error y le señala esto al investigador, el investigador amablemente revisa el código y lo corrige (virtudes responsabilistas).

El ejemplo anterior pretende ilustrar un caso en el que ambos tipos de virtudes contribuyen a evitar errores y obtener una creencia verdadera.
Por lo que no es claro que haya una distinción de los diferentes tipos si para distinguirlos echamos mano de que las virtudes del fiabilista constituyen conocimiento, mientras que las virtudes del responsabilista son auxiliares \parencite[][p. 144]{sosa2017}
Greco \parencite{greco2002} señala que, si bien es verdad que esto sólo parece un debate terminológico, hay casos interesantes en los que ambas teorías difieren.
En particular, los marcos de virtudes responsabilistas han servido para debatir problemas que se alejan de la epistemología clásica, por ejemplo, las injusticias epistémicas.
Pero en principio no son teorías excluyentes.


\paragraph{Solución de Pritchard}

Con este marco explicado, Pritchard resuelve los dos problemas que se le achacan al veritismo.
Recordemos que hay al menos dos problemas que Pritchard discute para señalar defender al veritismo.
Señalé ambos problemas al inicio: el problema de las verdades irrelevantes [VI] y el problema del drenado [PD].

El problema de las verdades irrelevantes nos dice que si lo único que nos importa en nuestras empresas epistémicas es la verdad, entonces antes dos proposiciones verdaderas: una de peso y una irrelevante; deberíamos creer ambas.

Por ejemplo, hay una respuesta correcta sobre el número de hojas que tiene un árbol y una respuesta correcta a si $a² + b² = c²$ cuando a y b son los catetos de un triángulo rectángulo.
Intuitivamente es más valiosa la segunda proposición que la primera.
Pero si el veritismo es correcto, entonces deberíamos creer ambas proposiciones.
Pero esto es claramente absurdo, por tanto, el veritismo es falso.

El problema del drenado [PD], argumentan, muestra que el veritismo no puede ser verdadero.
Los veritistas nos dicen que la verdad es el valor final para diferentes estados epistémicos.
Pero si esto es verdad, entonces el conocimiento no tiene un valor diferente a la creencia verdadera.
Entonces, según el veritismo, no hay una diferencia de valor entre tener conocimiento de $p$ y una creencia verdadera de que $p$.
Si ambos estados son igualmente valiosos por ser verdaderos, entonces el valor reside en otros factores distintos a la verdad.
La utilidad ha drenado todo valor que pudiéramos darle al conocimiento.
Por tanto el veritismo es falso.

Es interesante la analogía que Pritchard nos presenta para ilustrar lo que tiene en mente.
La analogía es la siguiente: que un chef haga comida deliciosa y luego la pruebe para saber si es deliciosa, no significa que su fin era probarla y no hacerla deliciosa.
El probarla sólo es una manera de asegurarse que es deliciosa.
La analogía indica que si obtener verdades es el fin de nuestras empresas epistémicas, eso no implica que sólo con la mera creencia verdadera, hemos llegado a nuestro objetivo, ni que los factores que no están vinculados a la verdad sean en lo que reside el valor del conocimiento.
Cualquier consecuencia práctica sirve sólo para asegurarse de que nuestro conocimiento es verdadero, es decir \emph{certero}. 

Ahora, una parte crucial de la solución a estos problemas, consiste en que el veritismo no implica que hay que maximizar el número de proposiciones verdaderas.
Lo que buscamos es que nuestro conocimiento tenga contacto sustantivo con la realidad.
Además de involucrar el marco que nos ofrece la teoría de las virtudes epistémicas para señalar que la justificación es un esfuerzo de los agentes.

Si es verdad que el valor fundamental de cualquier empresa epistémica es la verdad, eso explicaría por qué nos interesa obtener verdades: si bien esto tiene implicaciones prácticas, cualquier consecuencia depende de que de hecho nuestro conocimiento apunte a la verdad y la consiga.

La mera creencia verdadera no está motivada por las virtudes del agente.
Como bien señala Elgin, tener creencias verdaderas es muy barato.
Pero es falso que el conocimiento deba tener el mismo valor, porque es más complicado obtener conocimiento.
El conocimiento no sólo depende de la verdad, sino de que haya sido producida por un agente virtuoso.
Un agente que se esfuerza en aprender y ejercer sus virtudes.
Si esto es verdad, entonces el veritismo sí puede distinguir entre la mera creencia verdadera y el conocimiento sin depender de factores como la utilidad.
Resolviendo el problema con PD.

Resolver VI sigue una estrategia parecida. 
Una vez que nutrimos el veritismo con la teoría de las virtudes epistémicas, VI deja de ser problemático.
Como agentes virtuosos, queremos no sólo que nuestro conocimiento sea certero, sino además que el proceso para llegar a la verdad sea fiable.
En particular que esté guiado por características virtuosas de un agente.
Un agente virtuoso puede sopesar entre dos verdades: una irrelevante, la otra de más peso.
Recordemos también que Pritchard caracteriza al veritismo, de tal manera que no importa el \emph{número} de proposiciones verdades, sino que nuestras creencias verdaderas tengan contacto sustantivo con la realidad.

Una vez que Pritchard introduce a la teoría de las virtudes, VI no es un problema. 

\begin{quote}
    So once we unpack EVTM properly, in line with the intellectual virtues, then it follows that it isn’t committed to a view according to which any true belief is thereby of final epistemic value; rather, it is only those true beliefs that offer one genuine cognitive contact with reality $\ldots$ \parencite[][p. 11]{pritchard2021}. 
\end{quote}

Si podemos resolver estos problemas para el veritismo, entonces no es claro que debamos abandonar a la verdad como el valor fundamental del conocimiento.

Sin embargo, cuando empatamos el veritismo con la práctica científica, parece que entramos en problemas.
Uno de los casos más claros de logro epistémico son nuestras teorías científicas.
Regularmente evaluamos teorías científicas si nos permiten explicar fenómenos. 
Las explicaciones de los fenómenos tienen que ser correctas para contar como una buena explicación.
Sin embargo, la historia de la ciencia 

Creo con convicción que el objetivo de la indagación, en contextos de investigación, es obtener conocimiento. Esto implica obtener verdades.

\subsection{Veristismo en problemas: dos casos históricos}

Hasta este punto desarrollé los argumentos de Pritchard. 
Las razones expuestas están a favor de que la verdad es lo que hace valioso al conocimiento.
La verdad es un factor que nos asegura que vamos por el camino correcto, cuando nuestras creencias están bien justificadas.
Las explicaciones y demás consecuencias del conocimiento también dependen de la verdad. 
Más aún, las virtudes intelectuales nos ofrecen una manera de evitar el conocimiento espurio: queremos no sólo que nuestras creencias sean verdaderas, sino que además el proceso para llegar a la verdad sea fiable y guiado por características de un agente. 
Un agente virtuoso puede sopesar entre verdades relevantes e irrelevantes.

Estoy de acuerdo con lo que señala Pritchard, que la verdad es de dónde el conocimiento obtiene su valor (dejemos por el momento si la verdad es una motivación para la indagación). 
Este marco, incluso encaja bien con cómo podemos evaluar la investigación científica a la luz de la teoría presentada: evaluamos sus logros cognitivos las virtudes involucradas en el proceso y que sus resultados sean correctos. 

Sin embargo, no es claro que la tesis veritista sea adecuada para evaluar teorías científicas. 
La investigación científica es probablemente la manera más sistematizada que tenemos los seres humanos para producir conocimiento.
Muchas de nuestras explicaciones dependen de saber algunos hechos acerca de las diferentes disciplinas científicas. 
Sabemos, por ejemplo, que para que haya una combustión se necesita combustible y oxígeno.
Si cualquiera de estos factores está ausente, entonces no hay combustión.
Esto es una explicación del fenomeno de la combustión, sabemos que es una buena explicación que involucra muchos casos particulares de el mismo proceso: la combustión interna, que no seamos capaces de encender una fogata bajo el agua; que cuando la combustión agota el oxígeno a su alrededor, se apaga, etc.

Por supuesto, una explicación que utilice información falsa no es una buena explicación\footnote{Por el momento esto es una suposición, hay literatura que afirma la tesis contraria. Diré un poco más sobre esto en la sección final del capítulo y en el capítulo siguiente.}. 
Consideremos, por ejemplo, que tiro mi taza de café al piso.
Y que explico esto con base en señalar (falsamente) que justo antes de que mi brazo golpeara la taza, el viento la empujo y fue esta la razón por la que cayó al piso.
Esto es una mala explicación.
A lo menos es una mentira para evadir culpabilidad, pero es incorrecta en muchos sentidos.
Alguien que haya escuchado lo que dije puede preguntar por el peso de la taza y la velocidad del viento.
Fenómenos que sin las condiciones adecuadas, no pueden tirar una taza.
Luego concluir que mi explicación fue mala.
De manera sucinta, las teorías y explicaciones parecen tener que representar adecuadamente el mundo.
Todo esto es consistente con la tesis veritista.

Supongamos por un momento que las explicaciones verdaderas son las únicas explicaciones que nos interesan. 
La suposición anterior es claramente falsa. 
Podemos incluso citar el uso contemporáneo de teorías, por ejemplo, la mecánica Newtoniana o la teoría de la selección natural de Darwin. 
Usamos cotidianamente ambas teorías para explicar diferentes fenómenos. 
La teoría de la selección natural nos ayuda a explicar fenómenos biológicos como la adaptación y la especiación. 
Con base en esta teoría podemos explicar, por ejemplo, por qué un grupo dentro de una población tiene más descendencia que otro grupo dentro de la misma población; podemos también saber, por ejemplo, cuáles son los ancestros comunes de especies contemporáneas, \textit{e. g.}, las aves de los dinosaurios. 

Por otra parte, utilizamos la teoría newtoniana para explicar el movimiento de los astros y hacer predicciones de qué posición tomarán en un momento dado, podemos explicar la fuerza que se imprime en una superficie cuando es golpeada por una masa con cierta aceleración y nos sirve para explicar el movimiento de objetos apelando a la inercia.

Ambas teorías tienen grandes ventajas: nos permiten explicar un amplio rango de fenómenos.
A pesar de que estas teorías nos permiten explicar una amplia variedad de fenómenos, no es claro que sean \emph{literalmente} verdaderas. 
En lo siguiente presentaré la teoría newtoniana y la teoría de la selección natural de Darwin, luego expondré algunos de los problemas que varios investigadores han detectado en ellas.

\paragraph{Mecánica newtoniana y selección natural darwiniana}

Tanto Darwin como Newton son dos personajes históricos que asociamos con logros científicos.
Para poder explicar el movimiento, Newton desarrolló una teoría que tiene como entidades el \emph{tiempo absoluto} y el \emph{espacio absoluto}. 
Un cuerpo se mueve o permanece en reposo con respecto al espacio y tiempo absolutos. 
Todos los puntos en el espacio absoluto permanecen constantes durante diferentes intervalos temporales. 
Para explicar la llamada "primera ley" de Newton (inercia), es necesario definir qué significa que un cuerpo esté en reposo o en movimiento. 
Siguiendo a Newton, sabemos que un cuerpo está en movimiento, porque ocupa distintos puntos del espacio en diferentes intervalos de tiempo.
Si el movimiento entre los puntos se da en intervalos iguales de tiempo, entonces el movimiento es uniforme.
Como señala Freudenthal "The distinction between 'rest' and 'uniform motion' implies, however, an absolutely resting frame of reference, and this can only be absolute space'' \parencite{freudenthalNewtonJustificationTheory1986}.
El movimiento en el espacio absoluto no puede ser percibido, mientras que las posiciones relativas entre los cuerpos sí.

La mecánica newtoniana explica el movimiento de los cuerpos con masa. 
La selección natural explica los cambios y las diferencias entre los del planeta. 
Darwin fue quien desarrolló la idea de la selección natural. 
Si bien, antes de la teoría de la selección natural, ya había intentos por explicar cómo se modifican los organismo, Darwin presentó un mecanismo mediante el cuál esto sucede.

Darwin no fue el primero en formular la diea de que los organismos se modifican durante el transcurso del tiempo.
Lamarck tenía una teoría de que los organismos se transforman.
La idea de Lamarck era que el "uso y desuso" de ciertos rasgos de los organismos, los hacía adaptarse mejor a su ambiente.
Pero Lamarck creía que las especies se generaban espontáneamente y luego se adaptaban a su ambiente. 
Fue Darwin quien tuvo la idea del mecanismo de la selección natural.

Darwin fue un personaje interesante, hizo un viaje en barco para recabar información acerca de las corrientes, información meteorológica y la profundidad del mar \parencite{allen2014}.
Darwin originalmente estaba convencido del transformacionismo lamarckiano, pero durante el viaje modificó sus creencias al descubrir evidencia fósil y patrones idénticos en diferentes regiones geográficas. 
Al regresar a Inglaterra, comenzó e ditar el libro que después se publicaría bajo el nombre "El origen de las especies".

A grandes rasgos, la teoría de la selección natural de Darwin la podemos formular de la siguiente manera:

\begin{enumerate}
    \item Hay variación entre los organismos de una población.
    \item La variación es heredable, por lo que la descendencia se parece más a los padres que otros organismos dentro de la misma población.
    \item Las variantes mejor adaptadas tienden a tener más descendencia que las otras.\footnote{Estos tres puntos son una ligera modificación de los puntos que presenta Peter Godfrey-Smith en su libro "Philosophy of Biology" \parencite[][p. 30]{godfrey-smith2014}. El lector puede encontrar una formulación parecida en \parencite{lloyd1988}. Originalmente esta formulación se la debemos a Lewontin \parencite[][]{lewontin1970}.}
\end{enumerate}



Esto expica por qué los descendientes se parecen más a los padres que a otros organismos de la población, algo que le preocupaba a Buffon \parencite{buffon1885} y a otros naturalistas.
Además, explica la variación de las especies a partir del mecanismo de la selección natural.
Muchas dudas se sucitaron con respecto a la teoría, en particular, hay personas que dudan de que la teoría sea correcta porque la selección natural no puede "ser vista"

\paragraph{Problemas con las teorías}

Ambas teorías explican un amplia variedad de fenómenos naturales.
Dada nuestra suposición inicial, es de esperarse que sean verdaderas.
Sin embargo, esta última afirmación dista de ser obvia.

Señalé que la teoría newtoniana sugiere la existencia de una entidad de la que no es claro que tengamos certeza de que existe: el espacio absoluto. 
Algunos teóricos como Leibniz sugirieron que algo extraño estaba sucediendo en la física newtoniana: si sólo tenemos evidencia de los movimientos relativos y un cuerpo en el espacio absoluto no se mueve con respecto a nada, entonces no tenemos evidencia de que exista el espacio absoluto. 
Newton en la \emph{principia} presenta dos experimentos mentales para señalar la existencia del espacio absoluto.
Uno de ellos es en el cual señala que si atamos dos globos con un cordón y los hacemos rotar sobre su eje en direcciones contrarias, la cuerda se tensa.
Si aceptamos que es verdad que la cuerda se tensa incluso en el espacio absoluto, entonces tenemos que aceptar que sí hay movimiento en el espacio absoluto, los globos se mueven con respecto a dicho marco, aunque estén en reposo uno respecto del otro \parencite[][pp. 6-12]{newton1966}.

Leibniz no fue el único en sospechar de la teoría newtoniana. Físicos como Ernst Mach (véase especialmente el capítulo 2 \parencite{mach2013} señala que debido a que diferentes marcos de referencia inerciales tienen las mismas consecuencias empíricas que hablar de espacio absoluto, entonces no es necesario apelar a dicha entidad, algo que sabemos gracias a la relatividad galileana\footnote{Por supuesto, Newton no hablaba de un marco privilegiado, sino que se refería al espacio absoluto como una entidad física. Pero lo que es importante notar es que podemos tener las mismas consecuencias sin apelar a dicha entidad.}.
Más aún, la teoría de la relatividad moderna, señala que donde Newton distinguía dos entidades, realmente sólo hay una: espacio-tiempo.
Incluso en la teoría de la relatividad, no es necesario apelar al tiempo absoluto para dar explicaciones. 
En la relatividad no hay un marco privilegiado y como consecuencia de la estructura del espacio-tiempo tenemos la dilatación temporal.

La discusión anterior está relacionada sólo con el problema del tiempo y el espacio absoluto.
No quisiera entrar en demasiado detalles de la teoría newtoniana (como la naturaleza de las fuerzas gravitacionales).
Pero el ejemplo anterior sirve para ilustrar que hubo dudas sobre las entidades de la teoría newtoniana y que, más aun, sabemos que no existen.

La teoría de la selección natural de Darwin tampoco está excenta de problemas.
Primero que todo, sabemos que Darwin no sugirió cuál era el mecanismo segun el cual hay herencia de caracteres.
Darwin creía, como también formuló Buffon, que la herencia de caracteres se daba a través de la sangre. 
En ese momento, la teoría de la herencia mendeleiana no había sido revisada por Darwin. 
Esta mezcla entre herencia medeleiana y selección natural, la hicieron los investigadores de la "síntesis moderna"\footnote{La síntesis moderna es un periodo de la historia de la teoría evolutiva que se desarrolló al principio del siglo XX. Los biólogos de este periodo incorporaron la genética mendeleiana a la teoría de la selección de Darwin.}. \parencite[][p. 50]{dieguez2012}
El anterior no es el único problema de la teoría evolutiva.
Darwin pensaba que la selección natural es gradual y que va a un paso considerablemente lento \parencite{losos2014}.

Hablando de Darwin, ahora sabemos que hay periodos de estásis, esto es, periodos donde no hay cambio evolutivo durante largos periodos de tiempo.
Sabemos, además, que la selección natural puede durar periodos cortos de tiempo, por ejemplo, en poblaciones de bacterias.

Durante la síntesis moderna, tuvimos una mejor formulación de la teoría darwinista.
Sin embargo, dada nueva evidencia, sabemos también que algunas afirmaciones de los investigadores de la síntesis moderna son erróneas. 
El programa adaptacionista de Mayr fue debatido por Lewontin y Gould.
Los adaptacionistas nos dicen que todos y cada uno de los rasgos de un organismo fueron seleccionados naturalmente.
Gould y Lewontin tienen un artículo famoso acerca de por qué este programa es falso \parencite{gould1979}.
Sabemos además que la herencia genética no es la única forma de selección natural.
La plasticidad fenotípica y otras formas de herencia son evidencia en contra de la herencia genética \parencite{uller2019}. 

A partir de lo dicho aquí podemos obtener dos conclusiones.
La primera de ellas es que ambas teorías son muy útiles.
Nos permiten describir una amplia variedad de fenómenos. 
La mecánica newtoniana permite explicar la aceleración en caída libre de cuerpos en la tierra, el movimiento de los cuerpos celestes e incluso nos permitió llegar a la luna \parencite{nasa}.
Con la teoría de la selección natural de Darwin podemos explicar la variación de las especies, los cambios que han sufrido a lo largo del tiempo e incorporar especies bajo el mismo clado de manera que expliquemos las relaciones entre diferentes organismos.

Sin embargo, teorías físicas más recientes niegan que haya \emph{espacio absoluto}, incluso señalan que la distinción entre dos entidades: espacio y tiempo, no es la más adecuada para describir la estructura del mundo.
Nueva evidencia recabada, nos ha mostrado que la teoría darwiniana no es correcta: hay un mecanismo para la herencia, la herencia no es sólo genética y el gradualismo de la teoría de Darwin es incorrecto.
Dado esto, parece que podemos concluir que las teorías, tanto de Newton como de Darwin, son falsas.
Lo que contradice nuestra suposición de que sólo las explicaciones verdaderas son genuinamente explicaciones, nuestra suposición es falsa.

Pero a lo largo de este capítulo he tratado de defender que la verdad es el valor fundamental del conocimiento, aquello por lo que el conocimiento es valioso.
La tesis veritista, entonces, está en serios problemas.

Ambas conclusiones no pueden ser verdaderas: afirmar la tesis veritista (cuya consecuencia es que sólo las explicaciones verdaderas son útiles) y al mismo tiempo afirmar que teorías literalmente falsas son explicativas.
Personalmente, creo que ambas afirmaciones son verdaderas. 
Claramente hay una inconsistencia en afirmar ambas.
Pero sabemos que si sospechamos que hay una inconsistencia: o bien hay que elegir sólo una afirmación, o bien no son afirmaciones exluyentes, o bien no son afirmaciones exhaustivas. 

Creo que hasta ahora he dado razones a favor de ambas afirmaciones y no creo que ninguna de ellas sea falsa.
No estoy seguro de que sean afirmaciones exhaustivas, pero confieso que no me parece que sean afirmaciones excluyentes.
Lo que resta es decir por qué creo que no son afirmaciones excluyentes.
En la siguiente sección trataré de dar razones para esto.

\subsection{Promesas: veritismo, virtudes y explicaciones falsas}

Es ecesario dar razones a favor de cómo ambas conclusiones pueden ser compatibles.
Si tomamos la segunda estrategia, hasta donde veo, tenemos dos opciones: o bien ambas teorías son verdaderas después de todo, o bien la verdad es fudamental para el conocimiento.

Tomemos el segundo disyunto: que la verdad no es fundamental para para el conocimiento.
El argumento que ofrece Pritchard, y que expusimos líneas arriba, señala que la verdad es el valor fundamental.
Pero esto también es problemático para cualquier teoría en filosofía de la ciencia que se tome en serio que somos agentes falibles. 
El argumento de Laudan \parencite{laudan1981} nos enseñó que las buenas explicaciones no están necesariamente conectadas a la verdad.
La teoría de Newton funciona, aunque no es claro que debamos asumir que es verdadera.

Pero el marco que presenta Pritchard tiene la ventaja de resolver los problemas con los que comúnmente se embiste esta tesis.
Pero este marco no tiene por qué restringir las explicaiones con teorías que no son literalmente verdaderas.
Lo únicop necesario es que tengamos un "contacto cognitivo con la realidad".
No es obvio lo que Pritchard quiere decir con esto, pero al menos, creo que podemos entenderlo con el supuesto de que las virtudes intelectuales de los agentes ofrecen dicho contacto.

En general, justificar nuestras creencias es complicado, en particular, la investigación científica no es empresa fácil.
Hay bastantes detalles que debemos en cuenta en el proceso de justificación de hipótesis, en especial cuando usamos modelos, leyes, teorías literalmente falsas para describir fenómenos. 

La verdad sí es el valor fundamental del conocimiento. 
Señalé que el mundo impone cierta fricción en nuestras creencias, el conocimiento es fáctico. 
Es más, no creo que haya duda de que Newton y Darwin son agentes virtuosos, cuyo uso de virtudes intelectuales no los llevó a la verdad.
Pero las virtudes epistémicas juegan un papel fundamental en la justificación de sus hipótesis.
Ambos son personajes que se involucraron con sus teorías y que ofrecieron las mejores razones que tuvieron para justificar sus hipótesis.

El conocimiento tiene como componente a la justificación y justificar hipótesis es un proceso dinámico. 
Pero podemos afirmar que los agentes sabemos que algo es verdadero aún cuando no podamos afirmar con 100\% de seguridad que no habrá nueva evidencia en contra de nuestras creencias.
Lo que importa durante el proceso es el ejercicio de las virtudes intelectuales.

Desde hace algunos años, los epistemólogos han tratado de modificar el objeto de estudio de la epistemología.
En lugar de tratar de explicar el valor del conocimiento proposicional, han sugerido un cambio hacia el entendimiento. E
ste cambio tiene algunas motivaciones: entre ellas está la falla del programa del análisis del conocimiento y la falta de una respuesta contundente para el escéptico. 

El cambio que proponen aquellas filósofas que sugieren el cambio de conocimiento a entendimiento depende de que no juzgamos las proposiciones, sino que tenemos más objetos que somos capaces de juzgar: relaciones entre diferentes fenómenos, estructuras, pedazos de información, relaciones de dependencia, etc.

Las epistemólogas que han transitado de un enfoque proposicional a uno que toma otro tipo de objetos, señalan también que tenemos más formas de juzgar creencias además de ser verdaderas o falsas. 
Entre estas diferentes maneras de juzgar creencias tenemos: creencias justificadas, racionales, fiablemente formadas, virtuosamente formadas, etc. 
Al mismo tiempo este enfoque no renuncia a la verdad como uno de los objetivos de nuestras empresas epistémicas \parencite{grimm2012}.

Si bien las motivaciones para realizar este cambio responden a problemas puramente epistémicos, este cambio de enfoque nos permite incorporar el hecho de que el número de proposiciones no es lo único a lo que los investigadores deberían prestar atención.
Somos agentes falibles que pueden tener entendimiento de un fenómeno, aun cuando no es claro que hemos llegado a proposiciones verdaderas.

Esto indica que si bien nuestro objetivo en la investigación es la verdad, no es algo que podamos obtener tan fácilmente. 
No parece que esto sea una afirmación problemática: somos agentes falibles y las herramientas que tenemos disponibles para justificar hipótesis no son perfectas. 
Por ejemplo, una de las herramientas más utilizadas en investigación empírica es la estadística.
Sabemos que un resultado estadístico no justifica al 100\% una hipótesis. 

Esto parecería ir en contra de nuestra intuición inicial: que el objetivo de la investigación científica es la verdad. 
Pero como he señalado, la verdad es difícil de obtener: nuestras capacidades cognitivas y herramientas para justificar hipótesis no son perfectas.
Aun así, nuestro problema es cómo obtener la verdad a pesar de nuestra falibilidad. 
Que nuestros métodos sean falibles, no implica que no haya manera de obtener conocimiento.
Tenemos maneras para modificar nuestras creencias con base en nueva evidencia.

Si bien es verdad que somos agentes falibles, de esto no se sigue que debamos abandonar por completo nuestros compromisos veritistas. 
En el libro citado anteriormente de Deborah Mayo, hay una frase que encaja muy bien con la teoría falibilista de Peter Klein \parencite{klein2019} que quiero presentar a continuación. 
Mayo nos dice "We set sail with a simple tool: If little or nothing has been done to rule out flaws in inferring a claim, then it has not passed a severe test." \parencite[][p. xii]{mayo2018}.

Lo que señala Mayo es una intuición que sólo se puede descartar con razones de peso. 
Como agentes en investigación, sabemos que somos falibles.
Sabemos que nuestras creencias pueden ser falsas, pero también sabemos que al justificar creencias, deberíamos ser capaces de eliminar alternativas contrarias (derroteros) a nuestras creencias.

La pregunta importante es cómo a pesar de nuestra falibilidad podemos obtener conocimiento. 
Afortunadamente, Peter Klein \parencite{klein2019} ha desarrollado una teoría del conocimiento que encaja muy bien con lo que he señalado hasta ahora.

De acuerdo con lo que Klein llama "infinitismo derrotable" [defeasible infinitism], los seres humanos somos agentes falibles.
Pero valoramos el conocimiento, el "conocimiento de verdad". 
Esto quiere decir: conocimiento para el que tenemos suficientes razones.

El punto de Klein es que, como agentes, justificamos nuestras creencias con base en las mejores razonez disponibles.
Si tenemos a nuestra disposición razones para sostener una creencia cualquiera $x$, y no tenemos a nuestra disposición una creencia $y$ que disminuya nuestra justificación de $x$, tenemos conocimiento certero.
Si obtenemos nueva información $z$ que haga que dudemos de $x$ y nos haga retractarnos de nuestra creencia original, entonces hay que evaluar nuestra creencia a la luz de la nueva evidencia.

Por ejemplo, supongamos que leemos un estudio que señala una fuerte correlación entre el omeparazol y problemas cardiacos.
Luego descubrimos que el estudio está sesgado.
En este caso dos opciones se abren ante nosotros: o bien retractamos nuestra creencia, o bien ofrecemos razones para señalar por qué no hay sesgo después de todo.
Si somos capaces de dar razones de por qué no hay sesgo en el estudio, entonces tenemos conocimiento certero.

Para que esta tesis sea plausible, Klein defiende su teoría en contra de lo que él llama "el riesgo de desconfirmación empírica".
Este problema, señala Klein, embiste a las teorías epistémicas que afirman que una creencia debe tener una cadena causal adecuada.

El argumento de Klein descansa en lo siguiente: supongamos que tenemos una creencia falsa causada por evidencia empírica.
Al momento de corregir nuestra creencia con base en nueva evidencia empírica, no podemos estar seguro de si la nueva creencia modifica la cadena causal o es parte de la misma cadena.

Una sugerencia es señalar que lo que hice fue corregir mi creencia.
Pero la creencia formada por nueva evidencia tiene una historia causal completamente diferente a la anterior.
No sé si ahora tengo dos creencias causalmente formadas, o una creencia causalmente revisada. 

Siempre es difícil saber si hay una relación causal entre eventos (es necesario investigar empíricamente), más difícil aún saber si nuestras creencias fueron causadas por diferentes eventos o hemos modificado la cadena. 
En palabras de Klein:

\begin{quote}
    I take that as a good prima facie reason for thinking that the difference between real knowledge and less paradigmatic forms of knowledge or ignorance depend on the quality of reasons for the belief, not the etiology of the belief. \parencite[][p. 403]{klein2019}
\end{quote}

Esto hace sentido del hecho de que tanto Newton como Darwin \emph{entendían} tenían conocimiento certero. 
Esto es la calidad de sus razones era adecuada. 
Lo cual indica que hay valor en las investigaciones que realizaron, aun cuando nueva información nos dice que estaban equivocados. 

Siempre podemos encontrar derroteros para muchas de nuestras razones para sostener creencias.
Nueva información hace que nos retractemos de nuestras creencias, a menos que podamos derrotar la información.
En la época de Newton hubo investigadores que dudaron de la existencia del espacio absoluto.

Ahora sabemos que podemos hacer mucho trabajo en física sin la necesidad de postular dicha entidad. 
Pero Newton tenía buenas razones para sostener sus creencias, además la investigación racional opera de esa manera: desarrollamos teorías y nueva información es capaz de derrotar nuestras creencias. 
A menos que tengamos razones para desechar esa nueva información, estamos justificados y tenemos conocimiento.

Por supuesto, esto sólo indica que Newton tenía conocimiento, pero nosotros sabemos que la teoría es falsa.
Me parece que esto no es problemático en absoluto.
Como agentes preocupados por tener creencias justificadas, usamos modelos, abstracciones e idealizaciones para justificar dichas creencias.

Estos métodos no reflejan de manera precisa los fenómenos \parencite{bokulich2016}.
Pero son métodos que justifican lo suficiente como para decir que hay conocimiento.
Lo que importa es que estemos conscientes de que no reflejan de manera precisa su objetivo y esto puede ser problemático.
En estos métodos, ampliamente usados en investigación, fácilmente podemos cometer errores.
Errores que pueden llevarnos a conclusiones equivocadas, pero que somos capaces de solventar.
Pensemos por ejemplo en la definición de agente de la economía clásica y las diferentes maneras en la que los economistas han tratado de resolver.
Esto no quiere decir que la economía clásica sea algo que deberíamos abandonar por completo, sólo que hay que ser cuidadosos y tener en cuenta los diferentes factores involucrados \parencite{sen1977}.

El problema es, me parece, como entendemos la naturaleza de la verdad y cómo estos métodos encajan con lo que he dicho hasta ahora.
Quiero discutir estos temas en el siguiente capítulo porque, al final, me interesa discutir cómo la verdad juega un papel en nuestros modelos causales.

\subsection{Conclusiones}

En este capítulo defendí la tesis veritista: que el conocimiento es valioso porque es verdadero.
Presenté dos problemas que algunas filósofas señalan en contra de la tesis.
A estos problemas, expuse las soluciones que ofrece Pritchard a favor del veritismo. 

Sin embargo, no quedaba claro que si la verdad hace que el conocimiento sea valioso, cómo dicha tesis es compatible con casos de teorías que son literalmente falsas.
Presenté dos casos históricos que son casos de éxito epistémico, pero no son teorías verdaderas. 
Para hacer compatible la tesis veritista con estos casos, presenté la teoría de Klein que nos permite rescatar nuestras intuiciones originales: el veritismo y una teoría de la certeza basada en justificación falible.

Quisiera terminar señalando dos cosas que restan por hacer. 
Mi objetivo final es decir cómo los modelos causales sirven en la investigación, incluso cuando son abstracciones de los fenómenos que represenmtan.
Creo que el marco presentado en este capítulo es útil para continuar argumentando que los compromisos que tienen los investigadores cuando utilizan modelos es menos problemático de lo que pudiera parecer a primera vista.

% chapter chapter_4 (end)
