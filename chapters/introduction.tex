% !TEX root = ../main.tex
% LTeX: language=es, en

\chapter{Introducción}\label{ch:introduction}

\section*{Parte I}


\noindent El Doctorado ha sido un proceso extenso
durante el cual he aprendido muchísimo. La parte más
sustanciosa de lo que he aprendido está relacionado
con la filosofía, en particular filosofía de la ciencia.
\footnote{
	Juro que este breve relato tiene un punto.
}
La alta calidad del posgrado me ha permitido
profundizar en los temas que me interesaban, interés
que me motivó a entrar a este doctorado.

La fascinación que tengo por los temas en filosofía de
la ciencia nació durante mi licenciatura, porque
durante este periodo tomé cursos con dos profesores
que se dedicaban al área\footnote{
	Al menos, lo que en ese momento pensaba que era la
	filosofía de la	ciencia. Si les pregunto, estoy
	seguro de que su respuesta sería que no se dedican
	a la filosofía de la ciencia.
}
Ambos excelentes maestros y filósofos.

Durante este periodo de mi formación (licenciatura) mi
atención estuvo enfocada a los temas de la \emph{explicación
científica} y la \emph{naturaleza de la causalidad}. Mis
intuiciones---todavía queda un resabio de esto---sugerían
que señalar cuál es la causa de un fenómeno es la mejor
manera que tenemos de
explicarlo\footnote{
	Estoy usando de manera coloquial el término. Uso el
	término para hablar de aquello que hacen las
	personas en cualquier ámbito de la vida en el que se
	requiere ofrecer una explicación. Por ejemplo: que la
	carne se echó a perder porque el refrigerador
	dejó de funcionar, que la casa se incendió a causa
	del corto en el enchufe, etc. Ambos son casos en los
	cuáles decimos que A \emph{es una explicación de} B.
}.
Por ejemplo, parece claro que la mejor explicación de
por qué tengo cáncer de pulmón se debe a mis hábitos
de fumador, y fumar \emph{es una causa} de cáncer.

Que la causalidad sea parte de nuestra teoría de la
\emph{explicación científica,} es una intuición que no
está alejada de algunas teorías de la explicación
clásicas discutidas en filosofía de la ciencia. Me
parece además que la motivación más clara para incluir
una relación causal en nuestras teorías de la
explicación científica, se debe a que nos parece
necesario que la explicación de un fenómeno es \emph{
asimétrica}: que haya explotado el alimentador de
electricidad, explica por qué mi computadora se apagó
y perdí toda mi tesis. Pero claramente perder mi tesis
no explica por qué explotó el alimentador eléctrico.
Llamemos a esto \emph{condición de asimetría de la
explicación} [CA].

Salmon hizo notar en \citetitle{Salmon1970} que una teoría
de la explicación científica debe satisfacer CA. Salmon
argumenta que el modelo clásico que Hempel $\&$ Oppenheim
desarrollan en \citetitle{Hempel1948} no satisface CA y, por
lo tanto, es necesario modificar la teoría para dar cuenta
de la explicación científica. Quiero además resaltar que la
discusión que Salmon hace del modelo de Hempel $\&$
Oppenheim nace de un problema que  Sylvain Bromberger
discute en su artículo \citetitle{Bromberger1966}.

Para apreciar la discusión en torno a las motivaciones de CA
asimetría de la explicación, voy a comenzar exponiendo
brevemente en qué consiste el modelo de Hempel $\&$
Oppenheim.Este va a ser el punto de partida desde el cual
quiero trazar una historia del papel que, me parece, juega
la causalidad en una teoría de la explicación científica.
Elijo este punto de partida porque (i) es una teoría clásica
en filosofía de la ciencia y (ii) pretendo justificar que
\emph{la asimetría de la explicación es un reflejo de la
dirección causal.} Habré logrado el objetivo de esta
sección, si el lector está convencido de que la causalidad
es la mejor forma de dar cuenta de CA. 
En principio, no me parece controvertido afirmar que las
explicaciones no son simétricas. Mas controvertido es
afirmar que las relaciones causales explican perfectamente
por qué esperamos esta asimetría. Es por esto último que
en esta primera parte,---que además sirve como introducción
a mi trabajo---comienzo haciendo un breve repaso por los
títulos mencionados en el párrafo anterior, terminando con
una caracterización de la teoría \emph{intervencionista} de
la causalidad.


\paragraph{Hempel y Oppenheim (\citeyear{Hempel1948}),}
caracterizaron  a la explicación científica como un
argumento deductivo. Sabemos, por ejemplo, que \say{todos
los metales se dilatan cuando se calientan} y esta oración
cuantificada universalmente implica deductivamente cualquier
instancia particular. Si las explicaciones son todas
argumentos deductivos, entonces lo que explica por qué
aumenta ligeramente el tamaño de este pedazo de cobre cuando
lo dejo sobre una hoguera, se debe a que es un caso
particular de la oración cuantificada. El análisis que 
ofrecen estos autores sugiere que \emph{explicar} consiste 
en ofrecer un argumento y, en principio, este análisis suena
sumamente plausible, sin embargo, hay un par de detalles que
merman esta plausibilidad.

En primer lugar, hay que notar que la oración cuantificada
es crucial. Esta oración es lo que nos permite que el 
argumento sea deductivo\footnote{
  Estos autores también caracterizaron una forma inductiva
  de explicación. Para fines explicativos, en este texto
  sólo caracterizaré la forma deductiva. 
}: 
eliminando el cuantificador universal y sustituyendo la 
variable por una constante\footnote{
	Por si acaso, el argumento se formaliza
  $\forall{x}(M(x)$ \& $C(x) \rightarrow
  D(x))$ \& $(M(a)$ \& $C(a))
  \Rightarrow	D(a)$. 
}. 

En segundo lugar,esta oración cuantificada debería ser la 
formulación lin-güística de una ley natural. El hecho de que
la oración sea una ley natural es lo que asegura que el
argumento sea explicativo. Para ilustrar esto, consideremos
la oración \say{todas las monedas en el bolsillo de Oscar 
son de $\$1$.} Sucede que la oración es verdadera\footnote{
	Además resulta que es siempre verdadera por antecedente
	falso. Pero me parece que no hace diferencia para el
	ejemplo. 
},
dado que está cuantificada  universalmente, si tomo una
moneda de mi bolsillo, entonces será de $ \$1 $. Sin
embargo, que todas las monedas de mi bolsillo tengan una
denominación de $\$1$, no implica que hemos explicado por
qué las monedas que he sacado para pagar, son todas de
$ \$1 $. La conexión faltante en el ejemplo de las monedas,
parece sí estar presente en el caso del cobre.

El modelo sugerido por H \& O---también llamado modelo por
ley de co-bertura---requiere que la oración cuantificada
cumpla con las características para obtener el título de
\emph{ley de la naturaleza}\footnote{ 
  Se suele caracterizar a las leyes de la naturaleza como un
  fenómeno natural que es verdadero en todos los casos
  posibles, e	independiente del tiempo y lugar en el que
  ocurre. Esta afirmación debe ser verdadera en todas las
  situaciones e independiente del tiempo y del lugar.
  Condiciones que dicha oración cuantificada debe	cumplir si
  es que va a contar como una genuina explicación. Uno de
  los problemas que surgen de esto es que debemos ofrecer
  criterios parra distinguir una ley de la naturaleza de
  generalizaciones	universales espurias
  \parencite{Nagel1962}. Ahora, es una cuestión distinta, si
  de hecho hay algo que cumpla las condiciones para ser una
  ley de la naturaleza. Ambos problemas son tangenciales a
  mi	proyecto, y no discutiré más aquí. El lector
  interesado puede revisar	\parencite{sep-laws-of-nature}
}, 
de otro modo, el argumento carecería de cualquier valor
explicativo. Si estas condiciones se satisfacen, entonces
tendremos una oración cuantificada universalmente, que
presumiblemente es la formulación lingüística de una ley
natural, y que implica como instancia particular el fenómeno
en cuestión que queremos explicar. Si esto sucede, entonces
hemos tenido éxito ofreciendo una explicación.

La necesidad de caracterizar a las leyes de la naturaleza es
un problema de suyo, ya que no es obvio que todas las
disciplinas científicas  disponen de leyes naturales para
ofrecer explicaciones. Además, es necesario caracterizar la
naturaleza de la relación entre las leyes naturales y los
fenómenos particulares que son sus instancias. Si, por
ejemplo, es una relación causal, o es una relación de
necesitación entre antecedente y consecuente, etc. Pero
probablemente lo más grave, sea que esta
caracterización implica que la explicación es una relación
\emph{simétrica.}


\paragraph{Sylvain Bromberger (\citeyear{Bromberger1966})}
detecta este problema en la caracterización que Hempel y
Oppenheim ofrecen. Para defender su argumento, Bromberger
comienza señalando que hay muchas similitudes entre explicar
que este pedazo de cobre se dilata al calentarse---digamos, 
el fenómeno de la dilatación de los metales---y responder a
la pregunta \say{¿por qué mi cuchara se dilata cuando la 
caliento?} Por ejemplo, el autor señala que

	\begin{quote}
    $ \ldots $ there are issues in the philosophy of science
    that warrant an interest in the nature of why-questions.
    The most obvious of these issues are whether science (or
    some branch of science or some specific scientific
    doctrine or some approach) ought to, can, or does
    provide answers to why-questions, and if so, to which
    ones.
	\end{quote}

La afirmación de Bromberger es altamente intuitiva. No
es exagerado suponer que la investigación científica
se dedica a responder preguntas de este tipo\footnote{
  Por supuesto, es una afirmación	completamente distinta si
  la	investigación científica \emph{puede} responder a
  \emph{todas} las preguntas de este tipo; otra cuestión
  distinta es si la	ciencia \emph{debe} responder a
  \emph{todas} las preguntas de este tipo. Sería muy triste
  una respuesta afirmativa a la segunda cuestión.
}:
es más o menos claro que una explicación suele ser una
respuesta a una pregunta \emph{por-qué.}

Sin embargo, si  enmarcamos el problema en los términos de
Bromberger, entonces lo que el autor llama \say{la doctrina
hempeliana} es claramente incorrecta. El argumento procede
así: supongamos que lo que dice Bromberger es verdad y que
la doctrina hempeliana es correcta. De forma inocente, le
preguntamos a Cristian \say{¿por qué la altura del Empire
State es de 381 metros metros?} Cristian, muy versado en
funciones trigonométricas, responde de la siguiente manera:
hay un punto en quinta  avenida que está a $ x $ distancia
del Empire State. Dado el ángulo que forman los rayos de sol
desde el punto más alto del edificio hasta ese punto,
podemos derivar fácilmente la altura del edificio usando la
función tangente.

El argumento en este caso es claramente deductivo. Si
sustituimos las variables pertinentes, podemos adecuadamente
concluir cuál es la altura del Empire State: resultado nada
trivial. Pero esto significaría que usando este argumento,
Cristian ha respondido adecuadamente \emph{por qué} la
altura del Empire State es de 381 metros, pero nadie diría
que el hecho de que el ángulo que forman los rayos del sol
sea $ \theta $ y que la distancia entre el edificio y quinta
avenida sea $ x $, explica por qué el edificio mide 381
metros. Que el edificio tenga esa altura particular es algo
que debería explicar la persona que diseñó el edificio. Esto
no es culpa de Cristian, sino de quién afirme que dicha
respuesta es una genuina explicación de la altura del
edificio.

Son contraejemplos como éste, los que muestran que hay que
abandonar la doctrina hempeliana, y que por lo menos es
necesaria una caracterización distinta de la explicación
científica. Hay que notar que Bromberger discute
este\footnote{ 
  El ejemplo del poste no aparece en ninguno de sus escritos
  (el ejemplo original usa al Empire State) porque
  originalmente Bromberger no estaba atacando la asimetría
  que surge de la doctrina hempeliana. La historia muestra
  que el ejemplo del poste se lo comentó a Hempel en una
  conferencia, como señalan en \parencite{mitBromberger},
  anécdota mencionada también en
  \parencite[p.~81]{Dewulf2022}. La intención de Bromberger
  estaba enfocada en la semántica de las oraciones
  interrogativas y la ignorancia racional, tal como muestran
  sus trabajos posteriores \parencite{Bromberger1992}.
},
y otros contraejemplos en \citeyear{Bromberger1966}, pero
Wesley Salmon fue quien se concentró en mostrar que el
contraejemplo de la sombra implica que, si la doctrina
hempeliana es correcta, entonces la explicación científica
es simétrica. Una clara falta a CA y si este argumento es
sólido: el modelo por ley de cobertura de Hempel y Oppenheim
debe ser incorrecto. Durante este periodo de la filosofía,
muchos esfuerzos se concentraron en resolver este y otros
problemas problemas derivados de la doctrina hempeliana.

Wesley Salmon fue quien empleó el contraejemplo que presentó
Bromberger, y mostró que el modelo por ley de cobertura
implica esta simetría. Pero Salmon no sólo se dedicó a
señalar los problemas del modelo por ley de cobertura, sino
que formuló una teoría de la explicación que lidia con dicha
asimetría y que se presenta como una mejor alternativa a la
teoría de la explicación desarrollada por Hempel y 
Oppenheim.

\paragraph{Wesley Salmon (\citeyear{Salmon1970})} comienza
por afirmar que la asimetría temporal es crucial para la
explicación científica. Polemizando el contraejemplo de
Bromberger, Salmon señala que

 \begin{quote} 
    Although the sun, flagpole, and shadow are perhaps
    commonsensically regarded as simultaneous, a more
    sophisticated analysis shows that physical processes
    going on in time are involved.	Photons are emitted by
    the sun, they travel to the	vicinity of the flagpole
    where some are absorbed and some are not, and those
    which are not go on to illuminate the ground
    \parencite[p.~72]{Salmon1970}
 \end{quote}

Esto, por supuesto, resolvería el problema de la asimetría.
A primera vista, es obvio que los procesos físicos tienen
cierta dirección temporal.\footnote{ 
  Hay una discusión en torno a si hay una dirección
  privilegiada del tiempo, o si las relaciones temporales
  pueden revertirse. Esta discusión se ha retomado para, a
  su vez, definir la dirección en la que opera la
  causalidad. Esto es un tema en su propio derecho y
  necesitaría un trabajo mucho más largo que le hiciera
  justicia al tema. El lector interesado puede consultar el
  artículo clásico de Russell para un argumento
  antirrealista de la causalidad
  \parencite{onthecauserussell}. El argumento de Russell
  descansa en el hecho de que las igualdades en física son
  reversibles. Con respecto a Hempel y Oppenheim, se suele
  asumir que los empiristas lógicos basaron sus argumentos
  escépticos, en los argumento clásicos de Hume, argumentos
  que aparecen en \parencite{hume1784}. Para una discusión
  general en torno a la dirección del tiempo, la llamada
  \say{flecha del tiempo}, el lector puede revisar
  \parencite{utmArrowTime}, un argumento a favor de la
  retrocausalidad lo ofrece Dummet en
  \parencite{dummetcause}.
}
Suena sensato pensar que cualquier explicación que podamos
ofrecer de un fenómeno, debe respetar dicha dirección.

Sin embargo, basar la asimetría de la explicación en la
asimetría temporal de los procesos físicos parece que va
colando a las relaciones causales en nuestro análisis. Pero
Hempel y Oppenheim---se afirma tradicionalmente---querían 
excluir a las relaciones causales
de su análisis de la explicación científica, porque la
causalidad, es una noción oscura esto significa que debo
que justificar dicho cambio. Esto significa que debo
justificar por qué las relaciones causales comienzan a tomar
un papel prominente en la explicación científica. Para hacer
esto con éxito, debo recordar al lector que la discusión de
Salmon está enmarcada dentro de la teoría de la explicación
científica que él mismo desarrolla. Salmon busca resolver
los problemas de la teoría de Hempel a la luz de considerar
la asimetría de la explicación.

Para explicar brevemente en qué consiste la solución de
Salmon, voy a comenzar con un ejemplo. Supongamos que
queremos saber si un dado está cargado. Si lanzamos un dado
icosaédrico, el dado sólo va a mostrar una de sus caras en
cada lanzamiento y el resultado será un número entre 1 y 20.
Este es el espacio de \emph{eventos} definido para nuestro
experimento.

Supongamos que lanzamos el dado 50 veces y generamos una
lista de los resultados obtenidos. Obviamente el dado sólo
puede mostrar una cara a la vez, lo que implica que
cualquier otro evento---las caras que no se muestran---
están excluidas una a una. Es decir, si una cara cualquiera
del dado aparece, entonces el dado no muestra ninguna otra
cara. En este caso, el conjunto de los posibles resultados
está definido por el conjunto de números entre 1 y 20.
Sabemos además que cualquiera de las caras puede aparecer
por lo menos 1 de cada 20 tiradas.

Sucede que he estado usando dados icosaédricos de vez en
cuando---jugando D\&D--- y supongamos que visito a Mar. Mar
toma un dado de su colección y me propone una apuesta: cada
vez que el dado muestre un número non, ella me paga $ \$10
$, mientras que yo le pago la misma cantidad cada que sale
un número par. Por alguna razón, durante las 50 tiradas, los
números pares aparecen con más frecuencia---digamos $
\frac{15}{20} $---por lo que estoy comenzando a perder
grandes cantidades de dinero: tengo la sospecha de que el
dado está cargado.

Recapitulando el ejemplo: lo que quiero explicar es por qué
los números pares están apareciendo con más frecuencia, y
parece sensato pensar en que una explicación posible es que
el dado está cargado. Siguiendo las etiquetas clásicas, el
\emph{explanans}---aquello que quiero explicar---es por qué
(M) Mar está ganando 15 de 20 tiradas; mientras que el 
\emph{explanandum}---aquello que presentamos como evidencia,
justificación, hipótesis, la respuesta a la pregunta, 
etc.---es que (D) el dado está cargado. Llamemos \emph{clase
de referencia} a las condiciones que parten el espacio de
probabilidad en dos clases mutuamente excluyentes: que el
dado esté cargado $P(D) $ o que el dado no está cargado 
$P(\neg D)$. Suele suceder que los dados cargados muestran
algunos números con más frecuencia que otros, entonces $ C $
es estadísticamente relevante para $ M $, y si esto es
verdad, entonces, $ P( M | C ) $ debería ser mayor que $ P(M 
| \neg C ) $.

En corto, la \emph{clase de referencia} parte el espacio de
eventos en las variables que sospechamos hacen una
diferencia estadística. Una parte de la solución, que
desarrolla Salmon, comienza por hacer clara la definición de
la clase de referencia. Brevemente, la clase de referencia
es el conjunto que usamos para definir el espacio de eventos
y seleccionar las variables estadísticamente relevantes.

Por ejemplo, supongamos que queremos explicar por qué hoy
hubo una tormenta. Nuestra clase de referencia es el
conjunto de días en los que hubo tormentas y las condiciones
que estuvieron presentes mientras el fenómeno ocurría. Que
disminuya la presión atmosférica suele indicar que se
aproxima una tormenta, lo que significa que podemos dividir
la clase de referencia en los días en los cuáles hubo una
tormenta y disminuyó la presión atmosférica, y aquellos días
en los que hubo una tormenta y no disminuyó la presión
atmosférica. Si condicionamos el evento a una de estas dos
clase de referencia y descubrimos que una de ellas es
relevante para el hecho de que llueva, entonces esto explica
por qué hoy hubo una tormenta, cuya respuesta es, porque
disminuyó la presión atmosférica.

Salmon considera que la clase de referencia que hemos
definido, debe ser \emph{homogénea} y debe ser la \emph{más
amplia,} es decir, que incluya todas las variables
estadísticamente relevantes---mientras excluimos las
variables irrelevantes---para que el fenómeno en cuestión
ocurra. Retomando el ejemplo, podemos hacer más estrecha la
clase de referencia al introducir variables como la
velocidad del viento, las condiciones de humedad, la época
del año, etc. Para mi ejemplo simple, la frecuencia de (A)
los días en que hubo lluvia, (B) disminuyó la presión
atmosférica y (C) hubo velocidades de viento arriba de los $
80km/hr $, resultando en $ P( A | B$ \& $ C ) $. Agregar
estas variables, vuelve más estrecha a la clase de
referencia.

Regresando a la teoría de la explicación que Salmon
desarrolla, hay ciertas condiciones que la respuesta a la
pregunta \say{¿Por qué llovió hoy?} debe cumplir para que
cuente como una explicación genuina. En primer lugar,
debemos definir la clase de referencia. En segundo lugar,
tenemos que elegir qué condiciones son estadísticamente
relevantes para que ocurra el fenómeno, de manera que seamos
capaces de incluir todas las variables. Por último, las
variables deben ser estadísticamente relevantes para que
ocurra el
fenómeno en cuestión. De esta manera hemos elegido las
variables que \emph{hacen una diferencia} para que el
fenómeno ocurra. Esta diferencia está definida de la
siguiente manera: si $ P( A | B ) \neq P( A | \neg{ B } )
$\footnote{
  Estoy asumiendo que de hecho hay una diferencia en la
  frecuencia. Si no la hubiera, es decir $ P( A | B ) = P( A
  | \neg{ B } ) $, entonces la relevancia sería la misma que
  una tirada de moneda, por los axiomas 2 y	3 de Kolmogorov. 
},
y $C$ es una variable estadísticamente relevante---una
variable que hace una diferencia---entonces la respuesta a
\say{¿Por qué llovió el día de hoy?,} sería \say{Porque
  disminuyó la presión atmosférica y hubo vientos arriba de
80 $Km/h$.} Dado que estamos asumiendo que estas variables son
estadísticamente relevantes, entonces debería cumplirse que
$ P( A | B$ \& $C )$ es mayor que cualquiera entre $ P( A |
\neg{ B }$ \& $C ) $, $ P( A | B$ \& $\neg{ C } ) $, $ P( A
| \neg{ B }$ \& $\neg{ C } ) $. Es gracias a que
seleccionamos una clase de referencia adecuada---en la cuál
están incluidas las variables que de hecho \emph{hacen una
diferencia}---que podemos explicar la ocurrencia del evento
$ A $ a partir de la ocurrencia de $ B $ y $ C $.

Antes de continuar con la exposición, quiero recapitular
los detalles del modelo de Salmon. Primero, debemos ser
capaces de hacer una \say{partición homogénea} del
fenómeno a explicar. Que sea una partición homogénea
significa que dado el fenómeno que queremos explicar,
debemos seleccionar todas las variables relevantes que
hagan una diferencia para que el evento ocurra. Esta
partición debe ser exhaustiva, es decir, que no falte
ninguna variable relevante por incluir y que no sea
posible agregar nuevas variables al conjunto. En palabras
de Galavotti \say{[e]n esta perspectiva lo que cuenta para
la explicación no es la alta probabilidad, como lo
requería Hempel, sino estar en posición de afirmar que
la  distribución de probabilidad asociada con el
\emph{explanandum} refleja la información más completa y
detallada	que podamos obtener}\parencite{Galavotti2018}\footnote{
  En el original \say{In this perspective what counts	for
  the sake of explanation is not high probability, as
  required by Hempel, but being in a position to assert
  that the probability distribution associated with the
  explanandum reflects the most complete and detailed
  information attainable.}
} 

Hasta aquí, el modelo de explicación defendido por Salmon
tiene una serie de ventajas. El modelo resuelve muchos de
los problemas que tiene el modelo por ley de cobertura: la
noción de explicación que desarrolla Salmon no echa mano de
leyes y, por tanto, no tiene el problema de distinguir entre
leyes y generalizaciones accidentales. Además, evitar que la
explicación descanse en la noción de \emph{ley de la
naturaleza}, se antoja más adecuado para analizar las
explicaciones en aquellas ciencias en las que no es obvio
que haya tal cosa (como la biología o la economía).

Sin embargo, aún resta un problema por sortear: que seamos
capaces de seleccionar las variables pertinentes, no implica
que la relación temporal definida sea la correcta.
Supongamos que hay una cadena temporal entre el aumento de
la velocidad del viento, la disminución en la presión
atmosférica y, por último, que llueva. Llamemos a esta
descripción $ D $. Sin embargo, hay otra
manera~---perfectamente sensata---de describir esta relación
temporal: supongamos ahora que tanto la disminución en la
presión atmosférica y el aumento del viento ocurren al mismo
tiempo, luego llueve. Llamemos a esta descripción $ D' $.

Si asumimos que seleccionar las variables relevantes es
\emph{suficiente} para caracterizar a la explicación
científica, entonces ambas descripciones constituyen
explicaciones correctas. Más aún, esto implica que $ D  $ y
$  D' $ son explicaciones equivalentes. Sin embargo,
intuitivamente, $ D $ y $ D' $ describen fenómenos
completamente distintos. Mientras que $ D $ parece sugerir
que la baja en la presión atmosférica \emph{causa} el
aumento en la velocidad del viento y esto a su vez
\emph{causa} que llueva, $ D' $ parece sugerir que tanto la
baja en la presión atmosférica y el aumento en la velocidad
del viento \emph{causan} que llueva.

Usando sólo el modelo de Salmon, no seremos capaces de
distinguir entre ambas descripciones. En ambos casos están
involucradas las mismas variables: la presión atmosférica y
la velocidad. Además ambas son estadísticamente relevantes. 
Esto implica que el modelo de Salmon no ha resuelto los
problemas relacionados a la dirección
temporal. Este no es un problema de simetría tal como el que
tenía el modelo por ley de cobertura. El problema del
modelo de Salmon se debe a que hay distintas formas de
describir un fenómeno y esas distintas formas de describir
el fenómeno estructuran a las variables de maneras
temporalmente distintas. Esto es relevante para satisfacer 
CA, porque nos importa la dirección temporal de CA.  


Claramente este problema le preocupa al autor, como muestra
la cita casi al inicio de esta sección. Salmon reconoce que
el modelo tal como está expuesto, es claramente sensible a
hacer pasar las correlaciones estadísticas por relaciones
de dependencia causal.\footnote{
  Salmon discute e intenta resolver esto apelando a los
  estados de baja entropía y a la relación temporal entre el
  evento a explicar y la clase de referencia. No voy a
  detallar la solución que Salmon propone,	porque no es
  necesario para fines del proyecto. El lector	interesado
  puede leer la última sección de \parencite{Salmon1970}. 
}

Lo anterior es básicamente una forma de retratar un problema
más general. En forma condensada, este problema se refleja
en el más que común el eslogan \say{correlación no implica
causalidad.} El problema con el análisis de Salmon, tal como
lo he presentado hasta aquí, es que no hay manera de
distinguir entre las correlaciones que tienen una causa
común---tal como sucede en $D$---y aquellas que son
relevantes por separado---como en $D'$. El ejemplo del
barómetro es un caso en el cuál hay una causa común. 

Sé que para determinar cuál es la presión atmosférica, puedo
utilizar un barómetro. Sé además que cada vez que el
barómetro se comprime, es porque ha aumentado la presión
atmosférica. Obviamente la lectura del barómetro es
relevante para que se den condiciones de lluvia, pero la
variable causalmente relevante es la presión atmosférica, no
la lectura del barómetro. Si ingenuamente comienzo a mover
con mi mano la aguja del barómetro,---mientras todas las
demás condiciones permanecen constantes---es claro que no
he aportado a la producción de lluvia: el barómetro no es un
factor causal relevante.

Sin embargo, sabemos que la lectura del barómetro \emph{es}
una variable estadísticamente relevante, que \emph{hace una
diferencia} (tal como lo he definido líneas arribas) en el
resultado esperado,\footnote{
  No estoy seguro de que todas las personas que se dedican
  al clima tengan barómetros funcionales que usan
  cotidianamente para medir la presión atmosférica, pero
  puedo	apostar a que usan las lecturas de sus barómetros
  para determinar si va a llover o no, y que cuando hay una
  baja en la lectura del barómetro (y otras condiciones
  relevantes para el caso), es porque de	hecho va a llover.
  Confío en los expertos. 
}
pero es claro que he seleccionado un factor
\emph{causalmente} irrelevante, aún cuando es un factor
\emph{estadísticamente} relevante. Si sólo caracterizamos a
la explicación como he hecho hasta aquí, entonces vamos a
ganar casos que cumplen las condiciones, pero que no son
explicaciones.\footnote{
	Esto muestra que obviamente no siempre es bueno ganar,
	especialmente si lo que hemos ganado es basura.
} 
Para resolver este problema, Salmon hace notar que su modelo
de explicación puede naturalmente incorporar
\emph{relaciones causales.} De este modo, en torno al
ejemplo de la sombra, Salmon comenta que

	\begin{quote}
    The reason that the explanation of the length of the
    shadow in terms of the height of the flagpole is
    acceptable, whereas the \say{explanation} of the height
    of the flagpole in terms of the length of the shadow is
    not acceptable, seems to me to hinge directly upon the
    fact that there are causal processes with earlier and
    later temporal stages. \parencite[p.~72]{Salmon1970}
	\end{quote}
  
Lo que sugiere la cita anterior es que para responder a
estos problemas, deberíamos incluir factores causales en
nuestro análisis. Esto es una clara desviación de lo que se
suele asumir eran tesis básicas del antaño positivismo
lógico. Durante este periodo, mucho esfuerzo filosófico se
dedicó al tema de la causalidad y la explicación. Relevante
para este contexto, por ejemplo, es que Lewis publicó
\citetitle{Lewis1973a} y \citetitle{Lewis1973b}; mientras
que en torno al fenómeno de la explicación, Friedman publicó
\citetitle{Friedman1974}. Otros esfuerzos fueron dedicados a
hacer claro si es que la causalidad juega un papel en la
explicación científica, y cuál es dicho papel, como hace
Kitcher en \citetitle{Kitcher1962}. Por el momento no voy a
entrar en detalles en torno a los títulos mencionados. Lo
que voy a hacer es concluir señalando algunas de las
ventajas del modelo de Salmon.

El modelo de Salmon tiene ventajas sobre el modelo por ley
de cobertura. El modelo resuelve el problema de caracterizar
a las \emph{leyes de la naturaleza,} porque no son
necesarias para ofrecer explicaciones. Esto implica además
que ofrece una mejor caracterización de la investigación
científica, porque puede caracterizar fácilmente a la
explicación en aquellas disciplinas en las cuáles no es
claro que haya \emph{leyes de la naturaleza}---como en
economía o biología. Pero si queremos resolver CA, el modelo
aún no es suficiente, pero Salmon apunta a que la solución
del problema está en involucrar relaciones causales. Esto,
por supuesto, tiene el costo de involucrar a la causalidad
en nuestra teoría de la explicación. A pesar de esto, muchos
filósofos han tratado de ofrecer una teoría de la
explicación científica que incluya un factor causal en su
análisis para resolver CA. Tal es el caso de James
Woodward.

\paragraph{James Woodward (\citeyear{Woodward2004})} 
ha desarrollado una teoría de la explicación científica que
hace exactamente esto: incluir relaciones causales. Además,
esta teoría tiene una serie de ventajas que describiré más
adelante. Para enfatizar dichas ventajas, comenzaré
exponiendo la teoría de Woodward. 

La teoría de Woodward---por razones que espero queden claras en
los siguientes párrafos---ha recibido el nombre de
\emph{manipulabilista}. Brevemente, las teorías
\emph{manipulabilistas}\footnote{
  La siguiente  caracterización está basada en
  \parencite{sep-causation-mani,
             sep-causal-explanation-science, 
             Woodward2004,
             Woodward2000-WOOEAI}. 
}
afirman que dadas dos variables $ A $ y $ B $, decimos que $
A $ \emph{causa} $ B $ si al intervenir (manipular) la
variable $ A $, la variable $ B $ cambia en consecuencia. Me
voy a permitir formular un ejemplo que, espero, haga claros
los detalles de esta teoría.

Asumamos que Miranda me pregunta por qué hoy llovió y estoy
tratando de formular una respuesta. Retomando las
descripciones $ D $ y $ D' $ definidas anteriormente, puedo
formular dos respuestas. Suponiendo $ D $, debería buscar
información de cómo se producen los vientos y si la presión
atmosférica está relacionada en su producción,\footnote{
  Resulta que es la termodinámica del ambiente jugando un
  papel más complejo. Lo que sucede es que vientos veloces
  distribuyen mejor el calor, lo que enfría el aire y lo
  vuelve más denso, lo que aumenta la presión atmosférica
  \parencite{Spiridonov2021}. Esto implica que la respuesta
  que voy a dar es incorrecta, pero esto no afecta los fines
  expositivos. 
}
después de tener la información pertinente, entonces podría
responderle a Miranda señalando que cuando la (A) presión
atmosférica \emph{aumenta}, \emph{produce} cambios en la (B)
velocidad del viento, lo que a su vez \emph{causa} que (C)
llueva. Las palabras enfatizadas en esta oración, son
palabras que intuitivamente relacionamos con procesos
causales: palancas, poleas y manivelas. Al ofrecer esta
respuesta, parece que asumo que hay un proceso causal
participando. Es decir que son variables que
considero\footnote{ 
  Insisto aquí en la palabra \emph{considero.} Esto quiere
  decir que no significa que \emph{sé} que dichas variables
  son causalmente relevantes, sino que \emph{especulo} que
  lo son.
} 
son causalmente	relevantes para dar una explicación de por
qué llovió hoy.

Si modeláramos el evento\footnote{
	Algo que saben hacer los meteorólogos profesionales. Yo
	carezco del conocimiento para hacer un modelo mínimamente
	informativo, por tanto, el ejemplo es de juguete. Lo
	cuál  no afecta los fines explicativos. 
} 
asumiendo que hay una relación \emph{proporcional} entre la
presión atmosférica y la velocidad del viento, de tal suerte
que un aumento de $1~\text{atm}$ en la presión atmosférica,
implica un aumento de $ 10~\text{km/h} $ en la velocidad del
viento; además, sucede que siempre llueve cada que la
velocidad del viento es mayor a los $ 85~\text{km/h} $, o
dicho de otro modo, \say{si aumentáramos arriba de
$9~\text{atm}$ unidades la presión atmosférica, entonces
llovería.}

Con el ejemplo anterior, sólo pretendo señalar la parte más
intuitiva de las teorías manipulabilistas, que ciertos
cambios en los valores de las variables provocan cambios en
los valores de otras variables. Es importante enfatizar que
este cambio debe ser sólo a través de una variable
particular. Es decir que podemos aislar todas las variables
distintas a la presión atmosférica, incluso si hacen una
diferencia causalmente relevante. Esto lo hacemos porque la
variable que esta bajo escrutinio es la aportación causal
que hace el cambio en la presión atmosférica, no otras
variables. 

Esto, por supuesto, sólo es una parte de la historia de las
teorías manipulabilistas. Esta teoría tiene fuertes
implicaciones en torno a cómo la teoría explica nuestra
búsqueda de relaciones causales. Me parece que el ejemplo
de la lluvia funciona para señalar algunas de estas
consecuencias. La teoría implica que las relaciones causales
son modulares, es decir, que en el sistema que distingue
adecuadamente entre $D$ y $D'$, somos capaces de
distinguir por completo todas las variables que hacen una
aportación causal. Digamos qué tanto aporte causal tiene la
presión atmosférica en $D$, esto es qué tanto aumenta la
velocidad del viento, mientras que en $D'$ nos preguntamos
que aporte causal tiene la presión atmosférica para que
llueva el día de hoy, en suma con el aporte causal de la
velocidad del viento. Ambas descripciones claramente
implican una diferencia en cómo tratamos a las variables
involucradas. Además, esta modularidad implica que podemos
distinguir claramente entre ambas descripciones y ajustar el
modelo en consecuencia, porque, dado que el sistema es
modular, podemos aislar las variables relevantes siempre
independientes de cualquier otra variable del sistema. El
problema resulta de que no todos los sistemas son modulares,
como argumenta Cartwright en \parencite{Cartwright2002}. Por
el momento dejaré la discusión de los problemas que tiene
la teoría manipulabilista, porque en la página~\pageref{pii}
voy a presentar un problema distinto al que esta teoría se
enfrenta y es el que me importa discutir en este proyecto de
investigación. Para fines explicativos, por ahora voy a formular
un ejemplo más manejable, intentando por el momento evitar
estas implicaciones.    

Olvidémonos por un momento de la
presión atmosférica y supongamos que el fenómeno a analizar
es uno completamente estable, es decir, que cualquier cambio
de valor en la variable $ A $, implica un cambio
\emph{proporcional} en $ B $. Si hay una función bien
definida sobre un conjunto, la tarea es ligeramente
sencilla. Definamos una función cualquiera de una variable
tal que $ f(x) = 2x $, donde $ x \in  \mathbb{N} $. Esto
significa que si la función toma el $ 9 $ como valor de
entrada, el valor de salida será $ 18 $. Supongamos más aún
que estos son los valores de una medida de fuerza y que la
relación es la siguiente: si la fuerza $ A $ es de 1 unidad,
entonces la distancia $ B $ trasladada de un proyectil es de
2.\footnote{ 
	Por supuesto esto es puramente ficticio, se
	trata sólo de un ejemplo ilustrativo, el punto es
	que la función de distancia es proporcional a la
	medida de fuerza.
} 
Digamos entonces que aplicar \emph{3 medidas de fuerza},
producen \emph{6 metros de distancia} de un proyectil. Dado
que la variable bajo escrutinio es el aporte causal de la
fuerza aplicada, supongamos además que esta función es
\emph{independiente} de cualquier fuerza externa: como  la
resistencia del medio, en este caso el aire, o el ángulo de
lanzamiento porque, dependiendo del ángulo, la gravedad
comienza a jugar un papel, etc. Si logramos satisfacer estas
condiciones, entonces tenemos un sistema modular, en el cual
podemos aislar la variable \say{fuerza,} de todas las demás
variables: hemos creado un entorno en el cuál todas las
variables se mantienen constantes y sólo manipulamos la
fuerza.

Un teórico manipulabilista diría que esto significa que la
fuerza impresa en el proyectil \emph{causa} que el proyectil
alcance más distancia. Además diría que es posible que la
función definida anteriormente no describa esta relación
para algunos de los valores de $ A $, pero no para otros.
Por ejemplo, si imprimo una fuerza de $ 1*10^{-7} $
unidades, seguramente el proyectil no se trasladará $
2*10^{-7} $ unidades.

El ejemplo anterior me sirve para describir algunas de las
características de la teoría \emph{manipulabilista.} Que la
función describa correctamente el comportamiento del
proyectil, \emph{independiente} de otras fuerzas, quiere
decir que la fuerza impresa en el proyectil es la única
variable causalmente relevante que afecta la distancia del
mismo. Que la \emph{intervención} en la fuerza impresa
cause que el proyectil llegue a mas
distancia---\emph{independiente} de cualquier otra
fuerza---significa que en la medida de los posible hemos
decidido dejar fuera del modelo a las variables
\emph{exógenas,} variables que sabemos pueden afectar el
comportamiento del modelo, pero al decidir concentrarnos en
la fuerza y la relación con la distancia, dejamos dichas
variables fuera. Si además podemos excluir todas las
variables, excepto la fuerza, entonces es posible hacer una
intervención \emph{quirúrgica}, porque podemos aislar esa
variable particular, de las variables de las cuáles a su vez
dependen y que sólo haya ciertos valores que describen
correctamente el comportamiento del proyectil, significa que
es \emph{invariante} bajo un rango de intervenciones. Es en
esto, a grandes rasgos, en lo que consiste una teoría
\emph{manipulabilista} de la causalidad.

El problema que he estado discutiendo hasta aquí es que las
relaciones causales son centrales para resolver CA. Si esto
es verdad, entonces una buena teoría de la explicación
científica \emph{debe} incluir a la explicación causal como
uno de sus componentes. Por supuesto, en lo personal, me
parece claro que incluir un factor causal en nuestra teoría
de la explicación es la mejor manera de resolver CA, y
espero que con lo dicho hasta aquí, el lector
también esté convencido de esto. Eso fue lo que
intenté mostrar con cómo se dio la dialéctica de la discusión
que nos llevó hasta este punto.

Ahora quiero hacer clara la conclusion a la que pretendo
llegar en la siguiente sección: me parece la teoría
\emph{manipulabilista} es la mejor manera de caracterizar a
la causalidad si vamos a resolver adecuadamente la
asimetría. Lo que pretendí lograr hasta aquí es que el
lector este convencido de que la causalidad es al menos una
salida que tenemos para resolver CA. Pero que la teoría
manipulabilista sea la mejor descripción de la causalidad es
algo que necesita un argumento ulterior. 

Es por ello que en los siguientes párrafos, me voy a detener
un poco para discutir algunas de las virtudes e
implicaciones de la teoría manipulabilista. La siguiente
discusión será somera\footnote{
  mi objetivo es llegar al tema de los \emph{condicionales
  contrafácticos,} que es el tema que me ocupa en este
  proyecto.  
}
y por ello el argumento que voy a presentar es genuinamente
débil. Si el siguiente argumento a favor de la teoría
manipulabilista tiene una forma, lo llamaría un argumento de
\emph{indispensabilidad metodológica,} en analogía con el
argumento de indispensabilidad ontológica de Quine. Lo que
pretendo es mostrar que es plausible asumir que la teoría
manipulabilista de la causalidad es la teoría que debemos
asumir si queremos una visión más parsimoniosa de la
explicación científica. Y si esto es verdad, entonces
tenemos que comprometernos con la teoría manipulabilista y
su caracterización de la causalidad. Los siguientes párrafos
estarán poblados de ejemplos de la literatura contemporánea
que asumen una teoría manipulabilista de la causalidad. Diré
que estos ejemplos de la adopción de la teoría y las
virtudes que tiene, hacen que pague su cuota metodológica y
que esto es suficiente para la conclusión que mencioné líneas arriba. Esto es importante porque a lo largo
de este trabajo voy a asumir que dicha teoría es correcta.

La siguiente sección procederá de la siguiente manera:
comenzaré despachando rápidamente dos de las razones
clásicas que se han esgrimido en contra de la teoría
manipulabilista: que el énfasis en la manipulabilidad hace
que la teoría sea \emph{antropocéntrica} y como tal, no es
una caracterización general de la causalidad. El otro
problema clásico es que el concepto de causa del cuál echa
mano la teoría es inevitablemente \emph{circular.} Después
de esto, señalaré por qué la adopción de la teoría en un
rango de de prácticas hace que pague su \emph{cuota
metodológica,} lo que implica que debemos aceptar su
caracterización de la causalidad. La parte final de la
siguiente sección la dedico a presentar un problema que, me
parece, es aún más grave para la teoría manipulabilista.


\section*{Parte II}
\label{pii}


\paragraph{La cuota metodológica} que tiene que pagar la
teoría manipulabilista queda saldada por su adopción
práctica, esto es lo que voy a defender en los siguientes
párrafos. Para hacer mi caso más contundente, quiero
comenzar mencionando que se han presentado argumentos
para mostrar las deficiencias de la teoría. Por un lado, se
ha argumentado que la noción de intervención (de la cuál
echa mano la teoría) es \emph{antropocéntrica} y, por otro
lado, que la teoría manipulabilista es inevitablemente
\emph{circular.} Ambas razones son mencionadas por Hausman
en \citetitle{Hausman1986}.	

Ahora bien, como no entraré en muchos detalles, me gustaría
rápidamente sortear ambas objeciones, para luego proceder a
mencionar algunas de las virtudes de la teoría. En torno al
problema de si la noción es \emph{antropocéntrica,} me
parece que sólo es una crítica espuria de una teoría por un
uso poco afortunado de la palabra \say{intervención.} Por
ejemplo, decimos que el rayo causó que se incendiara el
árbol, aún cuando es claro que no hay un ser humano
controlando el clima. Esto naturalmente se puede representar
en la teoría manipulabilista diciendo que \say{los rayos
causan incendios forestales} y determinar si durante
periodos de tormentas eléctricas, aumentan los casos de
incendios forestales. 

Si queremos trabajar con casos mucho más específicos,
relaciones causales entre variables que sólo existen porque
existen los seres humanos, tales como determinar si el
consumo de tabaco causa cáncer, o si las condiciones de
vivienda causan problemas en la salud de las personas,
primero deben existir las condiciones que lo permiten: deben
existir las tabacaleras y el ritual humano en torno al
consumo de cigarros, o las viviendas con diferentes
condiciones estructurales---cerca de la costa o en un clima
desértico---y diferentes condiciones materiales---si son
techos de lámina o si las paredes son de ladrillo. Para
estos casos, existen los \emph{experimentos naturales,} que
suceden cuando algún evento crea las condiciones aleatorias
que distinguen entre un grupo de control y un grupo de
tratamiento. Sucede que hay personas fumadoras y personas no
fumadoras. Nadie (espero) diseñó un experimento en el cuál
se le dio a un grupo de personas la tarea de fumar una
cajetilla al día durante un periodo largo de tiempo. Pero
podemos estudiar los efectos del cigarro a largo plazo
gracias a que hay personas fumadoras. En este caso no hay
una intervención humana, aún cuando son las condiciones
humanas las que permiten este tipo de escenarios. 

El segundo problema es el de la circularidad. Me queda claro
que el problema de la circularidad es más apremiante, pero
discutir el problema de si la causalidad puede \emph{ser
reducida} a conceptos más básicos es un problema que
requiere un trabajo completamente dedicado a ello, en donde
se discuta la \emph{aridad} de la relación causal (si es que
es una relación,) y cuáles son los \emph{relata:} eventos,
hechos, objetos (y si son \emph{types} o \emph{tokens}) y no
lo voy a hacer aquí. Voy a asumir que las relaciones son
variables-\emph{type}, por razones que quedarán más claras
en \ref{piii}, donde explico lo que
se ha llamado \emph{el problema fundamental de la inferencia
causal}

% Sin embargo, antes de termina, quisiera hacer un comentario
% general sobre la teoría manipulabilista de la causalidad. 

Dicho lo anterior, me parece claro que la
literatura contemporánea ha adoptado una postura
manipulabilista de la causalidad, no sólo en filosofía, sino
también en investigación científica. Potochnik, hablando en
particular de la causalidad, señala que debemos incluir factores causales si es que queremos una visión más
parsimoniosa de la ciencia, Potochnik nos dice que
\say{[$\ldots$] defiendo que la búsqueda de patrones
causales es la manera provechosa de entender a la
ciencia.}\footnote{
  En el original \say{[$\ldots$] I make the case that much
  of	science is profitably understood as the search for 
  causal patterns.}
}
\parencite[][p.~24]{Potochnik2017-POTIAT-3}

Mas importante es su claro endorsamiento de la teoría
manipulabilista, por ejemplo, en la página 29, Potochnik
señala que 


\begin{quote}
  A manipulability approach is appropriate for my purposes for
  two reasons. First it puts human action at the center stage.
  [$\ldots$] Second, for Woodward, causal relationships among
  types are fundamental and the basis for causal attributions
  in singular instances \parencite[p.~29]{Potochnik2017-POTIAT-3}
\end{quote}

Este \say{espíritu causal manipulabilista} no sólo aparece en la literatura
filosófica, sino también en libros de texto que discuten el
papel que juega la causalidad en la investigación
científica. Por ejemplo, cabe notar que en \parencite{Pearl2016},
los autores  inician señalando que la causalidad
\emph{debería} ser empleada en nuestras explicaciones
estadísticas, los autores comentan que:

	\begin{quote}
  	We study causation because we need to make sense of
	  data, to guide actions and policies, and to learn from
	  our success and failures. We need to estimate the effect
	  of smoking on lung cancer, of education on salaries, of
    carbon emissions on the climate. \parencite[p.~1]{Pearl2016}
	\end{quote}

Lo que significa que la causalidad es central en
los estudios de este tipo. En torno a la teoría
\emph{manipulabilista,} las autoras señalan, por ejemplo

	\begin{quote}
	  When we collect data on factors associated with
    wildfires in the west, we are actually searching for
    something we can intervene upon in order to decrease
    wildfire frequency. \parencite[p.~53]{Pearl2016}
	\end{quote}

Es cierto que no hay un consenso universal de qué papel
juega la causalidad en la explicación científica, si es que
juega papel alguno. Dicho de otro modo, si la causalidad
\emph{debe} ser parte de nuestras teorías de la
explicación científica. Sin embargo, con la discusión
anterior quiero sugerir que la causalidad es la mejor manera
de dar cuenta de CA y que, por razones metodológicas, la
teoría \emph{manipulabilista} de la causalidad es la mejor
forma de enmarcar a la causalidad y su papel en la
explicación científica. Por tanto la teoría ha pagado su
cuota metodológica. Si la causalidad es una característica
fundamental del proceso de explicación científica, entonces
la \emph{explicación causal,} en términos manipulabilistas,
merece un análisis filosófico.

A lo largo de los párrafos anteriores, he sugerido que la
teoría ha \say{pagado} su cuota metodológica. 
Sin embargo, a pesar de esto, la teoría 
manipulabilista padece de un problema más grave al cuál 
llamaré la \say{deuda normativa.} En la siguiente sección me
dedicaré describir el problema. La sección \ref{piii} la
dedico a señalar por qué el problema es importante y cómo este
trabajo de investigación se presenta como una respuesta a este problema. 

Para comenzar, voy a
discutir un trabajo reciente \citetitle{caushuman}, en el
cuál Woodward ha defendido el papel que juega su teoría en
el debate filosófico en torno a la explicación científica.
Me parece que el autor despacha muy rápidamente la cuestión
de como interpretar la normatividad de su teoría y, me
parece también, que la interpretación que ofrece Woodward no
es suficiente para resolver el problema al que se enfrenta.
La razón por la cuál no es suficiente es porque da paso a
una mala interpretación filosófica de la ciencia. Si lo que
el autor dice es verdad, tendremos conclusiones indeseables
en nuestra mejor teoría de la explicación científica.
Llamaré a esto el problema de la \emph{deuda
normativa.}

Para presentar claramente el problema, comienzo
caracterizando las tesis que sostiene
Woodward en \emph{Causation with a Human Face} y luego problematizo la
solución que ofrece el autor. 

\paragraph{Woodward en (\citeyear{caushuman})} comienza
retomando una serie de distinciones clásicas en filosofía,
en particular, me quiero concentrar en la distinción entre lo \emph{descriptivo} y lo
\emph{normativo}. Woodward apuesta a que la distinción es
espuria si entendemos que hay una brecha entre ambos
dominios. La respuesta del autor es que no existe
tal brecha y ofrece una serie de argumentos mediante los
cuales traza un puente entre lo normativo y lo descriptivo
señalando que ambos dominios deben estar influenciados
mutuamente.

Para caracterizar su teoría, debo aclarar a qué se refiere
el autor con \emph{lo normativo} y \emph{lo descriptivo}. La
respuesta corta es que entiende a lo descriptivo como la
investigación empírica en psicología sobre como razonamos
causalmente y a lo normativo con la investigación en torno a
como \emph{deberíamos} razonar para rastrear relaciones
causales. En palabras del autor:

\begin{quote} 
  $\ldots$ what I will call \emph{normative/theoretical}
  work on causal representation and inference, of a sort
  conducted in statistics, computer science, and philosophy,
  and, on the other hand \emph{descriptive/empirical} work
  on causal cognition of the sort conducted by
  psychologists, among others. \parencite[p.~15]{caushuman}
\end{quote}

El autor procede a señalar que una de las maneras que dicta
cómo deberíamos razonar, es su teoría manipulabilista. El
autor procede a rastrear la normatividad de su teoría en la
noción de \emph{invarianza,} y en la noción de
\emph{proporcionalidad.} En breve, la noción de
\emph{invarianza} en la teoría manipulabilista se refiere a
qué tanto una relación se mantiene bajo diferentes rangos
de valores, mientras que la noción de
\emph{proporcionalidad} se relaciona con si los cambios en
la variable $ A $ tienen siempre la misma cantidad de efecto
en la variable $ B $, esto es,  la misma proporcionalidad
bajo un rango suficientemente grande de valores. 


Regresando a mi ejemplo del proyectil y la función inventada
$ f(x) = 2x $, esta función describía la distancia
trasladada de un proyectil para cierto rango de valores de
fuerza. Es seguro que la función describirá correctamente la
distancia del proyectil bajo un rango de valores, además
bajo ese rango de valores será proporcional. Esto es verdad,
dado que estoy
asumiendo que la función está bien definida y es correcta.
Pero es completamente seguro de que será necesario definir
una función distinta para velocidades cercanas a la
velocidad de la luz, donde el espacio-tiempo se curva.
Entonces decimos que esta función es \emph{invariante} y
\emph{proporcional} para valores desde al menos desde el 1, hasta
valores siempre debajo de la velocidad de la luz. Por otra
parte, la función que describe el precio de la gasolina
durante los últimos 20 años, ajustado a la inflación, puede
ser verdadera para un rango de valores. Sin embargo, la
función no es
proporcional porque siempre hay que ajustar el porcentaje de
inflación de cada año (que fluctúa por año).

Decimos entonces  que la relación es \emph{invariante} bajo
el rango que correctamente describe la distancia en función
de la fuerza. En palabras del autor \say{La invarianza tiene
que ver con si las relaciones causales se mantienen
mientras varios cambios ocurren}\footnote{
    En el original \say{Invariance has to do, roughly, with
    the extent to which causal relationship continues to
    hold as various changes occur $\ldots$}
}
\parencite[p.~15]{caushuman} 

Ahora resta caracterizar una razón por la cuál la
\emph{invarianza} y la \emph{proporcionalidad} son
componentes suficientes para cumplir como componentes
normativos. Woodward comienza por señalar su único
compromiso metafísico: que las relaciones causales son
objetivas. Lo que el autor llama \emph{metafísica mínima}

 \begin{quote}
    Aknowledging that how we think about causal relations is
    shaped by human concerns does not mean that there is
    nothing in the world outside of this that is tracked by
    such relations or that these relations are \say{mind
    independent} in some problematic way
    \parencite[p.~12]{caushuman}
  \end{quote}


Woodward procede a señalar que los estudios empíricos en
psicología muestran que razonamos causalmente en términos de
manipulabilidad. La apuesta del autor es que como los
humanos razonamos por naturaleza tal como lo describe su
teoría, entonces esto es un argumento suficiente para cerrar
la brecha entre lo \emph{normativo} y lo \emph{descriptivo:}
cualquier información acerca de cómo razonamos debería
informar cómo deberíamos razonar. Esto nos presenta un
marco que se alinea con los objetivos de la teoría
manipulabilista. Woodward además pretende revelar cómo el
razonamiento causal se adapta evolutivamente para servir
fines pragmáticos, y esto incorpora al mismo tiempo
nuestras limitaciones epistémicas.

Si al lector no le parece sospechoso hasta este punto, lo
haré explícito: el argumento es circular. Al basar sus
conceptos normativos en hechos descriptivos sobre cómo
razonamos, Woodward se enfrenta a un problema: la
evaluación normativa se vuelve totalmente dependiente de los
rasgos descriptivos. Y estos rasgos descriptivos no ofrecen
una buena aproximación a cómo \emph{deberíamos} razonar,
porque la inferencia causal no es una herramienta infalible.
Y si asumimos lo que nos pide el autor, entonces tendremos
una muy mala imagen de lo que hacen los investigadores
cuando explican fenómenos. Para mostrar este punto, pensemos
en un ejemplo para hacer intuitivo a lo que me refiero. 

Supongamos que creemos que los test de IQ de hecho miden una
característica de los seres humanos. Supongamos además que
hay una alta correlación entre el nivel de IQ y los ingresos
mensuales: más nivel de IQ, más nivel de ingresos. Luego,
especulamos que hay una relación causal entre ambas
variables. Para mejorar los ingresos de la población,
entonces parece buena idea manipular el nivel de IQ y en
correspondencia vamos a lograr que más personas tengan
mejores ingresos. Siguiendo el análisis causal de Woodward,
esto implica que quien sea que esté llevando a cabo está
investigación debería buscar relaciones de
\emph{invarianza} y \emph{proporcionalidad} entre ambas
variables. Relaciones que de hecho existen entre dichas
variables. La conclusión debería ser entonces que el aumento
en nivel de IQ, causa que aumente el ingreso mensual.   

Espero que el lector esté en shock ante el ejemplo anterior.
El ejemplo anterior, lamentablemente, no es del todo
inventado, sino que está ligeramente basado en el que se
discute en \parencite{Lewontin2017}. Pero asumamos por un
momento que de hecho nos creemos que hay una relación causal
que cumple con \emph{invarianza} y \emph{proporcionalidad}
entre el IQ y el ingreso mensual. Entonces al variar el
nivel de IQ, \emph{debería} variar el ingreso. Ante esta
situación, donde la evidencia sugiere que en efecto existen
estas relaciones, ¿Qué debería hacer entonces un
investigador en este caso? Parece que si continuamos
asumiendo la teoría manipulabilista, lo que debería hacer un
investigador sería buscar intervenir en la variable IQ y
puede entonces buscar más relaciones causales que aseguren
la proporcionalidad y la invarianza. Tristemente esta fue la
idea que tuvieron muchos investigadores al tratar de trazar
el IQ con algún factor genético. Algo que resulta aberrante.
Pero esta es, me parece, la conclusión indeseable a la que
nos lleva el proyecto que intenta trazar Woodward entre lo
normativo y lo descriptivo. Por tanto, la teoría
manipulabilista no es normativamente adecuada, porque lleva
a resultados indeseables. Esta es la deuda normativa.


\paragraph{La deuda normativa} resulta de intentar hacer que
el proyecto normativo dependa del proyecto descriptivo en
torno a cómo razonamos causalmente. Woodward menciona que su
teoría no es la única en el mercado que puede trazar
adecuadamente criterios normativos, pero el problema que
describí, hasta donde alcanzo a ver, aplica para todas las
teorías que pretendan basar los criterios normativos en
torno a cómo razonamos. 

Siguiendo el ejemplo de Lewontin, de hecho sucede que la
variación en el IQ implica una variación en el ingreso
mensual. Pero la razón por la cuál existe esta relación no
se debe a que haya una relación causal entre variables. Sino
que se debe a cómo el test esta diseñado con sesgos claros:
que están diseñados para medir la comprensión de una persona
cuya lengua nativa es el inglés\footnote{ 
  Burt tradujo los test de Binet en Inglaterra y tenía
  influencias galtonianas: eugenesia.
} 
y que tiene cierto entorno social particular. Sucede además
que no hay evidencia de que el IQ tenga bases genéticas.
Para sorpresa de nadie, sucede que las diferencias en los 
resultados se debe a las condiciones sociales. En
breve: variar el IQ no implica una variación en los ingresos
mensuales. 

Una primer objeción que podría tener el autor sería que el
ejemplo anterior sólo muestra que la hipótesis original era
falsa y que, por lo tanto, no existe tal relación causal.
Debido a que no hay tal relación causal, desechamos la
hipótesis. En este sentido,  el ejemplo no es una excepción
a la regla, sino que la endorsa. Lo que sucede en este caso
es que no rastreamos una relación causal real.

La respuesta a esto radica en la segunda parte del ejemplo.
Desechamos la hipótesis de que hay una relación causal
\emph{real} entre ambas variables. Sin embargo, aún existe
la evidencia que sustenta que de hecho hay una relación, aún
si no es causal. Si un investigador sabe que existe dicha
evidencia ¿Qué debería hacer este investigador particular?
Asumamos que desecha la hipótesis, por lo que concluye que
no hay tal relación causal. Ante la evidencia de que hay una
correlación entre variables, o bien puede continuar su
búsqueda de patrones \emph{invariantes} y
\emph{proporcionales} en otro lugar, lo que nos lleva a la
conclusión aberrante: si la relación no está directamente
relacionada con el IQ, puede llegarse a pensar que el IQ es
una expresión de algo mas fundamental. Queremos evitar esta
conclusión, entonces la otra opción es, o bien buscar
otras relaciones \emph{normativamente adecuadas} que describa una teoría diferente del
razonamiento causal.


El primer cuerno echa por tierra la teoría manipulabilista
como una teoría normativa. Porque si como he descrito el
proyecto del autor, nuestras teorías de cómo de hecho
razonamos---la tarea descriptiva---y que seamos capaces de
rastrear cuáles son las relaciones causales que de hecho
existen en la naturaleza, a través de nuestras prácticas de
cómo deberíamos razonar---la tarea normativa---no tiene el
resultado armónico que el autor prometió.   

Es decir que si nuestro objetivo es buscar relaciones
\emph{invariantes} y \emph{proporcionales,} esto no será
suficiente para asegurarnos de que nuestras hipótesis
causales---que presumiblemente involucran relaciones
causales reales---sean verdaderas. Además, la teoría ni
siquiera ofrece una guía de qué deberíamos hacer en estos
casos. Esta segunda parte es importante, porque el autor en
efecto se compromete a una suerte de realismo causal. Lo
que lo compromete a ofrecer una caracterización de cómo nos
aseguramos que dichas relaciones causales existen y son
\say{objetivas.} El autor llama a este compromiso
\emph{metafísica mínima}, en sus palabras


\begin{quote}

  Moreover, if we ask \emph{why} these particular
  intervention-supporting relationships obtain, the
  explanations will be ordinary scientific explanations
  appealing to, for example, the chemical structure os
  aspirin and the way in which interferes with the enzymes
  that synthesize prostaglandins. We don't require in
  addition some variety of \say{metaphysical} explanation.
  This view amounts to what I will call \say{minimalist}
  metaphyisics or \say{minimal realism.}
  \parencite[p.~7]{caushuman} 

\end{quote}


Además, quiero recapitular el proyecto de Woodward usando
sus propias palabras, en torno a cómo influye el ámbito
normativo al ámbito descriptivo y viceversa. El autor
menciona que

  \begin{quote}
    When it comes to causation, my working assumption will
    be that \emph{how} we find about causal relations can
    tell us something about what those relationships
    involves. Conversely, a satisfactory account of what
    causal relations involve can help us to understand how
    our procedures for finding out about them succeed to the
    extent that they do. 
    \parencite[Énfasis en el original,~][p.~10]{caushuman} 
  \end{quote}


Por las razones mencionadas anteriormente, estos compromisos
no se satisfacen. El segundo cuerno del dilema es buscar una
teoría de la causalidad que defina criterios normativos
distintos a la teoría manipulabilista. Este segundo cuerno,
me parece, es suficiente para generalizar el problema. Como
dije anteriormente, el autor está al tanto de la pluralidad
de teorías de la causalidad, teorías que definen relaciones
diferentes a la \emph{invarianza} y \emph{proporcionalidad}.
Asumamos que una teoría de estas es correcta. Asumamos
además, como traté de motivar en la sección anterior, que la
explicación científica debe incluir patrones causales.
Siguiendo el argumento del autor, esta nueva teoría basada
en cómo de hecho razonamos, debería fungir como teoría
normativa. Pero si las normas se infieren de prácticas
cognitivas reales, ¿cómo pueden servir luego para evaluar
críticamente esas mismas prácticas? Esta circularidad es la
cruz de la normatividad que estos modelos persiguen. Por
tanto, mientras no se explique cómo los orígenes
descriptivos son suficientes para dar una base normativa, la
propuesta debería ser rechazada. Me parece que Psillos tiene
esto en mente cuando señala que:

\begin{quote}
  This step is a double-edged sword for Woodward’s account,
  since it takes the risk of his normative notions being
  entirely dependent on the descriptive features, thereby
  depriving them of their normative status. The grounding or
  justification of our evaluating norms in his account is a
  point that needs some special attention.
  \parencite[p.~305]{Manola2023}.
\end{quote}

El argumento de circularidad\footnote{
  Nótese que este problema de circularidad es distinto al
  mencionado en la sección donde discutí dos de los
  problemas clásicos contra la teoría manipulabilista. La
  circularidad involucrada aquí es metodológica: justificar
  nuestras normas de  investigación, la circularidad
  involucrada en la sección mencionada es si la causalidad
  puede ser definida en términos no causales. 
} 
que he ofrecido líneas arriba implica que podemos
generalizar el problema a otras teorías del razonamiento
causal, incluso asumiendo los criterios del autor. Aún
cuando haya a nuestra disposición diferentes teorías del
razonamiento causal, si lo que he descrito líneas
arriba---que me parece es el mismo problema que señalan
Manola \& Psillos---es correcto, entonces no podemos trazar
la normatividad de la causalidad en estos términos
descriptivos que Woodward presenta. 

El objetivo de esta sección fue comenzar ofreciendo una
somera defensa de la teoría manipulabilista en tanto que es
metodológicamente adecuada. Después de presentar esto, que
llame \emph{cuota metodológica,} me dediqué a presentar el
problema de la \emph{deuda normativa.} A pesar de que la
teoría padece este problema, me parece que es un error
simplemente desechar a la teoría manipulabilista. Además,
estoy de acuerdo con mucho de lo que señala Woodward en
torno al proyecto de trazar una ruta entre los dominios
\emph{normativo} y \emph{descriptivo.} Sin embargo, me
parece que el problema que he señalado es de suma
importancia, porque creo que no es un problema sólo para las
teorías manipulabilistas, sino de nuestra caracterización
de cuáles son los objetivos de la investigación científica.
Antes de dar razones a favor de la afirmación anterior,
quiero hacer explícito que---como dije lineas
arriba--asumir que la teoría manipulabilista es correcta,
estoy comprometido con tratar de resolver las consecuencias
indeseables que tiene la teoría. 

Si bien trataré de lidiar con estas consecuencias, primero
quiero hacer explícito que hasta este punto he evitado
discutir uno de los compromisos que están en el centro de la
teoría manipulabilista. El compromiso del que hablo es que
la teoría depende de una caracterización de los \emph{condicionales
contrafácticos.} Comeinzo la siguiente sección señalando
qué papel desempeñan los contrafácticos en la teoría
manipulabilista y, más general, el papel que juegan los
contrafácticos en nuestra teoría de la \emph{explicación
científica.} Al final diré por qué esta tesis se presenta
como una solución a la \emph{deuda normativa.}

% TODO:
  % Why we need and independent account of what causation is
  % and the aim of knowledge in scientific theorizing should
  % be the truth And if I can show that we indeed need that
  % conception, the my work is done. I will not discuss in
  % detail different theories of truth, what I will say
  % about the

  % And what the hell i'm going to achieve with that. That's
  % the question for which i still have no answer for (This
  % is what i'm doing right now 11-06-25). That we need to
  % still keep
  % a gap between normative and descriptive issues, because,
  % as I will argue, truth is the central aim of scientific
  % research and truth is a normative concept. 

\section*{Parte III}\label{piii}


\noindent Hasta este momento de la anécdota, me he
concentrado en repasar rápidamente algunas de las teorías de
la explicación clásicas en filosofía de la ciencia. Como
bien confesé al inicio, tengo mis intuiciones en una canasta
particular. Pero quiero llamar la atención en torno a que en
la última de las terorías expuestas, el tema viró hacia los
\emph{condicionales contrafácticos.} Porque Son este tipo de
condidionales y el papel que juegan en la investigación
científica, los que ocupan mi trabajo de investigación
actual.

que he se sin embargo, me parece que es falso que sólo podemos
entender la llamada \say{mind independence} al hablar de
objetividad, nos acercamos al terreno de la teología. Porque
autor nos presenta un dilema: o bien debemos negar los datos
empíricos que ofrecen evidencia de cómo razonamos, o bien
conceder los datos empíricos y afirmar como cuestión
normativa, que al hablar de causalidad tal como existe en el
mundo (o sobre lo que la causalidad \say{es}), estamos
apuntando hacia una caracterización de la causalidad que se
abstrae totalmente de dichas influencias humanas---una
caracterización de las relaciones causales, tal como dios
las describiría, o algo por el estilo. Dicho de otra manera, 
que al hablar de lo que es causalidad, nos acercamos
peligrosamente al terreno de la teología. 

Porque un claro error que comete Woodward es evitar hablar
de la evidencia empírica en torno a los sesgos de
razonamiento que como seres humanos cometemos a diario. Y,
por tanto, no podemos tomar como base normativa sólo la
evidencia psicológica, 

Estamos asumiendo que nuestras concepciones de cómo
descubrimos que ciertas relaciones causales se mantienen es
se debe en parte a nuestros intereses, objetivos y
limitaciones cognitivas

Woodward tiene muchos desiderata con los que concuerdo
totalmente. Que la evidencia psicológica debe informar cómo
debemos razonar. Que la empresa metafísica y la empresa
epistémica no deberían perseguirse una independiente de la
otra. También que esto no implica ning ún tipo de
subjetivismo o proyectivismo acerca de la causalidad
Causal claims are the sort of things that are true or false
, and which they are turns on how matters stand in nature
and not on such factors as our projective activities. 

The relationships between "is" and "ought" arise throughout
philosophy. I use empirical work on causal cognition to
illustrate some possible answers to this question, some of
which may also apply elsewhere

Primero que la verdad es la fuente normativa que estamos
buscando y que depende de cómo debemos justificar causal
claims Mientras que estamos informados acerca de cómo de
hecho nuestras prácticas nos informan sobre cómo debemos
justificar causal claims. Ok Woodward, me estás diciendo que
los "hacedores-de-verdad" de las relaciones causales deben
estar informadas por cómo de hecho los seresd humanos
hacemos inferencias causales. Digamos, la psicología de la
causalidad. 





\subsection{citas Psillos y Violetta Manola}

Sí, el capítulo 3 es el que me aventé ayer. 

n his fascinating new book, Causation with a Human Face (henceforth CHF),
James Woodward develops a novel idea about how to think about and study causa-
tion by starting from how we humans reason causally and then moving to accounting
for the normative aspects of causation, thereby offering a human-centered shift in
perspective on these issues. 


As we have noted, Woodward digs his favored “normative notions”, invariance
and proportionality, out of empirical work on human causal cognition, thereby
situating the source of these formalized norms into the descriptive/empirical.





enfatiza que
la teoría depende de los condicionales
\emph{contrafácticos}, por ejemplo, en la página 81,
Woodward nos dice que \say{Debería ser obvio que una teoría
intervencionista de la causalidad necesita ser formulada
en términos contrafácticos.}\footnote{ 
    En el original \say{That a plausible version of an interventioonist theory of causation needs to be
    formulated in terms of counterfactuals should be 
    obvious.}
}
 


\subsection{Deuda normativa}

Me parece que la discusión anterior es suficiente para
asumir que la teoría manipulabilista, al menos en términos
de criterios clásicos de lo que es que algo sea
indispensable\footnote{
  Véase la sección 2. \say{What is it to be Indispensable?}
  de \parencite{indis}
},
ha pagado su \say{cuota metodológica.} Pero algo que no he
mencionado explícitamente, es que una teoría de la
explicación intervencionista depende de una caracterización
de los condicionales contrafácticos. Lo cuál, siguiendo el
argumento de indispensabilidad, nos compromete con que
tenemos manera de \emph{determinar la verdad de dichos
condicionales}. Antes de continuar con los condicionales,
quiero hacer un comentario respecto al alcance de este
proyecto dentro de la discusión en torno a la discusión que
hizo Woodward recientemente.

Estos condicionales son aquellos condicionales, cuyo
antecedente evaluamos asumiendo que es \say{contrario a los
hechos.} Que algo que de \emph{hecho} sucede, no sucede en
absoluto (sé que la oración suena sintácticamente
incorrecta.) Son condicionales subjuntivos de la forma
\say{Si $  p $ hubiera sido el caso, entonces $ q $ sería el
caso.} Por ejemplo, nos obligan a evaluar qué condiciones de
verdad tiene la oración \say{si México no hubiera sido
conquistado por los españoles, entonces México tendría
cantidades exorbitantes de oro.}
 

  % pero quizás el problema más controversial---al menos el
  % problema que me interesa y del cuál trata mi
  % trabajo---es que la teoría hace uso de condicionales
  % \emph{contrafácticos.}

  % PERO DE NUEVO, PARECE QUE WOODWARD SÓLO está evitando el
  % tema. No debería, porque tiene consecuencias
  % indeseables. AAjustamos sobre la marcha es una frase que
  % deberías explicar, oscarcito. 


Parece más que natural implementar estos condicionales en la
teoría intervencionista de la causalidad. Una vez que hemos
definido las variables causalmente relevantes,---y debido a
que podemos incluir tanto variables discretas como variables
continuas---podríamos analizar condicionales del tipo
\say{si imprimiéramos 20 unidades de fuerza, en lugar de 10
unidades de fuerza, entonces el proyectil se hubiera
trasladado 40 metros.} Si la función está bien definida,
como en mi ejemplo, entonces es relativamente sencillo
determinar valores de distancia para diferentes valores de
fuerza.

Estos condicionales se han vuelto parte de los modelos
causales que suelen hacerse en investigación científica.
Por ejemplo, en modelos estadísticos, especialmente al hacer
inferencia causal, los condicionales \emph{contrafácticos}
aparecen como parte de este análisis. La  \emph{inferencia
causal} es una herramienta usada para derivar conclusiones
causales a partir de agregación de datos e inferencia
estadística. Estos métodos sirven, sobre todo, para
\emph{determinar} cuáles factores son causalmente relevantes
de manera tal que al intervenir en la variable $A$, la
variable $B$ cambie en consecuencia \parencite{Pearl2016,
Pearl2018, llaudet2023}. Pensemos, por ejemplo, en las
\emph{Pruebas Controladas Aleatorizadas} (RCT, por sus
siglas en inglés). En estas pruebas---típicamente
farmacéuticas---los investigadores interesados en saber cuál
es la efectividad de un medicamento, intervienen en una de
las variables---dándole un placebo a un grupo de
control---mientras que otro grupo recibe el medicamento a
testar. Asumiendo que no hay sesgos en la muestra, si el
grupo de prueba responde al medicamento, entonces (en
principio) seremos capaces de detectar si el medicamento
hace una aportación causal. Podemos pensar a este tipo de
pruebas, como sí el grupo de control funcionara como una
especie de imagen alterna de lo que \emph{habría} pasado de
no haber tomado los medicamentos.



De esta manera he consolidado mi intuición original de que 
la causalidad de hecho juega un papel en la
explicación científica. Pero hay otro factor importante que
resta discutir, qué papel juega la verdad en la
investigación. Porque, supongamos que hemos adquirido la
mejor manera de caracterizar a la explicación científica, lo
que en parte es un trabajo descriptivo, y ofrecemos una 
teoría de la explicación que describe adecuadamente el
proceso de \say{explicar científicamente} por qué $ A $
es $ B $. Tener esto no nos informa, ni asegura que las
explicaciones de hecho tienen algún tipo de valor
epistémico. Porque sabemos que muchas buenas explicaciones
se han ofrecido a lo largo de la historia de las disciplinas
científicas, que han resultado inútiles a la luz de nueva
evidencia. 


Lo que yo voy a hacer es trazar la normatividad de las
explicaciones causales en términos de los desiderata que
presenta Woodward. O sea, estoy pensando en lo giguiente.
Quiero mencionar que hay que mantener la distinción entre
cuestiones normativas vs. cuestiones descriptivas, y que
puedo describir una teoría de la verdad bajo los mismos
desiderata que presenta el autor. Es deseable mantener ambas
cuestiones separadas porque 

% Might be too compromising. Maybe is not a good idea to say
% that. At least as you have in mind right now. Because we
% do not have a clear normative criterion. I will argue that
% the good old truth concept does the job. And we do not
% have to look outside philosophical terrain to discuss and
% address these issues. 

% And we should care about that because if we do not track
% truth, then we're going to be prey of confusing models
% with the world. Which has a lot of implications that as
% researchers we do not want in our theories

se 

% there is a strong prima facie caase that it should be
% unserstandable given our conception of what causal
% relationships involve, how the kinds of evidence and
% strategies we employ to discover such relationships
% sometimes actually successfully lead to their discover

% Such a connection between the method employed in the
% experminet and the interpretation of the causal claim it
% establishes is one thing that we are looking for when we
% ask for a connection between the epistemology/methodology
% and metaphysics of causation. 


% Given such a failure of synchronization, bot our
% methodology and our conception of what is for causal
% claims to be true would be completely useless---the former
% because unreliable, the latter because, given this
% unreliability, we have no way of telling which causal
% claims are true.

% There is a tendency among philosophers to be rathe
% abstract and nonspecific wwhen ir comes to
% epistemology/methodology: it may be thouyght sufficient,
% for example, to say that causal claims are established by
% "our usual inductive practices" or inference to the best
% explanation," or by "making adjustments in our web of
% belief" without further specifying what these involve.

% TODO: I use the example of causal reasoning to explore
% these mopre general issues. First , there is the general
% question of how, if at all, empirical investigation into
% how people reason and behave might be relevant to the more
% "theoretical" or "conceptual" or "normative"
% investigations carried out by philosophers and others with
% similar interests. 

% Althought the contrast may not be sharp or clear one, I
% would distinguish the minimalism just described from more
% ambitious or expansive forms of metaphysical theorizing,
% which (it seems to me) go beyond what science tells us
% about nature or involve forms of explanation that are
% something other than scientific. 
% or clear one

% I said earlier that it is an important fact that we have
% procedures for learning about and reasoning with causal
% relations that are sometimes successful and that to
% characterize such success and to explain 

Este dilema es falso y es importante señalar por qué. Este trabajo está
dedicado a resolver el dilema. 



Pero me parece claro que los teóricos de la teoría causal
manipulabilista han evitado un problema crucial. Problema
que es parte fundamental de esta teoría filosófica de la
causalidad y que los mismos autores han reconocido en 
distintas ocasiones.

Woodward lo expresa mejor cuando señala que \say{There is
vry srtong empirical evidence that these conceptions are
influenced or shaped by features of our psychology and
situation in the world---including our epistemics
limitations, goals, and interests. (This is true, I claim,
of both ordinary lay thinking about causartion and causal
notions deployed in the various sciences.) Denying this
empirical facts is not a viable strategy. An alternative
possibility is to concede the empirical facts, but to claim,
as a normatiuve matter, that in talking about causation as
it exists in the world (or what causation \say{is}) we
should be aiming at an account of causation that completely
abstracts away from such human influences---an account of
causal relations as God would describe them or some such.
But I see noi reason to suppose that we have access to such
a godlike conception of causation.}


% This is the response of the question asked lines above
% Part of understandin causation as it exists in the world
% involves undesrtandig these categoriues and modes of
% thought and what work they do for us. 




\parencite{} %TODO: Aquí debes citar las partes pertinentes
             %      de Suárez y las de Potochnik y las de
             %      Woodward.




Es más que obvio que estos condicionales están en la caja de
herramientas de cualquier persona que se dedica a hacer
estadística en su trabajo y, por tanto, los condicionales
contrfácticos son un elemento de la \emph{investigación
científica}. Estos condicionales sirven para
\emph{justificar} relaciones causales, por ejemplo, pensemos
en la oración \say{comer dulces causa caries,} donde la
intuicón es la siguiente: \say{si una persona no hubiera
consumido dulces, entonces no tendría caries} y \say{si una
persona hubiera consumido dulces, entonces tendría caries.}

Pensemos en ejemplos donde aparecen herramientas de este
tipo, como en los métodos de \emph{inferencia causal.}
Herramientas como esta, son usadas para aislar factores
causales a partir de agregados de datos e inferencia
estadística. Los investigadores (creo que el mundo entero)
tienen un interés paraticular por aislar factores causales,
por ejemplo, para saber qué factores afectan causalmente el
desempeño académico de los estudiantes, o para intervenir en
factores que promuevan el aumento de los salarios y la
calidad de vida, etc. La inferencia causal es una
herramienta usada para \emph{modelar} fenómenos y a partir
de las relaciones entre las variables, descifrar si en
efecto hay una relación causal. Estos modelos hacen uso de
\emph{contrafácticos,} presentando un grupo de control---las
personas que no comen dulces,---en contraste con un grupo de
tratamiento---las personas que comen dulces---y evaluando
cuál grupo tiene más incidencia de caries en los dientes. En
principio, las personas en cada uno de los grupos tiene
diferentes características (no sólo el consumo de dulces,
sino también la higiene bucal, la deficiencia de calcio o la
edad, que también son características que propician la
aparición de caries,) es posible que en ambos grupos haya
personas que no siguen las instrucciones de sus dentistas y
su higiene bucal es precaria. Esto quiere decir
que---incluso si las personas del grupo de control no comen
dulces si tienen pésima higiene bucal, entonces propicia que
tengan caries. Incluso, si las personas que consumen altas
cantidaes de azúcar tienen una excelenter higiene bucal,
entonces algunas no tendran caries. Para asegurar que sólo
estamos evaluando la variable relevante (el consumo de
dulces,) tanto las personas en el grupo de control como el
grupo de tratamiento son seleccionados aleatoriamente de un
conjunto de la población (digamos que el conjunto completo
es la población total de una escuela primaria.)

Si sucede que las personas en el grupo de tratamiento tienen
más caries que las del grupo de control, entonces deberíamos
concluir que el consumo de dulces \emph{causa} que las
personas tengan caries. La variable en la que estamos
intervieniendo es el consumo de dulces. El amplio uso de
estos métodos, así como algunas de las aplicaciones que
tienen en la investigación científica es algo que queda
reflejado en \parencite{Pearl2016}, en \parencite{Pearl2018}
y especialmente en \parencite{llaudet2023}.

Sin embargo, el hecho de que los contrafácticos sean parte
de estos métodos y que su uso sea cotidiano en
investigación, no significa que estén resueltos todos los
problemas filosóficos que acarrean. En este trabajo de
investigación, me voy a dedicar a discutir el problema que
ha recibido el nombre de \say{el problema fundamental de la
inferencia causal.} \parencite{Holland01121986} Dicho en una
oración: que no es posible saber el valor de verdad de nico
que realmente podemos observar son los resultados actuales,
no los resultados contrafácticos. Si por definición del
condicional, estamos evaluando algo que no sucede, pero que
pudo haber sucedido: ¿exactamente qué evidencia podemos
ofrecer para justificar condicionales como \say{si los
españoles no hubieran conquistado México, entonces
conservaríamos todas nuestras reservas de oro}?

Esto, por supuesto, es un problema relacionado con la
epistemología de la modalidad. Para ejemplificar el
problema, supongamos que queremos saber si los estudiantes
que leen más de un libro al año tienen un mejor desempeño
académico que aquellos alumnos que leen menos de un libro al
año. Supongamos, más aún, que hemos diseñado una prueba
aleatoria randomizada y que hemos aislado existrosamente la
variable \say{leer más de un libro al año}, que en la
encuesta probablemente midamos con el con la pregunta proxy
\say{¿lee usted más de un libro al año? }

Ahora, para descifrar qué hubiera sucedido si los alumnos no
leyeran al menos un libro al año, debemos evaluar el valor
de verdad del enunciado \say{si un alumno no hubiera leído
más de un libro al año, entonces su desempeño académico
sería peor.} Prima facie, esto significa que tenemos que
evaluar el desempeño del grupo de control, es decir: aquel
grupo de personas que leyó menos de un libro al año. Sin
embargo, la única información que tenemos es el agregado
estadístico de los alumnos que leyeron menos de un libro al
año, lo que implica que nunca seremos capaces de observar,
digamos, qué hubiera sido de Ana si hbiera leído menos de un
libro al año; ni siquiera seremos capaces de observar que
hubiera sido de \emph{cualquier} alumno dentro del grupo de
control: no podemos hacer que Ana lea más de un libro al
año, rebobinar, luego incentivar a Ana a leer menos y
observar en resultado en ambos casos.

Si somos personas trabajando en una hipótesis para saber
cómo la lectura afecta el desempeño de los estudiantes,
estas estadísticas sólo muestran que los alumnos que leyeron
más de un libro al año tuvieron un mejor desempeño académico
que aquellos alumnos que no lo hicieron. Pero esto no es
suficiente para justificar que la directora de la escuela le
recomiende a Ana leer más de un libro al año porque
\emph{mejoraría el desempeño académico de Ana:} descrito de
esta manera, es al menos físicamente imposible observar a
Ana leyendo al menos un libro al año, y luego observarla
haciendo exactamente lo contrario.

Entonces, ¿cómo sabemos que este modelo implica
correctamente que incentivar el hábito de la lectura genera
un mejor desempeño académico? Si no es posible observar
casos contrafácticos, significa que hemos perdido una
explicación causal de por qué Ana Sandoval tuvo un mejor
desempeño que Jorge Monreal. Simplemente porque no podemos
observar que habría sido de Ana si hubiera leído menos de un
libro al año, o bien, qué hubiera sido de Jorge si hubiera
leído al menos un libro al año.

Esto implica un problema acerca del significado o valor de
verdad de condicionales contrafácticos que tienen como
argumento un caso particular y no un agregado de datos:
podemos hacer inferencias de cómo el grupo de prueba y elm
grupo de control difieren, y podemos determinar si esto
implica una relación causal entre las variables \emph{leer}
y \emph{desempeño académico,} pero no una relación causal
entre la variable leer y un alumno particular. Resta además
señalar exactamente cuál es el papel que juegan estas
variables y cómo. Por ejemplo, es completamente plausible
que \emph{obligar} a los estudiantes tenga una efectividad
diferente a \emph{incentivar} a los estudiantesa leer. Por
supuesto, quienes usan este tipo de métodos en su
investigación tienen presente que con esto hemos perdido
causas particulares.

Por el momento, no voy a entrar en detalles sobre qué son y
cual es la naturaleza los condicionales contrafácticos,
porque eso es lo que pretendo hacer en mi proyecto, por
ahora sólo basta mencionar que los condicionales
contrafácticos encuentran su aplicación en estos contextos
para, digamos \say{crear} un evento alterno\footnote{
	Estoy voluntariamente evitando decir \emph{mundo posible.}
} 
(si asumimos que no hay sesgos en los grupos de control y
de prueba, y que hemos eliminado causas comunes dentro de
nuestro análisis), en principio, seremos capaces de derivar
una conclusión causal \cite{Otsuka2023}.


\paragraph{No voy a hacer una defensa} de la teoría
\emph{manipulabilista} de la causalidad. Eso lo hice en mi
proyecto de maestría---sea cual sea el éxito que haya tenido
esta defensa---en particular, intenté elucidar cómo una
teoría manipulabilista lidia con explicaciones causales en
biología evolutiva. Mi proyecto actual nace de
preocupaciones que restaron de ese proyecto. En particular,
si es verdad que en la investigación científica se hace uso
de condicionales contrafácticos, entonces vale la pena
preguntarse (i) \emph{¿cómo sabemos qué condicionales
contrafácticos son \say{verdaderos}?}

Hay mucho que desempaquetar en la pregunta (i). Primero, la
pregunta es parte de un condicional, esto es intencional. De
este condicional quiero extraer la conclusión de que de
hecho vale la pena esa pregunta. Presentar este condicional
es parte de la motivación del proyecto. Hasta ahora he
mencionado las aplicaciones que hacen mención de
contrafácticos, lo cual ofrece una motivación para preguntar
(i). Supongamos por un momento que dichos condicionales de
hecho son parte de la investigación---a estas alturas el
lector debería estar de acuerdo con esto. Ahora, responder a
(i) equivale a explicar por qué tiene sentido determinar los
valores de verdad de dichos condiocionales. Como estos
condicionales son parte de las herramientas de
\emph{modelado} en ciencias, mi pregunta está enmarcada en
términos de cómo dichos modelos representan el mundo.

Si asumimos que los modelos tienen condiciones de verdad, y
que los condicionales contrafácticos son parte de las
herramientas de modelado, mi proyecto se concentra en
resolver por qué \emph{la verdad} juega un papel importante
en la investigación científica.  y por qué el asunto de so
si  y luego en cómo \emph{justificar} qué condicionales
contrafácticos cumplen este criterio. Mi proyecto se
concentrará en el contexto de las inferencias causales, y
cómo echan mano de \emph{condicionales contrafácticos.} Si
son estos condicionales los que nos permiten trazar
relaciones causales según la teoría \emph{manipulabilista},
entonces más nos vale saber cómo determinar el valor de
verdad. Una motivación que se suele mencionar en la
literatura para introducir estos métodos es la llamada
\say{paradoja de Simpson. } Esto no es estrictamente una
paradoja, pero aparece cuando los datos se comportan de
manera poco intuitiva \parencite[p.~13]{Hajek2016-HAJOHO},
Algunos autores han señalado que la solución radica en ser
capaces de ofrecer un método para obtener relaciones
causales a partir de datos estadísticos.

Este tema --el de la explicación causal-- está fuertemente
vinculado con el uso de herramientas y \emph{modelos}
estadísticos en investigación. Mi propósito en este trabajo
es hacer claro cómo funcionan dichos modelos y qué
herramientas nos sirven para justificar que en efecto dos
variables están causalmente relacionadas. Esto es
importante, porque en muchos casos se suele decir que los
\emph{modelos} usados en investigación, sea cual sea su uso,
son \say{falsos}, sentido que suele expresar como
\say{[$\ldots$] models typically only hold \emph{approximately,}
in some ranges of circumstances, and they liberally employ
idealizations to accomplish this \emph{partial} fit.} 
\parencite[p.~18, énfasis agregado]{Potochnik2017-POTIAT-3}
Potochnik también afirma que buena parte de la investigación se
centra en ofrecer explicaciones causales y, más aún, afirma que
una buena manera de lidiar con explicaciones causales en 
investigación científica es adoptando una teoría causal
\emph{manipulabilista,} algo con lo que estoy completamente
de acuerdo, como espero haya quedado claro. Potochnik
también defiende que el uso de modelos en investigación---y
cualquier supuesto que imponga ciertas condiciones en el
modelo---surge de la necesidad de que seres con capacidades
cognitivas limitadas queremos representar fenómenos
naturales para aislar los factores causales bajo escrutinio.
Es importante poner énfasis en esto porque implica que los
modelos \emph{son generados} con ciertos objetivos en mente,
o como ella misma señala \say{[$\ldots$] the nature of these
idealizations is relative to the aim of research, as is most
clearly demonstrated by the different idealizations involved in
the various approaches to human aggression research. }

Mi investigación actual lidia con cómo \emph{determinar} el
valor de verdad de condicionales contrafácticos, usados para
formular hipótesis estadísticas, hipótesis que se justifican
haciendo uso de modelos---a veces modelos de inferencia
causal, o de otro tipo---mientras afirmamos que los
modelos son falsos. En principio, este no es un problema
grave si la \say{verdad} no es una parte central de la
investigación científica (en cualquier disciplina).

Pero si la investigación se propone \emph{justificar}
hipótesis para obtener \emph{conocimiento} de relaciones
causales reales, entonces tenemos un problema. ¿Cómo los
modelos, que no representan adecuadamente el fenómeno,
pueden dar como resultado nuevo conocimiento? El problema,
me parece, radica enla discusión en torno a la
representación. La relación que guarda el modelo con el
objetivo que pretende representar.

Mi hipótesis es que cierta forma de verdad debe estar
involucrada en nuestra definición de \emph{conocimiento,} si
es que queremos que las intervenciones en un sistema sean de
alguna utilidad. Por ejemplo, si el aumento del salario
mínimo causa un aumento en el desempleo y qué condiciones
juegan un papel causal en dicha relación.

\paragraph{Regresando a la anécdota,} el hecho de que estos
dos profesores buenos trabajaran filosofía de la ciencia---al 
menos lo que en ese momento creía que era la filosofía de la
ciencia---sesgó mis intereses a lo que hasta ahora ha sido mi
trabajo. La influencia de estos dos profesores me hizo tomar
la decisión de entrar al posgrado en Filosofía de la Ciencia.

En las clases de maestría hubo profesores excelentes. Mucho de lo que
aprendí en estos cursos me sirvió para enmarcar de diferente manera
las preguntas que me preocupaban, lo cuál me permitió entender
--mejor, me parece-- la naturaleza de la investigación científica. En
particular quería responder preguntas sobre la epistemología de la
ciencia, preguntas del tipo: \say{¿qué estamos justificados a
	creer?}, \say{¿cuándo podemos afirmar que una hipótesis ha sido
	corroborada? }, \say{¿es la explicación o compresión más básica que
	el conocimiento?}, etc. Preguntas que surgen a partir de mi interés
por la explicación causal.

Ahondar en estas preguntas y tratar de responderlas es una tarea más
que complicada. Muchos de los obstáculos se deben a que la práctica
científica es más caótica y heterogénea de lo que parece en un
principio. Incluso si nos concentramos en el uso de herramientas
estadísticas o herramientas que echan mano de la teoría de la
probabilidad, en casi todos los casos, ciertas propiedades del
\emph{fenómeno en cuestión} son idealizadas o ignoradas para hacer
que el modelo estadístico sea de utilidad.

Esta representación parcial del fenómeno se debe a que en muchos
casos buscamos destacar algunas propiedades particulares del
fenómeno, sin que todas las variables estén representadas en el
modelo. Que no involucremos todas y cada una de las variables que
afectan el fenómeno es un reflejo de nuestra limitada capacidad
cognitiva. A lo largo de su escrito, Potochnik nos recuerda que los
modelos diseñados en la investigación científica tienen un propósito
particular. Son usados para destacar algunas propiedades
particulares, especialmente, especialmente si lo que nos interesa son
las relaciones causales. Que simplifiquemos de ciertas manera el
modelo se debe a que los investigadores intentan resaltar las
características de un fenómeno en un momento particular.

Potochnik no es la única que ha señalado esto,
\textcite[][p.24]{abrams2023evolution}, por ejemplo, nos recuerda que
\say{So-called infinite-population models are simply models that have
	no role for drift. Biologists do often say things about evolution in
	infinite populations, and these claims are usually correct: the role
	of this terminology \emph{in practice} implies that it should not be
	understood literally. } (énfasis agregado)

A pesar de que los modelos no son completamente adecuados, --en el
sentido en el que no representan totalmente el fenómeno en
cuestión,-- son ampliamente utilizados en la investigación
científica. Si estas herramientas son usadas ampliamente, entonces
como filósofos de la ciencia hay que ofrecer un análisis de cómo
exactamente los modelos sirven para justificar hipótesis si parten de
supuestos obviamente falsos.

Me parece que buena parte del problema está relacionado con
determinar cuál es el papel que juega la \emph{representación
	adecuada} y la \emph{verdad} en la investigación científica. Para
motivar mi afirmación anterior, quiero dar un repaso por un periodo
de la filosofía de la ciencia, un periodo en el cuál --suele
decirse-- la \emph{verdad} jugaba un papel central en los análisis
filosóficos de la ciencia: me refiero al periodo en el que los
miembros del Círculo de Vienna estuvieron vivos. Este repaso sirve
para motivar la afirmación de que la \emph{verdad} juega un papel en
la investigación científica --por el momento no asumiré cuál es ese
papel,-- mientras que esto es al mismo tiempo compatible con la
\say{representación parcial} que hacen los modelos de un fenómeno.

Para comenzar este repaso, debo señalar que en la maestría aprendí
nuevas metodologías de investigación, además de diferentes maneras de
plantear y entender las preguntas que me preocupaban. Pero lo más
valioso que aprendí fue la importancia que tiene la historia de la
ciencia en la filosofía de la ciencia. Me voy a permitir hacer una
breve caracterización, exageradamente general, de dos posturas en
historia de la ciencia.

En la sección 4 de su artículo, Salmon motiva su discusión al tratar
de resolver el problema del caso único. Este problema parece funesto
para la interpretación frecuentista de la probabilidad (que es la
interpretación que Salmon favorece a lo largo de su artículo) y su
uso en explicación. Salmon cree que no es así. Para justificar esto,
desarrolla un aparato teórico para tratar con dicho problema. Una
parte importante de De acuerdo con Salmon, explicar un fenómeno
consiste en detectar las diferentes variables ---estadísticamente
relevantes--- para que un fenómeno ocurra.

Ese es el tema del siguiente capítulo: qué papel juega la
verdad en investigación científica



https://plato.stanford.edu/entries/causation-mani/
