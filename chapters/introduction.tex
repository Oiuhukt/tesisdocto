% !TEX root = ../main.tex
% LTeX: language=es, en

\chapter{Introducción}
%\label{ch:introduction}

\section{Una breve anécdota}

\noindent El doctorado ha sido un largo viaje durante el cuál
aprendí mucho. La parte más importante del aprendizaje está
relacionado con la filosofía,\sidenote{Juro que este breve relato
tiene un punto.} y la alta calidad del posgrado me permitió profundizar en los temas que me interesaban. Mi pasión en ellos nació durante mi licenciatura. Me fascinaron los temas que vimos en clase de  filosofía de la ciencia, en particular, debido a que porque durante este periodo tuve dos profesores que se dedicaban a la filosofía de la ciencia\sidenote{Lo que en ese momento pensaba que era la filosofía de la ciencia}. Ambos excelentes maestros y filósofos.

Mi atención la dirigí especialmente a los temas de la
\emph{explicación científica} y la \emph{naturaleza de la
causalidad}. Mis intuiciones --todavía queda un resabio de esto--
sugerían que encontrar la causa de un fenómeno es la mejor manera de
explicarlo. Por ejemplo, la mejor explicación de por qué tengo cáncer
de pulmón se debe a mis hábitos de fumador, y fumar es una causa de
cáncer.

Esta intuición no está muy alejada de algunas teorías clásicas de la
\textit{explicación científica}. La motivación más clara para
incluir una relación causal en nuestras teorías de la explicación
científicas se debe a que la explicación de un fenómeno es
\textit{asimétrica}, algo que Sylvain Bromberger hizo notar en su
artículo clásico "Why-Questions" \cite{Bromberger1966-BROW-2}.
Para presentar el ejemplo de manera clara, primero quiero
caracterizar brevemente lo que Hempel y Oppenheim dicen en torno a
la explicación científica\sidenote{Estoy utilizando \emph{explicación científica} como un término técnico. En este texto,  uso el término para referirme a lo que hacen las personas que se dedican a la investigación en cualquier área del conocimiento, cuando decimos que estas personas \emph{explican} un fenómeno.}.

\paragraph{Hempel y Oppenheim \citeyear{Hempel1948}} caracterizaron
a la explicación científica como un argumento deductivo. El argumento
debe tener como premisa una ley general, de la cuál podamos derivar
el fenómeno en cuestión. Sabemos, por ejemplo, que todos los metales
se dilatan cuando se calienta. Si calentamos un pedazo de cobre,
entonces debería dilatarse, porque pertenece a las  pedazo
particular de cobre, entonces se dilatará. Esto explica por qué este
pedazo de cobre se dilata cuando lo caliento.

Sin embargo, Bromberger \citeyear{Bromberger1950} detecta un problema con esta caracterización
de la explicación científica. Bromberger comienza argumentando que
hay muchas similitudes entre explicar que este pedazo de cobre se
dilata y responder a la pregunta "¿por qué este pedazo de cobre se
dilata cuando lo caliento?" El problema con lo que Bormberger llama
"La doctrina hempleiana" es que sólo puede responder a un conjunto
de preguntas. Dicho de otro modo, explica algunas de las
características de la explicación científica, pero no todas.

El ejemplo de Bormberger es el siguiente: supongamos que hay un
punto en quinta avenida que está a $x$ distancia del Empire State.
Dadas las leyes de la óptica y el ángulo que forma el rayo de sol
desde el punto más alto del edificio hasta el punto $x$, podemos
derivar fácilmente la altura del Empire State. A pesar de que este
es un argumento deductivo que se adhiere a la caracterización
hempeliana, nadie diría que el hecho de que conozcamos la altura del
edificio explica por qué el edificio tiene la altura que tiene.

Si lo que señala Bromberger es verdad, entonces las condiciones de Hempel no son suficientes para caracterizar el fenómeno de la explicación científica. Pero este no es el único problema con la teoría, sino que además no recupera la asimetría que solemos esperar de la explicación científica, algo que Wesley Salmon discute \citeyear{Salmon1970}.

Discutiendo el ejemplo de Bormberger, Salmon menciona que la asimetría temporal es crucial para explicar la altura del edificio, en la página 72, Salmon dice que

    \begin{quote}
    Although the sun, flagpole, and shadow are perhaps 
    commonsensically regarded as simultaneous, a more sophisticated 
    analysis shows that physical processes going on in time are 
    involved. Photons are emitted by the sun, they travel to the 
    vicinity of the flagpole where some are absorbed and some are 
    not, and those which are not go on to illuminate the ground
    \end{quote}

En breve, la afirmación principal que defiende Salmon es que si una
variable hace una diferencia en el resultado esperado, entonces
tenemos una explicación del resultado. Salmon reconoce, sin embargo,
que su teoría de la explicación científica es sensible al problema
de hacer pasar las correlaciones por relaciones de dependencia. En
estadística es común el eslogan \say{correlación no implica
causalidad} y el problema que señala el autor es que si sólo
caracterizamos a la explicación como estadísticamente relevante,
entonces vamos a perder casos claros de explicación científica. Por
ejemplo, si tengo un dolor de cabeza debido a que no he tomado agua, 
y sin darme cuenta me tomo una pastilla del frasco --que resulta que 
es un dulce--, no puedo afirmar que fue la pastilla la que me quitó 
el dolor de cabeza. En este caso fue el agua, que es el factor 
causal relevante.

Para resolver esto, Salmon hace notar que su modelo de explicación
puede naturalmente incorporar \emph{explicaciones causales}, que es
un requerimento para resolver el problema de la asimetría de la
explicación, Salmon menciona, por ejemplo que 

    \begin{quote}
    The reason that the explanation of the length of the shadow in
    terms of the height of the flagpole is acceptable, whereas the
    \say{explanation} of the height of the flagpole in terms of the
    length of the shadow is not acceptable, seems to me to hinge
    directly upon the fact that there are causal processes with
    earlier and later temporal stages.
    \end{quote}


Podemos encontrar en la literatura contemporánea, son destacables
las teorías de la explicación científica que incluyen factores
causales. Por ejemplo, Potochnik nos dice que \say{[$\ldots$] I make
the case that much of science is profitably understood as the search
for causal patterns.} \parencite[][p.~24]{Potochnik2017-POTIAT-3}

Este \say{espíritu causal} nota incluso en libros de texto. Por ejemplo, cabe notar que en \parencite{Pearl2016}, los autores inician señalando que la causalidad \emph{debería} ser empleada en nuestras explicaciones estadísticas, en la página 1, los autores comentan

    \begin{quote}
    We study causation because we need to make sense of data, to 
    guide actions and policies, and to learn from our success and 
    failures. We need to estimate the effect of smoking on lung 
    cancer, of education on salaries, of carbon emissions on the 
    climate.
    \end{quote}

Un poco más adelante, nos dicen que

    \begin{quote}
    When approached rigorously, causation is not merely an aspect of
    statistics; it is an addition to statistics, an enrichment that
    allows statistics to uncover workings of the world that 
    traditional methods alone cannot.
    \end{quote}

Dentro de las teorías de la explicación que involucran un componente causal, ha emergido la teoría manipulabilista de la explicación causal
    

    

Potochnik es una filósofa que afirma que son las teorías \emph
{manipulabilistas} \parencite{sep-causation-mani} de la causalidad
las que mejor vinculan a la práctica de la investigación científica
con la búsqueda de patrones causales.

Si bien no hay un consenso universal sobre la relevancia de la
causalidad en el fenómeno de la explicación científica, con la
discusión anterior quiero sugerir que la causalidad
es una característica fundamental del proceso de explicación 
científica.

\paragraph{Brevemente, las teorías \emph{manipulabilistas}\sidenote{
	La siguiente caracterización está basada en \parencite{sep-causation-mani, sep-causal-explanation-science, Woodward2004, Woodward2000-WOOEAI}
	} afirman} que dadas dos variables $A$ y $B$, decimos que $A$ \emph{causa} $B$ si al intervenir (manipular) la variable $A$, la variable $B$ cambia en consecuencia.
Esto, por supuesto sólo es una parte de la historia de las teorías
\emph{manipulabilistas}. Si hay una función bien definida sobre un
conjunto, la tarea es ligeramente sencilla. Pensemos, por ejemplo,
en una función cualquiera de una variable, digamos que la función
está definida de la siguiente manera $f(x) = 2x$, donde $x \in \mathbb
{N}$. Esto significa que si la función toma el $3$ como valor de
entrada, el valor de salida será $6$. Supongamos que estos son los
valores que puede tomar una medida de fuerza.

Digamos que \emph{dos centímetros} de presión en un resorte producen
\emph{cuatro metros} de distancia de un proyectil, \emph{independiente
} de cualquier fuerza externa, por ejemplo, la fuerza del aire
--insisto en que esto es sólo un caso para ejemplificar.

Esto significaría que la compresión del resorte \emph{causa} que el proyectil alcance más distancia.
Además es posible que la función definida anteriormente no sea correcta para algunos de los valores de $x$, por ejemplo, si el resorte sólo mide $8$ cm, llegará un momento en que el resorte esté completamente comprimido como para que más compresión haga una diferencia.

El ejemplo anterior sirve para describir algunas de las características de la teoría \emph{manipulabilista.}
Que la función describa correctamente el comportamiento del proyectil, \emph{independiente} de otra fuerza, quiere decir que la compresión en el resorte es la única variable que afecta el traslado del proyectil.

Que la \emph{intervención} en el resorte cause que el proyectil llegue a mas distancia, \emph{independiente} de cualquier otra fuerza, significa que es posible hacer una intervención \say{quirúrgica} y que sólo haya ciertos valores que describen correctamente el comportamiento del proyectil, significa que es \emph{invariante} bajo un rango de intervenciones.
En esto, a grandes rasgos, consiste una teoría \emph{manipulabilista} de la causalidad.

Hay otras características integran la teoría \emph{manipulabilista} de la causalidad.
Quizás la más controversial de ellas es que la teoría hace uso de condicionales \emph{contrafácticos.}
Estos condicionales sirven para responder preguntas del tipo \say{¿qué hubiera sucedido si $x$ hubiera tomado un valor distinto al que tomó?}
En investigación, se suele mencionar el uso de condicionales \emph{contrafácticos}, principalmente en inferencias causales.

Si tomamos en serio el trabajo que realizan las personas cuando hacen investigación, es claro que el uso de estas herramientas juega un papel central y, de manera particular, los modelos estadísticos hacen uso de condicionales \emph{contrafácticos.}
Los condicionales \emph{contrafácticos} son aquellos condicionales que evaluamos bajo el supuesto de que algo no sucede, pero que nos obligan a preguntarnos que pasaría si se cumplieran ciertas condiciones particulares.
Dicho de otro modo, nos preguntamos cómo sería el mundo si, por ejemplo, México no hubiera sido conquistado por los españoles.

Estos métodos sirven, sobre todo, cuando utilizamos modelos estadísticos para justificar hipótesis causales \parencite{Pearl2016, Pearl2018}.
Un ejemplo de esto son las \emph{Pruebas Controladas Aleatorizadas} (RCT, por sus siglas en inglés), donde se interviene en una de las variables --típicamente un grupo de control al que se le da un placebo,-- mientras que otro grupo recibe el medicamento a testar.
Este esperando que seamos capaces de detectar si el medicamento está jugando un papel causal.
Los condicionales contrafácticos encuentran su aplicación en estos contextos para \say{crear} un evento alterno.

\paragraph{No voy a hacer una defensa} de la teoría \emph{manipulabilista} de la causalidad.
Eso lo hice en mi proyecto de maestría, en particular tratando de elucidar cómo esta teoría lidia con explicaciones causales en biología evolutiva --sea cual sea el éxito que haya tenido esta defensa,-- con
mi proyecto actual quiero hacer claro cómo sabemos qué condicionales \emph{contrafácticos} son \say{verdaderos}.

Parece que queda claro que herramientas estadísticas, en el contexto de las inferencias causales, echan mano de \emph{condicionales contrafácticos.}
Además, son estos condicionales, los que nos permiten trazar relaciones causales según la teoría \emph{manipulabilista}
Una motivación que se ha mencionado en la literatura para introducir métodos que nos permitan trazar relaciones causales es la llamada \say{paradoja de Simpson.}
Esto no es estrictamente una paradoja, pero aparece cuando los datos se comportan de manera poco intuitiva \parencite[p.~13]{Hajek2016-HAJOHO},
Algunos autores han señalado que la solución radica en ser capaces de ofrecer un método para obtener relaciones causales a partir de datos estadísticos.

Este tema --el de la explicación causal-- está fuertemente vinculado con el uso de herramientas y \emph{modelos} estadísticos en investigación.
Mi propósito es hacer claro cómo funcionan dichos modelos y qué herramientas nos sirven para justificar que en efecto dos variables están causalmente relacionadas.
Esto es importante, porque en muchos casos se suele decir que los \emph{modelos} usados en investigación, sea cual sea su uso, son \say{falsos}

\paragraph{Se suele señalar} que hay un sentido en el cuál los modelos son \say{falsos}, sentido que suele expresar como \say{[$\ldots$] models typically only hold \emph{approximately,} in some ranges of circumstances, and they liberally employ idealizations to accomplish this \emph{partial} fit.} \parencite[p.~18, énfasis agregado]{Potochnik2017-POTIAT-3}

Potochnik también afirma que buena parte de la investigación se centra en ofrecer explicaciones causales y, más aún, afirma que una buena manera de lidiar con explicaciones causales en investigación científica es adoptando una teoría causal \emph{manipulabilista.}
Los argumentos de Potochnik que vinculan a la práctica de investigación con las teorías \emph{manipulabilistas}, dejan claro que hay una genuina preocupación sobre cómo de hecho se realiza investigación.

Potochnik también defiende que el uso de modelos en investigación --y cualquier supuesto que imponga ciertas condiciones en el modelo-- surge de la necesidad de que seres con capacidades cognitivas limitadas queremos representar fenómenos naturales para aislar los factores causales bajo escrutinio.
Es importante poner énfasis en esto porque implica que los modelos \emph{son generados} con ciertos objetivos en mente, o como ella misma señala \say{[$\ldots$] the nature of these idealizations is relative to the aim of research, as is most clearly demonstrated by the different idealizations involved in the various approaches to human aggression research.}

Mi investigación actual lidia con la afirmación de que los modelos son falsos.
Este no es un problema grave si la \say{verdad} no es una parte central de la investigación científica (en cualquier disciplina).
Pero si la investigación se propone \emph{justificar} hipótesis para obtener \emph{conocimiento} de relaciones causales reales, entonces tenemos un problema.
¿Cómo los modelos, que no representan adecuadamente el fenómeno, pueden dar como resultado conocimiento?
Si suponemos que el conocimiento es fáctico, entonces entramos en problemas.

El problema, me parece, radica en cómo representamos adecuadamente un fenómeno.
Debo aclarar también cómo es que funciona la representación científica y cómo la adecuación parcial de modelos que usan contara fácticos, implica fenómenos fácticos.

Mi hipótesis es que cierta forma de verdad debe estar involucrada en nuestra definición de \emph{conocimiento,} si es que queremos que las intervenciones en un sistema sean de alguna utilidad.
Por ejemplo, si el aumento del salario mínimo causa un aumento en el desempleo y qué condiciones juegan un papel causal en dicha relación.

\paragraph{Regresando a la anécdota,} el hecho de que estos dos profesores buenos trabajaran filosofía de la
ciencia --al menos lo que en ese momento creía que era la filosofía de la ciencia-- sesgó mis intereses a lo que hasta ahora ha sido mi trabajo.
La influencia de estos dos profesores me hizo tomar la decisión de entrar al posgrado en Filosofía de la Ciencia.

En las clases de maestría hubo profesores excelentes.
Mucho de lo que aprendí en estos cursos me sirvió para enmarcar de diferente manera las preguntas que me preocupaban, lo cuál me permitió entender --mejor, me parece-- la naturaleza de la investigación científica.
En particular quería responder preguntas sobre la epistemología de la ciencia, preguntas del tipo: \say{¿qué estamos justificados a creer?}, \say{¿cuándo podemos afirmar que una hipótesis ha sido corroborada?}, \say{¿es la explicación o compresión más básica que el conocimiento?}, etc.
Preguntas que surgen a partir de mi interés por la explicación causal.

Ahondar en estas preguntas y tratar de responderlas es una tarea más que complicada.
Muchos de los obstáculos se deben a que la práctica científica es más caótica y heterogénea de lo que parece en un principio.
Incluso si nos concentramos en el uso de herramientas estadísticas o herramientas que echan mano de la teoría de la probabilidad, en casi todos los casos, ciertas propiedades del \emph{fenómeno en cuestión} son idealizadas o ignoradas para hacer que el modelo estadístico sea de utilidad.

Esta representación parcial del fenómeno se debe a que en muchos casos buscamos destacar algunas propiedades particulares del fenómeno, sin que todas las variables estén representadas en el modelo.
Que no involucremos todas y cada una de las variables que afectan el fenómeno es un reflejo de nuestra limitada capacidad cognitiva.
A lo largo de su escrito, Potochnik nos recuerda que los modelos diseñados en la investigación científica tienen un propósito particular.
Son usados para destacar algunas propiedades particulares, especialmente, especialmente si lo que nos interesa son las relaciones causales.
Que simplifiquemos de ciertas manera el modelo se debe a que los investigadores intentan resaltar las características de un fenómeno en un momento particular.

Potochnik no es la única que ha señalado esto,
\textcite[][p.24]{abrams2023evolution}, por ejemplo, nos recuerda que \say{So-called infinite-population models are simply models that have no role for drift.
Biologists do often say things about evolution in infinite populations, and these claims are usually correct: the role of this terminology \emph{in practice} implies that it should not be understood literally.} (énfasis agregado)

A pesar de que los modelos no son completamente adecuados, --en el sentido en el que no representan totalmente el fenómeno en cuestión,-- son ampliamente utilizados en la investigación científica.
Si estas herramientas son usadas ampliamente, entonces como filósofos de la ciencia hay que ofrecer un análisis de cómo exactamente los modelos sirven para justificar hipótesis si parten de supuestos obviamente falsos.

Me parece que buena parte del problema está relacionado con determinar cuál es el papel que juega la \emph{representación adecuada} y la \emph{verdad} en la investigación científica.
Para motivar mi afirmación anterior, quiero dar un repaso por un periodo de la filosofía de la ciencia, un periodo en el cuál --suele decirse-- la \emph{verdad} jugaba un papel central en los análisis filosóficos de la ciencia: me refiero al periodo en el que los miembros del Círculo de Vienna estuvieron vivos.
Este repaso sirve para motivar la afirmación de que la \emph{verdad} juega un papel en la investigación científica --por el momento no asumiré cuál es ese papel,-- mientras que esto es al mismo tiempo compatible con la \say{representación parcial} que hacen los modelos de un fenómeno.

Para comenzar este repaso, debo señalar que en la maestría aprendí nuevas metodologías de investigación, además de diferentes maneras de plantear y entender las preguntas que me preocupaban.
Pero lo más valioso que aprendí fue la importancia que tiene la historia de la ciencia en la filosofía de la ciencia.
Me voy a permitir hacer una breve caracterización, exageradamente general, de dos posturas en historia de la ciencia.
