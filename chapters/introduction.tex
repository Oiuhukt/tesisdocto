% !TEX root = ../main.tex
% LTeX: language=es, en

\chapter*{Introducción}\label{ch:introduction}

\section*{Una breve anécdota}

\subsection{Parte I}

\noindent El Doctorado ha sido un proceso extenso
durante el cual he aprendido muchísimo. La parte más
sustanciosa de lo que he aprendido está relacionado
con la filosofía, en particular filosofía de la ciencia
\footnote{
	Juro que este breve relato tiene un punto.
}.
La alta calidad del posgrado me ha permitido
profundizar en los temas que me interesaban, interés
que me motivó a entrar a este doctorado.

La fascinación que tengo por los temas en filosofía de
la ciencia nació durante mi licenciatura, porque
durante este periodo tomé cursos con dos profesores
que se dedicaban al área\footnote{
	Al menos, lo que en ese momento pensaba que era la
	filosofía de la	ciencia. Si les pregunto, estoy
	seguro de que su respuesta sería que no se dedican
	a la filosofía de la ciencia.
}
Ambos excelentes maestros y filósofos.

Durante este periodo de mi formación (licenciatura) mi
atención estuvo enfocada a los temas de la \emph{explicación científica}
y la \emph{naturaleza de la causalidad}. Mis intuiciones---todavía queda un
resabio de esto---sugerían que señalar cuál es la
causa de un fenómeno es la mejor manera que tenemos de
explicarlo\footnote{
	Estoy usando de manera coloquial el término. Uso
	el término para hablar de aquello que hacen las
	personas en cualquier ámbito de la vida en el que
	se requiere ofrecer una explicación. Por ejemplo
    Que la carne se echó a perder porque el	refrigerador dejó de funcionar
    Que la casa se incendió a causa del	corto en el enchufe
	Ambos son casos en los cuáles decimos que A \emph{es una
	explicación de} B.
}.
Por ejemplo, parece claro que la mejor explicación de
por qué tengo cáncer de pulmón se debe a mis hábitos
de fumador, y fumar \emph{es una causa} de cáncer.

Que la causalidad sea parte de nuestra teoría de la
\emph{explicación científica,} es una intuición que no
está alejada de algunas teorías de la explicación
clásicas discutidas en filosofía de la ciencia. Me
parece además que la motivación más clara para incluir
una relación causal en nuestras teorías de la
explicación científica, se debe a que nos parece
necesario que la explicación de un fenómeno es \emph{
asimétrica }: que haya explotado el alimentador de
electricidad, explica por qué mi computadora se apagó
y perdí toda mi tesis. Pero claramente perder mi tesis
no explica por qué explotó el alimentador eléctrico.
Llamemos a esto \emph{condición de asimetría de la
explicación} [CA]. \citetitle{Salmon1970}

Salmon hizo notar en \citetitle{Salmon1970} que una
teoría de la explicación científica debe satisfacer CA
\citetitle{Salmon1970}. Salmon argumenta que el
modelo clásico que Hempel $\&$ Oppenheim desarrollan en
\citetitle{Hempel1948} no satisface CA y, por lo
tanto, es necesario modificar la teoría para dar
cuenta de la explicación científica. Quiero resaltar
que la discusión que Salmon hace del modelo de Hempel
$\&$ Oppenheim y por qué no satisface CA, nace de un
problema que Sylvain Bromberger discute en su artículo
\citetitle{Bromberger1966}.

Para apreciar la discusión en torno a la asimetría de la explicación, voy a comenzar exponiendo brevemente en qué consiste el modelo de Hempel $\&$ Oppenheim.Este va a ser el punto de partida desde el cual quierotrazar una historia del papel que, me parece, juega la
causalidad en una teoría de la expicación científica.
Elijo este punto de partida porque (i) es una
teoría clásica en filosofía de la ciencia y (ii)
pretendo justificar que \emph{la asimetría de la
explicación es un reflejo de la dirección causal.}
En principio, no me parece controvertido afirmar que
las explicaciones no son simétricas. Mas controvertido
es afirmar que las relaciones causales explican
perfectamente por qué esperamos esta asimetría. Es por
esto último que en esta primera parte de mi anécdota
---que además sirve como introducción a mi trabajo---
voy a justificar la afirmación en itálicas. Para esto,
comienzo haciendo un breve repaso por los títulos
mencionados en el párrafo anterior.



\paragraph{Hempel y Oppenheim \citeyear{Hempel1948},}
caracterizaron  a la explicación científica como un
argumento deductivo.  Sabemos, por ejemplo, que \say{todos
los metales se dilatan cuando se calientan.}
Esta oración cuantificada universalmente implica
deductivamente cualquier instancia particular en la
que se cumplan las condiciones iniciales. Si las
explicaciones son todas argumentos deductivos, entonces
lo que explica por qué crece este pedazo de cobre
cuando lo dejo sobre una hoguera, se debe a que es un
caso particular de la oración cuantificada
universalmente. El análisis que ofrecen estos autores
sugiere que \emph{explicar} consiste en ofrecer un
argumento en el cuál derivemos una instancia
particular de un cuantificador universal. En
principio, esto suena un proceso simple, sin embargo,
hay un par de detalles que resta especificar

En primer lugar, la oración cuantificada es crucial.
Esta oración es lo que nos permite que el argumento
sea deductivo: eliminando el cuantificador universal y
sustituyendo la variable por una constante\footnote{
	Por si acaso, el argumento se formaliza
	$ \forall{ x }( \text{ M }( x ) \& \text{ C }( x )
	\rightarrow \text{ D }( x ) ), ( \text{ M }( a )
	\& \text{ C }( a ) ) \Rightarrow \text{ D }( a ) $.
}.
En segundo lugar, esta oración cuantificada debería
ser la formulación lingüística de una ley natural. El
hecho de que la oración sea una ley natural es lo que
asegura que el argumento sea explicativo. Para
ilustrar esto, consideremos la oración \say{todas las
monedas en el bolsillo de Oscar son de $ \$ $1.} Sucede que la oración es verdadera\footnote{
    Además resulta que es siempre verdadera por
    antecedente falso. Pero me parece que no hace
    diferencia para el ejemplo.
}.

Supongamos que la oración es verdadera y, dado que
está cuantificada  universalmente, si tomo una moneda
de mi bolsillo, entonces será de $ \$ $1. Pero que todas
las monedas de mi bolsillo tengan una denominación de
$ \$ $1, pero esto no implica que hemos explicado por qué
la moneda vale $ \$ $1. Pero esto es algo que sí parece
estar presente en el caso del cobre.

El modelo sugerido por H$ \& $O---también llamado modelo
por ley de cobertura---requiere que la oración
cuantificada cumpla con las características para
obtener el título de \emph{ley de la naturaleza}
\footnote{
	Se suele caracterizar a las leyes de la naturaleza
	como un fenómeno natural que es verdadero en todos
	los casos posibles, e	independiente del tiempo y
	lugar en el que ocurre. Esta afirmación debe ser
	verdadera en todas las situaciones e independiente
	del tiempo y del lugar. Condiciones que dicha
	oración cuntificada debe	cumplir si es que va a
	contar como una genuina explicaión. Uno de los
	problemas que surgen de esto es que debemos
	ofrecer criterios parra distinguir una ley de la
	naturaleza de generalizaciones	universales
	espurias \parencite{Nagel1962}. Ahora, es una
	cuesión distinta, si de hecho hay algo que cumpla
	las condiciones para ser una ley de la naturaleza.
	Ambos problemas son tangenciales a mi	proyecto,
	y no discutiré más aquí. El lector interesado
	puede	revisar	\parencite{sep-laws-of-nature}
},
de otro modo, el argumento carecería de cualquier
valor explicativo. Si estas condiciones se satisfacen,
entonces tendremos una oración cuantificada
universalmente---que presumiblemente es la
formulación lingüística de una ley natural,---que
implica como instancia particular el fenómeno en
cuestión que queremos explicar. Si esto sucede,
entonces hemos tenido exito ofreciendo una
explicación.

\paragraph{Sylvain Bromberger \citeyear{Bromberger1966}} detecta un problema
con la caracterización que Hempel y Oppenheim ofrecen.
Bromberger comienza argumentando que hay muchas
similitudes entre explicar que este pedazo de cobre se
dilata al calentarse---digamos, el fenómeno de la
dilatación de los metales---y responder a la pregunta
\say{¿por qué mi cuchara se dilata cuando la
caliento?} Por ejemplo, el autor señala en la página
que

    \begin{quote}
    $ \ldots $ there are issues in the philosophy of
	science that warrant an interest in the nature of
	why-questions.	The most obvious of these issues
	are whether science (or some branch of science or
	some specific scientific doctrine or some approach
	) ought to, can, or does provide answers to
	why-questions, and if so,
	to which ones.
	\end{quote}

La afirmación de Bromberger es altamente intuitiva. No
es exagerado suponer que la investigación científica
se dedica a responder preguntas de este tipo\footnote{
	Por supuesto, es una afirmación	completamente
	distinta si la	investigación científica \emph{puede}
    responder a \emph{todas} las	preguntas
	de este tipo; otra cuestión	distinta es si la
	ciencia \emph{debe} responder a \emph{todas}
	las	preguntas de este tipo. Sería muy	triste una
	respuesta afirmativa a	la segunda cuestión.
}:
es más o menos claro que una explicación suele ser una
respuesta a una pregunta \emph{por-qué }.

Sin embargo, si  enmarcamos el problema en los
términos de Bromberger, entonces lo que el
autor llama \say{la doctrina hemepeliana} es
claramente incorrecta. El argumento procede así:
supongamos que lo que dice Bormberger es verdad
y que la doctrina hempeliana es correcta. De forma
inocente, le preguntamos a Cristian \say{¿por qué la
altura del Empire State es de 381 metros metros?}
Cristian, muy versado en funciones trigonométricas,
responde de la siguiente manera: hay un punto en
quinta  avenida que está a $ x $ distancia
del Empire State. Dado el ángulo que forman los rayos
de sol desde el punto más alto del edificio hasta ese
punto, podemos derivar fácilmente la altura del
edificio usando la función tangente.

El argumento en este caso es claramente deductivo. Si
sustituimos las variables pertinentes, podemos
adecuadamente concluir cuál es la altura del Empire
State: resultado nada trivial. Pero esto significaría
que usando este argumento, Cristian ha respondido
adecuadamente \emph{por qué} la altura del Empire
State es de 381 metros, pero nadie diría que el hecho
de que el ángulo que forman los rayos del sol sea $
\theta $ y que la distancia entre el edificio y quinta
avenida sea $ x $, explica por qué el edificio mide
381 metros. Que el edificio tenga esa altura
particular es algo que debería explicar la persona que
diseñó el edificio. Esto no es culpa de Cristian, sino
de quién afirme que dicha respuesta es una genuina
explicación de la altura del edificio.

El problema que Bromberger detecta es que
contraejemplos como éste muestran que hay que
abandonar la doctrina hempeliana, y que por lo menos
es necesaria una caracterización distinta de la
explicación científica. Hay que notar que Bromberger
discute este\footnote{
	El ejemplo del poste no aparece en ninguno de sus escritos, porque originalmente Bromberger no estaba atacando la asimetría que surge de	la doctrina Hempeliana. La historia muestra que el 	ejemplo del poste se lo comentó a Hempel en una	conferencia, como señalan en \parencite{mitBromberger}, anécdota mencionada también en	\parencite[p.~81]{Dewulf2022}. La intención de	Bromberger estaba enfocada en la semántica de las 	oraciones	interrogativas y la ignorancia racional, tal como muestran sus trabajos	posteriores \parencite{Bromberger1992}.
},
y otros contraejemplos en \parencite{Bromberger1966},
pero Wesley Salmon fue quien se concentró en mostrar
que el contraejemplo de la sombra implica que, si la
doctrina hempeliana es correcta, entonces la
explicación científica es simétrica. Una clara falta a
CA y si este argumento es sólido: el modelo por ley de
cobertura de Hempel y Oppenheim debe ser incorrecto.
Durante este periodo de la filosofía, muchos esfuerzos
se concentraron en resolver este y otros problemas
problemas derivados de la doctrina hempeliana.

Wesley Salmon empleó el contraejemplo presentado por
Bromberger y mostró que el modelo por ley de cobertura implica esta simetría. Pero Salmon no sólo se dedicó a señalar los problemas del modelo por ley de cobertura, sino que formuló una teoría de la explicación que lidia con dicha asimetría y que una mejor alternativa a la teoría de la explicación desarrollada por Hempel y Oppenheim.

\paragraph{Wesley Salmon \citeyear{Salmon1970}}
comienza por afirmar que la asimetría temporal es crucial
para la explicación científica. Polemizando el
contraejemplo de Bromberger, Salmon señala que

\begin{quote}
	Although the sun, flagpole, and shadow are perhaps
	commonsensically regarded as simultaneous, a more
	sophisticated analysis shows that physical processes
	going on in time are involved.	Photons are emitted by
	the sun, they travel to the	vicinity of the flagpole where
    some are absorbed and some are	not, and those which are
    not go on to  illuminate the ground \parencite[p.~72]{Salmon1970}
\end{quote}

Esto, por supuesto, resolvería el problema de la asimetría. A primera vista, es obvio que los procesos físicos tienen cierta dirección temporal\footnote{
	Hay	una discusión en torno a si hay una dirección privilegiada del	tiempo, o si las relaciónes temporales pueden revertirse. Esta discusión se ha retomado para, a su vez, definir la dirección en la que opera la causalidad. Esto es un tema en su propio derecho y necesitaría un trabajo mucho más largo que le hiciera justicia al tema. El lector interesado puede consultar	el artículo clásico de Russell para un argumento antirrealista de la	causalidad \parencite{onthecauserussell}. El argumento	de Russell descansa en el hecho de que las igualdades en	física son reversibles. Con respecto a Hempel y Oppenheim, Se suele asumir que los empiristas lógicos basaron sus	argumentos escépticos, en los argumento clásicos de Hume,	argumentos que aparecen en	\parencite{hume1784}. Para	una discusión general en torno a la	dirección del tiempo,	la llamada \say{flecha del tiempo}, el lector	puede	revisar \parencite{utmArrowTime}, un argumento a favor de la	retrocausalidad lo ofrece Dummet en \parencite{dummetcause}.
}.
Suena sensato pensar que cualquier explicación que podamos ofrecer de un fenómeno, debe respetar dicha dirección.

Sin embargo, basar la asimetría de la explicación en
la asimetría temporal de los procesos físicos parece
que va colando a las relaciones causales en nuestro
análisis. Pero Hempel y Oppenheim querían excluir a
las relaciones causales de su análisis de la
explicación científica, por lo que hay que justificar
dicho cambio. Esto significa que debo justificar por
qué las relaciones causales comienzan a tomar un papel
prominente en la explicación científica. Para hacer
esto con éxito, debo recordar al lector que la
discusión de Salmon está enmarcada dentro de la teoría
de la explicación científica que él mismo desarrolla.
Salmon busca resolver los problemas de la teoría
de Hempel a la luz de considerar la asimetría de la
explicación.

Para explicar brevemente en qué consiste la solución
de Salmon, voy a comenzar con un ejemplo. Supongamos
que queremos saber si un dado está cargado. Si
lanzamos un dado icosaédrico, el dado sólo va a
mostrar una de sus caras en cada lanzamiento y el
resultado será un número entre 1 y 20. Este es el
espacio de \emph{eventos} definido para nuestro
experimento.

Supongamos que lanzamos el dado 50 veces y generamos una lista de los resultados obtenidos. Obviamente el dado sólo puede mostrar una cara a la vez, lo que implica que cualquier otro evento---las caras que no se muestran---están excluidas una a una. Es decir, si una cara cualquiera del dado aparece, entonces el dado no muestra ninguna otra cara. En este caso, el conjunto de los posibles resultados está definido por el conjunto de números entre 1 y 20. Sabemos además que cualquiera de las caras puede aparecer por lo menos 1 de cada 20 tiradas.

Sucede que he estado usando dados icosaédricos de vez en cuando---probable-mente jugando D$ \& $D---y supongamos que visito a Mar. Mar toma un dado de su colección y me propone una apuesta: cada vez que el dado muestre un número non, ella me paga $ \$ $10, mientras que yo le pago la misma cantidad cada que sale un número par. Por alguna razón, durante las 50 tiradas, los números pares aparecen con más frecuencia---digamos $ \frac{ 15 }{ 20 } $---por lo que estoy
comenzando a perder grandes cantidades de dinero: tengo la sospecha de que el dado está cargado.

Recapituando el ejemplo: lo que quiero explicar es por qué los
números pares están apareciendo con más frecuencia, y parece sensato pensar en que una explicación posible es que el dado está cargado. Siguiendo las etiquetas clásicas, el \emph{explanans}---aquello que quiero explicar---es por qué (M) Mar está ganando 15 de 20 tiradas; mientras que el \emph {explanandum}---aquello que presentamos como evidencia, justificación, hipótesis, la respuesta a la pregunta,
etc---es que (D) el dado está cargado. Llamemos \emph{clase de
referencia} a las condiciones que parten el espacio de probabilidad en dos clases mutuamente excluyentes: que el dado esté cargado $ P( D ) $ o que el dado no está cargado $ P( \neg D ) $. Suele sucedcer que los dados cargados muestran algunos números con más frecuencia que otros, entonces $ C $ es estadísticamente relevante para $ M $, y si esto es
verdad, entonces, $ P( M | C ) $ debería ser mayor que $  P( M | \neg C ) $.

En corto, la \emph{clase de referencia} parte el espacio de eventos en las variables que sospechamos hacen una diferencia estadística. Una parte de la solución, que desarrolla Salmon, comienza por hacer clara la definición de la clase de referencia. Brevemente, la clase de referencia es el conjunto que usamos para definir el espacio de eventos y seleccionar las variables estadísticamente relevantes.

Por ejemplo, supongamos que queremos explicar por qué hoy hubo una tormenta. Nuestra clase de referencia es el conjunto de días en los que hubo tormentas y las condiciones que estuvieron presentes mientras el fenómeno ocurría. Que disminuya la presión atmosférica suele indicar que se aproxima una tormenta, lo que significa que podemos dividir la clase de referencia en los días en los cuáles hubo una tormenta y disminuyó la presión atmosférica, y aquellos días en los que hubo una tormenta y no disminuyó la presión atmosférica. Si condicionamos el evento a una de estas dos clase de referencia y descubrimos que una de ellas es relevante para el hecho de que llueva, entonces esto explica por qué hoy hubo una tormenta, cuya respuesta es, porque disminuyó la presión atmosférica.

Salmon considera que la clase de referencia que hemos definido, debe ser \emph{homogénea} y debe ser la \emph{más amplia,} es decir, que incluya todas las variables estadísticamente relevantes---mientras excluímos las variables irrelevantes---para que el fenómeno en cuestión ocurra. Retomando el ejemplo, podemos hacer más estrecha la clase de referencia al introducir variables como la velocidad del viento, las condiciones de humedad, la época del año, etc. Para mi ejemplo simple, la frecuencia de (A) los días en que hubo lluvia, (B) disminuyó la presión atmosférica y (C) hubo velocidades de viento arriba de los $ 80 km/hr $, resultando en $ P( A | B \& C ) $. Agregar estas variables, vuelve más estrecha a la clase de referencia.

Regresando a la teoría de la explicación que Salmon
desarrolla, hay ciertas condiciones que la respuesta a la pregunta \say{¿Por qué llovió hoy?} debe cumplir para que cuente como una explicación genuina. En primer lugar, debemos definir la clase de referencia. En segundo lugar, tenemos que elegir qué condiciones son estadísticamente relevantes para que ocurra el fenómeno, de manera que seamos capaces de incluir todas las variables. Por último, las variables deben ser estadísticamente relevantes para el fenómeno en cuestión. De esta manera hemos elegido las variables que \emph{hacen una diferencia} para que el fenómeno ocurra. Esta diferencia está definida de la siguiente manera: si $ P( A | B ) \neq P( A | \neg{ B } ) $\footnote{
	Estoy asumiendo que de hecho hay una diferencia en la frecuencia. Si no la hubiera, es decir $ P( A | B ) = P( A | \neg{ B } ) $, entonces la relevancia sería la misma que una tirada	de moneda, por los axiomas 2 y	3 de Kolmogorov.
},
y $C$ es una variable estadísticamente relevante---una variable que hace una diferencia---entonces la respuesta a \say{¿Por qué llovió el día de hoy?}, sería \say{Porque disminuyó la presión atmosférica y hubo vientos arriba de 80 Km/h}. Dado que estamos asumiendo que estas variables son estadísticamente relevantes, entonces debería cumplirse que $ P( A | B \& C )$ es mayor que cualquiera entre $ P( A | \neg{ B } \&	C ) $, $ P( A | B \& \neg{ C } ) $, $ P( A | \neg{ B } \& \neg{ C } ) $. Es gracias a que seleccionamos una clase de referencia adecuada---en la cuál están incluídas las variables que de hecho \emph{hacen una diferencia}---que podemos explicar la ocurrencia del evento $ A $ apartir de la ocurrencia de $ B $ y $ C $.

Antes de continuar con la exposicón, quiero recapitular los detalles del modelo de Salmon. Primero, debemos ser capaces de hacer una \say{partición homogénea} del fenómeno a explicar. Que sea una partición homogénea significa que dado el fenómeno que queremos explicar, debemos seleccionar todas las variables relevantes que hagan una diferencia para que el evento ocurra. Esta partición debe ser exhaustiva, es decir, que no falte ninguna variable relevante por incluir y que no sea posible agregar nuevas variables al conjunto. En palabras de Galavotti \say{En esta perspectiva lo que cuenta para la explicación no es la alta probabilidad, como lo requería Hempel, sino estar en posición de afirmar que la  distribución de probabilidad asociada con el	\emph{explanandum} refleja la información más completa y detallada	que podamos obtener}\footnote{
    In this perspective what counts	for the sake of    explanation is not high probability, as required by Hempel, but being in a position to assert that the probability distribution associated with the explanandum reflects the most complete and detailed information attainable.
}
\parencite{Galavotti2018}.

Hasta aquí, el modelo de explicación defendido por Salmon tiene una serie de ventajas. El modelo resuelve muchos de los problemas que tiene el modelo por ley de cobertura: la noción de explicación que desarrolla Salmon no echa mano de leyes y, por tanto, no tiene el problema de distinguir entre leyes y generalizaciones accidentales. Además, evitar que la explicación descanse en la noción de \emph{ley de la naturaleza}, se antoja más adecuado para analizar las explicaciones en aquellas ciencias en las que no es obvio que haya tal cosa (como la biología o la economía).

Sin embargo, aún resta un problema por sortear: que seamos capaces de seleccionar las variables pertinentes, no implica que la relación temporal definida sea la correcta. Supongamos que hay una cadena temporal entre el aumento de la velocidad del viento, la disminución en la presión atmosférica y, por último, que llueva. Llamemos a esta descripción $ D $. Sin embargo, hay otra manera---perfectamente sensata---de describir esta relación temporal: supongamos ahora que tanto la disminución en la presión atmosférica y el aumento del viento ocurren al mismo tiempo, luego llueve. Llamemos a esta descripción $ D' $.

Si asumimos que seleccionar las variables relevantes es \emph{suficiente} para caracterizar a la explicación científica, entonces ambas descripciones constituyen explicaciones correctas. Más aún, esto implica que $ D  $ y $  D' $ son explicaciones equivalentes. Sin embargo, intuitivamente, $ D $ y $ D' $ describen fenómenos completamente distintos. Mientras que $ D $ parece sugerir que la baja en la presión atmosférica \emph{causa} el aumento en la velocidad del viento y esto a su vez \emph{causa} que llueva, $ D' $ parece sugerir que tanto la baja en la presión atmosférica y el aumento en la velocidad del viento \emph{causan} que llueva.

Esto significa que el modelo de Salmon aún tiene problemas relacionados con distinguir cadenas temporales. Este problema obviamente le preocupa al autor, como muestra la cita casi al inicio de esta sección, porque Salmon reconoce que el modelo tal como está expuesto, es claramente sensible a hacer pasar las correlaciones estadísticas por relaciones de dependencia causal\footnote{
	Salmon discute e intenta resolver esto apelando a los estados de baja entropía y a la relación temporal entre el evento a explicar y la clase de referencia. No voy adetallar la solución que Salmon propone,	porque no es necesario para fines del proyecto. El lector interesado	puede leer la última sección de \parencite{Salmon1970}~.
}

En estadística, es más que común el eslogan \say{correlación no implica causalidad} y el problema con el análisis de Salmon, tal como lo he presentado hasta aquí,  es que no hay manera de distinguir entre las correlaciones que tienen una causa común. Sé que para determinar cuál es la presión atmosférica, puedo utilizar un barómetro. Sé además que cada vez que el barómetro se comprime, es porque ha aumentado la presión atmosférica. Obviamente la lectura del barómetro es relevante para que se den condiciones de lluvia, pero la variable causalmente relevante es la presión atmosférica, no la lectura del barómetro. Si ingenuamente comienzo a mover con mi mano la aguja del barómetro---mientras todas las demás condiciones permanecen cosntantes,---es claro que no he aportado a la producción de lluvia: el barómetro no es un factor causal relevante.

Sin embargo, sabemos que la lectura del barómetro \emph{es} una variable estadísticamente relevante, que \emph{hace una diferencia} (tal como lo he definido líneas arribas) en el resultado esperado\footnote{
	No estoy seguro de que todas las personas que se dedican al clima tienen barómetros funcionales que usan cotidianamente para medir la presión atmosférica, pero puedo	apostar a que usan las lecturas de sus barómetros para determinar si	va a llover o no, y que cuando hay una baja en la lectura del 	barómetro (y otras condiciones relevantes para el caso), es porque de	hecho va a llover.
},
pero es claro que he seleccionado un factor \emph{causalmente} irrelevante, aún cuando es un factor \emph{estadísticamente} relevante. Si sólo caracterizamos a la explicación como he hecho hasta aquí---exponiendo la teoría que Salmon presenta en \parencite{Salmon1970}---entonces vamos a ganar casos que cumplen las condiciones, pero que no son explicaciones.\footnote{
	obviamente no siempre es bueno ganar, especialmente si lo que hemos ganado es basura.
}
Para resolver este problema, Salmon hace notar que su modelo de explicación puede naturalmente incorporar \emph{relaciones causales.} De este modo, en torno al ejemplo de la sombra, Salmon comenta que

\begin{quote}
	The reason that the explanation of the length of the shadow in terms of the height of the flagpole is acceptable, whereas the \say{explanation} of the height of the flagpole in terms of the	length of the shadow is not acceptable, seems to me to hinge directly upon the fact that there are causal processes with earlier and later temporal stages. \parencite{Salmon1970}
\end{quote}

Lo que sugiere la cita anterior es que para responder a estos problemas, deberíamos incluir factores causales en nuestro análisis. Esto es una clara desviación de lo que se suele asumir eran tesis básicas del antaño positivismo lógico. Mucho esfuerzo filósfico se dedicó al tema de la causalidad y la explicación. Relevante para este contexto, por ejemplo, es que Lewis publicó \citetitle{Lewis1973-LEWC, Lewis1973-LEWC-2}; mientras que en torno al fenómeno de la explicación, Friedman publicó \citetitle{Friedman1974}. Otros esfuerzos fueron dedicados a hacer claro si es que la causalidad juega un papel en la explicación científica, y cuál es dicho papel \parencite{Kitcher1962-KITEUA}. Siguiendo algunas de las soluciones que ofreció Salmon al resolvcer el problema de la asimetría---entre ellas el hecho de que su modelo de explicación no necesita definir qué es una ley de la naturaleza---muchos filósofos han tratado de ofrecer una teoría de la explicación científica que incluya un factor causal en su análisis. Tal es el caso de James Woodward.

\paragraph{James Woodward \citeyear{Woodward2004}} ha
desarrollado una teoría de la explicación científica que hace exactamente esto: incluir relaciones causales. La teoría de Woodward, por razones que espero queden claras en los siguientes párrafos, ha recibido el nombre de \emph{manipulabilista}. Brevemente, las teorías \emph{manipulabilistas}\footnote{
	La	siguiente caracterización está basada en \parencite{sep-causation-mani, sep-causal-explanation-science,
	Woodward2004, Woodward2000-WOOEAI}.
}
afirman que dadas dos variables $ A $ y $ B $, decimos que $ A $
\emph{causa} $ B $ si al intervenir (manipular) la variable $ A $, la variable $ B $ cambia en consecuencia. Me voy a permitir formular un ejemplo que, espero, haga claros los detalles de esta teoría.

Asumamos que Miranda me pregunta por qué hoy llovió y estoy tratando de formular una respuesta. Retomando las descripciones $ D $ y $ D' $ definidas anteriormente, puedo formular dos respuestas. Suponiendo $ D $, debería buscar información de cómo se producen los vientos y si la presión atmoférica está relacionada en su producción\footnote{
	Resulta que es la termodinámica del ambiente jugando un papel más complejo. Lo que sucede es que vientos veloces distribuyen mejor el calor, lo que enfría el aire y lo vuelve más denso, lo que aumenta la presión atmosférica \parencite{Spiridonov2021}. Esto implica que la respuesta que voy a dar es incorrecta, pero esto no afecta los fines expositivos.
},
después de tener la información pertinente, entonces podría responderle a Miranda señalando que cuando la (A) presión atmosférica \emph{aumenta}, \emph{produce} cambios en la (B) velocidad del
viento, lo que a su vez \emph{causa} que (C) llueva. Las palabras enfatizadas en esta oración, son palabras que intuitivamente relacionamos con procesos causales: palancas, poleas y manivelas. Al ofrecer esta respuesta, parece que asumo que hay un proceso causal participando. Es decir que son variables que considero\footnote{
    Insisto aquí en la palabra \emph{considero.} Esto quiere decir que no significa que \emph{sé} que dichas variables son causalmente relevantes, sino que \emph{especulo} que son variables pertinentes.
}
son causalmente relevantes para dar una explicación de por qué llovió hoy.

Si modeláramos el evento\footnote{
	Algo que saben hacer los meteorólogos profesionales. Yo carezco del conocimiento para hacer un modelo  mínimamente informativo, por tanto, el ejemplo es de juguete. Lo cuál  no afecta los fines explicativos.
}
señalando que hay una relación proporcional entre la presión atmosférica y la velocidad del viento, de tal suerte que un aumento de 1 en la presión atmosférica, implica un aumento de 10km/h en la velocidad del viento; además, sucede que siempre llueve cada que la velocidad del viento es mayor a  los 85km/h, o dicho de otro modo, \say{si aumentáramos en 9 unidades la presión atmosférica, entonces llovería.}

Con el ejemplo anterior, sólo pretendo señalar la parte más intuitiva de las teorías manipulabilistas, que ciertos cambios en los valores de las variables provocan cambios en los valores de otras variables: pero sólo a través de este cambio. Esto, por supuesto, sólo es una parte de la historia de las dichas teorías. Olvidémonos por un momento de la presión atmosférica y supongamos que el fenómeno a analizar es uno completamente estable, es decir, que cualquier cambio de valor en la variable $ A $, implica un cambio \emph{proporcional} en $ B $. Si hay una función bien definida sobre un conjunto, la tarea es ligeramente sencilla. Definamos una función cualquiera de una variable tal que $ f( x ) = 2x $, donde $ x	\in{ N } $. Esto significa que si la función toma el $ 9 $ como valor de entrada, el valor de salida será $ 18 $. Supongamos más aún que estos son los valores de una medida de fuerza y que la relación es la siguiente: si la fuerza $ A $ es de 1 unidad, entonces la distancia $ B $ trasladada de un proyectil es de 2.\footnote{
	Por supuesto esto es púramente ficticio, se trata sólo de un ejemplo ilustrativo, el punto es que la función de distancia es proporcional a la medida de fuerza.
}
Digamos entonces que aplicar 3 medidas de fuerza, producen \emph{6 metros} de distancia de un proyectil. Supongamos además que esta función es \emph{independiente} de cualquier fuerza externa, por ejemplo, de la resistencia del medio (en este caso el aire), el ángulo de lanzamiento (dependiendo del ángulo, la gravedad comienza a jugar un papel), etc. Un teórico de la aproximación causal intervencionista, diría que esto significa que la fuerza impresa en el proyectil, \emph{causa} que el proyectil alcance más distancia. Además diría que es posible que la función definida anteriormente no describa esta relación para algunos de los valores de $ A $, por ejemplo, si imprimo una fuerza de $ 1*10^{ -7 } $ unidades, seguramente el proyectil no se traslade $ 2*10^{ -7 } $ unidades.

El ejemplo anterior me sirve para describir algunas de las características de la teoría \emph{manipulabilista.} Que la función
describa correctamente el comportamiento del proyectil,
\emph{independiente} de otras fuerzas, quiere decir que la fuerza
impresa en el proyectil es la única variable causalmente relevante
que afecta la distancia del mismo. Que la \emph{intervención} en la
fuerza impresa cause que el proyectil llegue a mas distancia
---\emph{independiente} de cualquier otra fuerza---significa que en
la medida de los posible hemos decidido dejar fuera del modelo,
variables \emph{exógenas,} variables que sabemos pueden afectar el comportamiento del modelo, pero al decidir concentrarnos en la fuerza y la relación con la distancia, dejamos dichas variables fuera. Si además podemos excluir todas las variables,
excepto la fuerza, entonces es posible hacer una intervención
\emph{quirúrgica} y que sólo haya ciertos valores que describen
correctamente el comportamiento del proyectil, significa que es
\emph{invariante} bajo un rango de intervenciones. Es en esto, a
grandes rasgos, en lo que consiste una teoría \emph{manipulabilista}
de la causalidad.

Por supuesto, otras características integran a la teoría, además de que
muchos argumento se han presentado para mostrar sus deficiencias
de la causalidad. Quizás la más controversial de ellas es que la teoría
hace uso de condicionales \emph{contrafácticos.}

Los condicionales \emph{contrafácticos} son aquellos condicionales
que evaluamos asumiendo que que algo de \emph{hecho}, no sucede. Esto
nos obliga a preguntarnos qué pasaría si se cumplieran ciertas
condiciones particulares. Dicho de otro modo, nos preguntamos cómo
sería el mundo si, por ejemplo, México \emph{no hubiera sido
conquistado} por los españoles.

Parece más que natural implementar estos condicionales en la teoría
intervencionista de la causalidad. Al definir las variables
causalmente relevantes, podemos incluir tanto variables discretas
como variables continuas y podríamos analizar preguntas del tipo
\say{¿qué hubiera sucedido si $ A $ hubiera tomado un valor distinto al
que tomó, digamos 0.08?} Si la función está bien definida, como en
mi ejemplo, entonces es relativamente sencillo determinar valores de
distancia para diferentes valores de fuerza.

Esto es algo que no sólo es parte de la teoría manipulabilista de la
causalidad, sino que suele mencionarse en contextos de investigación
científica. Cuando se analizan modelos estadísticos, los autores
suelen mencionar condicionales \emph{contrafácticos} como parte de
sus análisis. Entre estas herramientas, resalta especialmente
\emph{la inferencia causal}, que es una herramienta estadística usada
para derivar conclusiones causales a partir de agregación de datos e
inferencia estadística. Estos métodos sirven, sobre todo, para
determinar cuáles factores son causalmente relevantes \parencite{Pearl2016, Pearl2018}.

Pensemos, por ejemplo, en las \emph{Pruebas Controladas
Aleatorizadas} (RCT, por sus siglas en inglés), donde típicamente,
los investigadores interesados en saber cuál es la efectividad de un
medicamento, intervienen en una de las variables---dándole un placebo
a un grupo de control, mientras que otro grupo recibe el medicamento
a testar. Asuminedo que no hay sesgos en la muestra, si el grupo de
prueba responde al medicamento, entonces, en principio, seremos
capaces de detectar si el medicamento hace una aportación causal. En
corto: que podemos pensar que el grupo de control funciona como una
especie de imagen alterna de lo que \emph{habría} pasado de no haber
tomado los medicamentos.

Si bien, aún no hay un consenso universal de que la causalidad
\emph{deba} ser parte de nuestras teorías de la explicación
científica, con la discusión anterior quiero sugerir y motivar que la
causalidad es una característica fundamental del proceso de
explicación científica. Consolidando mi intuición original de que la
causalidad de hecho juega un papel en la explicación.

\subsection{Parte II}

\noindent Hasta este momento de la anécdota, me he concentrado en repasar
rápidamente algunas de las teorías de la explicación clásicas en
filosofía de la ciencia. Como bien confesé al inicio, tengo mis
preferencias en torno a qué teoría de la explicación es la más
adecuada. Pero quiero llamar la atención en torno a que en la última
de las terorías expuestas, el tema viró hacia los condicionales
contrafácticos. Son este tipo de condidionales los que ocupan mi trabajo de investigación actual.

Es más que obvio que los condicionales contrafácticos están en la
caja de herramientas de cuclquier persona que se dedica a hacer
estadística en su trabajo diario, por tanto, los \emph{condicionales
contrfácticos} son un elemento de la \emph{investigación científica}.

Pensemos en ejemplos donde aparecen herramientas de este tipo, como
en los métodos de \emph{inferencia causal.} Herramientas como los
modelos de inferencia causal, son usados regularmente para aislar
factores causales a partir de agregados de datos e inferencia
estadística. Los investigadores tenemos un interés paraticular por
aislar factores causales, por ejemplo, para saber qué factores
afectan causalmente el desempeño académico de los estudiantes, o para
intervenir en factores que promuevan el aumento de los salarios y la
calidad de vida, etc. La inferencia causal modela fenómenos causales
haciendo uso de \emph{contrafácticos,} que se presentan como eventos
alternos que tienen exactamente las mismas condiciones, pero variamos
la cantidad de la variable que creemos está haciendo el trabajo
causal. El amplio uso de estos métodos, así como algunas de las
aplicaciones que tienen en la investigación científica es algo que
queda reflejado en \parencite{Pearl2016}, en \parencite{Pearl2018} y
especialmente en \parencite{llaudet2023}.

Sin embargo, el hecho de que los contrafácticos sean parte de estos
métodos y que su uso sea cotidiano en investigación, no significa que
estén resueltos todos los problemas filosóficos que acarrean. En este trabajo de
investigación, me voy a dedicar a discutir el problema que ha recibido el nombre de \say{el
problema fundamental de la inferencia causal.} Dicho en una oración: que no es posible saber
el valor de verdad de un condicional \emph{contrafáctico}. Si por definición del condicional,
estamos evaluando algo que no sucede, pero que pudo haber sucedido: exactamente qué evidencia
podemos ofrecer para justificar condicionales como \say{si los españoles no hubieran
conquistado México, entonces conservaríamos todas nuestras reservas de oro.}

Esto, por supuesto, es un problema relacionado con la epistemología de la modalidad. Para ejemplificar el problema, supongamos que queremos saber si los estudiantes que leen más de un libro al año tienen un mejor desempeño
académico que aquellos alumnos que leen menos de un libro al año.
Supongamos, más aún, que hemos diseñado una prueba aleatoria
randomizada y que hemos aislado existrosamente la variable \say{leer
	más de un libro al año}, que en la encuesta probablemente midamos con
el con la pregunta proxy \say{¿lee usted más de un libro al año? }

Ahora, si nos preguntamos qué hubiera sucedido si los alumnos no
leyeran al menos un libro al año, debemos responder a la pregunta
\say{¿qué habría pasado si $\ldots$? }, lo que significa que tenemos que
evaluar el desempeño del grupo de control, es decir: aquel que leyó
menos de un libro al año. Sin embargo, la única información que
tenemos es el agregado estadístico de los alumnos que leyeron menos
de un libro al año, lo que implica que nunca seremos capaces de
observar, digamos, qué hubiera sido de Ana si hbiera leído menos de
un libro al año; ni siquiera seremos capaces de observar que hubiera sido
de \emph{cualquier} alumno dentro del grupo de control: en breve, no podemos hacer que Ana lea más de un libro al año, rebobinar, incentivar a Ana a leer menos, luego observar en resultado en ambos casos.

Si somos personas trabajando en una hipótesis para saber cómo la lectura afecta el desempeño de los estudiantes, estas estadísticas sólo muestran que los alumnos que leyeron más de un libro al año tuvieron un mejor desempeño académico que aquellos alumnos que no lo hicieron. Pero esto no es suficiente para justificar que la directora de la escuela le recomiende a Ana leer más de un libro al año \emph{mejora}: descrito de esta manera, es al menos físicamente imposible observar a Ana leyendo al menos un libro al año, y haciendo exactamente lo contrario.

Entonces, ¿cómo sabemos que este modelo implica correctamente que
incentivar el hábito de la lectura genera un mejor desempeño
académico? Si no es posible observar casos contrafácticos, significa
que hemos perdido una explicación causal de por qué Ana Sandoval tuvo
un mejor desempeño que Jorge Monreal. Simplemente porque no podemos
observar que habría sido de Ana si hubiera leído menos de un libro al
año, o bien, qué hubiera sido de Jorge si hubiera leído al menos un
libro al año.

Esto implica un problema acerca del significado o valor de verdad de
condicionales contrafácticos que tienen como argumento un nombre
propio: podemos hacer inferencias de cómo el grupo de prueba y el
grupo de control difieren, y podemos determinar si esto implica una
relación causal entre las variables \emph{leer} y \emph{desempeño
	académico}. Resta además señalar exactamente cuál es el papel que
juegan estas variables y cómo. Por ejemplo, es completamente
plausible que \emph{obligar} a los estudiantes tenga una efectividad
diferente a \emph{incentivar} a los estudiantes. Por supuesto,
quienes usan este tipo de métodos en su investigación tienen presente
que con hemos perdido causas particulares, pero creo que todos
podemos acordar que estos métodos son en efecto explicativos y que es
importante aislar factores causales.

Por el momento, no voy a entrar en detalles sobre qué son y cual es
la naturaleza los condicionales contrafácticos, porque eso ya es
parte de mi proyecto, por ahora sólo basta mencionar que los
condicionales contrafácticos encuentran su aplicación en estos
contextos para, digamos \say{crear} un evento alterno (si asumimos
que no hay sesgos en los grupos de control y de prueba, y que hemos
eliminado causas comunes dentro de nuestro análisis), en principio,
seremos capaces de derivar una conclusión causal \cite{Pearl2016,
	Otsuka2023}.

Podemos encontrar en la literatura contemporánea teorías de la
explicación científica que incluyen factores causales. Por ejemplo,
Potochnik nos dice que \say{[$\ldots$] I make the case that much of
	science is profitably understood as the search for causal patterns. }
\parencite[][p.~24] {Potochnik2017-POTIAT-3}

Este \say{espíritu causal} nota incluso en libros de texto. Por
ejemplo, cabe notar que en \cite{Pearl2016}, los autores inician
señalando que la causalidad \emph{debería} ser empleada en nuestras
explicaciones estadísticas, en la página 1, los autores comentan

\begin{quote}
	We study causation because we need to make sense of data, to guide
	actions and policies, and to learn from our success and failures.
	We need to estimate the effect of smoking on lung cancer, of
	education on salaries, of carbon emissions on the climate.
\end{quote}

Un poco más adelante, nos dicen que

\begin{quote}
	When approached rigorously, causation is not merely an aspect of
	statistics; it is an addition to statistics, an enrichment that
	allows statistics to uncover workings of the world that traditional
	methods alone cannot.
\end{quote}

\paragraph{No voy a hacer una defensa} de la teoría \emph{manipulabilista} de la causalidad. Eso lo hice en
mi proyecto de maestría---sea cual sea el éxito que haya tenido esta
defensa---, en particular, intenté elucidar cómo una teoría
manipulabilista lidia con explicaciones causales en biología
evolutiva. Mi proyecto actual nace de preocupaciones que restaron de
ese proyecto. En particular, si es verdad que en la investigación
científica se hace uso de condicionales contrafácticos, entonces vale
la pena preguntarse (i) \emph{¿cómo sabemos qué condicionales
	contrafácticos son \say{verdaderos}? }.

Hay mucho que desempaquetar en la pregunta (i). Primero, la pregunta
es parte de un condicional, esto es intencional. De este condicional
quiero extraer la conclusión de que de hecho vale la pena esa
pregunta. Presentar este condicional es parte de la motivación del
proyecto. Hasta ahora he mencionado brevemente las aplicaciones que
hacen mención de contrafácticos, lo cual ofrece una motivación para
preguntar (i). Supongamos por un momento que dichos condicionales de
hecho son parte de la investigación --digamos que son parte de los
métodos que usan los investigadores. El capítulo 1 lidia con cómo
estos condicionales son y 2 lidian con No sólo eso, sino que
responder a (i) equivale a explicar por qué la conclusión tiene
sentido, de acuerdo al marco teórico en el que se desarrollará.

Mi proyecto se concentra en resolver por qué \emph{la verdad} juega
un papel importante en la investigación científica y luego en cómo
\emph{justificar} qué condicionales contrafácticos cumplen este
criterio. Mi proyecto se concentrará en el contexto de las
inferencias causales, y cómo echan mano de \emph{condicionales
	contrafácticos.} Si son estos condicionales los que nos permiten
trazar relaciones causales según la teoría \emph{manipulabilista},
entonces más nos vale saber cómo determinar el valor de verdad. Una
motivación que se suele mencionar en la literatura para introducir
estos métodos es la llamada \say{paradoja de Simpson. } Esto no es
estrictamente una paradoja, pero aparece cuando los datos se
comportan de manera poco intuitiva
\parencite[p.~13]{Hajek2016-HAJOHO}, Algunos autores han señalado que
la solución radica en ser capaces de ofrecer un método para obtener
relaciones causales a partir de datos estadísticos.

Este tema --el de la explicación causal-- está fuertemente vinculado
con el uso de herramientas y \emph{modelos} estadísticos en
investigación. Mi propósito en este trabajo es hacer claro cómo
funcionan dichos modelos y qué herramientas nos sirven para
justificar que en efecto dos variables están causalmente
relacionadas. Esto es importante, porque en muchos casos se suele
decir que los \emph{modelos} usados en investigación, sea cual sea su
uso, son \say{falsos}, sentido que suele expresar como
\say{[$\ldots$] models typically only hold \emph{approximately,} in
	some ranges of circumstances, and they liberally employ idealizations
	to accomplish this \emph{partial} fit.} \parencite[p.~18, énfasis
	agregado]{Potochnik2017-POTIAT-3} Potochnik también afirma que buena
parte de la investigación se centra en ofrecer explicaciones causales
y, más aún, afirma que una buena manera de lidiar con explicaciones
causales en investigación científica es adoptando una teoría causal
\emph{manipulabilista,} algo con lo que estoy completamente de
acuerdo, como espero haya quedado claro. Potochnik también defiende
que el uso de modelos en investigación --y cualquier supuesto que
imponga ciertas condiciones en el modelo-- surge de la necesidad de
que seres con capacidades cognitivas limitadas queremos representar
fenómenos naturales para aislar los factores causales bajo
escrutinio. Es importante poner énfasis en esto porque implica que
los modelos \emph{son generados} con ciertos objetivos en mente, o
como ella misma señala \say{[$\ldots$] the nature of these
	idealizations is relative to the aim of research, as is most clearly
	demonstrated by the different idealizations involved in the various
	approaches to human aggression research. }

Mi investigación actual lidia con cómo \emph{determinar} el valor de
verdad de condicionales contrafácticos, usados para formular
hipótesis estadísticas, hipótesis que se justifican haciendo uso de
modelos ---a veces modelos de inferencia causal, o de otro tipo---,
mientras afirmamos que los modelos son falsos. En principio, este no
es un problema grave si la \say{verdad} no es una parte central de la
investigación científica (en cualquier disciplina).

Pero si la investigación se propone \emph{justificar} hipótesis para
obtener \emph{conocimiento} de relaciones causales reales, entonces
tenemos un problema. ¿Cómo los modelos, que no representan
adecuadamente el fenómeno, pueden dar como resultado nuevo
conocimiento? El problema, me parece, radica enla discusión en torno
a la representación. La relación que guarda el modelo con el objetivo
que pretende representar.

Mi hipótesis es que cierta forma de verdad debe estar involucrada en
nuestra definición de \emph{conocimiento,} si es que queremos que las
intervenciones en un sistema sean de alguna utilidad. Por ejemplo, si
el aumento del salario mínimo causa un aumento en el desempleo y qué
condiciones juegan un papel causal en dicha relación.

\paragraph{Regresando a la anécdota,} el hecho de que estos dos profesores buenos trabajaran filosofía de
la ciencia --al menos lo que en ese momento creía que era la
filosofía de la ciencia-- sesgó mis intereses a lo que hasta ahora ha
sido mi trabajo. La influencia de estos dos profesores me hizo tomar
la decisión de entrar al posgrado en Filosofía de la Ciencia.

En las clases de maestría hubo profesores excelentes. Mucho de lo que
aprendí en estos cursos me sirvió para enmarcar de diferente manera
las preguntas que me preocupaban, lo cuál me permitió entender
--mejor, me parece-- la naturaleza de la investigación científica. En
particular quería responder preguntas sobre la epistemología de la
ciencia, preguntas del tipo: \say{¿qué estamos justificados a
	creer?}, \say{¿cuándo podemos afirmar que una hipótesis ha sido
	corroborada? }, \say{¿es la explicación o compresión más básica que
	el conocimiento?}, etc. Preguntas que surgen a partir de mi interés
por la explicación causal.

Ahondar en estas preguntas y tratar de responderlas es una tarea más
que complicada. Muchos de los obstáculos se deben a que la práctica
científica es más caótica y heterogénea de lo que parece en un
principio. Incluso si nos concentramos en el uso de herramientas
estadísticas o herramientas que echan mano de la teoría de la
probabilidad, en casi todos los casos, ciertas propiedades del
\emph{fenómeno en cuestión} son idealizadas o ignoradas para hacer
que el modelo estadístico sea de utilidad.

Esta representación parcial del fenómeno se debe a que en muchos
casos buscamos destacar algunas propiedades particulares del
fenómeno, sin que todas las variables estén representadas en el
modelo. Que no involucremos todas y cada una de las variables que
afectan el fenómeno es un reflejo de nuestra limitada capacidad
cognitiva. A lo largo de su escrito, Potochnik nos recuerda que los
modelos diseñados en la investigación científica tienen un propósito
particular. Son usados para destacar algunas propiedades
particulares, especialmente, especialmente si lo que nos interesa son
las relaciones causales. Que simplifiquemos de ciertas manera el
modelo se debe a que los investigadores intentan resaltar las
características de un fenómeno en un momento particular.

Potochnik no es la única que ha señalado esto,
\textcite[][p.24]{abrams2023evolution}, por ejemplo, nos recuerda que
\say{So-called infinite-population models are simply models that have
	no role for drift. Biologists do often say things about evolution in
	infinite populations, and these claims are usually correct: the role
	of this terminology \emph{in practice} implies that it should not be
	understood literally. } (énfasis agregado)

A pesar de que los modelos no son completamente adecuados, --en el
sentido en el que no representan totalmente el fenómeno en
cuestión,-- son ampliamente utilizados en la investigación
científica. Si estas herramientas son usadas ampliamente, entonces
como filósofos de la ciencia hay que ofrecer un análisis de cómo
exactamente los modelos sirven para justificar hipótesis si parten de
supuestos obviamente falsos.

Me parece que buena parte del problema está relacionado con
determinar cuál es el papel que juega la \emph{representación
	adecuada} y la \emph{verdad} en la investigación científica. Para
motivar mi afirmación anterior, quiero dar un repaso por un periodo
de la filosofía de la ciencia, un periodo en el cuál --suele
decirse-- la \emph{verdad} jugaba un papel central en los análisis
filosóficos de la ciencia: me refiero al periodo en el que los
miembros del Círculo de Vienna estuvieron vivos. Este repaso sirve
para motivar la afirmación de que la \emph{verdad} juega un papel en
la investigación científica --por el momento no asumiré cuál es ese
papel,-- mientras que esto es al mismo tiempo compatible con la
\say{representación parcial} que hacen los modelos de un fenómeno.

Para comenzar este repaso, debo señalar que en la maestría aprendí
nuevas metodologías de investigación, además de diferentes maneras de
plantear y entender las preguntas que me preocupaban. Pero lo más
valioso que aprendí fue la importancia que tiene la historia de la
ciencia en la filosofía de la ciencia. Me voy a permitir hacer una
breve caracterización, exageradamente general, de dos posturas en
historia de la ciencia.

En la sección 4 de su artículo, Salmon motiva su discusión al tratar
de resolver el problema del caso único. Este problema parece funesto
para la interpretación frecuentista de la probabilidad (que es la
interpretación que Salmon favorece a lo largo de su artículo) y su
uso en explicación. Salmon cree que no es así. Para justificar esto,
desarrolla un aparato teórico para tratar con dicho problema. Una
parte importante de De acuerdo con Salmon, explicar un fenómeno
consiste en detectar las diferentes variables ---estadísticamente
relevantes--- para que un fenómeno ocurra.
