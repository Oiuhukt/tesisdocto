% !TEX root = ../main.tex
% LTeX: language=es, en

\chapter*{Introducción}
\label{ch:introduction}

\section{Una breve anécdota}

\noindent El doctorado ha sido un largo viaje durante el cuál
aprendí mucho. La parte más importante del aprendizaje está
relacionado con la filosofía,\sidefootnote{
	Juro que este breve relato tiene un punto.
} y la alta calidad del posgrado me permitió profundizar en los
temas que me interesaban. Mi pasión en ellos nació durante mi
licenciatura. Me fascinaron los temas que vimos en clase de
filosofía de la ciencia, en particular, debido a que durante este
periodo tuve dos profesores que se dedicaban a la filosofía de la
ciencia\sidefootnote{
	Al menos, lo que en ese momento pensaba que era la filosofía
	de la ciencia
}. Ambos excelentes maestros y filósofos.

Mi atención la dirigí especialmente a los temas de la \emph
{explicación científica} y la \emph{naturaleza de la causalidad
}. Mis intuiciones ---todavía queda un resabio de esto--- sugerían
que encontrar la causa de un fenómeno es la mejor manera de
explicarlo. Por ejemplo, que la mejor explicación de por qué tengo
cáncer de pulmón se debe a mis hábitos de fumador, y fumar \emph{es
	una} causa de cáncer.

Que la causalidad debería ser parte de nuestra teoría de la
\emph{explicación científica,} es una intuición que no está muy
alejada de algunas teorías clásicas. La motivación más clara para
incluir una relación causal en nuestras teorías de la explicación
científicas se debe a que la explicación de un fenómeno es \emph
{asimétrica}, algo que Sylvain Bromberger hizo notar en su artículo
\citetitle{Bromberger1966}. Para presentar el argumento de
Bromberger, primero quiero describir brevemente lo que Hempel y
Oppenheim dijeron en torno a la explicación científica\sidefootnote{
	Uso el término para referirme a lo que hacen las personas que se
	dedican a la investigación en cualquier área del conocimiento,
	cuando decimos que estas personas \emph{explican} un fenómeno.
}.

\paragraph{Hempel y Oppenheim} en \citetitle{Hempel1948},
caracterizaron a la explicación científica como un argumento
deductivo. Sabemos, por ejemplo, que todos los metales se dilatan
cuando se calientan. Si calentamos un pedazo de cobre particular,
entonces debería dilatarse. Si esto es verdad, entonce lo que
explica por qué este pedazo de cobre se dilata cuando lo caliento es
el hecho de que es una instancia particular de un cuantificador
universal. Nótese que es crucial la oración cuantificada
universalmente. En principio, esta oración debe tener la forma de
una \emph{ley general}. Si estas condiciones se cumplen, entonces
podamos derivar como caso particular el fenómeno en
cuestión.

Sin embargo, Bromberger en \citetitle{Bromberger1966} detecta un
problema con esta caracterización de la explicación científica.
Bromberger comienza argumentando que hay muchas similitudes entre
explicar que este pedazo de cobre se dilata y responder a la
pregunta \say{¿por qué este pedazo de cobre se dilata cuando lo
	caliento?}, señala, por ejemplo, que

\begin{quote}
	$\ldots$ there are issues in the philosophy of science that
	warrant an interest in the nature of why-questions. The most
	obvious of	these issues are whether science (or dome branch of
	science or	some specific scientific doctrine or some approach)
	ought to,	can, or does provide answers to why-questions, and if
	so, to	which ones. (p.~89)
\end{quote}


Hay un fuerte sentido intuitivo en la afirmación de Bromberger. No
es exagerado suponer que la investigación científica se dedica a
responder respuestas de este tipo\sidefootnote{Es una cuestión
	distinta si la investigación científica puede responder a \emph
	{todas} las preguntas de este tipo.
}: es claro que una explicación suele ser una respuesta a una
pregunta \emph{por-qué}. Sin embargo, si la doctrina hemepeliana es
correcta, entonces la  respuesta a la pregunta \say{¿por qué la
	altura del Empire State es de 381 metros?} debería responderse de la
siguiente manera: supongamos que hay un punto en quinta avenida que
está a $x$ distancia del Empire State. Dadas las leyes de la óptica
y el ángulo que forma el rayo de sol desde el punto más alto del
edificio hasta el punto $x$, podemos derivar fácilmente la altura
del Empire State.

Este es un argumento deductivo del cual ---sustituyendo las
variables pertinentes,--- podemos adecuadamente concluir la altura
del Empire State. Si la doctrina hempeliana es correcta, entonces
esto significa que este argumento explica la altura del Empire
State, algo que claramente es incorrecto: nadie diría que el hecho
de que el ángulo del sol y la distancia al punto $x$ de Quinta
Avenida explica por qué el edificio mide 381 metros. Eso
probablemente debería explicarlo un arquitecto o quien sea que haya
planeado el edificio.

\paragraph{El problema con} lo que Bormberger llama \say{La doctrina
	hempleiana,} es que contraejemplos como este muestran que la teoría
es incorrecta. Estos contraejemplos atacan la simetría de la
explicación que surge a partir de la doctrina hempeliana. Pero las
explicaciones no son simétricas.

Si lo que señala Bromberger es verdad, entonces las condiciones de
Hempel no son suficientes para caracterizar el fenómeno de la
explicación científica. Durante este periodo de la filosofía, muchos
esfuerzos sfilosóficos se concentraron en resolver el problema de la
asimetría ---y algunos otros problemas derivados de la doctrina
hempeliana.--- Wesley Salmon, por ejemplo, discute en \citetitle
{Salmon1970}, cómo podríamos resolver el problema de la asimetría.

\paragraph{Wesley Salmon} discute el ejemplo de Bormberger. En
particular, Salmon menciona que la asimetría temporal es crucial
para explicar la altura del edificio, en la página 72, Salmon dice
que

\begin{quote}
	Although the sun, flagpole, and shadow are perhaps
	commonsensically regarded as simultaneous, a more sophisticated
	analysis shows that physical processes going on in time are
	involved. Photons are emitted by the sun, they travel to the
	vicinity of the flagpole where some are absorbed and some are
	not, and those which are not go on to illuminate the ground
\end{quote}

En breve, la afirmación principal que defiende Salmon es que si una
variable hace una diferencia en el resultado esperado, entonces
tenemos una explicación del resultado. Salmon reconoce, sin embargo,
que su teoría de la explicación científica es sensible al problema
de hacer pasar las correlaciones por relaciones de dependencia. En
estadística es común el eslogan \say{correlación no implica
	causalidad} y el problema que señala el autor es que si sólo
caracterizamos a la explicación como estadísticamente relevante,
entonces vamos a perder casos claros de explicación científica. Por
ejemplo, si tengo un dolor de cabeza debido a que no he tomado agua,
y sin darme cuenta me tomo una pastilla del frasco --que resulta que
es un dulce--, no puedo afirmar que fue la pastilla la que me quitó
el dolor de cabeza. En este caso fue el agua, que es el factor
causal relevante.

Para resolver esto, Salmon hace notar que su modelo de explicación
puede naturalmente incorporar \emph{explicaciones causales}, que es
un requerimento para resolver el problema de la asimetría de la
explicación, Salmon menciona, por ejemplo que

\begin{quote}
	The reason that the explanation of the length of the shadow in
	terms of the height of the flagpole is acceptable, whereas the
	\say{explanation} of the height of the flagpole in terms of the
	length of the shadow is not acceptable, seems to me to hinge
	directly upon the fact that there are causal processes with
	earlier and later temporal stages.
\end{quote}

Podemos encontrar en la literatura contemporánea, son destacables
las teorías de la explicación científica que incluyen factores
causales. Por ejemplo, Potochnik nos dice que \say{[$\ldots$] I make
	the case that much of science is profitably understood as the search
	for causal patterns.} \parencite[][p.~24]{Potochnik2017-POTIAT-3}

Este \say{espíritu causal} nota incluso en libros de texto. Por ejemplo, cabe notar que en \cite{Pearl2016}, los autores inician señalando que la causalidad \emph{debería} ser empleada en nuestras explicaciones estadísticas, en la página 1, los autores comentan

\begin{quote}
	We study causation because we need to make sense of data, to
	guide actions and policies, and to learn from our success and
	failures. We need to estimate the effect of smoking on lung
	cancer, of education on salaries, of carbon emissions on the
	climate.
\end{quote}

Un poco más adelante, nos dicen que

\begin{quote}
	When approached rigorously, causation is not merely an aspect of
	statistics; it is an addition to statistics, an enrichment that
	allows statistics to uncover workings of the world that
	traditional methods alone cannot.
\end{quote}

Esto es sólo una muestra del vínculo que hay entre causalidad y
explicación. Ahora bien, dentro de las teorías de la explicación que
involucran un componente causal, ha emergido la teoría
\emph{manipulabilista} de la explicación causal. Woodward desarrolló
dicha teoría y actualmente es uno de sus mayores exponentes.

Hablando particularmente de las teorías \emph{manipulabilistas} de
la causalidad \parencite{sep-causation-mani}, Potochnik afirma que
son dichas teorías las que mejor parecen adecuarse a cómo de hecho
se trabaja en investigación científica. Esto es importante, ya que
como filósofos de la ciencia, nuestro análisis debería estar
informado por cómo de hecho se trabaja en investigación
científica.\sidefootnote{
	Abrams \citeyear{abrams2023evolution}, por ejemplo, discute esto en el primer capítulo de su libro.
}

Si bien no hay un consenso universal sobre la relevancia de la
causalidad en el fenómeno de la explicación científica, con la
discusión anterior quiero sugerir que la causalidad es una
característica fundamental del proceso de explicación científica.
Dicho esto, quiero presentar brevemente en qué consiste una teoría
manipulabilista de la causalidad.

Brevemente, las teorías \emph{manipulabilistas}\sidefootnote{
	La siguiente caracterización está basada en
	\cite{sep-causation-mani, sep-causal-explanation-science, Woodward2004, Woodward2000-WOOEAI}
} afirman que dadas dos variables $A$ y $B$, decimos que $A$ \emph{causa} $B$ si al intervenir (manipular) la variable $A$, la variable $B$ cambia en consecuencia.
Esto, por supuesto, sólo es una parte de la historia de las teorías
\emph{manipulabilistas}. Supongamos que el fenómeno a analizar es
estable, es decir., que cualquier cambio de valor en la variable $A$,
implica un cambio proporcional en $B$. Si hay una función bien
definida sobre un conjunto, la tarea es ligeramente sencilla.
Pensemos, por ejemplo, en una función cualquiera de una variable.
Digamos que la función está definida de la siguiente manera $f(x) =
	2x$, donde $x \in {N}$. Esto significa que si la función toma el $3$
como valor de entrada, el valor de salida será $6$. Supongamos que
estos son los valores de una medida de fuerza y que la relación es la
siguiente: Si la fuerza $A$ es de 1 unidad, entonces la distancia $B$
trasladada de un proyectil es de 2.\sidefootnote{
	Por supuesto esto es púramente ficticio, se trata sólo de un
	ejemplo ilustrativo, el punto es que la función de distancia es
	proporcional a la medida de fuerza
}

Digamos entonces que aplicar 3 medidas de fuerza, producen
\emph{cuatro metros} de distancia de un proyectil. Supongamos además
que esta función es \emph{independiente} de cualquier fuerza externa,
por ejemplo, la resistencia del medio (en este caso el aire)
--insisto en que esto es sólo un caso para ejemplificar.

Un teórico de la aproximación causal intervencionista, diría que
esto significa que la fuerza impresa en el proyectil, \emph{causa}
que el proyectil alcance más distancia. Además es posible que la
función definida anteriormente no sea correcta para algunos de los
valores de $A$, por ejemplo, si imprimo una fuerza de $1*10^{-7}$
unidades, probablemente el proyectil no se traslade $2*10^{-7}$
unidades.

El ejemplo anterior sirve para describir algunas de las características de la teoría \emph{manipulabilista.}
Que la función describa correctamente el comportamiento del
proyectil, \emph{independiente} de otra fuerza, quiere decir que la
fuerza impresa en el proyectil es la única variable que afecta la
distancia del mismo.

Que la \emph{intervención} en la fuerza impresa cause que el
proyectil llegue a mas distancia --\emph{independiente} de cualquier
otra fuerza,-- significa que en la medida de los posible hemos
decidido dejar variables (como la resistencia del aire) exógenas:
sabemos que pueden afectar el comportamiento, pero nos concentramos
en explicar la distancia en función de la fuerza. Si además podemos
excluir todas las variables, excepto la fuerza, entonces es posible
hacer una intervención \say{quirúrgica} y que sólo haya ciertos
valores que describen correctamente el comportamiento del proyectil,
significa que es \emph{invariante} bajo un rango de intervenciones.
En esto, a grandes rasgos, consiste una teoría \emph{manipulabilista}
de la causalidad.

Por supuesto, hay otras características integran la teoría \emph
{manipulabilista} de la causalidad. Quizás la más controversial de
ellas es que la teoría hace uso de condicionales
\emph{contrafácticos.} Los condicionales \emph{contrafácticos} son
aquellos condicionales que evaluamos bajo el supuesto de que algo no
sucede, pero que nos obligan a preguntarnos que pasaría si se
cumplieran ciertas condiciones particulares.
Dicho de otro modo, nos preguntamos cómo sería el mundo si, por
ejemplo, México no hubiera sido conquistado por los españoles.

Parece más que natural implementar estos condicionales en la teoría
intervencionista de la causalidad. Porque dado que las variables
pueden ser tanto discretas como continuas, podríamos analizar
preguntas del tipo \say{¿qué hubiera sucedido si $A$ hubiera tomado
	un valor distinto al que tomó?}. Si la función está bien definida,
como en mi ejemplo, entonces es relativamente sencillo determinar
valores de distancia para diferentes valores de fuerza\sidefootnote{
	En investigación, se suele mencionar el uso de condicionales
	\emph{contrafácticos}, principalmente en inferencias causales,
	por ejemplo \cite{Pearl2016, Pearl2018}.
}

Me parece que si tomamos en serio el trabajo que realizan las
personas cuando hacen investigación, es claro que el uso de estas
herramientas juega un papel en su trabajo y, de manera particular,
los modelos estadísticos mencionan condicionales
\emph{contrafácticos} como parte de sus análisis. Entre dichos
métodos, resalta \emph{la inferencia causal}, que es un método
estadístico que se utiliza para erivar conclusiones causales a
partir de agregación de datos e inferencia estadística. Estos
métodos sirven, sobre todo, cuando utilizamos modelos estadísticos
para justificar hipótesis causales \parencite{Pearl2016, Pearl2018}.

Un ejemplo de este tipo de métodos ---en el cuál se usan
métodos de inferencia causal---, es en las \emph{Pruebas Controladas
	Aleatorizadas} (RCT, por sus siglas en inglés), donde típicamente,
los investigadores interesados en saber cuál es la efectividad de un
medicamento, intervienen en una de las variables --dándole un
placebo a un grupo de control, mientras que otro grupo recibe el
medicamento a testar. Asuminedo que no hay sesgos en la muestra, si
el grupo de prueba responde al medicamento, entonces, en principio,
seremos capaces de detectar si el medicamento hace una aportación
causal. En corto: que podemos pensar que el grupo de control
funciona como una especie de imagen de lo que \emph{habría} pasado
de no haber tomado los medicamentos.

Hasta este momento de la exposición, estoy hablando de cómo los
\emph{condicionales contrfácticos} como un elemento de que es parte
de la \emph{investigación científica}. Lo que me interesa de estos
ejemplos es resaltar el hecho de que \emph{las inferencias causales,} que modelan fenómenos usando \emph{contrafácticos,} son parte del trabajo usual de los investigadores. Algo que queda reflejado en \parencite{Pearl2016}, en \parencite{Pearl2018} y especialmente en
\parencite{llaudet2023}.

Sin embargo, el hecho de que los contrafácticos sean parte de estos
métodos usados en in vestigación, no significa que todos los
problemas están resueltos. En particular, me importa señalar un
problema que ha recibido el nombre de \say{el problema fundamental
	de la inferencia causal.} Brevemente, el problema consiste en que no
es posible saber el valor de verdad de un \emph{contrafáctico} a
nivel del individuo. Por ejemplo, supongamos que queremos saber si
los estudiantes que leen un libro al año, tienen un mejor desempeño
académico que aquellos alumnos que leen menos de un libro al año.
Supongamos, aún más, que hemos diseñado una prueba aleatoria
randomizada y que hemos aislado existrosamente a los alumnos que
leen más de un libro al año. Si nos preguntamos qué hubiera sucedido
si los alumnos no leyeran al menos un libro al año. Para responder a
la pregunta \say{¿qué habría pasado si ...?} tenemos que ver cómo
fue el desempeño del grupo de control, es decir, aquel que leyó
menos de un libro al año. La única información que tenemos,
entonces, será el agregado estadístico de los alumnos que leyeron
menos de un libro al año. Sin embargo, esto significa que nunca
seremos capaces de observar, digamos, que hubiera sido de Ana
Sandoval si hbiera leído menos de un libro al año. Easto significa
que hemos perdido una explicación causal de por qué Ana Sandoval
tuvo un mejor desempeño que Jorge Monreal. Simplemente porque no
podemos observar que habría sido de Ana si hubiera leído menos de un
libro al año, o bien, qué hubiera sido de Jorge si hubiera leído al
menos un libro al año.

Esto implica un problema acerca del significado o calor de verdad de contrafácticos que tienen como argumento un nombre propio: podemos hacer inferencias de cómo el grupo de prueba y el grupo de control difieren, y si esto implica una relación causal entre \emph{leer} y \emph{desempeño académico}
Resta señalar exactamente cuál es el papel que juegan  Quienes  Además, quienes usan este tipo de métodos en su investigación tienen presente que con este método hemos perdido causas particulares \emph{tokens} que toman eventos como relata. Por ejemplo,  de la forma

Por el momento, no voy a entrar en detalles sobre qué son y cual es la naturaleza los condicionales contrafácticos, eso es algo que planeo hacer en un capítulo posterior. Para fines de exposición, sólo basta mencionar que los condicionales contrafácticos encuentran su aplicación en estos contextos para, digamos \say{crear} un evento alterno (Si asumimos que no hay sesgos en los grupos de control y de prueba, y que hemos eliminado causas comunes dentro de nuestro análisis), en principio, seremos capaces de derivar una conclusión causal \cite{Pearl2016, Otsuka2023}.

\paragraph{No voy a hacer una defensa} de la teoría
\emph{manipulabilista} de la causalidad. Eso lo hice en mi proyecto
de maestría--sea cual sea el éxito que haya tenido esta defensa--,
en particular, intenté elucidar cómo una teoría manipulabilista
lidia con explicaciones causales en biología evolutiva. Mi
proyecto actual nace de preocupaciones que restaron de ese proyecto.
En particular, si es verdad que en la investigación científica se
hace uso de condicionales contrafácticos, entonces vale la pena
preguntarse (i) \emph{¿cómo sabemos qué condicionales contrafácticos son \say{verdaderos}?}.

Hay mucho que desempaquetar en la pregunta (i). Primero, la
pregunta es parte de un condicional, esto es intencional. De este
condicional quiero extraer la conclusión de que de hecho vale la
pena esa pregunta. Presentar este condicional es parte de la motivación del proyecto. Hasta ahora he mencionado brevemente las aplicaciones que hacen mención de contrafácticos, lo cual ofrece una motivación para preguntar (i). Supongamos por un momento que dichos condicionales de hecho son parte de la investigación --digamos que son parte de los métodos que usan los investigadores. El capítulo 1 lidia con cómo estos condicionales son  y 2 lidian con   No sólo eso, sino que responder a (i) equivale a explicar por qué la conclusión tiene sentido, de acuerdo al marco teórico en el que se desarrollará.

Mi proyecto se concentra en resolver por qué \emph{la verdad} juega un papel importante en la investigación científica y luego en cómo \emph{justificar} qué condicionales contrafácticos cumplen este criterio.

Mi proyecto se concentrará Parece que queda claro que herramientas estadísticas, en el contexto de las inferencias causales, echan mano de \emph{condicionales contrafácticos.}
Además, son estos condicionales, los que nos permiten trazar relaciones causales según la teoría \emph{manipulabilista}
Una motivación que se ha mencionado en la literatura para introducir métodos que nos permitan trazar relaciones causales es la llamada \say{paradoja de Simpson.}
Esto no es estrictamente una paradoja, pero aparece cuando los datos se comportan de manera poco intuitiva \parencite[p.~13]{Hajek2016-HAJOHO},
Algunos autores han señalado que la solución radica en ser capaces de ofrecer un método para obtener relaciones causales a partir de datos estadísticos.

Este tema --el de la explicación causal-- está fuertemente vinculado con el uso de herramientas y \emph{modelos} estadísticos en investigación.
Mi propósito es hacer claro cómo funcionan dichos modelos y qué herramientas nos sirven para justificar que en efecto dos variables están causalmente relacionadas.
Esto es importante, porque en muchos casos se suele decir que los \emph{modelos} usados en investigación, sea cual sea su uso, son \say{falsos}

\paragraph{Se suele señalar} que hay un sentido en el cuál los modelos son \say{falsos}, sentido que suele expresar como \say{[$\ldots$] models typically only hold \emph{approximately,} in some ranges of circumstances, and they liberally employ idealizations to accomplish this \emph{partial} fit.} \parencite[p.~18, énfasis agregado]{Potochnik2017-POTIAT-3}

Potochnik también afirma que buena parte de la investigación se centra en ofrecer explicaciones causales y, más aún, afirma que una buena manera de lidiar con explicaciones causales en investigación científica es adoptando una teoría causal \emph{manipulabilista.}
Los argumentos de Potochnik que vinculan a la práctica de investigación con las teorías \emph{manipulabilistas}, dejan claro que hay una genuina preocupación sobre cómo de hecho se realiza investigación.

Potochnik también defiende que el uso de modelos en investigación --y cualquier supuesto que imponga ciertas condiciones en el modelo-- surge de la necesidad de que seres con capacidades cognitivas limitadas queremos representar fenómenos naturales para aislar los factores causales bajo escrutinio.
Es importante poner énfasis en esto porque implica que los modelos \emph{son generados} con ciertos objetivos en mente, o como ella misma señala \say{[$\ldots$] the nature of these idealizations is relative to the aim of research, as is most clearly demonstrated by the different idealizations involved in the various approaches to human aggression research.}

Mi investigación actual lidia con la afirmación de que los modelos son falsos.
Este no es un problema grave si la \say{verdad} no es una parte central de la investigación científica (en cualquier disciplina).
Pero si la investigación se propone \emph{justificar} hipótesis para obtener \emph{conocimiento} de relaciones causales reales, entonces tenemos un problema.
¿Cómo los modelos, que no representan adecuadamente el fenómeno, pueden dar como resultado conocimiento?
Si suponemos que el conocimiento es fáctico, entonces entramos en problemas.

El problema, me parece, radica en cómo representamos adecuadamente un fenómeno.
Debo aclarar también cómo es que funciona la representación científica y cómo la adecuación parcial de modelos que usan contara fácticos, implica fenómenos fácticos.

Mi hipótesis es que cierta forma de verdad debe estar involucrada en nuestra definición de \emph{conocimiento,} si es que queremos que las intervenciones en un sistema sean de alguna utilidad.
Por ejemplo, si el aumento del salario mínimo causa un aumento en el desempleo y qué condiciones juegan un papel causal en dicha relación.

\paragraph{Regresando a la anécdota,} el hecho de que estos dos profesores buenos trabajaran filosofía de la
ciencia --al menos lo que en ese momento creía que era la filosofía de la ciencia-- sesgó mis intereses a lo que hasta ahora ha sido mi trabajo.
La influencia de estos dos profesores me hizo tomar la decisión de entrar al posgrado en Filosofía de la Ciencia.

En las clases de maestría hubo profesores excelentes.
Mucho de lo que aprendí en estos cursos me sirvió para enmarcar de diferente manera las preguntas que me preocupaban, lo cuál me permitió entender --mejor, me parece-- la naturaleza de la investigación científica.
En particular quería responder preguntas sobre la epistemología de la ciencia, preguntas del tipo: \say{¿qué estamos justificados a creer?}, \say{¿cuándo podemos afirmar que una hipótesis ha sido corroborada?}, \say{¿es la explicación o compresión más básica que el conocimiento?}, etc.
Preguntas que surgen a partir de mi interés por la explicación causal.

Ahondar en estas preguntas y tratar de responderlas es una tarea más que complicada.
Muchos de los obstáculos se deben a que la práctica científica es más caótica y heterogénea de lo que parece en un principio.
Incluso si nos concentramos en el uso de herramientas estadísticas o herramientas que echan mano de la teoría de la probabilidad, en casi todos los casos, ciertas propiedades del \emph{fenómeno en cuestión} son idealizadas o ignoradas para hacer que el modelo estadístico sea de utilidad.

Esta representación parcial del fenómeno se debe a que en muchos casos buscamos destacar algunas propiedades particulares del fenómeno, sin que todas las variables estén representadas en el modelo.
Que no involucremos todas y cada una de las variables que afectan el fenómeno es un reflejo de nuestra limitada capacidad cognitiva.
A lo largo de su escrito, Potochnik nos recuerda que los modelos diseñados en la investigación científica tienen un propósito particular.
Son usados para destacar algunas propiedades particulares, especialmente, especialmente si lo que nos interesa son las relaciones causales.
Que simplifiquemos de ciertas manera el modelo se debe a que los investigadores intentan resaltar las características de un fenómeno en un momento particular.

Potochnik no es la única que ha señalado esto,
\textcite[][p.24]{abrams2023evolution}, por ejemplo, nos recuerda que \say{So-called infinite-population models are simply models that have no role for drift.
	Biologists do often say things about evolution in infinite populations, and these claims are usually correct: the role of this terminology \emph{in practice} implies that it should not be understood literally.} (énfasis agregado)

A pesar de que los modelos no son completamente adecuados, --en el sentido en el que no representan totalmente el fenómeno en cuestión,-- son ampliamente utilizados en la investigación científica.
Si estas herramientas son usadas ampliamente, entonces como filósofos de la ciencia hay que ofrecer un análisis de cómo exactamente los modelos sirven para justificar hipótesis si parten de supuestos obviamente falsos.

Me parece que buena parte del problema está relacionado con determinar cuál es el papel que juega la \emph{representación adecuada} y la \emph{verdad} en la investigación científica.
Para motivar mi afirmación anterior, quiero dar un repaso por un periodo de la filosofía de la ciencia, un periodo en el cuál --suele decirse-- la \emph{verdad} jugaba un papel central en los análisis filosóficos de la ciencia: me refiero al periodo en el que los miembros del Círculo de Vienna estuvieron vivos.
Este repaso sirve para motivar la afirmación de que la \emph{verdad} juega un papel en la investigación científica --por el momento no asumiré cuál es ese papel,-- mientras que esto es al mismo tiempo compatible con la \say{representación parcial} que hacen los modelos de un fenómeno.

Para comenzar este repaso, debo señalar que en la maestría aprendí nuevas metodologías de investigación, además de diferentes maneras de plantear y entender las preguntas que me preocupaban.
Pero lo más valioso que aprendí fue la importancia que tiene la historia de la ciencia en la filosofía de la ciencia.
Me voy a permitir hacer una breve caracterización, exageradamente general, de dos posturas en historia de la ciencia.
