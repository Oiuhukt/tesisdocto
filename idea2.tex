\documentclass[12pt]{article}
\usepackage{setspace}
\setstretch{1.5}
\usepackage[utf8]{inputenc}
\usepackage[T1]{fontenc}
\usepackage{url}
\usepackage[spanish]{babel}
\usepackage{apacite}
\usepackage{fullpage}

\begin{document}

\section{Idea 2}

A los seres humanos nos interesa la verdad. Nos sentimos, por ejemplo, molestos cuando alguien incumple una promesa. Esta importancia que le damos a la verdad guía también nuestras empresas epistémicas. Platón en su diálogo con Menón \cite[\P\P 97a-98b]{platonmeno} señala que hay un factor de seguridad ligado al conocimiento, que no se encuentra en la mera opinión verdadera\footnote{Por supuesto esto está más relacionado con la justificación que con la verdad propiamente. Sin embargo, cabe señalar que la verdad juega un papel importante en la explicación de lo que implica saber algo.}. Esta caracterización de conocimiento fue puesta en duda por Gettier en un famoso artículo \citeyear{Gettier}. La conclusión de Gettier es que la caracterización de Platón no captura condiciones suficientes para decir que un agente sabe.

A pesar de que las condiciones no sean suficientes, esto no implica que no sean necesarias. Aceptamos, por ejemplo, que la verdad es una condición necesaria para saber algo. Si algo es falso, entonces no lo sabes. Hablamos de creencias falsas, pero no de conocimiento falso. El ``saber que p'' tiene una estrecha conexión con la verdad. Por lo general admitimos que conocimiento implica verdad.

La condición de 'verdad' involucrada en el conocimiento es de un tipo diferente a las de creencia y justificación. Tanto creencia como justificación son condiciones epistémicas, la verdad, por otra parte, es una condición de otro tipo. Al menos creemos comúnmente que la verdad no es una condición epistémica, si lo fuera, nuestras creencias podrían modificar cómo es el mundo. El que yo crea que mi cortina es verde\footnote{De hecho es azul.}, no la hace verde. Adecuamos nuestras creencias y justificaciones para que encajen con el fenómeno que queremos comprender\footnote{En \cite{sep-knowledge-analysis} los autores señalan que esta es una condición metafísica. Trato de ser neutro con respecto a si es una condición metafísica, porque si asumimos una teoría pragmática de la verdad (al estilo de William James ), entonces la verdad es una condición epistémica. Como esto es apenas una presentación, evito el compromiso.}.  

Ahora bien, la investigación científica es probablemente la manera más sistematizada que tenemos los seres humanos para producir conocimiento. Muchas de nuestras explicaciones dependen de conocer algunos hechos acerca de las diferentes disciplinas científicas. Sabemos, por ejemplo, que para que haya una combustión se necesita combustible y oxígeno. Si cualquiera de estas variables está en 0, entonces no hay combustión. Sabemos, por tanto, que sin combustible u oxígeno, no hay combustión.

Podemos incluso citar ejemplos del uso contemporáneo de teorías, por ejemplo, la mecánica Newtoniana y la teoría de la selección natural de Darwin son teorías que usamos cotidianamente para explicar fenómenos. La teoría de la selección natural, por ejemplo, nos ayuda a explicar fenómenos biológicos como la adaptación y la especiación. Con base en esta teoría, podemos explicar, por ejemplo, por qué un grupo dentro de una población, tiene más descendencia que otro gurpo dentro de la misma población; podemos, también saber, por ejemplo, cuáles son los ancestros comunes de especies contemporáneas, \textit{e. g.}, las aves de los dinosaurios. Por su parte, utilizamos la teoría newtoniana para explicar el movimiento de los astros y hacer predicciones de qué posición tomarán en un momento dado, podemos explicar la fuerza que se imprime en una superficie cuando es golpeada por una masa con cierta aceleración.

A pesar de que ambas teorías nos permiten explicar un amplio rango de fenómenos, las dos tienen problemas. Sabemos que la distinción newtoniana: tiempo y espacio es falsa. Para poder explicar el movimiento, Newton desarrolló una teoría que tiene como entidades el \textbf{tiempo absoluto} y el \textbf{espacio absoluto}. Un cuerpo se mueve o permanece en reposo con respecto al espacio y tiempo absolutos. Todos los puntos en el espacio absoluto permanecen constantes durante diferentes intervalos temporales. Para explicar la llamada ``primera ley''  de Newton (inercia), es necesario definir qué significa que un cuerpo esté en reposo o en movimiento. Según Newton sabemos que un cuerpo está en movimiento, porque ocupa distintos puntos del espacio en diferentes intervalos de tiempo. Si el movimiento entre los puntos se da en intervalos iguales de tiempo, entonces el movimiento es uniforme. 

% El movimiento en el espacio absoluto, sin embargo, no puede ser percibido, mientras que las posiciones relativas de los cuerpos sí pueden ser percibidas. Esto, por supuesto no significa que haya que echar a la basura ambas entidades, ya que el espacio y tiempo absolutos juegan un papel en la explicación del movimiento newtoniana. 

Ambas entidades son parte fundamental de la teoría newtoniana, ya que permiten explicar los movimientos relativos \cite{Maudlin2014Filosofia:7985} La relatividad general señala que donde Newton disitnguía dos entidades, realmente sólo hay una: espacio-tiempo. 










\end{document}
