\documentclass[12pt]{article}
\usepackage{setspace}
\setstretch{1.5}
\usepackage[utf8]{inputenc}
\usepackage[T1]{fontenc}
\usepackage{url}
\usepackage[spanish]{babel}
\usepackage{apacite}
\usepackage{fullpage}

\begin{document}

\section{Idea 2}

A los seres humanos nos interesa la verdad. Nos sentimos, por ejemplo, molestos cuando alguien incumple una promesa. Esta importancia que le damos a la verdad guía también nuestras empresas epistémicas. Platón \cite{} definió al conocimiento como una creencia, verdadera y justificada. Por supuesto es debatible que esta sea una buena definición de conocimiento (no captura condiciones suficientes para decir que un agente sabe \cite{gettier}), sin embargo, parece que la verdad sí es una condición necesaria para saber algo \cite{pritchard}. El ``saber que p'' tiene una estrecha conexión con la verdad. en caso de que creas algo falso, entonces no sabes que 'p'. 

Ahora bien, la investigación científica es probablemente la manera más sistematizada que tenemos los seres humanos para producir conocimiento. Muchas de nuestras explicaciones dependen de conocer algunos hechos acerca de las diferentes disciplinas científicas. Sabemos, por ejemplo, que para que haya una combustión se necesita combustible y oxígeno. Si cualquiera de estas variables está en 0, entonces no hay combustión. Sabemos, por tanto, que sin combustible u oxígeno, no hay combustión.

Podemos incluso citar ejemplos del uso contemporáneo de teorías, por ejemplo, la mecánica Newtoniana y la teoría de la selección natural de Darwin son teorías que usamos cotidianamente para explicar fenómenos. La teoría de la selección natura, por ejemplo, nos ayuda a explicar fenómenos biológicos como la adaptación y la especiación. Con base en esto, sabemos, por ejemplo, cuáles son los ancestros comunes de especies contemporáneas, \textit{e. g.}, las aves de los dinosaurios.

Sin embargo, sabemos que la distinción newtoniana: tiempo y espacio es falsa. La relatividad general señala que donde Newton disitnguía dos entidades, realmente sólo hay una: espacio-tiempo. 










\end{document}
