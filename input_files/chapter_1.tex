%!TEX root = ../main.tex




    \dochaptoc

\chapter{What we can learn about abstraction from mathematics?}
\label{ch:wwlmath}

\section{Introduction}

Scientist use models in their research, in consequence, philosophers of science became interested\footnote{There has been interest in modeling strategies since the years of the Vienna Circle.} in the modeling strategies used in science.
Between the epistemological aspects of models, we have the debate on how models relate with their target, or how they "represent" their target, as \citeauthor{Frigg2020} say "If we want to understand how models allow us to learn about the world, we have to come to understand how they represent."\footcite[][p. x]{Frigg2020}

In this chapter, I want to address this question.
%This contribution is relevant for the workshop because I will discuss how the relation between model and world works in biology.
I will use two main examples: the "Hodgkin-Huxley model", as developed by themselves in a series of articles \parencite{Hodgkin1951, Hodgkin1952, Hodgkin1952a, Hodgkin1952b}, but I will focus on the paper \citetitle{Hodgkin1951} by Hodgkin.
Also I will contrast this paper with the work made by Piccolino and Bresadola\footcite{Piccolino2013}.

Also, I want to emphasise why having the relation between idealizations\/world wrong, leads researchers into trouble.
This will be argued having as example the reconstruction of the debate on IQ tests\footcite{Lewontin2017} by Lewontin, Rose and Kamin.

With these examples at hand, I want to argue against the artifactual theory of idealizations\footcite{Carrillo2021-CARAAP-12, Carrillo2022}.
Artifactualist argue that their theory is more adequate than others, and as such should be accepted.
The core motivation that the artifactualist has in favour of her theory is that it doesn't depend on the relation between idealizations and world.
The artifactualists argue that this point of departure has many advantages.

I will argue that actual research shows that the motivations behind the artifactualist theory are wrong, and that the reconstruction they make about the Hodgkin-Huxley model is misguided.
In this chapter I will try to deal with the need (or not) to clarify the relation between the models and the world.

\section{Main motivations}

It seems intuitive that if researchers care about natural phenomena, then they want their models to be somewhat accurate.
But this last point has been widely discussed.
Some philosophers have argued that this relation isn't necessary to give a good theory of models in science.
Models need to leave some properties of their targets out, if we want the model to be of any use. %13/06/24 Agregar referencias. Esto está en Potochnik, pero también se puede encontrar en Winther. Leer a Winther
So, they misrepresent their intended target.
And sometimes researchers misrepresents on purpose with the use of idealizations. %13/06/24 Agregar un par de referencias. Esto  lo dicen explícitamente la Natalia y la Tarja. Poner eso, también el libro de Suárez.
We have a problem here: Assuming that truth is valuable in research, if models misrepresents their target, it seems that they cannot be part of theories.
Because in some sense, we say that idealizations are "false", or that they "misrepresent" their target.

Philosophers have tried to account for this problem\sidenote{This is part of a larger problem about the value of Knowledge. Philosophers like \citeauthor{pritchard2021a} have argued that truth is indeed what gives value to our knowledge \citeyear{pritchard2021a, pritchardEpistemicValueCognitive2021}. While Philosophers like \citeauthor{elgin2017true}, have argued that truth is not what gives value to our knowledge \citeyear{elgin2017true, elgin2004}. This latter group of philosophers appeal to another kind of epistemic values like Understanding. I will talk of this topic in the next chapter. In this chapter I will deal with idealizations.}
If we got that truth is an important part of our theories, it seems that idealizations should be banned.
But idealizations are widely used in scientific research, so it seems that truth is not an important part of our theories.

We have a dilemma that we need to account: either truth isn't important or idealizations should be banned.
I personally think that this is a false dilemma, and that actual patterns of research show that it is.
So, we need to address why the use of models and a care for truth can be true at the same time.

Now, talking about models\/idealizations, there are different accounts on how to close this gap.
To close this gap, some philosophers have argued that idealizations serve as means.
The world is too complex for our human cognitive limitations, so we use idealizations to have a more manageable model.
This model use idealizations as a mean. As Potochnik says "to support human cognitive and practical ends." \footcite[][p. ix]{Potochnik2017-POTIAT-3}.

Within this stance, Potochnik says, we can idealize certain phenomena highlighting the causal structure of the phenomena at hand.
Idealizations are means to highlight the causal pattern we're interested in.
Let us call this the "causal relevance" theory.
It has to be clear that Potochnik affirms that there is no need to de-idealize the model: it has a lot of epistemic advantages to leave the model as it is, including the idealizations.

Other stance affirms that the idealized model is just temporal.
This group of philosophers argue that idealizations in the model function as aids: it has an epistemic advantage to distort the target phenomena momentarily, but new research will show how we must modify the idealization to make it more accurate.
Let us call this the "de-idealization" theory, or what \citeauthor{Weisberg2007} calls "galilean idealization" \footcite{Weisberg2007}.

Still another group of philosophers have argued that the previous stances are both wrong.
This philosophers argue that these stances are wrong because the "causal relevance" and the "de-idealization" stances, both make substantial assumptions on the relation between model and world.
This means that the causal relevance stance and the de-idealization stance\footnote{They use different labels for the "stances" that I have been characterizing here. What I call "causal relevance", they call it "epistemic benefits"; what I call "de-idealization", they call "distortion of reality".}, take too much care on the model and how accurate is representing the target.
They suggest that

\begin{quote}
	$\ldots$ in these and many other cases, idealization is better understood from an artifactual perspective that does not take the representational model-world relationship as a point of departure, presupposing the possibility of some straightforward comparisons between models and their supposed target systems \parencite[][p. 2]{Carrillo2021-CARAAP-12}.
\end{quote}

I have been using the word "stances" to describe the previous theories.
I have a reason for using this word.
I call them stances, because we are philosophers of science, trying to explain (i) why there are idealizations in scientific research, and (ii) which are the uses of those idealizations\footnote{I'm following Potochnik when she says that no actual philosopher of science will doubt that history of science and actual research activity are important to answer philosophical questions \citeyear[cfr.][p. 9]{Potochnik2017-POTIAT-3}. Also, the artifactualist seems to agree with this.}.
We take a stance that can help us to interpret the process of "making" idealizations in order to answer "(i)" and "(ii)".
Three stances are under debate here: "de-idealization" [DI], "causal relevance" [CR] and "artifactualist" [AR].

I will argue that the artifactualist doesn't have good answers to "(i)" and "(ii)".
I say that AR doesn't show how the DI and CR exclude what the artifactual stance achieves.
Second, I want to say that the motivation behind the artifactual stance is incorrect, because there is evidence showing that researchers do care about the relations between model and world.
Also, there is evidence about the dangers of getting this relation wrong.

I will also argue that the artifactualist is not really an alternative.
Much of the advantages that the artifactualist claim are advantages that we are already dealing in philosophy, almost since the time of the Vienna Circle.


\section{What the artifactualist claim}

As I said above, the artifactualist affirm that we do not need to have an accuracy measure between idealizations and the world to have a good theory of idealizations.
This change of stance, they argue, has a lot of benefits that the other stances lack of.
First they claim that "$\ldots$ the artifactual approach to modeling is to provide an alternative to the traditional accounts of models that assume that models give knowledge in virtue of accurately representing their target systems or their parts"\footcite[][pp. 51-52]{Carrillo2022}.

Here we are dealing with what I called the CR and DI stances.
Remember that it seems plausible that a good stance on idealizations can answer questions "(i)" and "(ii)"
So let's take a stance that can help us to understand the process of "making" idealizations in order to answer both questions.


\subsection{A brief characterization}

In this section I want to make a brief characterization of the three stances under discussion.
I want the order of the exposition to be 1) artifactualist, 2) deidealization and 3) causal relevance.
Let me begin with the artifacualist stance.

\paragraph{The artifactualist}

Within AR there are, what I think, three central thesis.
Two of them are encapsulated in the next quote "Accordingly, idealization tends to be holistic in that it is not often easily attributable to just some specific parts of the model" \footcite[][p. 3]{Carrillo2021-CARAAP-12}.

The last quote affirms two things, first that "idealizations tends to be holistic", second that "idealization is not often attributable to just some specific parts of the model".
Both affirmations are related to the "holistic" nature of idealizations that the artifactualist commits to.
Let me address the "holistic" nature of idealizations.

There is no single definition of what "holism" means.
The term is commonly attributed to Quine, affirming that our knowledge, and the meaning of the propositions that we hold, are contrasted with the world not one by one, but as a whole\footnote{"My countersuggestion, issuing essentially from  Carnap's doctrine of the physical world in the Aufbau, is that our  statements about the external world face the tribunal of sense experience not individually but only as a corporate body." \parencite[][p. 38]{Quine1951} }.
And this seems to be the kind of holism that the artifactualist claims.

In the paper \citetitle{Carrillo2021-CARAAP-12}, "holism" seems to be the connection that the idealization has with other idealizations already in the scientific community.
They say that "$\ldots$ the justification eventually boils down to coherence with earlier theoretical and methodological commitments as well as empirical results" \parencite[][p. 52]{Carrillo2022}

Also, this is related with their claim that idealization is attributable to specific parts of the model, they claim that "$\ldots$ many idealizations appear holistic, and not separable into assumptions whose distorting nature would be self-evident" \parencite[][p. 57]{Carrillo2022}.

The third thesis is about the methodology that the artifactualist uses.
They say that the artifactualist stance can answer questions "(i)" and "(ii)" by claiming that their theory doesn't builds up from the relation between the model and the world.

\begin{quote}
	While idealization has traditionally been understood as deliberate misrepresentation of a feature of the target system, the artifactual  approach is not hung up on the accuracy/distortion of a model or its parts. \parencite[][p. 8]{Carrillo2021-CARAAP-12}
\end{quote}

And in a latter paper, they defend their theory based on this idea.

\begin{quote}
	In this paper, we develop an artifactual view of holistic idealizations that does not start from the representational assumptions inherent in idealization-as-distortion accounts, but rather focuses on the processes trough which models are achieved, used, and further developed. \parencite[][p. 50]{Carrillo2022}
\end{quote}

This detachment of the world, they say, has certain benefits.
It seems to clarify how models are constructed, with certain limitations of what the researches can hypothesize based on the tools available.
As they urge "$\ldots$ what  mathematical methods can be used, what analogies are available, what can be measured)." \parencite[][p. 9]{Carrillo2021-CARAAP-12}

So they answer the questions by arguing that the model construction in actual researchers doesn't want to accurately represent the world, instead they want to use the tools\footnote{Tools in the artifactualist sense, \emph{i, e.} mathematical methods, analogies, etc.} available to render new models/idealizations/metaphors.
And this answers our two questions.

\paragraph{The de-idealizer}

The "de-idealized" stance affirms that the idealization is just a momentary aid for simplicity, but with scientific progress, we will eventually get rid of the idealization.
There is a characterization of this stance in \parencite{Weisberg2007}, Weisberg characterizes such stance as a "$\ldots$ idealization justified pragmatically", p. 641.
This means that this idealization helps to simplify certain phenomena.
And because the problem is one of computational tractability, new computational power will bring the idealization more accurate, with the goal of getting rid of it.

The de-idealizer says, then, that idealizations have the purpose of simplify computational tractability.
And, because the idealization is motivated by their use, better computational power will render the idealization obsolete.

It seems that the de-idealizing stance makes two affirmations.
First, that the model is justified pragmatically, in order to make the phenomena computationally tractable, answering question "(i)".
The de-idealizer also says that the model is used just temporarily.
Due to the computational complexity, the development of new computational tools, will give us a more accurate idealization.
Looking at the aim of getting rid of the idealization in the future, answering the question "(ii)".

\paragraph{The causal relevantist}

I think that the most detailed defense that we have of the causal relevance stance was made by \cite{Potochnik2017-POTIAT-3}.
The causal relevantist is motivated by "$\ldots$ that science is ultimately a human creation and, as such, responsive to particular human concerns" \parencite[][p. 11]{Potochnik2017-POTIAT-3}.
This has a commitment with another whole set of thesis tat the causal relevantist accepts.
But I think we can talk about how the causal relevantist deals with idealizations, without going into much detail about the other set of thesis.

The causal relevance stance, begins with the motivation that scientist idealize, because of the complexity of the world and the limited cognitive capabilities of humans.
Also, Potochnik says that reasons to idealize have multiple sources, and they not just reflect features of the world, but also researcher's interests.

This means that researchers idealize because they need to simplify causal complexities.
The different variables that can affect the phenomena are excluded.
Researchers just need to account for the relevant causal factor they are interested in.
And because researchers are looking for information according to a hypothesis, they know which causal factor is relevant.

But in order to know which causal factor is relevant, they need to represent the phenomena somewhat accurately, as Potochnik says "I intend causal patterns to be regularities in phenomena themselves." \parencite[][p. 25]{Potochnik2017-POTIAT-3}.
The search for causes in science depends on our interests as researchers.
But we still need to account for actual phenomena.

\begin{quote}
	But successful idealizations do bear certain similarities to the systems they help to represent.
	Idealizations represent phenomena as if they had features they don't, but those misrepresentations are useful insofar as there are functional similarities--similarities in causal role or behaviour-- $\ldots$ \parencite[][p. 53]{Potochnik2017-POTIAT-3}.
\end{quote}

So the causal relevantist answer question "(i)" arguing that scientists make idealizations because they want to isolate relevant causal factors they are particularly interested in.
This also answers why scientists use idealizations, and that gives an answer to question "(ii)".

We need to notice that the causal relevantist also affirms that researchers doesn't need to de-idealize, because the causal factor already isolated is of use still when we have more computational power.
And if we already have more computational power, researchers may want to account for more complexity: adding new variables, new causal factors, new types of idealizations, etc.
This in contrast with what the de-idealizer affirms.

Also, the causal relevantist claims that the idealization does bear some similarities with the phenomena represented.
Because causal patterns are regularities of real phenomena.
This in contrast with the artifacualist.





% chapter chapter_1 (end)
