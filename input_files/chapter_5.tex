%!TEX root = ../main.tex




\chapter{Prácticas biológicas}
\label{ch:practices}

Parece claro que los investigadores en general buscan explicaciones.
Dichas explicaciones lucen diferente en distintas disciplinas.
Por ejemplo, en matemáticas, buscamos una prueba para saber ciertas propiedades de los números.
Esto, sin embargo, es distinto en las ciencias empíricas.

Los físicos tratan de describir fenómenos como el movimiento, velocidad y aceleración de los objetos.
Tratan de descrtibir la fuerza necesaria para mover un objeto con cierto peso, etc.
Ahora bien, la física tiene descripciones muy precisas de estos fenómenos, sin embargo, otras disciplinas no tienen descirpciones tan precisas: hay más variables involucradas.
Comúnmente se acepta que la física usa leyes para describir fenómenos.
En mecánica clásica, dichas leyes están cuantificadas universalmente, son reversibles y deterministas.
De manera tal que sistemas que cumplan las misma propiedades pueden describirse con una misma ecuación.

Pienso, por ejemplo, en el movimiento uniformemente acelerado.
Este tipo de movimiento es tal que, la aceleración de un cuerpo con masa es constante en intervalos iguales de tiempo. 

En biología difícilmente tenemos este tipo de generalizaciones.
Por lo general se asume que en biología no hay leyes. 
En particular en biología evolutiva, sabemos que las condiciones no son lo sufcientementer estables para hacer generalizaciones.
Pensemo, por ejemplo, en las leyes de Mendel. 
Las ley de segreación de Mendel nos dice que los alelos de un gen están segregados entre ambos padres.
Esto significa que si ambos alelos son recesivos no codifican, mientras que si uno de ellos es dominante, entonces codifican para un rasgo fenotípico particular.





% chapter chapter_5 (end)
