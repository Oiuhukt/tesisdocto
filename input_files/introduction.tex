%!TEX root = ../main.tex

\chapter{Introducción}
\label{ch:introduction}

\dochaptoc % displays the toc of the chapter as margin note

El doctorado ha sido un largo viaje en el cuál aprendí muchas cosas, muchas de ellas relacionadas a la filosofía; otras, no tanto.\footnote{Juro que este breve relato tiene un punto.}
Mucho de eso que aprendí fue por qué decidí entrar a un posgrado en Filosofía de la Ciencia, y no un posgrado de Filosofía a secas.
Durante la Licenciatura tuve profesores muy malos, pero dos de los profesores que tuve son exclentes filósofos.
En ese momento, casi al final de la licenciatura, estaba teniendo severos problemas con la filosofía: los profesores malos fueron más que los buenos.

Tenía entonces esta "crisis existencial filosófica" y el hecho de que los dos profesores buenos trabajan filosofía de la ciencia\footnote{Al menos lo que en se momento creía que era la filosofía de la ciencia}.
Al buscar mejores métodos de investigación que los que me habían enseñado los profesores malos, decidí hacer un posgrado en Filosofía de la Ciencia.

Durante la maestría en el posgrado en Filosofía de la Ciencia de la UNAM, aprendí sobre nuevas metodologías de investigación, como lo esperaba; pero una de las más valiosas, fue el la importancia que tiene la historia de la ciencia en la filosofía de la ciencia.
Al menos dos corrientes son identificadas con posturas filosóficas sobre la historia de la ciencia, tajantemente divididas en "internalistas" y "externalistas".
Puesto de manera procaz, los internalistas argumentan que la historia de la ciencia debería interpretarse como un progreso de ideas.
Además, cada uno de los nuevos descubrimientos en la ciencia están motivados por intenciones internas a los investigadores: la búsqueda de la verdad, curiosidad natural, el gusto por explicar fenómenos, etc. 
Con mejores métodos de investigación, los investigadores ofreceran mejores respuestas a sus preguntas, siempre bajo la guía de sus intereses.
Creo que podemos identificar a Weinberg como un internalista, cuando dice que: 

\begin{quote}
	The word “discovery” in the subtitle is also problematic. I had thought of using The Invention of Modern Science as a subtitle. After all, science could hardly exist without human beings to practice it. I chose “Discovery” instead of “Invention” to suggest that science is the way it is not so much because of various adventitious historic acts of invention, but because of the way nature is. \parencite{Weinberg2015}
\end{quote}

En contraste con el internalista, quien adopte una metodología externalista, resalta los "actos históricos fortuitos" de los que habla Weinberg en la cita anterior.
Quien adopte una metodología externalista explicará los procesos de investigación científica a partir de las condiciones históricas.
Periodos de tiempo durante los cuales se fueron desarrollando los investigadores.
El externalista resalta que la investigación científica no es una actividad que pueda separase de su contexto "cultural"\footnote{
Por supuesto que la distincón cultural$\/$no cultural es, a su vez, problemática. 
Pero creo que puedo continuar la exposición con una idea intuitiva del término. 
No es necesario decir mucho más al respecto en lo que sigue.
}.

\section{deprueba}

Esto para ver el comportamiento del comando





