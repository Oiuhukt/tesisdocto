%!TEX root = ../main.tex

\chapter{Introducción}
\label{ch:introduction}

% displays the toc of the chapter as margin note

\section{Una breve anécdota}

\noindent El doctorado ha sido un largo viaje durante el que aprendí mucho.
Mucho del aprendizaje está relacionado con la filosofía; otras, no tanto.\footnote{Juro que este breve relato tiene un punto.}
Mucho de eso que aprendí fue por qué decidí entrar a un posgrado en Filosofía de la Ciencia, y no un posgrado de Filosofía a secas.
Durante la Licenciatura tuve profesores muy malos, pero dos de los profesores que tuve son exclentes filósofos.
En ese momento, casi al final de la licenciatura, estaba teniendo severos problemas con la filosofía: los profesores malos fueron más que los buenos.

Tenía entonces esta "crisis existencial filosófica" y el hecho de que los dos profesores buenos trabajan filosofía de la ciencia --al menos lo que en ese momento creía que era la filosofía de la ciencia.
Decidí entonces que quería entrar al posgrado en filosofía de la ciencia.
En las clases de maestría hubo profesores buenos, pero también los hubo excelentes.
Esto me motivó a responder las preguntas que me preocupaban
Decidí entonces buscar mejores maneras de entender la investigación científica, en particular para responder preguntas sobre la epistemología de la ciencia: \say{¿qué estamos justificados a creer?}, \say{¿cuándo podemos afirmar que una tesis ha sido corroborada?}, \say{¿es la explicación más básica que el conocimiento?}, etc.

Durante la maestría en el posgrado en Filosofía de la Ciencia de la UNAM, aprendí--bastante--sobre nuevas metodologías de investigación, maneras diferentes de plantearse y entender las preguntas que me preocupaban.
Pero lo más valioso que aprendí fue la importancia que tiene la historia de la ciencia en la filosofía de la ciencia.



\section{Un breve repaso}

\noindent En historia de la ciencia, comúnmenmte, se distingue entre dos corrientes, divididas en \emph{internalistas} y \emph{externalistas}.
De manera procaz, los internalistas argumentan que la historia de la ciencia debería interpretarse como un progreso de ideas.
Además, cada uno de los nuevos descubrimientos en la ciencia están motivados por intenciones internas a los investigadores: la búsqueda de la verdad, curiosidad natural, el gusto por explicar fenómenos, etc.
Con mejores métodos de investigación, los investigadores ofreceran mejores respuestas a sus preguntas, siempre guíados por sus intereses.
Creo que podemos identificar a Weinberg como un internalista, cuando dice que:

\begin{quote}
	The word \say{discovery} in the subtitle is also problematic. I had thought of using The Invention of Modern Science as a subtitle. After all, science could hardly exist without human beings to practice it. I chose \say{Discovery} instead of \say{Invention} to suggest that science is the way it is not so much because of various adventitious historic acts of invention, but because of the way nature is. \parencite{Weinberg2015}
\end{quote}

En contraste con el internalista, quien adopte una metodología externalista resalta los \say{actos históricos fortuitos} de los que habla Weinberg en la cita anterior.
Quien adopte una metodología externalista explicará los procesos de investigación científica a partir de las condiciones históricas en las que se desarrollaron.
Periodos de tiempo durante los cuales se desenvolvieron los investigadores y sus investigaciones.
El externalista resalta que la investigación científica no es una actividad que pueda separase de su contexto \say{cultural}, nos dice \textcite{Yturbe1995}\footnote{
	Por supuesto que la distincón cultural $|$ no cultural es, a su vez, problemática.
	Pero creo que puedo continuar la exposición con una idea intuitiva del término.
	No es necesario decir mucho más al respecto en lo que sigue.
}.



\begin{quote}
	The general tendency in the historiography of the sciences is to consider the social function of science, as weIl as so me aspects of the matrix from wh ich the problematic is formed, as external elements, while the conceptual apparatus and the problem field are treated as internal. \emph{But we should not think of science as having two independent histories.} \parencite[][p. 85. Énfasis agregado]{Yturbe1995}
\end{quote}

En la cita anterior, resalto la última oración porque implica que no puede haber distinción entre factores internos y externos.
En el artículo, la autora defiende la tesis que afirma que una teoría a la vez internalista como externalista, no es una teoría plausible.
La conclusión está basada en la premisa de que no es posible hacer la distinción entre \emph{factores internos} y \emph{factores externos} y
creo que la premisa es falsa.
Mi argumento descansa a su vez en la premisa de que \say{el \emph{contexto de descubrimiento} y el \emph{contexto de justificación} son distinciones equivalentes a \emph{factores externos} y \emph{factores internos}, respectivamente.}
Que es premisa para el argumento principal de la autora.



\section{¿Qué podemos aprender de la historia de la ciencia?}

\noindent Quiero comenzar esta sección confesando lo siguiente: estoy de acuerdo con la autora, en particular con el carácter de la conclusión anterior, esto es, que es complicado--si no es que imposible--, separar entre los procesos externos y los procesos internos que influyen en la investigación científica.
Pero aceptar esto --eso quiero argumentar-- no implica que debamos deshacernos de la distinción entre el \emph{contexto de investigación} y el \emph{contexto de descubrimiento}.
En lo que resta de la sección quiero hacer un breve repaso de los argumentos de la autora.

Yturbe nos dice que el contexto de descubrimiento, se dedica a analizar factores externos que fueron influyendo en los factores internos; mientras que el \emph{contexto de justificación} sólo se dedica a analizar los factores internos al desarrollo teórico.
Lo que ella llama \say{the doctrine of the two contexts}\footnote{
	\say{one of the philosophical theses in favor of the contraposition between the internalist approach and the externalist approach is the so-called doctrine of the two contexts, developed by positivism.} \cite[][p. 75]{Yturbe1995}
}
La autora nos dice que la doctrina de los dos contextos hace una distinción que no se puede hacer.
El contexto de justificación es siempre influenciado por el contexto de descubrimiento porque la ideología, la economía, las relaciones de poder, etc. influyen en la toma de decisiones\footnote{Por usar una caricatura: a quién se le asigna presupuesto}, por lo tanto la doctrina es incorrecta.

Como dije al principio de esta sección, estoy de acuerdo con que es casi imposible hacer la distinción entre \emph{historia interna} e \emph{historia externa}, pero aceptar esto, no implica deshacernos de la distinsión \emph{contexto de descubrimiento} y \emph{contexto de justificación}.
Para sustentar esto, dependo de la caracterización que hace la autora sobre \say{la doctrina de los dos contextos}, la caracterización de la doctrina está ligeramente sesgada, si no es que completamente inadecuada.
En la siguiente sección quiero ofrecer mis razones para esta conclusión.

\subsection{La doctrina}

\noindent En el texto de \textcite[][p.75]{Yturbe1995}, la autora ofrece una descripción de la doctrina, ella señala que los filósofos doctrinarios afirman que \say{according to this position, [factores externos], not only fails to increase our understanding of scientific development, but even obscures the fundamental question of the rational validation of scientific arguments.}
Además que \say{This conception constitutes the dominant framework in which the philosophy of science has been developed, and consists in drawing a radlcal distinction between the context of discovery and the context of validation.}


Su argumento para decir que hay un colapso entre factores externos (contexto de descubrimiento) y factores internos (contexto de justificación) se encuentra en la página 85

\begin{quote}

	External factors are not only found in the context of discovery, they are present also in the development of the concepts, problems, methods, problem fields, etc. of scientific discourses: that is, external factors are present in the context of validation itself.
	There are scientific discourses in which ideological conceptions pass on to form part of the body of the science itself, functioning as principles which define its field of study or guide its research; thus, external factors can become internal.

\end{quote}

La doctrina de los dos contextos implica que podemos claramente separar entre el \emph{contexto de descubrimiento} y el \emph{contexto de justificación}.
Pero los factores externos están presentes en el contexto de justificación, tanto así que pueden convertirse en factores internos.
Esto implica que los contextos no son claramente separables, y que, por tanto, la doctrina es incorrecta.

Contrario a lo que dice la autora

\begin{quote}
	The search for new explanatory programs is characterized above all by the attempt to reconcile the internalist and externalist approaches. But, in view of the fact that both approaches are committed to theses concerning the nature of science that are not only incompatible, but are unsustainable, their union without important changes in their philosophical presuppositions cannot result in a viable program. \parencite[][p. 79]{Yturbe1995}
\end{quote}

Creo que es verdad que no podemos distinguir entre factores externos y factores internos en la ciencia.
Sin embargo, me parece que lo que expresa la cita anterior es claramente erróneo: no se siguie la conclusión \say{que no pueden resultar en un programa viable}, porque depende de señalar que la distinción entre  \emph{contexto de descubrimiento} $|$ \emph{contexto de justificación} es equivalente a la distinción \emph{factores externos} $|$ \emph{factores internos} y estas distinciones no son equivalentes.



\subsection{En defensa de la doctrina}

\noindent En años recientes ha habido un creciente interés en discutir las publicaciones del Círculo de Vienna \parencite{Bentley2023, Richardson2023, Suarez2024, Riel2014}.
Si hago una tosca generalización, diría que literatura reciente se ha dedicado a señalar que los miembros del Círculo de Vienna, no eran tan herméticos como se pensaba.
Además, la cantidad de temas que discutieron es más vasta de lo que se creía.
Particularmente pensando en que cada uno de los miembros difería de los otros en tesis centrales.

\textcite{Suarez2024}, por ejemplo, discute que una preocupación genuina sobre los modelos en la ciencia, precede, data y prosigue a los años del Círculo de Vienna, el autor nos dice \say{I focus particularly on the nineteenth-century modelers and summarize their insights and contributions, which, I claim, remain essentially unsurpassed.} (p. 20)

Además, los miembros del Círculo de Vienna tuvieron una preocupación genuina por las prácticas científicas--Neurath, por ejemplo--
Tenían presente que ciertos factores externos pueden influenciar a la investigación científica.
\textcite[][p. 24]{Bentley2023} lo expresa mejor: \say{Frank's historical work, which frequently anticipates Kuhnian or post-Kuhnian themes, consistently emphasizes the significance of non-scientific, external factors on the decision making of scientists.}
Los miembros del Círculo de Vienna tenían claro que los factores externos influyen en el proceso de la investigación científica, presentes incluso en el \emph{contexto de justificación}.

Hablando en particular del trabajo de Neurath y las tesis que sostenía, Otto defendía una teoría coherentista de la justificación.
Neurath argumentaba que nuestras teorías científicas están subdeterminadas empíricamente.
En cualquier momento del tiempo hay hipótesis en pugna.
Pero si las teorías están empíricamente subdeterminadas, es posible que dos teorías internamente coherentes y externamente inconsistentes entre sí, convivan al mismo tiempo.

Pero si el escenario anterior es plausible, entonces es patente que aquél que defienda una teoría coherentista de la confirmación, no puede resolver el problema de la elección racional de teorías.
Debido a que todo el tiempo hay teorías en pugna, como filósofos, deberíamos poder explicar por qué es racional elegir entre una teoría y sus rivales. Para poder resolver este problema, Neurath defendía que los factores externos son información clave para explicar la racionalidad de la elección de los investigadores.
Factores externos como los eventos históricos fortuitos.

\textcite{Bentley2023} nos recuerda que Neurath fue falibilista sobre el conocimiento, es decir,  sostenía que cada una de nuestras creencias puede estar equivocada.
Sumado a esto, sostenía que la confirmación es holista: no son oraciones particulares, sino grupos de oraciones, los que contrastamos con el mundo.
Bentley lo expresa mejor \say{$\ldots$ it is always possible for a system of beliefs to be altered to accommodate a statement or for the statement to be rejected $\ldots$} (p. 21).



Esto último inició como un abstract que mandé a un taller y se fue volviendo más largo mientras leía.

Hablando de modelos en la filosofía de la ciencia durante el periodo del Círculo de Vienna, una figura importante que mencionar es Mary Hesse.
Mary Hesse trabajó con especial atención el uso de modelos en la ciencia.

``This historical chapter introduces the emergence in the nineteenth century of  what I call the modeling attitude.
This is a stance toward scientific work and  discovery, and it continues to this day." (p. 43)






