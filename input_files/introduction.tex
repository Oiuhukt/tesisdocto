%!TEX root = ../main.tex
% LTeX: language=es

\chapter{Introducción}
\label{ch:introduction}

% displays the toc of the chapter as margin note

\section{Una breve anécdota}

\noindent El doctorado ha sido un largo viaje durante el que aprendí mucho y mucho del aprendizaje está relacionado con la filosofía; otras, no tanto.\footnote{Juro que este breve relato tiene un punto.}
En el posgrado profundicé en los temas que me interesaban, y es debido a estos intereses por lo cuales decidí entrar a un posgrado en Filosofía de la Ciencia y no un posgrado de Filosofía a secas.
Quiero mencionar que mi interés en estos temas nació durante mi licenciatura.

Durante la Licenciatura tuve profesores muy malos y un puñado de muy buenos profesores.
Pero mis intereses se deben a dos de los profesores que tuve, quienes son excelentes filósofos.
Estos profesores me dieron clases en las que vimos temas que despertaron mis intereses: la explicación científica y la naturaleza de la causalidad.
Pero, casi al final de la licenciatura, estuve teniendo severos problemas con la filosofía, en especial porque los profesores malos fueron más que los buenos.

Tenía entonces esta \say{crisis existencial filosófica} y el hecho de que los dos profesores buenos trabajaran filosofía de la ciencia --al menos lo que en ese momento creía que era la filosofía de la ciencia-- sesgó mis intereses a lo que hasta ahora ha sido mi trabajo.
Esta influencia me hizo tomar la decisión de entrar al posgrado en filosofía de la ciencia, por supuesto, empezando con la maestría.

En las clases de maestría hubo profesores buenos, pero también los hubo excelentes.
Mucho de lo que aprendí me sirvió para enmarcar las preguntas que me preocupaban de diferentes maneras, lo que me permitió entender --mejor, me parece-- la naturaleza de la investigación científica.
En particular quería responder preguntas sobre la epistemología de la ciencia: \say{¿qué estamos justificados a creer?}, \say{¿cuándo podemos afirmar que una hipótesis ha sido corroborada?}, \say{¿es la explicación o compresión más básica que el conocimiento?}, etc.
En la maestría aprendí sobre nuevas metodologías de investigación, maneras diferentes de plantearse y entender las preguntas que me preocupaban.

Pero sin lugar a dudas, lo más valioso que aprendí, fue la importancia que tiene la historia de la ciencia en la filosofía de la ciencia.
Me voy a permitir hacer una breve caracterización, exageradamente general, de dos posturas en historia de la ciencia.


\section{Un breve repaso de dos posturas}

Prometí una breve caracterización y en esta sección voy a hacerla; este es un breve y burdo repaso de las \say{metodologías} \emph{externalistas} e \emph{internalistas}.
Las llamo \say{metodologías} porque difieren en términos sustanciales, por ejemplo, el de \emph{verdad}; y es a partir de estas diferencias que ambas metodologías ofrecen una interpretación distinta de la historia de la ciencia.
Además, al usar el término \emph{metodología}, no quiero comprometerme con que son teorías o diferentes maneras de hacer historiografía; esto es algo que no puedo discutir aquí y el nombre \say{metodologías}, me parece, es lo suficientemente neutro.

%La sección \ref{sbc:yturbe} la dedico a analizar una supuesta alternativa para el debate anterior.
%Además, esta alternativa, es más cercana a la filosofía de la ciencia.

\subsection{Dos metodologías: \emph{externalismo} e \emph{internalismo}}

\noindent Digamos que en historia de la ciencia, se suelen distinguir dos corrientes: los \emph{internalistas} y los \emph{externalistas}.
De manera procaz, los internalistas argumentan que la historia de la ciencia debería interpretarse como un progreso de ideas; cada uno de los nuevos descubrimientos en la ciencia están motivados por intenciones internas a los investigadores: la búsqueda de la verdad, curiosidad natural, el gusto por explicar fenómenos, etc.

Con mejores métodos de investigación, los investigadores ofrecerán mejores respuestas a sus preguntas.
Tendrán, por decirlo de alguna manera, respuestas más detalladas del fenómeno que les interesa y descubrirán nuevos fenómenos que no contemplaban originalmente; siempre guiados por sus intereses.
Al ser guiados por sus intereses, y sólo por sus intereses, los investigadores son inmunes a los fenómenos culturales que les rodean.
Digamos que la ciencia es inmune a la cultura y creo que podemos identificar a Weinberg\footnote{
	Esto es tramposo porque no sé si los hitoriadores de la ciencia estiman a Weinberg como uno de sus colegas.
	Aparece en este escrito para fines puramente explicativos.
} como un internalista, cuando dice que

\begin{quote}
	The word \say{discovery} in the subtitle is also problematic.
	I had thought of using The Invention of Modern Science as a subtitle.
	After all, science could hardly exist without human beings to practice it.
	I chose \say{Discovery} instead of \say{Invention} to suggest that science is the way it is not so much because of various adventitious historic acts of invention, but \emph{because of the way nature is}. \parencite[Énfasis agregado][Prefacio]{Weinberg2015}
\end{quote}

Quiero destacar el énfasis que hice en la cita anterior.
Creo que no hay mucho peligro en asumir que la investigación es, probablemente, la mejor manera que tenemos de producir conocimiento.
Creo, además, que la investigación tiene como meta ofrecer explicaciones \emph{verdaderas} de los fenómenos; entonces me alíneo con Weinberg en un sentido restringido de \say{descubrimiento}\footnote{
	El sentido restringido del que hablo es algo que quiero discutir a lo largo de este capítulo introductorio. Al menos es parte de lo que voy a defender en este documento: el lector tendrá que leer el trabajo completo y regresar a esto cuando termine.
}.
Dicho en términos más familiares: el internalista estima el papel que la \emph{verdad} juega en la ciencia; la ciencia \emph{descubre} fenómenos, no los \emph{inventa}.

Decirlo de esta manera, me resulta más familiar, pero \say{ciencia} no es un término que esté libre de debate; y una vez entrando en los detalles, el debate se torna bastante complejo.
Hay por lo menos dos maneras en las que el término es problemático.

\newthoght{En primer lugar}, en filosofía de la ciencia, los filósofos debaten si es posible distinguir entre ciencia$\/$no-ciencia.\footnote{
	El debate es complejo y no podría explicarlo con justicia en este documento.
	El lector interesado puede revisar la entrada \parencite{sep-pseudo-science}.
}
En segundo lugar, los historiadiores de la ciencia debaten si es posible definir en qué periodo comienza la \emph{ciencia} tal como la conocemos en tiempos contemporáneos.

Quiero hacer notar que introduje deliberadamente el término \say{verdad}, cuya naturaleza es una discusión que requiere más detalles y profundidad; pero este término es central para la discusión que seguiré a lo largo del documento, por lo que quiero rápidamente señalar cómo el término difiere en las dos metoddologías que he descrito.
Veamos que dice el externalista sobre este tema.

Si adoptamos una metodología externalista, tendríamos que resaltar los \say{actos históricos fortuitos} de los que habla Weinberg, para elucidar cómo proceden las personas cuando investigan y qué influencias externas afectan el proceso.
Un externalista explicará los procesos de investigación científica a partir de las condiciones históricas en las que se desarrollaron; periodos de tiempo durante los cuales se desenvolvieron los investigadores y sus investigaciones.
El externalista resalta que la investigación científica no es una actividad que pueda separase de su contexto \say{cultural}.

Hablando de la \say{verdad} o \say{veracidad}, me parece que Shapin lo expresa mejor cuando afirma que \say{All claims have to win credibility, and credibility is the outcome of contingent social and cultural practice.} \parencite[][Capítulo 2]{shapin2010never}
Adoptando, digamos, una metodología externalista, Shapin se ha dedicado a estudiar detalladamente como los fenómenos sociales irrigan conceptos tan centrales en la investigación científica como es el de \emph{verdad}; y para aclarar el punto de Shapin, la siguiente cita es ilustrativa

\begin{quote}
	The notion of truthfulness was thus central to the description of gentle qualities.
	Through the Renaissance and into the eighteenth century an honorable man and an honest man were interchangeable designations: \say{honesty} included the notion of truthtelling but was understood far more broadly to include concepts of probity, uprightness, fairdealing, and respectability. \parencite[][pp. 70-71]{Shapin1995}
\end{quote}

Dada mi caracterización, si asumimos que Weinberg es un internalista y que Shapin es un externalista, entonces lo que he descrito párrafos arriba es suficiente para saber cómo la \emph{verdad} juega un papel en ambas metodologías.

Parece que con esto me he desviado del tema, pero esta exposición es importante para lo que quiero defender a continuación; tenemos todavía un ejemplo que quiero discutir.
Este es un caso más cercano a la filosofía de la ciencia, que trata el tema que me incumbe en este capítulo.
Me dedico a este caso a continuación.

\subsection{Un caso más cercano a la filosofía de la ciencia}
\label{sbc:yturbe}

En el artículo \citetitle{Yturbe1995}, la autora argumenta que es un falso dilema tener que elegir entre las metodologías internalista y externalista.
Yturbe dice, por ejemplo que

\begin{quote}
	The general tendency in the historiography of the sciences is to consider the social function of science, as weIl as some aspects of the matrix from which the problematic is formed, as external elements, while the conceptual apparatus and the problem field are treated as internal. \emph{But we should not think of science as having two independent histories.} \parencite[][p. 85. Énfasis agregado]{Yturbe1995}
\end{quote}

La oración en itálicas implica que no deberíamos hacer una distinción entre factores internos y factores externos.
Además, Corina defiende que una teoría a la vez internalista como externalista no es una teoría plausible.
Al menos esto parece sugerir cuando dice que

\begin{quote}
	The search for new explanatory programs is characterized above all by the attempt to reconcile the internalist and externalist approaches.
	But, in view of the fact that both approaches are committed to theses concerning the nature of science that are not only incompatible, but are unsustainable, their union without important changes in their philosophical presuppositions cannot result in a viable program. \parencite[][p. 79]{Yturbe1995}
\end{quote}

Su conclusión se basa en que no es posible marginar ambos factores; además, la autora sugiere que esta supuesta distinción tiene sus orígenes en la distinción entre el \emph{contexto de descubrimiento} y el \emph{contexto de justificación} que trazaron los positivistas.
La autora señala, por ejemplo que

\begin{quote}
	External factors are not only found in the context of discovery, they are present also in the development of the concepts, problems, methods, problem fields, etc. of scientific discourses: that is, external factors are present in the context of validation itself.
	There are scientific discourses in which ideological conceptions pass on to form part of the body of the science itself, functioning as principles which define its field of study or guide its research; thus, external factors can become internal. \parencite[][p. 85]{Yturbe1995}
\end{quote}

La cita anterior sugiere que el contexto de justificación involucra sólo factores internos, mientras que el contexto de descubrimiento involucra sólo factores externos.
La autora nos dice que los factores externos contagian al contexto de justificación, cuando dichos factores se convierten en parte de las teorías --generando conceptos, principios centrales, etc.--

Creo que la premisa, en la que la autora nos dice que el contexto de justificación/descubrimiento es equivalente a los factores internos/externos, es falsa y que la equivalencia descansa en una confusión.
El argumento que voy a ofrecer para esta conclusión es que el \emph{contexto de descubrimiento} y el \emph{contexto de justificación} no es lo mismo que los \emph{factores externos} y \emph{factores internos}  respectivamente.

Para exponer este argumento, primero debo hacer un breve repaso de en qué consisten el \emph{contexto de descubrimiento} y \emph{el contexto de justificación.}



\newthoght{La distinción} entre el \emph{contexto de descubrimiento} y \emph{el contexto de justificación} la asociamos, generalmente, a Reichenbach.
En su libro \citetitle{reichenbach1938experience}, el autor usa esta distinción para \say{excluir} los así llamados factores externos (sean sociales, políticos o económicos)\footnote{
	El pasaje no dice exactamente esto, sino algo más cercano a que tenemos claro cómo usamos un análisis filosófico/formal para la justificación de teorías, mientras que no tenemos manera de hacer ese análisis en el contexto de descubrimiento.
	Este lado de la investigación científica, digamos, es muy heterogéneo como para ofrecer un análisis formal.
}.
Esta distinción, nos dice Reichenbach, sirve para ilustrar el hecho de que no tenemos a la mano un \say{método} para analizar el fenómeno del descubrimiento científico; lo que algunos filósofos \parencite{reichenbach1938experience, Seo2015} se refieren a esto como el momento \emph{Eureka}.
Pero no tener una herramienta de análisis formal, no implica que lo que sucede en el contexto de descubrimiento sea absolutamente deleznable.

Que los factores externos puede afectar el contexto de justificación, fue un tema que se discutió ampliamente durante los años del círculo de Viena (o positivismo lógico o empirismo lógico).
Entonces, tengo que ofrecer un breve repaso de cómo algunos de los miembros del círculo de Viena lidiaron con este problema.

\newthought{Los positivistas lógicos}\footnote{
	Me voy a referir con este término a los afilósofos que aparecen en la publicación de la \emph{International Encyclopedia of Unified Science}. Tanto del comité organizador, como el comité asesor.
}, estaban al tanto de cómo los \emph{factores externos} influyen en el \emph{contexto de justificación}.
Dicho de otra manera, que los aspectos sociales, políticos y económicos, afectan a las personas dedicadas a hacer investigación.

La afirmación de Reichernbach sobre el análisis formal del \emh{contexto de justificación} es algo que los miembros del círculo tenían en su agenda; con mayor o menor éxito.

\newthought{Neurath}, Joseph Bentley nos recuerda, sostenía que

\begin{quote}
	Despite his advocacy for scientific methods, Neurath never takes this method to be set in stone, nor does he attempt to portray science as an enterprise of purely objective methods, completely divorced from social, historical, and material contexts  or the personalities of scientific practitioners. \parencite[p.41][]{Bentley2023}
\end{quote}

Neurath sabía que los fenómenos sociales y políticos afectan las prácticas científicas.
Pero esto es algo que hay que estudiar de otra manera, la filosofía puede encargarse muy bien de cuestiones \emph{normativas}; y con un poco de cautela puede encargarse de cuestiones \emph{descriptivas}.

Y al hablar específicamente de la justificación de creencias, Bentley señala que \say{[science], Neurath maintains, it is still the best we have.} \parencite[p. 41][]{Bentley2023}
Y si somos capaces de reconocer cómo las prácticas juegan un papel, Bentley lo expresa mucho mejor cuando dice que \say{if we recognize the ... creation of the norms, methods and values of science, it can be made better.} \parencite[p. 41][]{Bentley2023}
Hay que recalcar que Neurath no fue el único miembro del círculo que reconoció el papel que juegan las personas que hacen investigación; y esto implica que los aspectos pragmáticos deben tomarse en cuenta.

\newthoght{Philipp Frank} expresa una actitud semejante a la de Neurath.
Ambos sostenían que los aspectos \say{externos} eran una parte crucial del análisis que los filósofos pueden ofrecer de las prácticas científicas.
Esto es más o menos claro cuando Frank nos dice que 

\begin{quote}
The special mechanism by which social powers bring about a tendency to acceptor reject a certain theory depends upon the structure of the society within which the scientist operate. \parencite[p. 143][]{Frank1954}
\end{quote}

Aceptar o rechazar una teoría es un proceso que involucra justificar ambas teorías y luego aceptar cuál es mejor para los propósitos que tenemos.
Es durante la fase de aceptación de la teoría, cuando los factores externos toman un papel central. 




%Ambos sostienen una forma de verificacionismo: que el significado de de una oración de


%Neurath tenía la intención de esclarecer esta relación, en particular porque estaba interesado en el fenómeno de la elección de teorías.
%Neurath nos recuerda que hay teorías que entran en competencia, sean viejas o nuevas; por lo que los investigadores deben elegir cuál teoría \emph{usar}.
%Los factores sociales y políticos son medios para sleccionar una teoría.
%De acuerdo a los objetivos que buscamos en la investigación.
%Y estos objetivos son influencias que imbuyen el contexto de justificación.

%Incluso Carnap, que es al que más le llueven vergazos y del que se sotiene tenía las ideas máws radicales deel círculo, llegó a aceptar que las teorías tienen sin duda un aspecto pragmático. 
%Neurath y Philipp Frank exponen este tipo de preocupaciones.


%Neurath fue falibilista, coherentista (sobre la justificación) y holista.
%Por supuesto que sostenía una forma de \emph{verificacionismo}, Bentley lo expresa más claramente \say{Theories must be sensitive to, grounded in and testible by experience} \parencite[p. 38][]{Bentley2023}.
%Este criterio es bastante vago, pero Bentley lo describe no como una tesis, sino como una postura; una postura que es difícil capturar con una sola afirmación \parencite[p.38][Cfr.]{Bentley2023}




\subsection{Citas Phillip Frank}

The misinterpretation of scientific principles, as will be shown, can be avoided if, in every statement found in books on physics or chemistry, one is careful to distinguish an experimentally testable assertion  about observable facts from a proposal to represent the facts in a certain way by word or diagram.
If this distinction is sharply drawn, there will no longer be any room for the interpretation of physics in favor of a spiritualistic or a materialistic metaphysics. [p. 4-5]

The views represented in the present book are closely associated with the movement now generally  called "logical empiricism" or "logical positivism."
I must confess that I do not like these words either.
But a long life among views and theories has shown me that if we want a view to be regarded as a respectable tree in the garden of opinions it must have a label just as much as the elms and oaks in our public gardens. [p. 5]

* recordar que Philipp Frank fue fundador del grupo original, que 20 años después se convirtió en el círculo de Vienna.

* Cualquier reconstrucción se le va a quedar corta a este libro.
But the outlook of the so-called "Vienna Circle"  has been only one particularly coherent doctrine  among the many intellectual fruits which have  emerged from the soil of Central-European positivism. Mathematicians such as R. von Mises, Κ. Menger, and Κ. Godei, physicists such as E. Schroedinger,  economists such as J. Schumpeter, lawyers such as  H. Kelsen, and sociologists such as E. Zilsel had their  roots in this environment. The whole intellectual  background of this general movement can be understood best through R. von Mises' textbook of positivism. [p. 9]

Several young American philosophers  traveled to Vienna and Prague in order to come into  scientific contact with Schlick and Carnap. Among  them were W . V . Quine (now at Harvard) and  E. Nagel (now at Columbia). In particular, Charles  W . Morris of Chicago recognized the connection  with American pragmatism and publicized the idea  of cooperation between the two groups. []

But what does it mean when we say that a question  is insoluble? Let us suppose, for example, that someone has asserted that the problem of a regular airplane  route to the planet Neptune is insoluble, or that the  production of a living organism from lifeless matter  is insoluble. Despite this assertion, the person making  it can describe quite accurately the concrete experience we should have if the problem were solved.

\section{¿Qué podemos aprender de la historia de la ciencia?}

\noindent Sin embargo,

\noindent Quiero comenzar esta sección confesando lo siguiente: estoy de acuerdo con la autora, en particular con el carácter de la conclusión anterior, esto es, que es complicado--si no es que imposible--, separar entre los procesos externos y los procesos internos que influyen en la investigación científica.
Pero aceptar esto --eso quiero argumentar-- no implica que debamos deshacernos de la distinción entre el \emph{contexto de investigación} y el \emph{contexto de descubrimiento}.
En lo que resta de la sección quiero hacer un breve repaso de los argumentos de la autora.

Yturbe nos dice que el contexto de descubrimiento, se dedica a analizar factores externos que fueron influyendo en los factores internos; mientras que el \emph{contexto de justificación} sólo se dedica a analizar los factores internos al desarrollo teórico.
Lo que ella llama \say{the doctrine of the two contexts}\footnote{
	\say{one of the philosophical theses in favor of the contraposition between the internalist approach and the externalist approach is the so-called doctrine of the two contexts, developed by positivism.} \cite[][p. 75]{Yturbe1995}
}
La autora nos dice que la doctrina de los dos contextos hace una distinción que no se puede hacer.
El contexto de justificación es siempre influenciado por el contexto de descubrimiento porque la ideología, la economía, las relaciones de poder, etc. influyen en la toma de decisiones\footnote{Por usar una caricatura: a quién se le asigna presupuesto}, por lo tanto la doctrina es incorrecta.

Como dije al principio de esta sección, estoy de acuerdo con que es casi imposible hacer la distinción entre \emph{historia interna} e \emph{historia externa}, pero aceptar esto, no implica deshacernos de la distinsión \emph{contexto de descubrimiento} y \emph{contexto de justificación}.
Para sustentar esto, dependo de la caracterización que hace la autora sobre \say{la doctrina de los dos contextos}, la caracterización de la doctrina está ligeramente sesgada, si no es que completamente inadecuada.
En la siguiente sección quiero ofrecer mis razones para esta conclusión.

\subsection{La doctrina}

\noindent En el texto de \textcite[][p.75]{Yturbe1995}, la autora ofrece una descripción de la doctrina, ella señala que los filósofos doctrinarios afirman que \say{according to this position, [factores externos], not only fails to increase our understanding of scientific development, but even obscures the fundamental question of the rational validation of scientific arguments.}
Además que \say{This conception constitutes the dominant framework in which the philosophy of science has been developed, and consists in drawing a radlcal distinction between the context of discovery and the context of validation.}


Su argumento para decir que hay un colapso entre factores externos (contexto de descubrimiento) y factores internos (contexto de justificación) se encuentra en la página 85

\begin{quote}

	External factors are not only found in the context of discovery, they are present also in the development of the concepts, problems, methods, problem fields, etc. of scientific discourses: that is, external factors are present in the context of validation itself.
	There are scientific discourses in which ideological conceptions pass on to form part of the body of the science itself, functioning as principles which define its field of study or guide its research; thus, external factors can become internal.

\end{quote}

La doctrina de los dos contextos implica que podemos claramente separar entre el \emph{contexto de descubrimiento} y el \emph{contexto de justificación}.
Pero los factores externos están presentes en el contexto de justificación, tanto así que pueden convertirse en factores internos.
Esto implica que los contextos no son claramente separables, y que, por tanto, la doctrina es incorrecta.

Contrario a lo que dice la autora

\begin{quote}
	The search for new explanatory programs is characterized above all by the attempt to reconcile the internalist and externalist approaches. But, in view of the fact that both approaches are committed to theses concerning the nature of science that are not only incompatible, but are unsustainable, their union without important changes in their philosophical presuppositions cannot result in a viable program. \parencite[][p. 79]{Yturbe1995}
\end{quote}

Creo que es verdad que no podemos distinguir entre factores externos y factores internos en la ciencia.
Sin embargo, me parece que lo que expresa la cita anterior es claramente erróneo: no se siguie la conclusión \say{que no pueden resultar en un programa viable}, porque depende de señalar que la distinción entre  \emph{contexto de descubrimiento} $|$ \emph{contexto de justificación} es equivalente a la distinción \emph{factores externos} $|$ \emph{factores internos} y estas distinciones no son equivalentes.



\subsection{En defensa de la doctrina}

\noindent En años recientes ha habido un creciente interés en discutir las publicaciones del Círculo de Vienna \parencite{Bentley2023, Richardson2023, Suarez2024, Riel2014}.
Si hago una tosca generalización, diría que literatura reciente se ha dedicado a señalar que los miembros del Círculo de Vienna, no eran tan herméticos como se pensaba.
Además, la cantidad de temas que discutieron es más vasta de lo que se creía.
Particularmente pensando en que cada uno de los miembros difería de los otros en tesis centrales.

\textcite{Suarez2024}, por ejemplo, discute que una preocupación genuina sobre los modelos en la ciencia, precede, data y prosigue a los años del Círculo de Vienna, el autor nos dice \say{I focus particularly on the nineteenth-century modelers and summarize their insights and contributions, which, I claim, remain essentially unsurpassed.} (p. 20)

Además, los miembros del Círculo de Vienna tuvieron una preocupación genuina por las prácticas científicas--Neurath, por ejemplo--
Tenían presente que ciertos factores externos pueden influenciar a la investigación científica.
\textcite[][p. 24]{Bentley2023} lo expresa mejor: \say{Frank's historical work, which frequently anticipates Kuhnian or post-Kuhnian themes, consistently emphasizes the significance of non-scientific, external factors on the decision making of scientists.}
Los miembros del Círculo de Vienna tenían claro que los factores externos influyen en el proceso de la investigación científica, presentes incluso en el \emph{contexto de justificación}.

Hablando en particular del trabajo de Neurath y las tesis que sostenía, Otto defendía una teoría coherentista de la justificación.
Neurath argumentaba que nuestras teorías científicas están subdeterminadas empíricamente.
En cualquier momento del tiempo hay hipótesis en pugna.
Pero si las teorías están empíricamente subdeterminadas, es posible que dos teorías internamente coherentes y externamente inconsistentes entre sí, convivan al mismo tiempo.

Pero si el escenario anterior es plausible, entonces es patente que aquél que defienda una teoría coherentista de la confirmación, no puede resolver el problema de la elección racional de teorías.
Debido a que todo el tiempo hay teorías en pugna, como filósofos, deberíamos poder explicar por qué es racional elegir entre una teoría y sus rivales. Para poder resolver este problema, Neurath defendía que los factores externos son información clave para explicar la racionalidad de la elección de los investigadores.
Factores externos como los eventos históricos fortuitos.

\textcite{Bentley2023} nos recuerda que Neurath fue falibilista sobre el conocimiento, es decir,  sostenía que cada una de nuestras creencias puede estar equivocada.
Sumado a esto, sostenía que la confirmación es holista: no son oraciones particulares, sino grupos de oraciones, los que contrastamos con el mundo.
Bentley lo expresa mejor \say{$\ldots$ it is always possible for a system of beliefs to be altered to accommodate a statement or for the statement to be rejected $\ldots$} (p. 21).



Esto último inició como un abstract que mandé a un taller y se fue volviendo más largo mientras leía.

Hablando de modelos en la filosofía de la ciencia durante el periodo del Círculo de Vienna, una figura importante que mencionar es Mary Hesse.
Mary Hesse trabajó con especial atención el uso de modelos en la ciencia.

``This historical chapter introduces the emergence in the nineteenth century of  what I call the modeling attitude.
This is a stance toward scientific work and  discovery, and it continues to this day." (p. 43)






