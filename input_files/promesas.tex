\chapter{Promesas}

\section{Promesas}

\noindent Este apartado es un mensaje para mi comité tutor escrito el 07 de junio del 2024. Sé que la fecha límite de evaluación se acerca y que hay muy poco tiempo para leer lo que les mando hoy.
Quería en este par de semanas escribir la introducción, porque me he dado cuenta de que no he sido muy claro con cuál quiero que sea la estructura de mi trabajo--me disculpo por eso.
Pero creo que ahora puedo hacer más explícita la estructura del proyecto.

Lo que quiero hacer es lo siguiente.
En el primer capítulo quiero argumentar por qué el \emph{contexto de descubrimiento} y el \emph{contexto de justificación} son genuinamente separables.
Con esto quiero decir que, para hablar de racionalidad científica, necesitamos ser capaces de vagamente aislar cuáles son los factores genuinamente epistemológicos y los no epistemológicos.
Y esto, me parece, es fácilmente capturado por la distinción.

Defiendo esto respondiendo a un artículo de \textcite{Yturbe1995}.
El artículo es viejo y creo que claramente es una caricatura de las posturas de \emph{los positivistas}
Pero para esas fechas Mary Hesse ya había publicado \say{Models and Analogies in Science}, la caricatura está injustificada.
Quiero aclarar que  al responder a este artículo tengo la intención de desarrollar las teorías filosóficas de \textcite{Hessen2009, Grossmann2009} tal como las reconstruye \textcite{Freudenthal2009} y las lecciones de \textcite{blasjo2022} sobre historia de la ciencia.
Porque me parece que una distinción importante porque, a diferencia de otras actividades humanas, la investigación tiene un interés por la Verdad.
Por ahora dejemos el término sin analizar.

En palabras muy generales, y poco adecuadas, Blasjo defiende que la historia de la ciencia debería verse como un proceso continuo entre las teorías del pasado y las teorías más recientes.
Esta postura tiene ciertos detalles que la hacen atractiva, por ejemplo, que asume que los investigadores del pasado se comportan más o menos como lo hacen los investigadores actuales--a mí me gustaría pensar eso.
En particular en que realmente buscan justificar sus teorías.
Esta parte es importante.

Sabemos que si bien había representaciones, digamos gráficas, en los primeros años del cálculo; y las representaciones analíticas aparecieron posteriormente, esto no implica que había una "barrera" en lo que podían pensar los investigadores.
No es que no les alcanzara para pensar en las soluciones analíticas: de hecho usaban las representaciones analíticas para sus apuntes, para hacerlos más breves.
Esta \emph{barrera} es algo de lo que hablan \textcite{Carrillo2021-CARAAP-12}  y que les sirve para argumentar que su postura artefactualista explica mejor cómo funcionan y se generan las idealizaciones, a diferencia de las posturas de los \emph{relevantistas causales} y lo que ellos llaman los \emph{desidealizadores}.
Su argumento o explicación, depende de señalar que su postura no parte de la relación modelo-mundo.

Pero esto no es una ventaja de ningún tipo.
Aquí es donde importan los autores raros de los que he estado hablando.
Su ejemplo principal es el del modelo Hodgkin-Huxley.
Su reconstrucción es equivocada.
Además, es falso que los investigadores no sean capaces de \emph{ver más allá} de sus compromisos teóricos.
Hodgkin y Huxley sabían qué posibles hipótesis podrían hacer que su modelo fuera más adecuado, pero siempre tomando en cuenta cuál es el objetivo específico.
Porque el modelo no está \emph{superado}, se sigue usando como metodología.
Pero me queda claro que H\&H no estaban haciendo lo que dice Carrillo y Knutiila.
Para esto estaba usando el libro \citetitle{Piccolino2013} que señalan algo muy parecido a lo que señalan Blasjo, Hessen, Grossman, etc., en corto, que la distinción \emph{factores externos} $|$ \emph{factores internos} es espuria.
Pero aún deberíamos poder explicar cómo diferentes aspectos sociales e históricos influyen en la justificación, pero centrándonos en la justificación.
No soy historiador y creo que esto lo expresan más bellamente Blasjo, Hessen y Grossman.
Pero como tengo a la mano el libro de \emph{shoking frogs} que toma una postura parecida a los autores rusos y de los paises bajos de los que he venido hablando.

Además, quiero decir que tomar el camino que sugieren los artefactualistas es erróneo porque no puede lidiar con casos como el de los estudios de IQ.
Sabemos que el IQ no mide nada y si mide algo, el estudio está completamente sesgado, ¿justificamos esto como lo dice Carrillo?
Por supuesto que no, porque tendríamos que decir que los involucrados en el diseño estaban decidiendo racionalmente con base en las analogías, modelos e idealizaciones, que tenían a la mano.
Esto es claramente falso, el estudio estuvo sesgado desde el inicio.
Revisar los estudios del pasado con las herramientas del presente no es más que hacer investigación.
Nunca me ha quedado claro exactamente como lidian los internalistas ni los externalistas con esto.
Afortunadamente, no tenemos que elegir, sólo justificar.

Para esto está el capítulo que ya les había mandado sobre Pritchard.
Realmente no es mucho más que una explicación de las diferentes posturas.
Como dije casi al inicio, la verdad es lo que le da valor al conocimiento, tal como defiende Pritchard.
Y creo que la noción de verdad es lo que falta por explicar: la metáfora de Pritchard de un \emph{contacto cognitivo profundo con la realidad} es muy vaga.
PEro creo que podemos analizar la naturaleza de la verdad en término distintos al de \say{mind independence}
Por eso quería decir algo al final sobre el artículo de \textcite{klein2019} y el trabajo de \textcite{friedman, friedmana}, en particular con los casos de Newton y Darwin.
Los historiadores mencionados y sus reconstrucciones, empatan muy bien con la teoría coherentista y con el trabajo que ha desarrollado Friedman.
Soy pésimo apostando, así que sólo apuesto si tengo la información completa.
Entonces me gustaría darles al menos la información necesaria para proseguir.
Creo que el conocimiento es al menos una creencia, verdadera y justificada.
Estoy al tanto del trabajo de Gettier, de las respuestas que se le dieron, del artículo de Zagzebski.
Pero no he encontrado una mejor manera de describir el conocimiento.
Estoy también al tanto de los trabajos posteriores acerca de la estructura de la justificación y estoy muy influenciado por el artículo de Sosa \textcite{Sosa1991}, por lo que creo que la estructura de la justificación es coherentista.
Me parece que esto está muy en línea con mis intereses en torno a la historia de la ciencia y que no podemos pedir mucho más para la justificación.
Estoy al tanto, también de los problemas que tienen las teorías coherentistas de la justificación.
Pero no quiero discutir esto porque el trabajo sería interminable: creo que con hacer referencia a los autores pertinentes es suficiente.

¿Qué tiene que ver todo esto con los modelos y la teoría de la verdad?, que debido a que los investigadores en el proceso de creación de modelos, sí parten de la intención de representar el fenómeno de alguna manera.
Además, esto está respaldado por la historia de la ciencia, tal como la describen \textcite{Hessen2009, Grossmann2009, blasjo2022}.
Que además los investigadores sí tiene la intención de \emph{representar adecuadamente} el fenómeno y que mucho de esto está guiado por sus intereses partículares \textcite{friedman}.
Que \emph{intereses particulares} no significa siempre completamente racionales o completamente irracionales, porque no hay distinción entre factores externos y factores externos.
Que los investigadores quieren \emph{justificar} sus creencias y por qué su investigación explica un fenómenode interés; también tienen en cuenta qué puede fallar con su modelo y pueden hipotetizar qué haría que su hipótesis o creencia sea falsa.
De hecho es un ejercicio que hacen \textcite{Hodgkin1951, Hodgkin1952a, Hodgkin1952b}.
El artículo de 1951 es genial porque el mismo Hodgkin reconstruye todo el debate que hay en su tiempo sobre el potencial eléctrico, cómo se le ocurrió la idea de usar los axones del calamar gigante, que muestra su modelo, cómo derivar las ecuaciones (usaron un método de \say{fuerza bruta} y no una solución analítica), etc.
Si les soy sincero, me parece extraño que Carrillo y Knuutiila no citen ese artículo.
Todo esto apunta a que el conocimiento de la ciencia es proposicional y son creencias verdaderas justificadas.
Porque es muy fácil hablar de conjuntos de proposiciones verdaderas o falsas.
Porque si hablamos en estos términos, podemos, por decir algo, aislar un subconjunto de las creencias de dos investigadores sosteniendo teorías distintas.
Y comprobar a la luz de nueva evidencia cómo corregir creencias.
Esto es algo que ya hace la teoría de la argumentació: sea formal o informal.
Y en este espacio no hay un terreno muy claro, y yo no soy un teórico de la argumentación, mis colegas lo hacen mucho mejor sin mi.
Además no puedo hablar de \emph{truthbearers} y \emph{truthmakers}, porque no hago metafísica, ellos ya lo hacen mucho mejor.
Por eso voy a tomar esto sin discutirlo: conjuntos de creencias, que son proposiciones, que son los \emph{truthbearers} de lo que sea que sean los \emph{truthmakers}.
Pero sea lo que sea ese truthmaker, debe estar relacionado de alguna manera con nuestras creencias: ya sea generalizar, hipotetizar, abstraer, idealizar, etc.
Y esta relación está de hecho en la historia de la ciencia y en las prácticas actuales: representar adecuadamente el fenómeno.
O como dice MClaughlin, allá donde nuestras creencias son exitosas, sabemos que estamos obedeciendo las leyes de la naturaleza.
Pero este conocimiento sólo es \say{parcial}, por usar un término poco adecuado.
Y que la única manera de saber cuándo una creencia es verdadera, es explicando mejor.
Y que el conocimiento es falibilista a la \emph{Klein} y que la teoría de virtudes nos ofrece la fuente de la normatividad para evaluar casos de conocimiento.
Y que podemos tener \citetitle{klein2019}.
Y que esto es lo más que le podemos pedir a la verdad con todos los constreñimientos anteriores: históricos, sociales, económicos en los que se desarrollaron las teorías.
Pero siempre buscando la justificación de hipótesis.

Esto es una imagen de lo que quiero decir.
Por supuesto, esta tesis sigue teniendo el en el título \emph{semántica de contrafácticos}
Pero con lo dicho hasta ahora sirve para dejar mis cartas sobre la mesa.
Entonces, asumiendo que todo lo dicho anteriormente es verdad--al menos tengo razonespara creerlo--la teoría más adecuada para repsentar las prácticas científicas bajo estos lentes es el empirismo modal \textcite{russelllogical}.
Que lidia con el problema del acceso: sabemos que los contrafácticos son verdaderos\footnote{en el torpe sentido en el que lo presenté} cuando funcionan\footnote{En el torpe sentido en el que lo presenté}.
Lo que justifica nuestro conocimiento causal basado en intervenciones\footnote{Ya deben estar cansado de esto}.
Y la teoría de Potochnik explica muy bien por qué \parencite{Potochnik2017-POTIAT-3}, y lo hace mucho mejor de lo que yo podría hacerlo.
También creo que es más amable con la postura de \textcite{suarez2004} del inferencialismo.


En corto, estoy tratando de participar en el desafío que presenta suárez.
Su libro no es sobre la verdad y otros conceptos semánticos.
Pero creo que modificando nuestro concepto de verdad\footnote{como la torpe presentación que realicé} de esta manera: tomando en cuenta la práctica y el uso de estas herramientas en la investigación científica.
Señalando cómo decimos que una creencia es verdadera por los constreñimientos a la mano y llenar el hueco semántico que señala \textcite[][p. 16]{suarez2004}

\begin{quote}
	The deflationary approach adopted in this book is aimed instead at scientific representation via modeling.
	It does advance and defend a deflationary account of the use of models in practice.
	It does not, however, advance  a thesis about semantics, truth, or any of the categories typical of analytic  philosophy of language.
\end{quote}


Ahora, si entendemos el debate bajo esta luz, es más claro el problema entre causalistas $|$ estadisticalistas.
Una de las premisas que he visto continuamente, pero que no he visto discutidas, es la de que los modelos estadísticos no pueden ser causales porque pertenecen al reino de lo abstracto.
Pero si nuestros modelos, al contrario, parten de un fenómeno y luego decidimos si es adecuada nuestra representación (decidiendo entonces si darle la etiqueta de verdadero), no hay misterio en fenómenos causando fenómenos.
Hacer cierta distinción entre la selección natural como proceso y la selección natural como causa, que es algo en lo que ha venido insitiendo \textcite{Millstein2006, Millstein2002}.
Pero no sé si quiero entrar a detalle en esto, fue lo que intenté hacer--fracasando abruptamente--en la maestría.
Sólo puedo señalar ciertas consecuencias de la imagen que les presento.

Creo que el proyecto no es muy original y creo que es hasta el día de hoy (08 de junio del 2024) que puedo llamarlo proyecto.
Me disculpo por esto.
Tengo buena parte de lo que aparece en el pdf como el capítulo 2, pero lo tengo en otra carpeta\footnote{está en el git}.
Ese documento necesita mucha edición, entonces no recomiendo que lo lean.
El capítulo está en mi mal inglés porque partió de un abstract que mandé a un evento, luego sólo continué.
Las partes relevantes para la evaluación son la Introducción  y el capítulo 2.
Todavía tengo que editar lo demás del texto, entonces tampoco recomiendo leerlo.
Recomiendo encarecidamente leer esta sección.

