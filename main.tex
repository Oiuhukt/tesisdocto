% Template inspired by
% https://tufte-latex.github.io/tufte-latex/
% http://c-elvira.github.io/ Ph.D. thesis
% https://jflamant.github.io/ Ph.D. thesis
% https://guilgautier.github.io/ Ph.D. thesis

\documentclass[letterpaper, twoside, table, justified, nofonts, nobib, nohyper]{tufte-book}


\usepackage{soul}

% Book metadata
\title{Title}
\author[Oscar Abraham Olivetti Alvarez]{Oscar Abraham Olivetti Alvarez}
\publisher{UNAM}

%%%%%%%%%%%%%%%%%%%%%%%%%%%%%%%%%%%%%%%%%%%%%
% default imports and commands are located in
% tufte-common-local.tex
%%%%%%%%%%%%%%%%%%%%%%%%%%%%%%%%%%%%%%%%%%%%%

%\input{input_files/acronyms_and_glossary}  % must be imported in preamble %movido a posopts

% The bibliography is managed with biblatexmain.tex
\addbibresource{bibliography.bib}

\usepackage{fontawesome}
\newcommand{\conferenceIcon}{\textcolor{myblue}{\faUsers}}
\newcommand{\paperIcon}{\textcolor{burgundy}{\faNewspaperO}}

\usepackage{amsthm}
\makeatletter
\let\c@theorem\relax
\let\c@corollary\relax
\let\c@definition\relax
\let\c@example\relax
\let\c@lemma\relax
\let\c@proposition\relax

\let\corollary\relax
\let\definition\relax
\let\example\relax
\let\lemma\relax
\let\proposition\relax

\newtheorem{theorem}{Theorem}[section]
\newtheorem{assumption}[theorem]{Assumption}
\newtheorem{corollary}[theorem]{Corollary}
\newtheorem{definition}[theorem]{Definition}
\newtheorem{example}[theorem]{Example}
\newtheorem{lemma}[theorem]{Lemma}
\newtheorem{proposition}[theorem]{Proposition}
\newtheorem{remark}[theorem]{Remark}
\makeatother

\usepackage{amsmath}
\counterwithin{equation}{section}

\usepackage{amsfonts, amssymb}

\begin{document}

%%%%%%%%%%%%%%%%%%%%%%%%%%%%%%%%%%%%%%%%%%%%%%%%%%%%%%%%%%%%%%%%%%%%%%%%%%%
% The front matter contains title page, acknowledgements, toc, nomenclature and dedication
%%%%%%%%%%%%%%%%%%%%%%%%%%%%%%%%%%%%%%%%%%%%%%%%%%%%%%%%%%%%%%%%%%%%%%%%%%%

\frontmatter

\maketitle

\setcounter{chapter}{-1}  % Start at Chapter 0 (Introduction)
\setcounter{secnumdepth}{3}
\setcounter{tocdepth}{3}

%\input{input_files/acknowledgements} %movido a posopts, es decir, posibles opciones

\begin{fullwidth}
	\tableofcontents
\end{fullwidth}

%\listoffigures

%\listoftables

%\input{input_files/nomenclature} %movido a posopts

%\input{input_files/dedication} %movido a posopts

%%%%%%%%%%%%%%%%%%%%%%%%%%%%%%%%%%%%%%%%%%%%%%%%%%%%%%%%%%%%%%%%%%%
% The main matter contains numbered chapter: introduction, chapters
%%%%%%%%%%%%%%%%%%%%%%%%%%%%%%%%%%%%%%%%%%%%%%%%%%%%%%%%%%%%%%%%%%%

\mainmatter

% !TEX root = ../main.tex
% LTeX: language=es, en

\chapter{Introducción}\label{ch:introduction}

\section{Una breve anécdota}

\noindent El doctorado ha sido un largo viaje durante el cuál aprendí mucho.
Una parte considerable del aprendizaje está relacionado con la filosofía\footnote{Juro que este breve relato tiene un punto.}
En el posgrado profundicé en los temas que me interesaban, y quiero mencionar que mi interés en estos temas nació durante mi licenciatura.

Durante la Licenciatura tuve profesores muy malos y un puñado de muy buenos profesores.
Pero mi interés en filosofía de la ciencia se debe a dos de ellos; ambos son excelentes maestros y excelentes filósofos.
En las clases que tomé con ellos vimos temas que despertaron mi curiosidad.
Todos los temas fueron sumamente interesantes, pero mi atención la dirigí especialmente a la \emph{explicación científica} y la \emph{naturaleza de la causalidad}.
Mis intuiciones --todavía queda un resabio de esto-- señalaban que encontrar la causa de un fenómeno es la mejor manera de explicar por qué sucede ese fenómeno.
Pero al finalizar mi licenciaturas tuve teniendo severos problemas con la filosofía.
Sentí que lo que había aprendido era fútil y creo que esto se debió a que los profesores malos fueron más que los buenos.

Tenía entonces esta \say{crisis existencial filosófica} y el hecho de que los dos profesores buenos trabajaran filosofía de la ciencia --al menos lo que en ese momento creía que era la filosofía de la ciencia-- sesgó mis intereses a lo que hasta ahora ha sido mi trabajo.
Esta influencia me hizo tomar la decisión de entrar al posgrado en filosofía de la ciencia, por supuesto, empezando con la maestría.

En las clases de maestría hubo profesores buenos, pero también los hubo excelentes.
Mucho de lo que aprendí en estos cursos me sirvió para enmarcar de diferente manera las preguntas que me preocupaban, lo cuál me permitió entender --mejor, me parece-- la naturaleza de la investigación científica.
En particular quería responder preguntas sobre la epistemología de la ciencia: \say{¿qué estamos justificados a creer?}, \say{¿cuándo podemos afirmar que una hipótesis ha sido corroborada?}, \say{¿es la explicación o compresión más básica que el conocimiento?}, etc.
Preguntas que surgen a partir de mi interés por la explicación causal.
En la maestría aprendí nuevas metodologías de investigación y maneras diferentes de plantearme y entender las preguntas que me preocupaban.

Pero sin lugar a dudas, lo más valioso que aprendí, fue la importancia que tiene la historia de la ciencia en la filosofía de la ciencia.
Me voy a permitir hacer una breve caracterización, exageradamente general, de dos posturas en historia de la ciencia.


\section{Un breve repaso de dos posturas}

\noindent El siguiente será un breve y burdo repaso de las \say{metodologías}\footnote{Esto que describo es igual a lo que Suárez llama \say{posturas}} \emph{externalistas} e \emph{internalistas}.
Las llamo \say{metodologías} porque difieren en términos sustanciales --por ejemplo, el de \emph{verdad}-- y es a partir de estas diferencias que ambas metodologías ofrecen una interpretación distinta de la historia de la ciencia.
Al usar el término \emph{metodología}, no quiero comprometerme con si son teorías en contradicción, o teorías complemetarias; esto es algo que no puedo discutir aquí y el nombre \say{metodologías}, me parece, es lo suficientemente neutro\footnote{
Lo que llamo aquí \emph{metodologías}, corresponde a lo que Suárez llama \emph{postura} \cite{suárez emergence}
}.

%La sección \ref{sbc:yturbe} la dedico a analizar una supuesta alternativa para el debate anterior.
%Además, esta alternativa, es más cercana a la filosofía de la ciencia.

\subsection{Dos metodologías: \emph{externalismo} e \emph{internalismo}}

\noindent Digamos que en historia de la ciencia se suelen distinguir dos corrientes: los \emph{internalistas} y los \emph{externalistas}.
De manera procaz, los internalistas argumentan que la historia de la ciencia debería interpretarse como un progreso de ideas; cada uno de los nuevos descubrimientos en la ciencia están motivados por intenciones internas a los investigadores: la búsqueda de la verdad, curiosidad natural, el gusto por explicar fenómenos, etc.

Con mejores métodos de investigación, los investigadores ofrecerán mejores respuestas a sus preguntas.
Tendrán, por decirlo de alguna manera, respuestas más detalladas del fenómeno que les interesa y descubrirán nuevos fenómenos que no contemplaban originalmente.
Todo este proceso guiado siempre por sus intereses.
Al ser guiados por sus intereses, y sólo por sus intereses, los investigadores son inmunes a los fenómenos culturales que les rodean.
Dicho de otra manera: que la ciencia es inmune a la cultura.
Creo que podemos identificar a Weinberg\footnote{
	Esto es tramposo porque no sé si los historiadores de la ciencia estiman a Weinberg como uno de sus colegas.
	Aparece en este escrito para fines puramente explicativos.
	} como un internalista, cuando dice que:

	\begin{quote}
		The word \say{discovery} in the subtitle is also problematic.
		I had thought of using The Invention of Modern Science as a subtitle.
		After all, science could hardly exist without human beings to practice it.
		I chose \say{Discovery} instead of \say{Invention} to suggest that science is the way it is not so much because of various adventitious 		historic acts of invention, but \emph{because of the way nature is}. \parencite[Prefacio, énfasis agregado]{Weinberg2015}
	\end{quote}

Quiero destacar el énfasis que hice en la cita anterior.
Creo que no hay mucho peligro en asumir que la investigación es la mejor manera que tenemos de producir conocimiento; y en particular, creo que parte de este conocimiento depende de \say{la forma en la que la naturaleza es.}
Creo, además, que la investigación tiene como meta ofrecer explicaciones \emph{verdaderas} de los fenómenos; puedo decir entonces que me alíneo con Weinberg en un sentido restringido de \say{descubrimiento}\footnote{
	El sentido restringido del que hablo es algo que quiero discutir a lo largo de este capítulo introductorio. Al menos es parte de lo que voy a defender en este documento.
	El lector tendrá que leer el trabajo completo y regresar a esto cuando termine.}.
Para regresar a la caracterización del \emph{internalismo}: el internalista estima el papel que la \emph{verdad} juega en la ciencia; la ciencia \emph{descubre} fenómenos, no los \emph{inventa}.

Por supuesto, decirlo de esta manera esconde varios problemas, por ejemplo, esconde que \say{ciencia} no es un término que esté libre de debate; debate que una vez que entramos en los detalles se torna bastante complejo.
Hay por lo menos dos maneras en las que el término es problemático.

En primer lugar, en filosofía de la ciencia, los filósofos debaten si es posible distinguir entre ciencia $/$ no-ciencia.\footnote{El debate es más sofisticado y no podría explicarlo con justicia en este documento.
	El lector interesado puede revisar la entrada \parencite{sep-pseudo-science}.}.
En segundo lugar, los historiadores de la ciencia debaten si es posible definir en qué periodo comienza la \emph{ciencia} tal como la conocemos en tiempos contemporáneos.

Y si adoptamos una metodología externalista, estas discusiones se vuelven cada vez más claras; y para lograr esta claridad debemos resaltar los \say{actos históricos fortuitos} de los que habla Weinberg; todo ello con la finalidad de elucidar cómo proceden las personas cuando investigan, qué influencias externas afectan el proceso de investigación; y que la historia debe jugar un papel central, si es que queremos responder estas preguntas.

Un externalista explicará los procesos de investigación científica a partir de las condiciones históricas en las que se desarrollaron; periodos de tiempo durante los cuales se desenvolvieron los investigadores y sus investigaciones.
El externalista resalta que la investigación científica no es una actividad que pueda separarse de su contexto \say{cultural}.

Quiero hacer notar que introduje deliberadamente el término \say{verdad.}
La naturaleza de la verdad es una discusión que requiere más detalle; pero este término es central para la discusión que seguiré a lo largo del documento.
Por ello, quiero rápidamente señalar cómo la \say{verdad} difiere en las dos metoddologías que he descrito.
Sobre la \say{verdad,} Shapin se expresa claramente cuando afirma que \say{All claims have to win credibility, and credibility is the outcome of contingent social and cultural practice.} \parencite[Capítulo 2]{shapin2010never}

Adoptando, digamos, una metodología externalista, Shapin se ha dedicado a estudiar detalladamente como los fenómenos sociales irrigan conceptos tan centrales en la investigación científica como es el de \emph{verdad}; y para aclarar el punto, la siguiente cita es ilustrativa

\begin{quote}
	The notion of truthfulness was thus central to the description of gentle qualities.
	Through the Renaissance and into the eighteenth century an honorable man and an honest man were interchangeable designations: \say{honesty} included the notion of truthtelling but was understood far more broadly to include concepts of probity, uprightness, fairdealing, and respectability. \parencite[pp. 70-71]{Shapin1995}
\end{quote}

Es decir que la verdad estaba asociada no sólo a ser honesto, sino con la figura del hombre honorable; la nobleza y la riqueza suelen estar asociadas al hombre honorable, por tanto su \say{palabra} valía más que las de otros.

Si lo que he descrito hasta ahora es ligeramente correcto y asumimos que Weinberg es un \emph{internalista} y que Shapin es un \emph{externalista}, entonces los párrafos de arriba son suficientes para saber cómo la \emph{verdad} juega un papel en ambas metodologías.
Para el internalista el objetivo de la ciencia es la verdad.
Esta verdad, por supuesto, debería estar acorde con la forma en la que la naturaleza es.
Por otro lado, el externalista nos dice que la verdad siempre está atravesada por un contexto histórico y que no siempre depende del modo en que la naturaleza es, sino también por consideraciones sociales como la honorabilidad.

Parece que esta breve caracterización me ha desviado del tema.
Pero el papel que juega la verdad en ambas metodologías es algo que los filósofos han discutido ampliamente.
Además, esta exposición es importante porque quiero convencer al lector de que hay que considerar seriamente la distinción entre el \emph{contexto de descubrimiento} y el \emph{contexto de justificación.}
¿Por qué lo anterior está relacionado con el \emph{contexto de justificación} y el \emp{contexto de descubrimiento}?
Porque hay una autora que señala que la distinción entre factores \emph{internos} y \emph{externos}, surge a partir de la distinción de los dos contextos.

Para lograr mi propósito, quiero discutir un artículo que presenta un caso más cercano a la filosofía de la ciencia.
Un artículo que relaciona por un lado a las metodologías que he discutido hasta aquí, y por otro cómo éstas juegan un papel en la filosofía de la ciencia.
Me dedico a este caso a continuación.

\subsection{Un caso más cercano a la filosofía de la ciencia}\label{sbc:yturbe}

\noindent En el artículo \citetitle{Yturbe1995}, la autora argumenta que es un falso dilema tener que elegir entre las metodologías \emph{internalista} y \emph{externalista.}
Yturbe dice, por ejemplo que

\begin{quote}
	The general tendency in the historiography of the sciences is to consider the social function of science, as weIl as some aspects of the matrix from which the problematic is formed, as external elements, while the conceptual apparatus and the problem field are treated as internal. \emph{But we should not think of science as having two independent histories.} \parencite[p.85, Énfasis agregado]{Yturbe1995}
\end{quote}

La oración en itálicas implica que no deberíamos hacer una distinción entre factores internos y factores externos.
Además, Corina defiende que una teoría a la vez internalista como externalista no es una teoría plausible.
Al menos esto parece sugerir cuando dice que

\begin{quote}
	The search for new explanatory programs is characterized above all by the attempt to reconcile the internalist and externalist approaches.
	But, in view of the fact that both approaches are committed to theses concerning the nature of science that are not only incompatible, but are unsustainable, their union without important changes in their philosophical presuppositions cannot result in a viable program. \parencite[p. 79]{Yturbe1995}
\end{quote}

Su conclusión se basa en que no es posible marginar ambos factores; además, la autora sugiere que esta supuesta distinción tiene sus orígenes en la distinción entre el \emph{contexto de descubrimiento} y el \emph{contexto de justificación} que trazaron los positivistas.
La autora señala, por ejemplo que

\begin{quote}
	External factors are not only found in the context of discovery, they are present also in the development of the concepts, problems, methods, problem fields, etc. of scientific discourses: that is, external factors are present in the context of validation itself.
	There are scientific discourses in which ideological conceptions pass on to form part of the body of the science itself, functioning as principles which define its field of study or guide its research; thus, external factors can become internal. \parencite[p. 85]{Yturbe1995}
\end{quote}

La cita anterior sugiere que el \emph{contexto de justificación} involucra sólo \emph{factores internos,} mientras que el \emph{contexto de descubrimiento} involucra sólo \emph{factores externos.}
La autora nos dice que los factores externos contagian al contexto de justificación, cuando dichos factores se convierten en parte de las teorías --generando conceptos, principios centrales, etc.--
La autora dice que esta ceguera a considerar que los \emph{factores externos} afectan al \emph{contexto de justificación,} surge de la \say{doctrina de los dos contextos.}
Corina nos dice que \say{One of the philosophical theses in favor of the contraposition between the internalist approach and the externalist approach is the so-called doctrine of the two contexts, developed by positivism.} \parencite[p. 75]{Yturbe1995}

Debido a que quiero convencer al lector de que deberíamos adoptar la distinción entre los dos contextos, debo señalar que la autora se equivoca.
Que es perfectamente plausible hacer la distinción entre el \emph{contexto de descubrimiento} y el \emph{contexto de justificación} sin que esto implique la distinción enctre factores \emph{internos} y \emph{externos.}

Para lograr esto, voy a señalar que la premisa, en la que la autora nos dice que el contexto de justificación$/$descubrimiento es equivalente a los factores internos$/$externos, es falsa y que la equivalencia descansa en una confusión.
Esta confusión se debe, en parte, a la mala caracterización de las tesis del \say{positivismo}; la otra parte, me parece, se debe a una interpretación poco caritativa de la historia de la ciencia y los propósitos de los investigadores.

El argumento de la mala caracterización del positivismo lo doy en este mismo capítulo.
El argumento sobre la interpretación y los propósitos de los investigadores será dado a lo largo del capítulo 2.
Pero, este argumento comienza en este capítulo, en especial la sección \ref{ssc:aprender}.

Recordemos que el argumento que voy a ofrecer es para concluir que la autora ofrece una mala caracterización del positivismo.
Esta conclusión depende de que el \emph{contexto de descubrimiento} y el \emph{contexto de justificación} no son equivalentes a los \emph{factores externos} y \emph{factores internos} respectivamente; lo que Yturbe llama \say{la doctrina de los dos contextos}.
Para este propósito quiero repasar brevemente en qué consisten el \emph{contexto de descubrimiento} y el \emph{contexto de justificación.}



\newthought{La distinción} entre el \emph{contexto de descubrimiento} y el \emph{contexto de justificación}, generalmente está asociada a Reichenbach\footnote{
	Stillwell comenta que la distinción puede trazarse hasta Arquímedes.
	Al menos eso entiendo cuando afirma que \say{Archimedes was probably the first mathematician candid enough to explain that there is a difference between the way theorems are discovered and the way they are proved.} \parencite[p. 56]{stillwell1989mathematics}
	}.
En su libro \citetitle{reichenbach1938experience}, el autor usa esta distinción para \say{excluir} los así llamados factores externos (sean sociales, políticos o económicos)\footnote{
	El pasaje no dice exactamente esto, sino algo más cercano a que tenemos claro cómo hacer un análisis filosófico$/$formal para la justificación de teorías, mientras que no tenemos manera de hacer ese análisis en el contexto de descubrimiento.
	Este lado de la investigación científica, digamos, es muy heterogéneo como para que sea posible ofrecer un análisis formal.
	}.
Esta distinción, nos dice Reichenbach, sirve para ilustrar el hecho de que no tenemos a la mano un \say{método} para analizar el fenómeno del descubrimiento científico; lo que algunos filósofos \parencite{reichenbach1938experience, Seo2015} llaman el momento \emph{Eureka}.
Pero no tener una herramienta de análisis formal, no implica que lo que sucede en el contexto de descubrimiento sea absolutamente deleznable.

Que los factores \emph{externos} pueden afectar el \emph{contexto de justificación}, fue un tema que se discutió ampliamente durante los años del círculo de Viena\footnote{
	O positivismo lógico o empirismo lógico, cada quien elige su etiqueta favorita.
} y no es una distinción que, digamos,  permaneció fija a lo largo de lo que, asumo, Yturbe llama \say{positivismo}.

Como me parece que la distinción juega un papel importante en (i) cómo interpretamos filosóficamente la historia de la ciencia y que juega un papel clave en (ii) cómo los investigadores realizan investigación.
Creo que ofrecer una justificación para (i) y (ii), es suficiente para convencer al lector que vale la pena adoptar la distinción.
Llamaré a (i) el objetivo metodológico y a (ii) el objetivo epistémico
Para lograr estos objetivos, quiero repasar brevemente cómo algunos de los miembros del círculo de Viena lidiaron con la distinción.


\subsection{El círculo de Viena}

\noindent Quiero comenzar señalando que los positivistas\footnote{
	Me voy a referir con este término a los filósofos que aparecen en la publicación de la \emph{International Encyclopedia of Unified Science} \parencite{Carnap1938-CARFOL-10}.
	Tanto del comité organizador, como el comité asesor.
	}, 
estaban al tanto de cómo los \emph{factores externos} influyen en el \emph{contexto de justificación}.
Dicho de otra manera, que los aspectos sociales, políticos y económicos afectan las prácticas de las personas dedicadas a hacer investigación.

La afirmación de Reichernbach sobre la carencia de un análisis formal del \emph{contexto de justificación} es algo que los miembros del círculo tenían en su agenda de investigación; 
y lograron dicho análisis con mayor o menor éxito.

\newthought{Neurath}, nos recuerda Joseph Bentley, sostenía que

\begin{quote}
	Despite his advocacy for scientific methods, Neurath never takes this method to be set in stone, nor does he attempt to portray science as an enterprise of purely objective methods, completely divorced from social, historical, and material contexts or the personalities of scientific practitioners. \parencite[p.~41]{Bentley2023}
\end{quote}

Neurath sabía que los fenómenos sociales y políticos afectan las prácticas científicas; sin embargo, sostenía también, que estudiar esta relación debe llevarse a cabo con otro tipo de herramientas.
Bentley lo expresa mucho mejor cuando señala que \say{As in the case of theory-choice, decision-making is central. But to make metatheoretical decisions, metatheoretical information is needed.} \parencite[p.~62]{Bentley2023}

Neurath llama \say{beahaviouristics of scholars} al conjunto de herramientas y métodos para reunir esta información metateórica; 
la cuál sería la disciplina con las herramientas para analizar aspectos históricos, políticos, económicos, etc;
que es a lo que estamos llamando factores externos.

Y al hablar específicamente de la justificación de creencias, Bentley señala que \say{[science], Neurath maintains, it is still the best we have} \parencite[p.~41]{Bentley2023}.
Y si somos capaces de reconocer cómo las prácticas juegan un papel, podemos evaluar y mejorar dichas prácticas.

Mejorar las herramientas que usamos en investigación, no es una tarea fácil. 
Porque modificar una herramienta tiene como consecuencia analizar detalladamente los resultados que dependen de dicha herramienta.

Para mejorar nuestras prácticas, no debemos olvidar que Neurath fue un \say{holista} confirmacional; esto significa que Neurath creía que no podemos confirmar o falsificar una proposición aislada, sino que todo el conocimiento es juzgado a la vez; 
cada vez que modifiquemos una hipótesis o alguna de nuestras herramientas, hay que hacer cambios en otras hipótesis y herramientas.\footnote{
	Tanto fue su interés por sistematizar la comunicación entre diferentes comunidades científicas que diseñó el proyecto de la Enciclopedia de la  Ciencia Unificada.
	De esta manera el trabajo se distribuye entre distintas comunidades y se coteja con el trabajo de otras comunidades de investigación.
	}

Sólo podemos hacer mejores nuestras herramientas y minimizar los errores cuando somos capaces de reconocer el papel que juegan los factores externos, Bentley expresa mejor el objetivo de Neurath diciendo \say{if we recognize the ... creation of the norms, methods and values of science, it can be made better.} \parencite[p.~41]{Bentley2023}

Quiero enfatizar que Neurath no fue el único miembro de los positivistas lógicos que reconoció el papel que juegan estos factores en la investigación;
las personas que realizan esta actividad tienen ciertos sesgos políticos y sociales, son personas que tienen que tomar decisiones y que tratan de justificar sus hipótesis con las mejores herramientas a la mano.
Philipp Frank expresa mucho mejor la relación entre los factores externos y el contexto de justificación.



\newthought{Philipp Frank} sostenía que los aspectos \say{externos} eran una parte crucial del análisis que los filósofos pueden ofrecer de las ciencias.
En su artículo \citetile{Frank1956}, Frank argumenta que los aspectos sociales jamás han estado excluidos del proceso de justificación de teorías.
Esto es más o menos claro cuando Frank nos dice que \say{The special mechanism by which social powers bring about a tendency to accept or reject a certain theory depends upon the structure of the society within which the scientist operate.} \parencite[p.~143]{Frank1954}

Para su argumento, Frank comienza enfatizando que a lo largo de la historia, la elección de teorías nunca es arbitraria;
supongamos, por ejemplo, que tenemos dos teorías en competencia $a$ y $b$.
Los investigadores no simplemente deciden entre $a$ o $b$ sopesando cuál de las dos tiene más consecuencias empíricas.
Frank nos recuerda que tomar una decisión entre $a$ y $b$ involucra diferentes aspectos políticos y sociales.
Estos aspectos son parte del proceso de justificación de una teoría, porque no sólo revisamos las consecuencias empíricas de la teoría, sino además qué tan coherente es con otros dominios (digamos biología, química, economía, etc.)
Estos aspectos están involucrados en la justificación de teorías  porque la investigación es un producto realizado por seres humanos;
como seres humanos, tenemos capacidades cognitivas limitadas \parencite{Potochnik2017-POTIAT-3}, además tenemos una gran cantidad de sesgos implícitos \parencite{nordell2021end}, etc.
Si para el proceso de elección de teorías sólo evaluamos a la teoría al medir qué tanto está \emph{de acuerdo con los hechos}, nunca podríamos decir si una teoría es \emph{mejor que} la otra.

Hay que aclarar que Frank expresa que una teoría \emph{es mejor que} otra, cuando medimos \say{mejor que} en términos de la \emph{utilidad} de la teoría.
Utilidad que depende de los propósitos que queremos lograr;
y si la teoría va a ser \emph{usada}, entonces debe ser \emph{simple.}
Esto lo expresa diciendo que \say{The final theory has to be in fair agreement with observations and also has to be sufficiently simple to be usable.} \parentcite[p.~14]{Frank1956}
Siguiendo esta línea de razonamiento, una teoría no se juzga sólo a partir de las consecuencias empíricas, sino también de acuerdo a qué uso queremos darle, es decir, cuando es usada para fines \emph{prácticos.}

Frank lo expresa mucho mejor al decir que \say{[h]owever, the situation becomes much more complex, if we mean by simplicity not only simplicity of the mathematical scheme but also simplicity of the whole discourse by which the theory is formulated.} \parencite[p.~4]{Frank1956}
Hay que saber, además, que el discurso bajo el que la teoría está formulada, en algunas ocasiones, sirve para fines propagandísticos.
Tanto Neurath como Frank condenaban dichos fines propagandísticos, pero la única manera de saber cuándo los fines son propagandísticos, debemos saber cuáles son los supuestos de la teoría y la finalidad que se le está dando.
Si involucramos la utilidad de la teoría y el discurso bajo el cual está formulada una teoría, entonces como filósofos de la ciencia,  podemos ofrecer un análisis más completo de la ciencia.

Las afirmaciones anteriores (i) que hay que considerar el discurso bajo el cuál la teoría está formulada y (ii) que el valor de la teoría depende también de los fines para los cuales va a ser usada, están inscritas en el problema de la \emph{elección de teorías} y en el problema de la \emph{subdeterminación empírica de las teorías.}
Supongamos que queremos decidir si una teoría $a$ es mejor que $b$. 
Según lo que nos dicen Neurath y Frank debemos involucrar los factores externos y es en este sentido en el que los factores externos son parte del contexto de justificación.
Aceptar o rechazar una teoría es un proceso que involucra \emph{justificar} las teorías y decidir cuál es \emph{mejor} para los propósitos que deseamos.
Es durante la fase de aceptación de la teoría, cuando los factores externos toman un papel central, porque de no ser por la finalidad práctica de una teoría, no tendríamos manera de decidir qué teoría es más adecuada.
Don Howard señala esto al decir que 

\begin{quote}
	On Neurath’s view, it is a contingent fact, well supported by historical evidence, that we do choose among empirically equivalent theories on the basis of our estimation of the likelihood of their serving our favored social and political ends, this especially in sciences like economics and sociology. \parencite[p.~5]{Howard2006}
\end{quote}

Muchas de estas afirmaciones se deben a que los miembros del círculo de Viena estaban interesados en saber si $a$ es equivalente a $b$;
y algunos de los miembros del círculo argumentaban que la única manera en que $a$ y $b$ sean equivalentes es si hacen las mismas predicciones, es decir, si tienen las mismas consecuencias empíricas.
O como también dice Don Howard

\begin{quote}
	The place of values in science is secured by the fact that, on Neurath’s view, logic and experience underdetermine theory choice.
	But turn that argument around and it implies that values come into play only within what I like to call the domain of underdetermination. 
	That is, logic and experience are first allowed to do all of the work they can do. 
	Only then do we ask which of several empirically equivalent theories is most conducive to the achievement of our social and political ends. \parencite[p.~10]{Howard2006}
\end{quote}

Señalo lo anterior porque no es completamente obvio por qué las consecuencias empíricas no son suficientes para resolver el problema de la elección de teorías;
No es obvio porque es completamente plausible que $a$ y $b$ tengan \emph{distintas} consecuencias empíricas;
además, suponiendo que de hecho hay subdeterminación empírica, es completamente plausible que $a$ y $b$ sean contradictorias entre sí y que tengan las \emph{mismas} consecuencias empíricas.

Para aclarar por qué no es suficiente, hay que saber que esta conclusión está motivada por el fenómeno de la \emph{subdeterminación empírica de las teorías.}
El fenómeno de la subdeterminación empírica sucede cuando en un punto del tiempo la evidencia no es suficiente para determinar qué teoría deberíamos aceptar.
Dos teorías contradictorias entre sí pueden hacer las mismas predicciones, en cuyo caso, la evidencia no es suficiente para decidir entre una teoría u otra.
Ahora, supongamos que en un momento dado sólamente existe una teoría.
Supongamos además que esta teoría ha hecho predicciones exactas hasta ahora y que dada la teoría predecimos que sucederá un fenómeno $f$.
Llega la fecha de la predicción y $f$ no sucede.

Si la descripción anterior le parece plausible al lector, entonces hay que investigar en qué consistió el error.
Sabemos además que una teoría depende de un conjunto de supuestos y que si $f$ no sucede, entonces hay que revisar cuál de los supuestos de la teoría falló.
Si el valor de la teoría depende sólo de las predicciones que podemos obtener al hacer uso de ella, entonces la teoría ha perdido su valor.

Pero no es obvio que las teorías que hacen predicciones erróneas pierden completamente su valor, hay al menos un ejemplo en el que modificar alguno de los supuestos ha llevado a descripciones exitosas: el caso del desccubrimiento de NEptuno  
Si hay casos reales de subdeterminación empírica en la historia de la ciencia no es algo que pueda discutir aquí.
Pero saber que puede suceder un caso como este es suficiente para concluir que si la evidencia es lo único que importa, entonces puede haber casos de subdeterminación empírica 



haciendo énfasis en que estos factores sociales jamás han desaparecido del \emph{contexto de justificación}.
En particular, ante dos teeorías en competencia, la decisión no puede tomarse sin este tipo de consideraciones prácticas.
Consideraciones prácticas que dependen del contexto histórico y sociocultural\footnote{En algunas ocasiones la utilidad de la teoría es funcionar como propaganda, algo que señalan los autores de \parencite{Lewontin2017}  y también Frank en \parencite{}.}.

Hasta este punto, me parece que la evidencia textual muestra suficiente información para concluir que hay una confusión en la caracterización de la llamada \emph{doctrina de los dos contextos} \parencite{Yturbe1995}.
Se suele caracterizar a los miembros del círculo de Viena defendiendo tesis exageradamente estrictas, Philipp Frank incluso lo menciona en \say{cita de Frank} y más recientemente, Bentley también cuando dice que \say{citga de bentley}.
Y esta no es la única mención a esto \say{cita del número especial}

Suele mencionarse también que los miembros del círculo no prestaron atención a ciertos temas por estar demasiado ocupados con la física [citar al de Sahorta] o que el problema de la reducción de teorías incluía el proyecto de reducción de las ciencias especiales a la física [citar a Raphael van Riel].
Afortunadamente, existen intentos recientes por ofrecer una caracterización justa de las afirmaciones de los miémbros del \emph{círculo de Viena.}

Quiero señalar que esto puede ser una preocupación más filosófica que histórica, y por lo tanto, parecer que como Yturbe \cite{Yturbe1995} estoy confundiendo una tesis filosófica y una tesis sobre historiografía.
No soy historiador, por lo que no puedo más que referirme a dos autores en particular que, me parece, vinculan las preocupaciones de los miembros del círculo con la historiografía de la ciencia.
A esto me deddico en la siguiente sección.


\subsection{Justificar una teoría es el objetivo principal de la investigación}

Llegado a este punto, he señalado que la caracterización que hace la autora de la \emph{doctrina de los dos contextos} es incorrecta; 
más aún, prometí que valía la pena recuperar esta distinción porque me parece que el objetivo principal de la investigación 
científica es el \emph{proceso} de \emph{justificar} teorías.
Quiero llegar a esta conclusión a partir de lo que dije anteriormente: como la caracterización de Yturbe es errónea, tenemos una alternativa; 
nuestra alternativa es que hay una manera de separar el \emph{contexto de justificación} del \emph{contexto de descubrimiento}.




\newthought{Justificar} una teoría, nos dicen ambos filósofos, involucra necesariamente aspectos pragmáticos.
La práctica científica no está separada de su contexto cultural.
Pero cuando lidiamos con procesos de justificación, lo mejor que tenemos son las herramientas que usamos en investigación.
Herramientas que pueden cambiar con el tiempo y que necesitarán una justificación más robusta.
Todo esto con el objetivo de señalar si una oración es verdadera o falsa.
Reconociendo que la verdad y la falsedad de las oraciones no está separada de otras oraciones.

Estos comentarios sobre el \say{materialismo dialéctico} y el \say{pragmatismo americano},
% Mencionar la mención al operacionalismo 
, pueden sonar fuera de lugar.
Pero es importante mencionar esto porque las fuentes históricas que voy a citar son, una de ellas soviéticas, usando la llamada metodología del \say{materialismo histórico}, mientras que la otra fuente ofrece una interpretación operacionalista de la matemática griega.

El pragmatismo americano se vincula al operacionalismo

Pero

Cómo juega el verificacionismo un papel?, esto es, por qué la manera de verificar si una oración es verdadera o falsa depende de cómo se verifica?
Lo único que nos piden los positivistas es que tengamos cuidado al juzgar si la oración es verdadera.
Al juzgar una oración, debemos tener en cuenta qué operaciones serían necesarias para saber si la oración es verdadera.
A los términos singulares de la oración, se les da una definición \emph{operacional}, donde por operacional, los positivistas se refieren a qué experiencias físicas serían necesarias para decir que una oración es verdadera.

Some has to do with it's sustainability.
Sólo para hacer la distinción entre el contexto de descubrimiento y el contexto de justificación.
Pero no hay que confundir a la operación con la cosa.
Hay que distinguir claramente entre la decisión de cómo representar un fenómeno y entre el fenómeno mismo.

Sobre la representación

pues introduciendo
Digamos que el operacion

Frank traza también un puente entre el pragmatismo, el materialismo histórico soviético y la filosofía del positivismo.


\section{citas}



Value considerations are not intended to trump considerations of logic and experience; they are intended to respect them. Don Howard p.10

The place of values in science is secured by the fact that, on Neurath’s view, logic and experience underdetermine theory choice. But turn that argument around and it implies that values come into play only within what I like to call the domain of underdetermination. That is, logic and experience are first allowed to do all of the work they can do. Only then do we ask which of several empirically equivalent theories is most conducive to the achievement of our social and political ends. Don Howard p. 10

The freedom of choice is not a freedom simply to deny the manifest evidence of the senses. One has to interpret, one has to tell a coherent story, and one has to tell a story that works. p. 14


This means there is no possibilityt of isolating a class of priviledged sentences, to act as a fixed foundation Naturalism and the Vienna Circle

Every term  introduced into the theory must be accompanied by  a description of the physical operations by which may  be tested the degree to which the property expressed  by this term may be attributed to a given physical  system. The description of these operations, the  "operational definition" of this term, may be more or  less direct; perhaps only a combination of terms will  correspond to a certain operation. Professor Bridgman's views have frequently been labelled "operationalism," although he himself is not pleased by this name.

But the outlook of the so-called "Vienna Circle"  has been only one particularly coherent doctrine  among the many intellectual fruits which have  emerged from the soil of Central-European positivism.

In particular, Charles  W . Morris of Chicago recognized the connection  with American pragmatism and publicized the idea  of cooperation between the two groups. It was decided, for the purpose of this cooperation, to call a  special congress, for which the name "Congress for  the Unity of Science" was coined by Otto Neurath.

The conception of the relative worthlessness of the theory in  comparison to the phenomenon gives to the theorizing  of such an investigator something especially free and  imaginative.

The known connections among  phenomena form a network; the theory seeks to pass  a continuous surface through the knots and threads  of the net. Naturally, the smaller the meshes, the  more closely is the surface fixed by the net. Hence,  as our experience progresses the surface is permitted  less and less play, without ever being unequivocally  determined by the net.

nce this possibility has been substantiated,  the whole of analysis can proceed to develop as usual.  But now when a theorem about derivatives is set up  and somebody begins to subtilize about it, asking  whether this theorem is really in agreement with the  "nature" of the differential and going into profound  and skeptical deliberations concerning this "nature,"  he can be told quite simply: "I could express this  theorem, if I took enough time, as a theorem about  integers; the nature of this theorem is hence no more  and no less mysterious than that of the natural  numbers."

The  atoms are auxiliary conceptions just like others which  can be employed advantageously in a limited domain.  They are not suitable for an epistemological foundation.

Más aún, Frank veía en la práctica científica una manera de resolver problemas como el de la representación y el de la elección, digamos \say{temporal}, de una teoría.
Esto queda bastante claro cuando Frank afirma

\begin{quote}
	If we look for an answer to the question of wether a certain theory, say the copernican system or the theory of relativity, is preferred or true, we have to ask the preliminary question: what purpose is the theory to serve? \parencite[p. 15]{Frank1954}
\end{quote}

Ambos filósofos estaban al tanto de que los factores externos afectan el contexto de justificación.
Ambos fueron miembros del círculo de Viena y si ambos sostenían que hay una distinción entre el \emph{contexto de descubrimiento} y \emph{el contexto de justificación}; entonces no son equivalentes el \emph{contexto de descubrimiento} y los \emph{factores externos}; y  tampoco son equivalentes el \emph{contexto de justificación} y \emph{los factores internos}.

Me parece que con señalar que los miembros del círculo de Viena que he discutido hasta ahora, no se alineaban con las tesis descritas por \parencite{Yturbe1995}, es suficiente para decir que hay una confusión en la premisa que usa la autora.
Sin embargo, quiero decir un poco más sobre las tesis a las que se adherían Neurath y Frank; además quiero señalar la relación que ve Frank entre la postura del \say{empirismo lógico} y la \say{filosofía de la Unión Soviética}, que es algo que quiero discutir en la siguiente sección.

\newthought{Tanto Neurath como Frank} sostienen una forma de verificacionismo: que el significado de de una oración depende del método de comprobación.
Pero, ambos están de acuerdo en que nunca tenemos una imagen perfecta, con la que podamos saber con completa seguridad, que una oración es verdadera o falsa.



%\section{Citas Amsterdamski}

%The sociology of science studies the evolution of science as a  social institution, the styles of scientific thinking, the reception of scientific  ideas and their social determination. Finally, logic and the methodology  of science (often called the philosophy of science), as a branch of epistemology, study the structure of scientific theories, their development, the  rules of concept and theory formation, the criteria of accepting and  refuting the claims of science, and the relations between theory and  empirical data.

%Anyone who has even a slight interest in the history of science must have  found himself faced by many bewildering problems. How was it possible,  for instance, that the most remarkable minds of their times held opinions  that could today be disproved by any high school student, often on the  basis of empirical facts known even then? It is just this last circumstance  which appears to be most striking, for it is one thing to accept theories  which later may be proved imprecise or simply false, but are not contradicted by known facts, and quite a different matter to maintain opinions regardless of known facts which clearly contradict them. If the first  instance seems quite natural to us, as we believe that it is the discovery of  new facts which usually compel us to revise old theories, then the second  instance appears to be perplexing; in fact, it seems to contradict the very  nature of scientific knowledge.

%the paintings of the Aztecs were more valuable than those of the European  Renaissance, the philosopher or historian of science sees nothing wrong  with comparing the theories of Ptolemy and Copernicus, Newton and  Einstein, or Darwin and Mendel. He is convinced that these theories had  to provide a coherent explanation of the same domains of phenomena (the  movement of celestial bodies, mechanics, the mechanism of heredity),  and that in attempting to formulate explanations they utilized the same  criteria for evaluating the results of their work.

%Although we are prone to evaluate works of art in a relativistic manner,  taking into account the aesthetic criteria and historical circumstances  which are particular to a given period, we tend to evaluate the achievements of scientists (just their achievements, not their merits) on the grounds  of some supra-historical criteria. More precisely, we tend to evaluate  scientific results on the grounds of those criteria which are accepted by  contemporary science and which we take (for better or for worse - we  will come back to this problem) as supra-historical.

%It was not only the Copernican theory that for years had to deal with a  mass of anomalies. The same thing happened with Newton's theory, with  quantum mechanics, and in general with any new theory which required a  radical reconstruction of large areas of acquired knowledge. This is the  fate of any new theory which attempts to introduce a new order into the  domain of phenomena which it describes and which were ordered in a  different way by its predecessors. Since every such theory must take into  account experiments and observations previously carried out, it must  therefore reinterpret them, make them fit into the new theoretical framework. Sometimes this reinterpretation is extremely complex and hardly  discernible to anyone but the specialist, especially in cases when the meaning of those basic concepts which became a part of everyday language  undergo a radical change. Thus after the rise of the theory of relativity, the  meaning of such notions as 'mass', 'velocity' or 'simultaneity' had  changed, and this fact should be taken in account when we investigate its  relation to Newtonian mechanics.

%such as, for instance, the contemporary state of mathematical knowledge,  the theoretical situation in other disciplines, the state of the technical  apparatus available or the accepted epistemological and ontological  beliefs, etc. To quote F. Jacob, there is" ... a domain which thought strives  to explore, where it seeks to establish order and attempts to construct a  world of abstract relationships in harmony not only with observations and  techniques but also with current practices, values and interpretations."15

%It is worth remembering that the empirical confirmation of the Copernican theory, which at the same time disproved the Ptolemaic system,  was only provided many years after the death of Copernicus. Copernicus  himself could offer no decisive fact which would confirm his own theory.  "No fundamental astronomical discovery, no new sort of astronomical  observation, persuaded Copernicus of ancient astronomy's inadequacy or  of the necessity for change."!7 The prognoses of celestial phenomena  provided by either theory proved to be essentially the same.

%The content of Newtonian mechanics not only stepped outside the  boundaries of empirical data and, therefore, could not be deduced from  them, but also was incompatible with observational data which were  known at the time when it was formulated - such as the impossibility of  distinguishing between absolute and relative motion.

%In any case, neither statements of scientists nor  reasonings of philosophers provide sufficient confirmation of the thesis,  that in posing experimental questions about nature, and reading the  answers by means of measuring instruments, scientists were free from  hypotheses and presuppositions and that in declaring the glory of radical  empiricism they were not subject to any philosophy or metaphysics. We  would indeed be quite naive if we took their words in this matter.


%Secondly, what is the true relationship between the empirical basis of  science (i.e., facts), and theory, and what changes of this basis can provide  a satisfactory foundation for the understanding of the process of the  evolution of scientific cognition?



%Los factores sociales y políticos son medios para sleccionar una teoría.
%De acuerdo a los objetivos que buscamos en la investigación.
%Y estos objetivos son influencias que imbuyen el contexto de justificación.
%
%Incluso Carnap, que es al que más le llueven vergazos y del que se sotiene tenía las ideas máws radicales deel círculo, llegó a aceptar que las teorías tienen sin duda un aspecto pragmático. 
%Neurath y Philipp Frank exponen este tipo de preocupaciones.


%Neurath fue falibilista, coherentista (sobre la justificación) y holista.
%Por supuesto que sostenía una forma de \emph{verificacionismo}, Bentley lo expresa más claramente \say{Theories must be sensitive to, grounded in and testible by experience} \parencite[p. 38][]{Bentley2023}.
%Este criterio es bastante vago, pero Bentley lo describe no como una tesis, sino como una postura; una postura que es difícil capturar con una sola afirmación \parencite[p.38][Cfr.]{Bentley2023} 


%\subsection{Citas Phillip Frank}

%The misinterpretation of scientific principles, as will be shown, can be avoided if, in every statement found in books on physics or chemistry, one is careful to distinguish an experimentally testable assertion  about observable facts from a proposal to represent the facts in a certain way by word or diagram.
%If this distinction is sharply drawn, there will no longer be any room for the interpretation of physics in favor of a spiritualistic or a materialistic metaphysics. [p. 4-5]

%The views represented in the present book are closely associated with the movement now generally  called "logical empiricism" or "logical positivism."
%I must confess that I do not like these words either.
%But a long life among views and theories has shown me that if we want a view to be regarded as a respectable tree in the garden of opinions it must have a label just as much as the elms and oaks in our public gardens. [p. 5]

%* recordar que Philipp Frank fue fundador del grupo original, que 20 años después se convirtió en el círculo de Vienna.

%* Cualquier reconstrucción se le va a quedar corta a este libro.

%But the outlook of the so-called "Vienna Circle"  has been only one particularly coherent doctrine  among the many intellectual fruits which have  emerged from the soil of Central-European positivism. Mathematicians such as R. von Mises, Κ. Menger, and Κ. Godei, physicists such as E. Schroedinger,  economists such as J. Schumpeter, lawyers such as  H. Kelsen, and sociologists such as E. Zilsel had their  roots in this environment. The whole intellectual  background of this general movement can be understood best through R. von Mises' textbook of positivism. [p. 9]

%Several young American philosophers  traveled to Vienna and Prague in order to come into  scientific contact with Schlick and Carnap. Among  them were W . V . Quine (now at Harvard) and  E. Nagel (now at Columbia). In particular, Charles  W . Morris of Chicago recognized the connection  with American pragmatism and publicized the idea  of cooperation between the two groups. []

%But what does it mean when we say that a question  is insoluble? Let us suppose, for example, that someone has asserted that the problem of a regular airplane  route to the planet Neptune is insoluble, or that the  production of a living organism from lifeless matter  is insoluble. 
%Despite this assertion, the person making  it can describe quite accurately the concrete experience we should have if the problem were solved.

\section{¿Qué podemos aprender de la historia de la ciencia?}\label{sub:aprender}


\noindent Quiero comenzar esta sección confesando lo siguiente: estoy de acuerdo con la autora, en particular con el carácter de la conclusión anterior, esto es, que es complicado--si no es que imposible--, separar entre los procesos externos y los procesos internos que influyen en la investigación científica.
Pero aceptar esto --eso quiero argumentar-- no implica que debamos deshacernos de la distinción entre el \emph{contexto de investigación} y el \emph{contexto de descubrimiento}.
En lo que resta de la sección quiero hacer un breve repaso de los argumentos de la autora.

Yturbe nos dice que el contexto de descubrimiento, se dedica a analizar factores externos que fueron influyendo en los factores internos; mientras que el \emph{contexto de justificación} sólo se dedica a analizar los factores internos al desarrollo teórico.
Lo que ella llama \say{the doctrine of the two contexts}\footnote{
	\say{one of the philosophical theses in favor of the contraposition between the internalist approach and the externalist approach is the so-called doctrine of the two contexts, developed by positivism.} \cite[p. 75]{Yturbe1995}
}
La autora nos dice que la doctrina de los dos contextos hace una distinción que no se puede hacer.
El contexto de justificación es siempre influenciado por el contexto de descubrimiento porque la ideología, la economía, las relaciones de poder, etc. influyen en la toma de decisiones\footnote{Por usar una caricatura: a quién se le asigna presupuesto}, por lo tanto la doctrina es incorrecta.

Como dije al principio de esta sección, estoy de acuerdo con que es casi imposible hacer la distinción entre \emph{historia interna} e \emph{historia externa}, pero aceptar esto, no implica deshacernos de la distinsión \emph{contexto de descubrimiento} y \emph{contexto de justificación}.
Para sustentar esto, dependo de la caracterización que hace la autora sobre \say{la doctrina de los dos contextos}, la caracterización de la doctrina está ligeramente sesgada, si no es que completamente inadecuada.
En la siguiente sección quiero ofrecer mis razones para esta conclusión.

\subsection{La doctrina}

\noindent En el texto de \textcite[p.75]{Yturbe1995}, la autora ofrece una descripción de la doctrina, ella señala que los filósofos doctrinarios afirman que \say{according to this position, [factores externos], not only fails to increase our understanding of scientific development, but even obscures the fundamental question of the rational validation of scientific arguments.}
Además que \say{This conception constitutes the dominant framework in which the philosophy of science has been developed, and consists in drawing a radlcal distinction between the context of discovery and the context of validation.}


Su argumento para decir que hay un colapso entre factores externos (contexto de descubrimiento) y factores internos (contexto de justificación) se encuentra en la página 85

\begin{quote}

	External factors are not only found in the context of discovery, they are present also in the development of the concepts, problems, methods, problem fields, etc. of scientific discourses: that is, external factors are present in the context of validation itself.
	There are scientific discourses in which ideological conceptions pass on to form part of the body of the science itself, functioning as principles which define its field of study or guide its research; thus, external factors can become internal.

\end{quote}

La doctrina de los dos contextos implica que podemos claramente separar entre el \emph{contexto de descubrimiento} y el \emph{contexto de justificación}.
Pero los factores externos están presentes en el contexto de justificación, tanto así que pueden convertirse en factores internos.
Esto implica que los contextos no son claramente separables, y que, por tanto, la doctrina es incorrecta.

Contrario a lo que dice la autora

\begin{quote}
	The search for new explanatory programs is characterized above all by the attempt to reconcile the internalist and externalist approaches. But, in view of the fact that both approaches are committed to theses concerning the nature of science that are not only incompatible, but are unsustainable, their union without important changes in their philosophical presuppositions cannot result in a viable program. \parencite[p. 79]{Yturbe1995}
\end{quote}

Creo que es verdad que no podemos distinguir entre factores externos y factores internos en la ciencia.
Sin embargo, me parece que lo que expresa la cita anterior es claramente erróneo: no se siguie la conclusión \say{que no pueden resultar en un programa viable}, porque depende de señalar que la distinción entre  \emph{contexto de descubrimiento} $|$ \emph{contexto de justificación} es equivalente a la distinción \emph{factores externos} $|$ \emph{factores internos} y estas distinciones no son equivalentes.



\subsection{En defensa de la doctrina}

\noindent En años recientes ha habido un creciente interés en discutir las publicaciones del Círculo de Vienna \parencite{Bentley2023, Richardson2023, Suarez2024, Riel2014}.
Si hago una tosca generalización, diría que literatura reciente se ha dedicado a señalar que los miembros del Círculo de Vienna, no eran tan herméticos como se pensaba.
Además, la cantidad de temas que discutieron es más vasta de lo que se creía.
Particularmente pensando en que cada uno de los miembros difería de los otros en tesis centrales.

\textcite{Suarez2024}, por ejemplo, discute que una preocupación genuina sobre los modelos en la ciencia, precede, data y prosigue a los años del Círculo de Vienna, el autor nos dice \say{I focus particularly on the nineteenth-century modelers and summarize their insights and contributions, which, I claim, remain essentially unsurpassed.} (p. 20)

Además, los miembros del Círculo de Vienna tuvieron una preocupación genuina por las prácticas científicas--Neurath, por ejemplo--
Tenían presente que ciertos factores externos pueden influenciar a la investigación científica.
\textcite[p. 24]{Bentley2023} lo expresa mejor: \say{Frank's historical work, which frequently anticipates Kuhnian or post-Kuhnian themes, consistently emphasizes the significance of non-scientific, external factors on the decision making of scientists.}
Los miembros del Círculo de Vienna tenían claro que los factores externos influyen en el proceso de la investigación científica, presentes incluso en el \emph{contexto de justificación}.

Hablando en particular del trabajo de Neurath y las tesis que sostenía, Otto defendía una teoría coherentista de la justificación.
Neurath argumentaba que nuestras teorías científicas están subdeterminadas empíricamente.
En cualquier momento del tiempo hay hipótesis en pugna.
Pero si las teorías están empíricamente subdeterminadas, es posible que dos teorías internamente coherentes y externamente inconsistentes entre sí, convivan al mismo tiempo.

Pero si el escenario anterior es plausible, entonces es patente que aquél que defienda una teoría coherentista de la confirmación, no puede resolver el problema de la elección racional de teorías.
Debido a que todo el tiempo hay teorías en pugna, como filósofos, deberíamos poder explicar por qué es racional elegir entre una teoría y sus rivales. Para poder resolver este problema, Neurath defendía que los factores externos son información clave para explicar la racionalidad de la elección de los investigadores.
Factores externos como los eventos históricos fortuitos.

\textcite{Bentley2023} nos recuerda que Neurath fue falibilista sobre el conocimiento, es decir,  sostenía que cada una de nuestras creencias puede estar equivocada.
Sumado a esto, sostenía que la confirmación es holista: no son oraciones particulares, sino grupos de oraciones, los que contrastamos con el mundo.
Bentley lo expresa mejor \say{$\ldots$ it is always possible for a system of beliefs to be altered to accommodate a statement or for the statement to be rejected $\ldots$} (p. 21).



Esto último inició como un abstract que mandé a un taller y se fue volviendo más largo mientras leía.

Hablando de modelos en la filosofía de la ciencia durante el periodo del Círculo de Vienna, una figura importante que mencionar es Mary Hesse.
Mary Hesse trabajó con especial atención el uso de modelos en la ciencia.

``This historical chapter introduces the emergence in the nineteenth century of  what I call the modeling attitude.
This is a stance toward scientific work and  discovery, and it continues to this day." (p. 43)






  % Chapter 0

%!TEX root = ../main.tex

\chapter{What we can learn about abstraction from mathematics?}
%\label{ch:wwlmath}

\section{Introduction}

Scientist use models in their research, in consequence, philosophers of science became interested\footnote{There has been interest in modeling strategies since the years of the Vienna Circle.} in the modeling strategies used in science.
Between the epistemological aspects of models, we have the debate on how models relate with their target, or how they "represent" their target, as \citeauthor{Frigg2020} say "If we want to understand how models allow us to learn about the world, we have to come to understand how they represent."\footcite[][p. x]{Frigg2020}

In this chapter, I want to address this question.
%This contribution is relevant for the workshop because I will discuss how the relation between model and world works in biology.
I will use two main examples: the "Hodgkin-Huxley model", as developed by themselves in a series of articles \parencite{Hodgkin1951, Hodgkin1952, Hodgkin1952a, Hodgkin1952b}, but I will focus on the paper \citetitle{Hodgkin1951} by Hodgkin.
Also I will contrast this paper with the work made by Piccolino and Bresadola\footcite{Piccolino2013}.

Also, I want to emphasise why having the relation between idealizations\/world wrong, leads researchers into trouble.
This will be argued having as example the reconstruction of the debate on IQ tests\footcite{Lewontin2017} by Lewontin, Rose and Kamin.

With these examples at hand, I want to argue against the artifactual theory of idealizations\footcite{Carrillo2021-CARAAP-12, Carrillo2022}.
Artifactualist argue that their theory is more adequate than others, and as such should be accepted.
The core motivation that the artifactualist has in favour of her theory is that it doesn't depend on the relation between idealizations and world.
The artifactualists argue that this point of departure has many advantages.

I will argue that actual research shows that the motivations behind the artifactualist theory are wrong, and that the reconstruction they make about the Hodgkin-Huxley model is misguided.
In this chapter I will try to deal with the need (or not) to clarify the relation between the models and the world.

\section{Main motivations}

It seems intuitive that if researchers care about natural phenomena, then they want their models to be somewhat accurate.
But this last point has been widely discussed.
Some philosophers have argued that this relation isn't necessary to give a good theory of models in science.
Models need to leave some properties of their targets out, if we want the model to be of any use. %13/06/24 Agregar referencias. Esto está en Potochnik, pero también se puede encontrar en Winther. Leer a Winther
So, they misrepresent their intended target.
And sometimes researchers misrepresents on purpose with the use of idealizations. %13/06/24 Agregar un par de referencias. Esto  lo dicen explícitamente la Natalia y la Tarja. Poner eso, también el libro de Suárez.
We have a problem here: Assuming that truth is valuable in research, if models misrepresents their target, it seems that they cannot be part of theories.
Because in some sense, we say that idealizations are "false", or that they "misrepresent" their target.

Philosophers have tried to account for this problem\sidenote{This is part of a larger problem about the value of Knowledge. Philosophers like \citeauthor{pritchard2021a} have argued that truth is indeed what gives value to our knowledge \citeyear{pritchard2021a, pritchardEpistemicValueCognitive2021}. While Philosophers like \citeauthor{elgin2017true}, have argued that truth is not what gives value to our knowledge \citeyear{elgin2017true, elgin2004}. This latter group of philosophers appeal to another kind of epistemic values like Understanding. I will talk of this topic in the next chapter. In this chapter I will deal with idealizations.}
If we got that truth is an important part of our theories, it seems that idealizations should be banned.
But idealizations are widely used in scientific research, so it seems that truth is not an important part of our theories.

We have a dilemma that we need to account: either truth isn't important or idealizations should be banned.
I personally think that this is a false dilemma, and that actual patterns of research show that it is.
So, we need to address why the use of models and a care for truth can be true at the same time.

Now, talking about models\/idealizations, there are different accounts on how to close this gap.
To close this gap, some philosophers have argued that idealizations serve as means.
The world is too complex for our human cognitive limitations, so we use idealizations to have a more manageable model.
This model use idealizations as a mean. As Potochnik says "to support human cognitive and practical ends." \footcite[][p. ix]{Potochnik2017-POTIAT-3}.

Within this stance, Potochnik says, we can idealize certain phenomena highlighting the causal structure of the phenomena at hand.
Idealizations are means to highlight the causal pattern we're interested in.
Let us call this the "causal relevance" theory.
It has to be clear that Potochnik affirms that there is no need to de-idealize the model: it has a lot of epistemic advantages to leave the model as it is, including the idealizations.

Other stance affirms that the idealized model is just temporal.
This group of philosophers argue that idealizations in the model function as aids: it has an epistemic advantage to distort the target phenomena momentarily, but new research will show how we must modify the idealization to make it more accurate.
Let us call this the "de-idealization" theory, or what \citeauthor{Weisberg2007} calls "galilean idealization" \footcite{Weisberg2007}.

Still another group of philosophers have argued that the previous stances are both wrong.
This philosophers argue that these stances are wrong because the "causal relevance" and the "de-idealization" stances, both make substantial assumptions on the relation between model and world.
This means that the causal relevance stance and the de-idealization stance\footnote{They use different labels for the "stances" that I have been characterizing here. What I call "causal relevance", they call it "epistemic benefits"; what I call "de-idealization", they call "distortion of reality".}, take too much care on the model and how accurate is representing the target.
They suggest that

\begin{quote}
	$\ldots$ in these and many other cases, idealization is better understood from an artifactual perspective that does not take the representational model-world relationship as a point of departure, presupposing the possibility of some straightforward comparisons between models and their supposed target systems \parencite[][p. 2]{Carrillo2021-CARAAP-12}.
\end{quote}

I have been using the word "stances" to describe the previous theories.
I have a reason for using this word.
I call them stances, because we are philosophers of science, trying to explain (i) why there are idealizations in scientific research, and (ii) which are the uses of those idealizations\footnote{I'm following Potochnik when she says that no actual philosopher of science will doubt that history of science and actual research activity are important to answer philosophical questions \citeyear[cfr.][p. 9]{Potochnik2017-POTIAT-3}. Also, the artifactualist seems to agree with this.}.
We take a stance that can help us to interpret the process of "making" idealizations in order to answer "(i)" and "(ii)".
Three stances are under debate here: "de-idealization" [DI], "causal relevance" [CR] and "artifactualist" [AR].

I will argue that the artifactualist doesn't have good answers to "(i)" and "(ii)".
I say that AR doesn't show how the DI and CR exclude what the artifactual stance achieves.
Second, I want to say that the motivation behind the artifactual stance is incorrect, because there is evidence showing that researchers do care about the relations between model and world.
Also, there is evidence about the dangers of getting this relation wrong.

I will also argue that the artifactualist is not really an alternative.
Much of the advantages that the artifactualist claim are advantages that we are already dealing in philosophy, almost since the time of the Vienna Circle.


\section{What the artifactualist claim}

As I said above, the artifactualist affirm that we do not need to have an accuracy measure between idealizations and the world to have a good theory of idealizations.
This change of stance, they argue, has a lot of benefits that the other stances lack of.
First they claim that "$\ldots$ the artifactual approach to modeling is to provide an alternative to the traditional accounts of models that assume that models give knowledge in virtue of accurately representing their target systems or their parts"\footcite[][pp. 51-52]{Carrillo2022}.

Here we are dealing with what I called the CR and DI stances.
Remember that it seems plausible that a good stance on idealizations can answer questions "(i)" and "(ii)"
So let's take a stance that can help us to understand the process of "making" idealizations in order to answer both questions.


\subsection{A brief characterization}

In this section I want to make a brief characterization of the three stances under discussion.
I want the order of the exposition to be 1) artifactualist, 2) deidealization and 3) causal relevance.
Let me begin with the artifacualist stance.

\paragraph{The artifactualist}

Within AR there are, what I think, three central thesis.
Two of them are encapsulated in the next quote "Accordingly, idealization tends to be holistic in that it is not often easily attributable to just some specific parts of the model" \footcite[][p. 3]{Carrillo2021-CARAAP-12}.

The last quote affirms two things, first that "idealizations tends to be holistic", second that "idealization is not often attributable to just some specific parts of the model".
Both affirmations are related to the "holistic" nature of idealizations that the artifactualist commits to.
Let me address the "holistic" nature of idealizations.

There is no single definition of what "holism" means.
The term is commonly attributed to Quine, affirming that our knowledge, and the meaning of the propositions that we hold, are contrasted with the world not one by one, but as a whole\footnote{"My countersuggestion, issuing essentially from  Carnap's doctrine of the physical world in the Aufbau, is that our  statements about the external world face the tribunal of sense experience not individually but only as a corporate body." \parencite[][p. 38]{Quine1951} }.
And this seems to be the kind of holism that the artifactualist claims.

In the paper \citetitle{Carrillo2021-CARAAP-12}, "holism" seems to be the connection that the idealization has with other idealizations already in the scientific community.
They say that "$\ldots$ the justification eventually boils down to coherence with earlier theoretical and methodological commitments as well as empirical results" \parencite[][p. 52]{Carrillo2022}

Also, this is related with their claim that idealization is attributable to specific parts of the model, they claim that "$\ldots$ many idealizations appear holistic, and not separable into assumptions whose distorting nature would be self-evident" \parencite[][p. 57]{Carrillo2022}.

The third thesis is about the methodology that the artifactualist uses.
They say that the artifactualist stance can answer questions "(i)" and "(ii)" by claiming that their theory doesn't builds up from the relation between the model and the world.

\begin{quote}
	While idealization has traditionally been understood as deliberate misrepresentation of a feature of the target system, the artifactual  approach is not hung up on the accuracy/distortion of a model or its parts. \parencite[][p. 8]{Carrillo2021-CARAAP-12}
\end{quote}

And in a latter paper, they defend their theory based on this idea.

\begin{quote}
	In this paper, we develop an artifactual view of holistic idealizations that does not start from the representational assumptions inherent in idealization-as-distortion accounts, but rather focuses on the processes trough which models are achieved, used, and further developed. \parencite[][p. 50]{Carrillo2022}
\end{quote}

This detachment of the world, they say, has certain benefits.
It seems to clarify how models are constructed, with certain limitations of what the researches can hypothesize based on the tools available.
As they urge "$\ldots$ what  mathematical methods can be used, what analogies are available, what can be measured)." \parencite[][p. 9]{Carrillo2021-CARAAP-12}

So they answer the questions by arguing that the model construction in actual researchers doesn't want to accurately represent the world, instead they want to use the tools\footnote{Tools in the artifactualist sense, \emph{i, e.} mathematical methods, analogies, etc.} available to render new models/idealizations/metaphors.
And this answers our two questions.

\paragraph{The de-idealizer}

The "de-idealized" stance affirms that the idealization is just a momentary aid for simplicity, but with scientific progress, we will eventually get rid of the idealization.
There is a characterization of this stance in \parencite{Weisberg2007}, Weisberg characterizes such stance as a "$\ldots$ idealization justified pragmatically", p. 641.
This means that this idealization helps to simplify certain phenomena.
And because the problem is one of computational tractability, new computational power will bring the idealization more accurate, with the goal of getting rid of it.

The de-idealizer says, then, that idealizations have the purpose of simplify computational tractability.
And, because the idealization is motivated by their use, better computational power will render the idealization obsolete.

It seems that the de-idealizing stance makes two affirmations.
First, that the model is justified pragmatically, in order to make the phenomena computationally tractable, answering question "(i)".
The de-idealizer also says that the model is used just temporarily.
Due to the computational complexity, the development of new computational tools, will give us a more accurate idealization.
Looking at the aim of getting rid of the idealization in the future, answering the question "(ii)".

\paragraph{The causal relevantist}

I think that the most detailed defense that we have of the causal relevance stance was made by \cite{Potochnik2017-POTIAT-3}.
The causal relevantist is motivated by "$\ldots$ that science is ultimately a human creation and, as such, responsive to particular human concerns" \parencite[][p. 11]{Potochnik2017-POTIAT-3}.
This has a commitment with another whole set of thesis tat the causal relevantist accepts.
But I think we can talk about how the causal relevantist deals with idealizations, without going into much detail about the other set of thesis.

The causal relevance stance, begins with the motivation that scientist idealize, because of the complexity of the world and the limited cognitive capabilities of humans.
Also, Potochnik says that reasons to idealize have multiple sources, and they not just reflect features of the world, but also researcher's interests.

This means that researchers idealize because they need to simplify causal complexities.
The different variables that can affect the phenomena are excluded.
Researchers just need to account for the relevant causal factor they are interested in.
And because researchers are looking for information according to a hypothesis, they know which causal factor is relevant.

But in order to know which causal factor is relevant, they need to represent the phenomena somewhat accurately, as Potochnik says "I intend causal patterns to be regularities in phenomena themselves." \parencite[][p. 25]{Potochnik2017-POTIAT-3}.
The search for causes in science depends on our interests as researchers.
But we still need to account for actual phenomena.

\begin{quote}
	But successful idealizations do bear certain similarities to the systems they help to represent.
	Idealizations represent phenomena as if they had features they don't, but those misrepresentations are useful insofar as there are functional similarities--similarities in causal role or behaviour-- $\ldots$ \parencite[][p. 53]{Potochnik2017-POTIAT-3}.
\end{quote}

So the causal relevantist answer question "(i)" arguing that scientists make idealizations because they want to isolate relevant causal factors they are particularly interested in.
This also answers why scientists use idealizations, and that gives an answer to question "(ii)".

We need to notice that the causal relevantist also affirms that researchers doesn't need to de-idealize, because the causal factor already isolated is of use still when we have more computational power.
And if we already have more computational power, researchers may want to account for more complexity: adding new variables, new causal factors, new types of idealizations, etc.
This in contrast with what the de-idealizer affirms.

Also, the causal relevantist claims that the idealization does bear some similarities with the phenomena represented.
Because causal patterns are regularities of real phenomena.
This in contrast with the artifacualist.





% chapter chapter_1 (end)


% !TEX root = ../main.tex

\chapter{La naturaleza de la \emph{verdad}}
\label{ch:natruth}
 



\section{Naturaleza}

La naturaleza de la verdad es un tema que sigue siendo motivo de debate entre filósofos.
Hay al menos tres posturas clásicas sobre la verdad: la pragmática, correspondentista y coherentista.
Por supuesto hay más detalles y más teorías sobre la naturaleza de la verdad, pero, por motivos expositivos, concentrémonos en las teorías clásicas.
Además, quiero hacer claros un par de supuestos.
En primer lugar, el mundo y los fenómenos que queremos estudiar son complejos.
La segunda y última suposición, los seres humanos tenemos capacidades cognitivas limitadas, muchas de las veces en las que queremos estudiar un fenómeno, limitamos las diferentes variables que pueden afectar el fenómeno que queremos estudiar 
%<--- Poner ejemplos y citar el trabajo de Cartwright y Potochnik ---!>

"If the world is highly complex relative to our cognitive capacities and we nevertheless seek to know it in its full complexity, this requires stretching our cognitive endowments, devising multiple means for reaching its less accessible regions, improvising, experimenting, tinkering, exercising our imagination, etc" Gila sher

It requires a balance between unity and diversity, between observing and proposing, between describing and constructing, between being critical and understanding. I will call a theory of truth that requires a substantial correspondence (of one kind or another) between true cognition and reality, allows multiple—including intricate—routes of correspondence from language to reality, yet seeks maximal unity and systematicity, a “composite correspondence” theory.

\subsection{Teoría pragmática de la verdad}


\section{Veritismo: limitaciones y alcances}

\subsection{Introducción}

En el capítulo anterior, prometí señalar cómo nuestro compromiso, el veritismo, es capaz de lidiar con los problemas de idealización y modelos en la ciencia.
El problema, a grandes rasgos, consiste en que en la investigación científica  se usan idealizaciones, abstracciones y modelos.
Los métodos anteriores difieren del fenómeno que pretenden representar.
Por decirlo de manera sucinta: idealizaciones, modelos y abstracciones son literalmente falsos (se desvían de la realidad).


Muchas de las teorías científica hacen uso de estos métodos.
El equilibrio de Hardy-Weinberg asume que una población de organismos es infinitamente grande y que la frequencia alélica permanece constante.

Para la caída libre, Galileo generalizó sus experimentos, asumiendo que los planos no tienen fricción (usó esferas y planos lo más pulidos posibles, pero 
esto no significa que haya ausencia de fricción).

%<!--- Más ejemplos aquí --->

Estos métodos, ampliamente usados, son desviaciones del fenómeno y muchas filósofas han dicho que esto es un problema para las aproximaciones veritistas en filosofía de la ciencia \parencite{elgin2004, Potochnik2017-POTIAT-3, bokulich2016}. 
%<!--- Agregar las citas --->
Un veritista como Strevens y compañía, dirán que el uso de estos métodos es temporal.
Esto es, que mientras más avance la investigación , vamos a deshacernos de tales idealizaciones, modelos y abstracciones.
Sin embargo, una amplia literatura, por ejemplo, las autoras anteriores, señalan que algunos de los casos son imprescindibles y que, de hecho, son explicativos porque son falsos.
La conclusión que hay que extraer de está literatura es que debemos relajar nuestros compromisos veritistas.

En el capítulo anterior, discutí por qué la verdad es un valor necesario en la investigación. 
Hay un par de teorías epistemológicas acerca de cómo se entrelazan la teoría de virtudes y la investigación. 
Entre estas, una desarrollada por Haomiao Yu (\citeyear{yu2021}).
Quisiera hacer una comparación entre lo que señalé en el capítulo anterior y las teorías mencionadas.

Haomiao Yu desarrolla una teoría parecida a lo que presenté en el capítulo anterior. Uno de los puntos cruciales de su teoría es que trata de hacer explícito qué es el "entendimiento". Su trabajo trata de hacer compatible una teoría epistémica con las teorías de la explicación científica. Yu detecta un problema en la literatura sobre explicación científica: muchos asumen que su modelo de explicación se conecta con cómo comprendemos [entendemos] el mundo. Sin embargo, sólo asumen tal conexión sin dar razones a favor de su afirmación.

Pera solventar esta brecha, Yu apuesta a favor de una teoría de las virtudes. Para esto es pertinente indtriducir un par de disttinviones. La primera distinción que hace Yu es entre teorías responsabilistas de las virtudes y teorías fiabilistas de las virtudes. Esto es algo que no señalé explícitamente en el capítulo anterior y vale la pena aclararlo aquí: las teorías fiabilistas de las virtudes se comprometen con que las virtudes de los agentes son constitutivas del conoicimiento. Por su parte, las teorías responsabilistas no se comprometen con esto. Lo único que señalan las teorías responsabilistas es que las virtudes son rasgos que tiene un buen conocedor. Yu nos dice que en su teoría el fiabilismo de virtudes es crucial; el responsabilismo de virtudes es sólo auxiliar.

Lsa virtudes espisémicas identifican diferentes habilidades de los agentes involucrados en la investigación. Yu señala que muchas de las capacidades cognitivas juegan un papel que cierra la brecha entre explicación y entendimiento. El autor señala que una de las teorías del entendimiento basadas en habilidades de los agentes es la de Khalifa. Sin embargo, Yu argumenta que la teoría de Khalifa sólo nos permite hacer distinciones de grado, mientras que hay claros casos de diferentes tipos de entendimiento. Hablidades comunes que señala Yu son: razonamiento deductivo/inductivo, razonamiento causal/mecánico, razonamiento contrafáctico, generalización/categorización y abtrsacción. 

Para aclarar su teoría, Yu nos presenta un caso de estudio: Galileo y su prueba de que el péndulo es isocrónico. Primero que nada, la recosntrucción del caso describe cómo Galileo hace uso de diferentes habilidades. Para su investigación, Galileo hizo uso de sus habilidades matemáticas. Sin embargo, Yu señala que el entendimiento de Galileo sólo es correcto hasta cierto grado. Esto se debe a que Galileo usó las leyes de Kepler para derivar su modelo, entonces sólo tenía disponible un mapeo de la estructura matemática del péndulo. Para un entendimiento completo del fenómeno, es necesario usar las leyes de Newton.

La primer diferencia entre lo que presenta Yu y lo que señalé en el capítulo anterior es que estoy en desacuerdo con su conclusión. Si bien estoy de acuerdo con su teoría de las habilidades/virtudes de los agentes, me parece que imputarle a Galileo que su entendimiento de un fenómeno es de un grado menor que quien tenga a su disposición las leyes de Newton. Esto puede parecer ser muy obvio, pero no estoy de acuerdo. No lo estoy porque es un caso de anacronismo.

En primer lugar Yu señala que lo que constituye el entendimiento es el uso de las habilidades de los agentes. Si esto es verdad, no es claro cómo una situación externa al agente (el hecho de que vivió antes del desarrollo de la física newtoniana), tiene cualqueir cosa que ver con el entendimiento que tuvo Galileo. Más aún, recordemos que, si un sistema es axiomático, una vez que aceptamos las premisas estamos condenados a aceptar la conclusión (suponiendo que el argumento es válido). Siempre podemos renegar de los axiomas: desecharlos por alguna razón, lo que implica que el argumento deja de ser sólido. Dado esto que señalo, si Galileo derivó la fórmula del péndulo, entonces no es claro cómo podía estar equivocado (suponiendo que las premisas son verdaderas). Más aún, debido al marco presentado en la sección anterior, no podemos imputarle errores a Galileo debido a que no tenía información disponible acerca de las leyes de Newton. Hasta el punto de su conocimiento, podemos afirmar que tenía conocimiento certero ``real knowledge''.

El caso del péndulo en Galileo es más intrincado. Para exponer este caso, usaré la reconstrucción que hace Ariotti en \citeyear{Ariotti1968}. Galileo usa un par de experimentos para motivar la intuición de que el periodo del péndulo es proporcional a la raíz cuadrada de la longitud de la cuerda. Galileo afirma que el péndulo es \emph{isocrónico}, esto quiere decir que el tiempo que tarda el péndulo en llegar al punto más bajo del círculo descrito es independiente al ángulo. Sin embargo, las pruebas de Galileo no son suficientes para mostrar la isocronía del péndulo, sólo su \emph{sincronía}. Es decir que el tiempo que tardan dos cuerpos de diferentes pesos en llegar al punto más bajo del círculo descrito es el mismo. El experimento es sencillo: tomemos dos cuerpos con diferentes pesos atados a cuerdas con la misma longitud. Si dejamos caer ambos cuerpos, llegarán al mismo tiempo al punto más bajo y su periodo será el mismo.

Lo anterior sólo muestra la sincronía del péndulo, no la isocronía. Para apoyar la hipótesis de que el péndulo es isocrónico, Galileo recurre a un modelo rígido del péndulo. En lugar de usar cuerdas (que pueden doblarse por no ser rígidas), Galileo utiliza una construcción de madera lo suficientemente pulida para que la fricción no afecte el tiempo de traslado. Para medir los tiempos en este experimento, Galileo usó agua. Dejaba caer agua en desde un jarrón con un orificio hacia un bote. Luego pesaba el agua para saber si la cantidad era la misma, por tanto los tiempos iguales en diferentes ángulos. Debido a la poca discrepancia entre los pesos del agua recolectada, Galileo concluye que el péndulo es isocrónico.

Pero este modelo rígido no se comporta igual que un péndulo de cuerda. En la cuerda hay discrepancias debido a que la cuerda hace una curva cuando dejamos caer un cuerpo atado a ella. Por tanto hay discrepancias entre los tiempos. El modelo rígido no es completamente exacto con el fenómeno a estudiar. Más aún, los resultados de Huygens muestran que para que un péndulo sea isocrónico, la curva no debe ser un círculo, sino un cicloide \cite{Ramond2023} pero las investigaciones de Galileo fueron un paso hacia el teorema de Huygens de que el péndulo es isocrónico. 

Cabe notar que en esta descripción del caso del péndulo en Galileo, es claro el uso de las habilidades que Yu señala. Habilidades para modelar, habilidades matemáticas y habilidades de deducción. También hay que destacar que son estas habilidades lo que lleva a Galileo a afirmar que el péndulo es sincrónico e isocrónico (casi isocrónico: para distancias de cuerda mayores a la distancia de donde liberamos la masa). 

Sin embargo, toda esta historia nos muestra más que un fallo, un éxito. Vaya, la incvestigación en general depende no sólo de un experimento, sino de diferentes personas trabajando en un mismo tema. Galileo fue un paso para el teorema de Huygens. No podía afirmar la isocronía del péndulo debido a que el experimento no es exacto: el modelo rígido es diferente al modelo con cuerdas. El método de medición por supuesto no es perfecto. Además el círculo no es la curva para que un péndulo sea isocrónico, la curva es un ciloide, como muestra Huygens. Cabe notar además que la solución de Huygens vino antes de la teoría de Newton, por tanto, tampoco son necesarias las leyes de Newton para mostrar la isocronía del péndulo. Los fallos de Galileo son fallos de los materiales a su disposición. Por supuesto, en estos materiales también incluyo la teoría newtoniana.

Pero no se sigue que cualquiera que tenga a su disposición mejores materiales, incluyendo la teoría de Newton, podría haber entendido completamente el fenómeno. Las virtudes y habilidades de Galileo deberían ser lo único relevante para decir si hizo uso de sus habilidades (aunque suene trivial), que es lo que señala Yu. Aunque siendo sinceros, no sé si Yu esté dispuesto a aceptar este argumento, quizás haya algo más en la teoría que diseña, de manera que, la comprensión completa del fenómeno sea un factor relevante.

%No son del todo precisas, claro qwue tenía entendimiento. Dos preguntas Juan. 

\section{Modelos}

Lo dicho anteriormente, por supuesto, se relaciona con el problema del realismo científico. De manera muy sucinta y poco precisa, los realistas científicos defienden la tesis de que las teorías científicas son literalmente verdaderas. Esto implica que las entidades que aparecen en la teoría de hecho existen y que las relaciones entre objetos que señala la teoría reflejan cómo de hecho es el mundo.

Esta tesis tiene muchas aristas. En primer lugar hay un aspecto epistémico involucrado. Si saber implica verdad, entonces el hecho de que alguien señale que Newton sabía que el espacio absoluto existe, el espacio absoluto de hecho existe. Otro aspecto del realismo científico es semántico. Saber si efectivamente los términos individuales que son parte de las oraciones de una teoría, de hecho refieren a un objeto. También hay un aspecto axiológico involucrado: que el objetivo de la ciencia es la verdad\footnote{Estas aristas corresponden más o menos a cómo Khalifa describe la tesis del realismo científico. Véase (\citeyear{khalifa2010})}.

Si bien todos estos problemas están relacionados (por ejemplo determinar que el espacio absoluto de hecho existe, haría que las oraciones donde aparece dicho término individual refiriera y que alguien que sepa la teoría sabe que el tiempo absoluto existe), vale la pena tener estos aspectos separados. 

Lo que he señalado hasta ahora está relacionado sólo con el problema epistémico. Nuestro problema es que no es claro que la teoría de Newton sea verdadera, ya que nuevas teorías han mostrado ser mejores descripciones del mundo que la teoría newtoniana, por tanto, Newton no sabía que el espacio absoluto existiese. 
La justificación de Newton no es suficiente. 
Esto es sólo una manera de exponer algo que Laudan ya había señalado: la historia de la ciencia ha mostrado que los términos individuales de teorías exitosas no siempre refieren, por lo que dichas teorías son falsas \cite{laudan1981}. 
Creo que la moraleja que nos da Laudan es que debemos ser cautelosos al formular una tesis realista sobre la ciencia\footnote{Más aún, el punto de Laudan es más débil que señalar que el realismo científico es falso. El punto de Laudan es señalar que no hay una relación tan fuerte entre verdad y poder describir y predecir correctamente. Es por eso que el argumento de los "no-milagros" falla, ya que supone que dicha conexión es más fuerte.}.

Pero hemos llegado de nuevo al punto inicial. Si la hipótesis del espacio absoluto es falsa, y la teoría newtoniana depende de dicha entidad y la estructura es de derivación, entonces los teoremas extraídos de dichas hipótesis son falsas. Por lo que nadie sabría la teoría newtoniana. Más aún, si suponemos que las buenas explicaciones son explicaciones verdaderas, entonces no podríamos explicar nada con la teoría newtoniana. Hay que desecharla por alguna teoría física más moderna.

Pero lo anterior claramente es falso, la teoría de Newton es explicativa. Tal como señala Laudan, hay teorías explicativas que postulan entidades falsas: aún cuando la teoría depende de dichas entidades, sigue siendo explicativa. El problema ahora está a nivel ontológico y relacionado con el argumento a la mejor explicación: del hecho de que una teoría sea explicativa, no podemos pasar a que las entidades postuladas de hecho existan. Recordemos que uno de los argumentos más usados para defender el realismo el así llamado ``argumento del no-milagro'' depende de la inferencia a la mejor explicación\footnote{No todos los argumentos a favor del realismo dependen de dicho tipo de inferencia. Si bien el argumento del ``no-milagro'' depende de aceptar que esta inferencia es válida, no quiere decir que este sea el único argumento a favor del realismo científico. Por otro lado, no es necesario negar que hay que desechar este tipo de inferencia. Cabe destacar que este tipo de inferencia no asegura que de premisas verdaderas pasemos a una conclusión verdadera, por lo que una manera de defender el realismo científico es hacer que las inferencias de este tipo sean más ``robustas'', es decir, que podamos asegurar la existencia de los objetos de acuerdo a una inferencia no-deductiva. Por supuesto, aún cuando podamos hacer esto todavía queda margen de error ya que estas inferencias son falibles. Para una exposición más detallada véase \cite{Saatsi2010-SAAFVC-2}.}. Este argumento no es deductivamente válido, por lo que siempre hay lugar para el error. Si esto es verdad, entonces el realista que dependa del argumento de los no-milagros está en problemas para justificar su tesis.

Parece entonces que el problema no está en cómo justificamos creencias, sino en el hecho de que esperamos demasiado de la justificación que podemos ofrecer y de cómo se conecta con la verdad. Incluso un argumento deductivo perfectamente válido puede no tener conclusión verdadera debido a que una de las premisas es falsa. Difícilmente en ciencias empíricas podemos ofrecer un grado de certeza del 100\%. Los métodos más utilizados, entre ellos las herramientas estadísticas, no nos dan ese grado de certeza. Aún cuando deseemos que toda la información que usemos en una inferencia sea verdadera y, por tanto, nuestras conclusiones sean verdaderas (utilizando métodos deductivos) si nuestras premisas son conclusiones de un argumento no-deductivo, entonces no hay manera de asegurar tal certeza.

%Alguien podría señalar que si bien la verdad no es algo que podamos obtener (en ciencias empíricas), sin duda es una motivación para los investigadores. Pero esto también es falso, véase por ejemplo el artículo de Boris Hessen (al menos creo que es una forma de obtener la negación de la hipótesis).

Pero entonces qué teoría de la verdad podríamos adoptar tal que no incluyamoLa teoría correspondentista de la verdad tiene una carga intuitiva muy fuerte, pero difícilmente da lugar a los errores que hay en la historia de la ciencia. Por otro lado, no es claro cómo podemos incluir proposiciones que sólo son plausibles en los razonamientos (como aquellas conclusiones de argumentos no-deductivos). Entonces, ¿deberíamos deshacernos de la verdad como una condición del conocimiento?

Si defendemos algo como lo qeu sugiere esta última pregunta

%La verdad juega un papel mediador. Si la verdad es un concepto que no involucra propiedades epistémicas, entonces tiene sentido que podamos corregir nuestras creencias con base en nuevos descubrimientos. Si tiene propiedades epistémicas, entonces la verdad cambia  en tanto nuestras creencias cambian. Esto último no sucede, de ser así, podríamos modificar a conveniencia el color, tamaño y forma de los objetos. Entonces la verdad no involucra propiedades epistémicas. 

% Habrá qué decir a este respecto de cómo la verdad por correspondencia es sumamente intuitiva cuando hablamos de conocimiento cotidiano, mientras que en ámbitos donde tenemos que evaluar entidades no-observables es más difuso cómo juega el mismo papel. PAra esto leer el artículo "A note on truth and reference" de Penelope Maddy. Aunque MAddy señala que podemos decir que no hay un concepto homogéneo de verdad, sino que en contextos cotidianos una teoría correspondentista de la verdad podría ser útil, mientras que en un contexto de investigación no parece muy útil, no me queda claro cómo defender esto. No me queda claro porque muchas de nuestras explicaciones cotidianas dependen de nuestras "empresas epistémicas". Por ejemplo explicar por qué el cielo es azul, o por qué mi cultivo murió ¿es conocimiento cotidiano o de investigación? ¿Cómo hacemos ese corte? Los ejemplos pueden ser ilustrativos "Consider, for example, Hartry Field’s story of the  ancient Greek whose directivities for ‘Zeus is throwing thunderbolts’ correlate neatly with the worldly support of lightning; his utetrance would serve  as a warning to others." MAddy dice que para decir que estos casos son claramente falsos debemos hablar de qué correlaciones vamos a tomar como genuinamente referenciales.  Aunque también menciona que "Here I’m not  out to define ‘truth’ or to provide any theoretical account at all; the whole  point of classifying ‘true’ and ‘refers’ with ‘excuses’ and ‘know’ rather than  with ‘time’ and ‘hardness’ is to deny that they’re things to be theorized about  rather than words whose complex usage is to be studied." Ahora creo que lo qeu defiende MAddy no es un pluralismo alético, sino una nueva manera de investigar el uso de la palabra verdad, que no es homogénea en todo contexto, ya que otras palabras pueden describir mejor la situación que sólo ser verdaderas. 



Otra postura que evita los problemas relacionados con el conocimiento (y con ello la verdad) involucrados en la investigación científica, es el antirrealismo. El argumento de Laudan me parece contundente en el debate entre realistas y antirrealistas. Dado lo expuesto hasta ahora, vale la pena tomar esta postura en serio.

Hasta ahora he descrito una gran variedad de tesis. 


\section{Citas}

But my aim will be to vindicate the possible-worlds theory while making minimal commitments about substantive metaphysical questions, for example, about whether there are things, or properties, that exist only contingently, whether there are individual essences that are irreducible to qualitative properties, whether there could be distinct but qualitatively indiscernible worlds. In the balance of costs and benefits, I give positive weight to this kind of neutrality. \cite{stalnaker2012}

But this would still be an insufficient criterion for we cannot
exclude that a breakthrough in a certain field makes more realistic
explanatory models available. Second, if we do not have a definitive
criterion of indispensability, some idealized models that we currently
deem necessary may turn out to be incompatible with realism.

Although the rule itself is not an explanation, it imposes on cognizers
epistemic constraints on explanations through two steps. First, it gives a
set of phenomena a common referent point or starting point for explanation. This is the constructive function of idealization. Employing HWE
is the starting point for all population-genetical investigation. In Kantian
terms, it gives ‘systematic unity’ to our particular cognitions as a "focus
imaginarius". Second, idealizations are not explanations of phenomena,
but rather normative standards for the assessment of explanations.
It would be misleading to say that HWE as such ‘explains’ anything. 
It furnishes instead a standard to which we can compare collected data. 
By doing so we further the unity and accuracy of our explanations. 
More specifically, we identify cases where deviations are small and thereby do
not reject HWE, and cases that deeply challenge HWE and thus lead us to
new findings.
Ojalá no sea el caso.

% chapter chapter_2 (end)


%!TEX root = ../main.tex


\chapter{Teorías de la verdad}
\label{ch:truththeories}

\section{Blah}

En el capítulo anterior, discutí por qué la verdad es un valor necesario en la investigación. Hay un par de teorías epistemológicas acerca de cómo se entrelazan la teoría de virtudes y la investigación. Entre estas, una desarrollada por Haomiao Yu \citeyear{yu2021}. Quisiera hacer una comparación entre lo que señalé en el capítulo anterior y las teorías mencionadas.

Haomiao Yu desarrolla una teoría parecida a lo que presenté en el capítulo anterior. Uno de los puntos cruciales de su teoría es que trata de hacer explícito qué es el ``entendimiento''. Su trabajo trata de hacer compatible una teoría epistémica con las teorías de la explicación científica. Yu detecta un problema en la literatura sobre explicación científica: muchos asumen que su modelo de explicación se conecta con cómo comprendemos [entendemos] el mundo. Sin embargo, sólo asumen tal conexión sin dar razones a favor de su afirmación.

Pera solventar esta brecha, Yu apuesta a favor de una teoría de las virtudes. Para esto es pertinente indtriducir un par de disttinviones. La primera distinción que hace Yu es entre teorías responsabilistas de las virtudes y teorías fiabilistas de las virtudes. Esto es algo que no señalé explícitamente en el capítulo anterior y vale la pena aclararlo aquí: las teorías fiabilistas de las virtudes se comprometen con que las virtudes de los agentes son constitutivas del conoicimiento. Por su parte, las teorías responsabilistas no se comprometen con esto. Lo único que señalan las teorías responsabilistas es que las virtudes son rasgos que tiene un buen conocedor. Yu nos dice que en su teoría el fiabilismo de virtudes es crucial; el responsabilismo de virtudes es sólo auxiliar.

Lsa virtudes espisémicas identifican diferentes habilidades de los agentes involucrados en la investigación. Yu señala que muchas de las capacidades cognitivas juegan un papel que cierra la brecha entre explicación y entendimiento. El autor señala que una de las teorías del entendimiento basadas en habilidades de los agentes es la de Khalifa. Sin embargo, Yu argumenta que la teoría de Khalifa sólo nos permite hacer distinciones de grado, mientras que hay claros casos de diferentes tipos de entendimiento. Hablidades comunes que señala Yu son: razonamiento deductivo-inductivo, razonamiento causal-mecánico, razonamiento contrafáctico, generalización-categorización y abtrsacción. 

Para aclarar su teoría, Yu nos presenta un caso de estudio: Galileo y su prueba de que el péndulo es isocrónico. Primero que nada, la recosntrucción del caso describe cómo Galileo hace uso de diferentes habilidades. Para su investigación, Galileo hizo uso de sus habilidades matemáticas. Sin embargo, Yu señala que el entendimiento de Galileo sólo es correcto hasta cierto grado. Esto se debe a que Galileo usó las leyes de Kepler para derivar su modelo, entonces sólo tenía disponible un mapeo de la estructura matemática del péndulo. Para un entendimiento completo del fenómeno, es necesario usar las leyes de Newton.

La primer diferencia entre lo que presenta Yu y lo que señalé en el capítulo anterior es que estoy en desacuerdo con su conclusión. Si bien estoy de acuerdo con su teoría de las habilidades/virtudes de los agentes, me parece que imputarle a Galileo que su entendimiento de un fenómeno es de un grado menor que quien tenga a su disposición las leyes de Newton. Esto puede parecer ser muy obvio, pero no estoy de acuerdo. No lo estoy porque es un caso de anacronismo.

En primer lugar Yu señala que lo que constituye el entendimiento es el uso de las habilidades de los agentes. Si esto es verdad, no es claro cómo una situación externa al agente (el hecho de que vivió antes del desarrollo de la física newtoniana), tiene cualqueir cosa que ver con el entendimiento que tuvo Galileo. Más aún, recordemos que, si un sistema es axiomático, una vez que aceptamos las premisas estamos condenados a aceptar la conclusión (suponiendo que el argumento es válido). Siempre podemos renegar de los axiomas: desecharlos por alguna razón, lo que implica que el argumento deja de ser sólido. Dado esto que señalo, si Galileo derivó la fórmula del péndulo, entonces no es claro cómo podía estar equivocado (suponiendo que las premisas son verdaderas). Más aún, debido al marco presentado en la sección anterior, no podemos imputarle errores a Galileo debido a que no tenía información disponible acerca de las leyes de Newton. Hasta el punto de su conocimiento, podemos afirmar que tenía conocimiento certero ``real knowledge''.

El caso del péndulo en Galileo es más intrincado. Para exponer este caso, usaré la reconstrucción que hace Ariotti en \citeyear{ariotti1968}. Galileo usa un par de experimentos para motivar la intuición de que el periodo del péndulo es proporcional a la raíz cuadrada de la longitud de la cuerda. Galileo afirma que el péndulo es \emph{isocrónico}, esto quiere decir que el tiempo que tarda el péndulo en llegar al punto más bajo del círculo descrito es independiente al ángulo. Sin embargo, las pruebas de Galileo no son suficientes para mostrar la isocronía del péndulo, sólo su \emph{sincronía}. Es decir que el tiempo que tardan dos cuerpos de diferentes pesos en llegar al punto más bajo del círculo descrito es el mismo. El experimento es sencillo: tomemos dos cuerpos con diferentes pesos atados a cuerdas con la misma longitud. Si dejamos caer ambos cuerpos, llegarán al mismo tiempo al punto más bajo y su periodo será el mismo.

Lo anterior sólo muestra la sincronía del péndulo, no la isocronía. Para apoyar la hipótesis de que el péndulo es isocrónico, Galileo recurre a un modelo rígido del péndulo. En lugar de usar cuerdas (que pueden doblarse por no ser rígidas), Galileo utiliza una construcción de madera lo suficientemente pulida para que la fricción no afecte el tiempo de traslado. Para medir los tiempos en este experimento, Galileo usó agua. Dejaba caer agua en desde un jarrón con un orificio hacia un bote. Luego pesaba el agua para saber si la cantidad era la misma, por tanto los tiempos iguales en diferentes ángulos. Debido a la poca discrepancia entre los pesos del agua recolectada, Galileo concluye que el péndulo es isocrónico.

Pero este modelo rígido no se comporta igual que un péndulo de cuerda. En la cuerda hay discrepancias debido a que la cuerda hace una curva cuando dejamos caer un cuerpo atado a ella. Por tanto hay discrepancias entre los tiempos. El modelo rígido no es completamente exacto con el fenómeno a estudiar. Más aún, los resultados de Huygens muestran que para que un péndulo sea isocrónico, la curva no debe ser un círculo, sino un cicloide \cite{ramond2023} pero las investigaciones de Galileo fueron un paso hacia el teorema de Huygens de que el péndulo es isocrónico. 

Cabe notar que en esta descripción del caso del péndulo en Galileo, es claro el uso de las habilidades que Yu señala. Habilidades para modelar, habilidades matemáticas y habilidades de deducción. También hay que destacar que son estas habilidades lo que lleva a Galileo a afirmar que el péndulo es sincrónico e isocrónico (casi isocrónico: para distancias de cuerda mayores a la distancia de donde liberamos la masa). 

Sin embargo, toda esta historia nos muestra más que un fallo, un éxito. Vaya, la incvestigación en general depende no sólo de un experimento, sino de diferentes personas trabajando en un mismo tema. Galileo fue un paso para el teorema de Huygens. No podía afirmar la isocronía del péndulo debido a que el experimento no es exacto: el modelo rígido es diferente al modelo con cuerdas. El método de medición por supuesto no es perfecto. Además el círculo no es la curva para que un péndulo sea isocrónico, la curva es un ciloide, como muestra Huygens. Cabe notar además que la solución de Huygens vino antes de la teoría de Newton, por tanto, tampoco son necesarias las leyes de Newton para mostrar la isocronía del péndulo. Los fallos de Galileo son fallos de los materiales a su disposición. Por supuesto, en estos materiales también incluyo la teoría newtoniana.

Pero no se sigue que cualquiera que tenga a su disposición mejores materiales, incluyendo la teoría de Newton, podría haber entendido completamente el fenómeno. Las virtudes y habilidades de Galileo deberían ser lo único relevante para decir si hizo uso de sus habilidades (aunque suene trivial), que es lo que señala Yu. Aunque siendo sinceros, no sé si Yu esté dispuesto a aceptar este argumento, quizás haya algo más en la teoría que diseña, de manera que, la comprensión completa del fenómeno sea un factor relevante.

%No son del todo precisas, claro qwue tenía entendimiento. Dos preguntas Juan. 

\section{Modelos}

Lo dicho anteriormente, por supuesto, se relaciona con el problema del realismo científico. De manera muy sucinta y poco precisa, los realistas científicos defienden la tesis de que las teorías científicas son literalmente verdaderas. Esto implica que las entidades que aparecen en la teoría de hecho existen y que las relaciones entre objetos que señala la teoría reflejan cómo de hecho es el mundo.

Esta tesis tiene muchas aristas. En primer lugar hay un aspecto epistémico involucrado. Si saber implica verdad, entonces el hecho de que alguien señale que Newton sabía que el espacio absoluto existe, el espacio absoluto de hecho existe. Otro aspecto del realismo científico es semántico. Saber si efectivamente los términos individuales que son parte de las oraciones de una teoría, de hecho refieren a un objeto. También hay un aspecto axiológico involucrado: que el objetivo de la ciencia es la verdad\footnote{Estas aristas corresponden más o menos a cómo Khalifa describe la tesis del realismo científico. Véase \citeyear{khalifa2010}}.

Si bien todos estos problemas están relacionados (por ejemplo determinar que el espacio absoluto de hecho existe, haría que las oraciones donde aparece dicho término individual refiriera y que alguien que sepa la teoría sabe que el tiempo absoluto existe), vale la pena tener estos aspectos separados. 

Lo que he señalado hasta ahora está relacionado sólo con el problema epistémico. Nuestro problema es que no es claro que la teoría de Newton sea verdadera, ya que nuevas teorías han mostrado ser mejores descripciones del mundo que la teoría newtoniana, por tanto, Newton no sabía que el espacio absoluto existiese. La justificación de Newton no es suficiente. Esto es sólo una manera de exponer algo que Laudan ya había señalado: la historia de la ciencia ha mostrado que los términos individuales de teorías exitosas no siempre refieren, por lo que dichas teorías son falsas \cite{laudan1981}. Creo que la moraleja que nos da Laudan es que debemos ser cautelosos al formular una tesis realista sobre la ciencia\footnote{Más aún, el punto de Laudan es más débil que señalar que el realismo científico es falso. El punto de Laudan es señalar que no hay una relación tan fuerte entre verdad y poder describir y predecir correctamente. Es por eso que el argumento de los ``no-milagros'' falla, ya que supone que dicha conexión es más fuerte.}.

Pero hemos llegado de nuevo al punto inicial. Si la hipótesis del espacio absoluto es falsa, y la teoría newtoniana depende de dicha entidad y la estructura es de derivación, entonces los teoremas extraídos de dichas hipótesis son falsas. Por lo que nadie sabría la teoría newtoniana. Más aún, si suponemos que las buenas explicaciones son explicaciones verdaderas, entonces no podríamos explicar nada con la teoría newtoniana. Hay que desecharla por alguna teoría física más moderna.

Pero lo anterior claramente es falso, la teoría de Newton es explicativa. Tal como señala Laudan, hay teorías explicativas que postulan entidades falsas: aún cuando la teoría depende de dichas entidades, sigue siendo explicativa. El problema ahora está a nivel ontológico y relacionado con el argumento a la mejor explicación: del hecho de que una teoría sea explicativa, no podemos pasar a que las entidades postuladas de hecho existan. Recordemos que uno de los argumentos más usados para defender el realismo el así llamado ``argumento del no-milagro'' depende de la inferencia a la mejor explicación\footnote{No todos los argumentos a favor del realismo dependen de dicho tipo de inferencia. Si bien el argumento del ``no-milagro'' depende de aceptar que esta inferencia es válida, no quiere decir que este sea el único argumento a favor del realismo científico. Por otro lado, no es necesario negar que hay que desechar este tipo de inferencia. Cabe destacar que este tipo de inferencia no asegura que de premisas verdaderas pasemos a una conclusión verdadera, por lo que una manera de defender el realismo científico es hacer que las inferencias de este tipo sean más ``robustas'', es decir, que podamos asegurar la existencia de los objetos de acuerdo a una inferencia no-deductiva. Por supuesto, aún cuando podamos hacer esto todavía queda margen de error ya que estas inferencias son falibles. Para una exposición más detallada véase \cite{saatsi2010}.}. Este argumento no es deductivamente válido, por lo que siempre hay lugar para el error. Si esto es verdad, entonces el realista que dependa del argumento de los no-milagros está en problemas para justificar su tesis.

Parece entonces que el problema no está en cómo justificamos creencias, sino en el hecho de que esperamos demasiado de la justificación que podemos ofrecer y de cómo se conecta con la verdad. Incluso un argumento deductivo perfectamente válido puede no tener conclusión verdadera debido a que una de las premisas es falsa. Difícilmente en ciencias empíricas podemos ofrecer un grado de certeza del 100\%. Los métodos más utilizados, entre ellos las herramientas estadísticas, no nos dan ese grado de certeza. Aún cuando deseemos que toda la información que usemos en una inferencia sea verdadera y, por tanto, nuestras conclusiones sean verdaderas (utilizando métodos deductivos) si nuestras premisas son conclusiones de un argumento no-deductivo, entonces no hay manera de asegurar tal certeza.

%Alguien podría señalar que si bien la verdad no es algo que podamos obtener (en ciencias empíricas), sin duda es una motivación para los investigadores. Pero esto también es falso, véase por ejemplo el artículo de Boris Hessen (al menos creo que es una forma de obtener la negación de la hipótesis).

Pero entonces qué teoría de la verdad podríamos adoptar tal que no incluyamoLa teoría correspondentista de la verdad tiene una carga intuitiva muy fuerte, pero difícilmente da lugar a los errores que hay en la historia de la ciencia. Por otro lado, no es claro cómo podemos incluir proposiciones que sólo son plausibles en los razonamientos (como aquellas conclusiones de argumentos no-deductivos). Entonces, ¿deberíamos deshacernos de la verdad como una condición del conocimiento?

Si defendemos algo como lo qeu sugiere esta última pregunta

%La verdad juega un papel mediador. Si la verdad es un concepto que no involucra propiedades epistémicas, entonces tiene sentido que podamos corregir nuestras creencias con base en nuevos descubrimientos. Si tiene propiedades epistémicas, entonces la verdad cambia  en tanto nuestras creencias cambian. Esto último no sucede, de ser así, podríamos modificar a conveniencia el color, tamaño y forma de los objetos. Entonces la verdad no involucra propiedades epistémicas. 

% Habrá qué decir a este respecto de cómo la verdad por correspondencia es sumamente intuitiva cuando hablamos de conocimiento cotidiano, mientras que en ámbitos donde tenemos que evaluar entidades no-observables es más difuso cómo juega el mismo papel. PAra esto leer el artículo "A note on truth and reference" de Penelope Maddy. Aunque MAddy señala que podemos decir que no hay un concepto homogéneo de verdad, sino que en contextos cotidianos una teoría correspondentista de la verdad podría ser útil, mientras que en un contexto de investigación no parece muy útil, no me queda claro cómo defender esto. No me queda claro porque muchas de nuestras explicaciones cotidianas dependen de nuestras "empresas epistémicas". Por ejemplo explicar por qué el cielo es azul, o por qué mi cultivo murió ¿es conocimiento cotidiano o de investigación? ¿Cómo hacemos ese corte? Los ejemplos pueden ser ilustrativos "Consider, for example, Hartry Field’s story of the  ancient Greek whose directivities for ‘Zeus is throwing thunderbolts’ correlate neatly with the worldly support of lightning; his utetrance would serve  as a warning to others." MAddy dice que para decir que estos casos son claramente falsos debemos hablar de qué correlaciones vamos a tomar como genuinamente referenciales.  Aunque también menciona que "Here I’m not  out to define ‘truth’ or to provide any theoretical account at all; the whole  point of classifying ‘true’ and ‘refers’ with ‘excuses’ and ‘know’ rather than  with ‘time’ and ‘hardness’ is to deny that they’re things to be theorized about  rather than words whose complex usage is to be studied." Ahora creo que lo qeu defiende MAddy no es un pluralismo alético, sino una nueva manera de investigar el uso de la palabra verdad, que no es homogénea en todo contexto, ya que otras palabras pueden describir mejor la situación que sólo ser verdaderas. 



Otra postura que evita los problemas relacionados con el conocimiento (y con ello la verdad) involucrados en la investigación científica, es el antirrealismo. El argumento de Laudan me parece contundente en el debate entre realistas y antirrealistas. Dado lo expuesto hasta ahora, vale la pena tomar esta postura en serio.

Hasta ahora he descrito una gran variedad de tesis. 


\section{Citas}

But my aim will be to vindicate the possible-worlds theory while making minimal commitments about substantive metaphysical questions, for example, about whether there are things, or properties, that exist only contingently, whether there are individual essences that are irreducible to qualitative properties, whether there could be distinct but qualitatively indiscernible worlds. In the balance of costs and benefits, I give positive weight to this kind of neutrality. (\cite{stalnaker2012})

% chapter chapter_3 (end)


%!TEX root = ../main.tex




   
\chapter{La verdad importa}
\label{ch:truthmatter}

\section{Introducción}
 
Los seres humanos valoramos la sinceridad.
Confiamos en los expertos y tratamos de intercambiar información con nuestros pares.
También valoramos que la información sea correcta.
Damos valor a las personas que saben algo que nosotros no.
Por supuesto aún cuando tengamos confianza en la información que nos proporcionan otros, queremos que la información sea acertada y dudamos de la información que sinceramente nos ofrecen cuando es errónea.
Confiamos en lo que \emph{saben} y confiamos que no nos están \emph{mintiendo} cuando nos dan información.
Por usar términos de Williams \parencite{williams2002} al darle valor a la información dada por otros, esperamos que sea \emph{Precisa} y \emph{Sincera}.
Dadas estas condiciones, es sensato esperar que las personas haciendo investigación sean sinceras y precisas.

Sin embargo, a pesar de lo intuitivo de estos puntos, es bastante más complicado evaluar nuestras teorías científicas con base en la precisión y la sinceridad.
La historia de la ciencia nos ha mostrado que los debates teóricos no están cerrados de una vez y para siempre: descubrimos nuevos fenómenos y modificamos nuestras treorías con base en la evidencia.
Aún así, la sinceridad y la precisión es algo esperable.
¿Cómo podemos reconciliar ambos fenómenos?
Por un lado los cambios en nuestras teorías y por otro la sinceridad y la precisión.
Partamos de estos puntos que creo que son bastante intuitivos y nada controvertidos para explicar la parte central de este capítulo.

En este capítulo, me propongo a exponer la tesis \textit{veritista} como la presenta Duncan Pritchard \parencite{pritchardEpistemicValueCognitive2021}.
La tesis veritista afirma que la verdad es el valor epistémico fundamental.
Es decir que valoramos los estados epistémicos de los agentes debido a que son verdaderos.
Esto es una forma de monismo sobre el valor: el único valor de cualquier estado epistémico depende de que sea verdadero.
En particular, filósofos veritistas que afirman que el \emph{conocimiento} es el estado epistémico por antonomasia, señalan que valoramos el conocimiento porque es verdadero y no porque sirva a otros fines.

Hay al menos dos maneras en las que se ha señalado que el conocimiento no es valioso. 
La primera línea de argumentos, por ejemplo, discute si la \textit{creencia verdadera} es distinta del conocimiento.
Al final, ambos estados epistémicos tienen los mismo fines intrumentales.
Que yo tenga hambre y que yo sepa que hay comida en el refrigerador, no hace diferencia con que crea verdaderamente que hay comida en el refrigerador: al final, saciaré mi hambre.
En este caso, lo valioso es saciar mi hambre, que es un valor de naturaleza instrumental.

La segunda línea de argumentos, descansa en el hecho de que \textit{la verdad} puede restringirnos de otros fines cognitivos valiosos \parencite{elgin2017true}.
Elgin sugiere que, cuando analizamos investigaciones científicas, debemos relajar nuestros compromisos con la verdad.
En ciencia encontramos idealizaciones y modelos que difieren de una representación precisa de la realidad, por lo que deberíamos abandonar el compromiso con la verdad.

Para este trabajo, me ocuparé de la segunda línea de argumentos. 
Para señalar esto, quiero exponer los argumentos de Pritchard contra Elgin, que consisten en mostrar que la tesis veritista no está comprometida con evaluar el número de proposiciones verdaderas.
Veritistas como Pritchard nos dicen que las verdades tienen que tener una conexión profunda con la realidad.
Después de esto, quiero señalar que aún en los casos que menciona Elgin, la verdad es un valor que no podemos rechazar.
Para esto, presento los argumentos de Klein que permiten lidiar con este tipo de casos. 

\subsection{Tensiones en la evaluación de teorías}

De Paul \parencite{depaul2001} ofrece una divertida historia sobre el problema del valor en epistemología.
De Paul señala que hay dos niveles con respecto al problema del valor.
Uno de ellos es señalar que el conocimiento no es el único estado epistémico que es valioso.
Valoramos la sabiduría, por ejemplo, y la sabiduría no es algo que esté relacionado con la verdad.
Si bien, parece haber acuerdo en que el conocimiento no es el único estado epistémico valioso, los filósofos que afirman la tesis veritista defienden que el valor que damos a los diferentes estados epistémicos depende de que sean verdaderos. 
Pero esto resulta polémic para estados epistémicos como la comprensión y la sabiduría.

Por mi parte, estoy de acuerdo con el veritismo: creo que el valor de cualquier estado epistémico depende de que sean verdaderos.
Por usar una metáfora de Sher (\citeyear{sher2016}), la verdad impone fricción en nuestras creencias, así es como sabemos que hicimos las cosas bien, es decir, que nuestras creencias son precisas.
En este capítulo quiero exponer la tesis de Pritchard que ha desarrollado en (\citeyear{pritchard2021, pritchard2021a}) defendiendo el veritismo.
No obstante, mis creencias personales no justifican esta tesis, falta por señalar qué papel juega la verdad y de qué manera conceptualizarla para explicar por qué es valiosa.
Por ejemplo, sabemos que hay errores en la historia de la ciencia.
Teorías "erróneas" que, sin embargo, explican fenómenos, hacen las cosas bien; son precisas\footnote{Dado el papel central que juega la verdad, el segundo capítulo de este trabajo estará dedicado a las teorías de la verdad.}.
Debido a esto, parece haber un conflicto entre la tesis veritista y el valor que damos a las teorías científicas.
Algunos filósofos señalan que la verdad no juega un papel prominente y que deberíamos evaluar a las teorías de acuerdo a otros valores.
Por ejemplo, evaluamos a las teorías científicas si salvan los fenómenos, representan bien su objeto de estudio e incluso la estética \parencite{ivanova2020}.
Mi pretensión es señalar que la verdad también es uno de los valores a tener en cuenta en esta evaluación y que el conflicto con el veritismo es sólo aparente.
Si tomamos en cuenta la justificación y las virtudes de los agentes que hacen investigación, el conflicto desaparece.

\subsection{El plan}

Para este capítulo, el plan es el siguiente: comienzo exponiendo algunas motivaciones, luego el problema del Menón y la tesis \textit{veritista} (recordemos que la tesis señala que el conocimiento es valioso porque es verdadero).
Después presento dos argumentos que se han usado en la literatura para tratar de mostrar que el veritismo es falso; a saber, el argumento de las \emph{verdades irrelevantes} [VI] y el \emph{problema del drenado}\footnote{En inglés "swamping problem". Decidí traducirlo como "problema del drenado", por la siguiente razón: los valores instrumentales consumen cualquier valor que asociemos a la verdad en el conocimiento. Esto en analogía con el artículo de Ned Block "¿se drenan hasta desaparecer los poderes causales?" de (\citeyear{block2013})} [PD].
Aunque pretendo defender la tesis veritista exponiendo la respuesta de Pritchard a VI y AD, los casos históricos parecen debilitar la plausibilidad de la tesis.\footnote{Por supuesto, el argumento más general para esto es la llamada meta-inducción pesimista \parencite{laudan1981}.}
Por lo que hay una tensión entre lo que nos ha enseñado la historia de la ciencia y el veritismo.
Menciono estos ejemplos porque mi interés general es la evaluación de teorías.\footnote{En términos muy generales, los realistas científicos señalan que hay que entender literalmente las oraciones afirmadas por las teorías científicas. Esta tesis se divide en varios compromisos: epistémicos, semánticos y metafísicos. Por ahora sólo voy a formular la tesis epistémica, que es lo que me concierne en este capítulo: un realista científico está comprometido con que nuestras mejores teorías científicas son descripciones exactas del mundo.Pero sabemos, gracias a muchos casos históricos, que las mejores teorías del pasado han sido revisadas, corregidas y —en algunos casos— descartadas. Es clara la tensión que hay entre el realismo científico, el compromiso del realista con la verdad y los múltiples casos históricos.} 

Para conciliar este conflicto quiero presentar un marco, desarrollado por Klein (\citeyear{klein2019})que me permite señalar que los casos históricos expuestos no presentan problema a la tesis veritista.
Esto porque lo que señalamos en los casos históricos depende no sólo de que las teorías sean verdaderas a secas, sino también de cómo justificamos hipótesis y qué papel juegan las virtudes epistémicas en la investigación científica.

\subsection{El problema del valor}

Quiero señalar que a los seres humanos nos interesa la verdad.
Esta importancia que le damos a la verdad guía también nuestras empresas epistémicas\footnote{Utilizo el concepto de ‘empresas epistémicas’ como algo muy general y no bien definido. El concepto puede ser usado para describir tanto a la investigación científica, como a la curiosidad de un niño que está aprendiendo a sumar.}.
Platón en su diálogo con Menón \parencite[][¶¶ 97a-98b]{platon2008}, señala que hay un factor de seguridad ligado al conocimiento, que no se encuentra en la mera opinión verdadera.
Sócrates pide a Menón señalar cuál es la diferencia entre creencia verdadera y conocimiento.

\begin{quote}
    Sóc. — Por lo tanto, la opinión verdadera, en relación con la rectitud del obrar, no será peor guía que el discernimiento; y es esto, precisamente lo que antes omitíamos al investigar acerca de cómo era la virtud, cuando afirmábamos que solamente el discernimiento guiaba correctamente al obrar.
\end{quote}
   
En efecto, también puede hacerlo una opinión que es verdadera.
Lo que señala Sócrates a Menón es que si sólo tomamos en cuenta el valor instrumental del conocimiento, no hay ninguna diferencia entre saber y creer verdaderamente.
En términos prácticos, no hay diferencia entre mi creencia verdadera de llegar a la Ciudad de México desde Aguascalientes, que sea diferente a mi conocimiento del trayecto.

Sin embargo, nos parece que el conocimiento tiene más valor que la mera creencia verdadera.
Platón hace una analogía con las estatuas de dédalo: así como el conocimiento y la creencia verdadera, el conocimiento es como las estatuas de dédalo que están sujetas.
La creencia verdadera, como las estatuas sin base, comienzan a moverse.\footnote{No es claro exactamente cuál es la sugerencia platónica. Williamson (\citeyear{williamson2002}) sugiere que el conocimiento es más estable que la creencia verdadera frente a nueva evidencia. Por utilizar un tecnicismo, los derrotadores para la creencia forman un conjunto de cardinalidad mayor que los derrotadores para el conocimiento.}
Distinguir entre por qué es más valioso el conocimiento que la creencia verdadera es lo que detona el problema del valor en epistemología.

Para fines narrativos, llamaré \emph{veritistas} a aquellos que defienden que el valor del conocimiento radica en que sea verdadero.
Hay al menos dos maneras de negar la tesis veritista.
Una de ellas es señalar que el conocimiento no es el único estado epistémico valioso.
Valoramos otros estados epistémicos como la comprensión, la sabiduría, la racionalidad, etc.
Estos estados son epistémicamente valiosos sin que su valor dependa de que sean verdaderos \parencite{kvanvig2003}, aun si valoramos el conocimiento porque es verdadero.

Otra manera de negar la tesis es señalar que la verdad no es el valor fundamental de ningún estado epistémico, por tanto, tampoco la característica por la cual el conocimiento tiene valor.
Catherine Elgin (\citeyear{elgin2004})ha desarrollado una epistemología que toma en serio las falsedades que encontramos comúnmente en las ciencias, afirmando que cuando realizamos investigación, la verdad es irrelevante.

Kvanvig, por ejemplo, señala que la comprensión es un estado epistémico que no tiene los problemas del conocimiento proposicional.

Si bien Kvanvig asume que el problema de la naturaleza del conocimiento involucra necesariamente a la \emph{verdad} como un componente, esto no implica que la comprensión involucre dicho componente.

Supongamos, por ejemplo, que tomo un libro que contiene una cantidad $n$ de proposiciones verdaderas, de manera tal que es una lista de proposiciones sin conexión alguna entre ellas: la suma de "$2+2 = 2²$", "Yo estoy aquí ahora", "Borges escribió 'El Aleph'", etc.
Luego me doy a la tarea de memorizar todas esas proposiciones.
Si bien tengo un conjunto de proposiciones verdaderas, no es el caso que comprendo lo que dice el libro.

Kvanvig señala que casos como el anterior fallan en ser casos de comprensión porque carezco de las relaciones estructurales relevantes entre las diferentes proposiciones verdaderas.

\begin{quote}
     For when understanding comes to mind, the central elements in focus are ones concerned with structural relationships between various pieces of information grasped by the possessor of understanding, unlike the central element of non-accidentality in focus when one is reflecting on the concept of knowledge. \parencite{kvanvig2009}
\end{quote}
   
Grimm en (\citeyear{grimm2012})hace un recuento de las tesis a favor de la comprensión, señalando que la comprensión puede suplantar al conocimiento proposicional.
Las motivaciones detrás de esto es que la comprensión puede evitar casos Gettier, que la comprensión es un estado epistémico mas transparente y que es claramente un logro cognitivo.

Estos comentarios parecen señalar que la comprensión es un estado claramente distinto al conocimiento proposicional.
Un estado que puede solucionar muchos de los problemas de la epistemología clásica y que ofrece una mejor imagen de los logros cognitivos de los agentes.

Que un nuevo concepto resuelva tantos problemas al mismo tiempo suena sospechoso. 
No es obvio que una teoría del conocimiento más robusta no pueda lidiar con algunos de estos problemas.
Pienso en particular en las relaciones estructurales de las que habla Kvanvig ¿El conocimiento proposicional excluye dichas relaciones?

Esto no es obvio. El debate entre coherentistas y fundacionalistas en teorías de la justificación nos ha enseñado que hay relaciones entre nuestras diferentes creencias.
Ambas teorías buscan explicar cuándo estamos justificados en afirmar que una proposición es verdadera.
Ambas teorías están de acuerdo con que justificamos nuestras creencias con base en otras creencias.
En lo que difieren es en la naturaleza de la justificación.
Los fundacionalistas nos dicen que cada creencia está justificada y que la justificación de que $p$ es verdadera depende de la justificación de que $q$ es verdadera.
La justificación está dada por cada una de las creencias que sostenemos .

Por su lado, los coherentistas señalan que la justificación no está dada por cada una de las proposiciones en nuestro sistema de creencias, sino que el sistema completo de creencias está justificado cuando el sistema es coherente. %<!--- Faltan las citas del libro verde --->

En epistemología clásica el conocimiento está separado en componentes: justificación, creencia y verdad.
Estas teorías de la justificación claramente abogan por relaciones estructurales en el conocimiento.
No es obvio que la comprensión esté dividida. Sólo parece agrupar estos componentes en uno y le pusieron un nuevo nombre.

Además, esto no resuelve los problemas que regularmente asociamos al concepto de "conocimiento".
Supongamos que la \emph{comprensión} efectivamente es distinta al \emph{conocimiento}, entonces el conocimiento sigue teniendo los mismos problemas.
Sólo hemos pasado el problema a otro lado.
Nuestra noción de conocimiento todavía tiene que ser analizada.

Entonces no es claro que el conocimiento no pueda involucrar dichas relaciones estructurales.
Más apremiante que los problemas anteriores, es la motivación para evitar casos Gettier que tiene la comprensión.
Me parece que podemos generar casos Gettier para el entendimiento.
Por lo que la comprensión no excluye casos accidentales.

Una actriz mexicana alguna vez dijo que sentía mucho pésame por las víctimas del "surimi" de Singapur.
Entendemos que la palabra que quiso usar era "Tsunami".
La actriz puede incluso comprender \emph{lo que quiso decir}, pero accidentalmente usó la palabra incorrecta.\footnote{Me encantaría decir que este ejemplo es original, pero está inspirado en la discusión de Davidson en "A nice derangement of epitaphs" (\citeyear{davidson1986}). Por supuesto, la discusión en este artículo es sobre fallos en composicionalidad lingüística. Pero me parece que podemos extraer una moraleja para el tema de la comprensión epistémica.}
Pero la actriz puede tardar en comprender \emph{lo que dijo}, incluso jamás comprenderlo, nunca volver a prestar atención a la afirmación que hizo.

El ejemplo anterior pretende señalar lo siguiente: en un sentido importante, la actriz comprende \emph{lo que quiso decir}.
Pero podemos estar de acuerdo en que la actriz no comprende \emph{lo que dijo}.
Si evaluamos la expresión, la afirmación es claramente incorrecta.
Pero somos capaces de comprender qué fue lo que quiso decir con su expresión.
Creo que este ejemplo sirve para dudar que la comprensión sea invulnerable a casos Gettier.


\subsection{La receta Gettier}

En esta sección quiero repasar brevemente el problema del análisis del conocimiento.
Este problema no es uno que quiera discutir a profundidad, sin embargo, esclarece los componentes que solemos esperar que tenga un agente que sabe que $p$
Además, sirve para contextualizar el ejemplo de la sección anterior.
Quisiera recordarle al lector el problema Gettier y la "receta" de Zagzebski para generar casos Gettier.

A partir de lo discutido con Menón y Teeteto, Platón sugirió que el conocimiento tiene tres componentes: el conocimiento es una “creencia, verdadera y justificada”.
De manera ingeniosa, esta caracterización del conocimiento fue puesta en duda por Gettier (\citeyear{gettier1963}).
El argumento de Gettier depende de presentar dos contraejemplos, lo que hace Gettier con estos contraejemplos es presentar dos casos que cumplen las tres condiciones establecidas y en los cuáles no diríamos que los agentes en cuestión "saben".
Incluso tenemos una receta para generar contraejemplos tipo Gettier \parencite{zagzebski1994}, sólo hay que agregar un evento fortuito que haga nuestras creencias verdaderas.

Por ejemplo, supongamos que voy viajando en el autobús.
Voy viendo la ventana, comienzo a sentirme somnoliento, volteo hacia el frente de la unidad y creo ver a una amiga (digamos Atziri, para que sea más sencilla la narración).
Sin embargo, la persona que vi no es Atziri, sino otra persona muy parecida a ella.
Me quedo dormido durante 10 minutos.
Durante el intervalo en el que me quedé dormido, la persona que vi se baja del autobús y se sube Atziri, quien casualmente lleva un atuendo idéntico a la persona que se bajó.
Despierto de mi sueño y al bajar de la unidad paso al lado de Atziri, la saludo y la invito a tomar un café (siempre es un gusto hablar con ella).

Este ejemplo es uno generado por la receta de Zagzebski.
La conclusión de Gettier es que la caracterización de Platón no ofrece condiciones suficientes para decir que S sabe que $p$.

Hay que recordar que el análisis de Gettier sólo aplica a uno de los lados del condicional material.
Gettier comienza señalando que el análisis de Platón pretende ofrecer condiciones necesarias y suficientes que señalan qué es el conocimiento\footnote{Caracterizar al conocimiento de manera tal que recuperemos todos y sólo los casos en los cuales podamos aplicar el término se conoce como el problema del análisis del conocimiento \parencite{ichikawa2018}.}.
Es decir, que captura todos y sólo los casos que constituyen la aplicación del término "saber".

Gettier utiliza para sus contraejemplos dos casos de inferencia defectuosa.
Pero incluso podemos generar casos para el conocimiento no inferencial, como mi falta de atención en el autobús.

La literatura posterior a Gettier trató de agregar condiciones que excluyeran  estos casos \parencite{zagzebski1994}.
Y si bien las condiciones no son conjuntamente suficientes, esto no señala un problema con que dichas condiciones sean necesarias.

En particular quiero señalar que, en principio, aceptamos que la verdad es una condición necesaria para decir que sabemos algo, digamos que una proposición es verdadera.
Si sabes que hay un gato sobre la alfombra, entonces hay un gato sobre la alfombra.
El conocimiento tiene una estrecha conexión con la verdad y la justificación es el adhesivo.

Si bien, la verdad es una condición necesaria del conocimiento (el conocimiento es fáctico), esto no implica que nuestro conocimiento sea valioso porque es verdadero.
Como prometí al inició del capítulo, quiero presentar los argumentos de Pritchard contra los problemas que han discutido algunos filósofos concluyendo que la verdad no es el componente valioso del conocimiento.
En particularm, hay dos argumentos para rechazar a la verdad como valor: el argumento de las \emph{verdades irrelevantes} [VI] y el \emph{problema del drenado} [PD].

Hasta este momento del capítulo, he señalado mis pretensiones, he dicho que encuentro un problema con la tesis veritista y casos de historia de la ciencia;he señalado, también, que pretendo ofrecer razones para decir que la tensión es aparente.
Además, he ofrecido razones en contra de remplazar al conocimiento por la comprensión.

En el siguiente apartado hago un breve repaso de por qué tenemos la intuición de que el conocimiento es valioso porque es verdadero.
Luego presentaré los argumentos de Pritchard contra VI y PD.
Al final diré por qué los argumentos de Pritchard me parecen relevantes para la evaluación de teorías.

\subsection{Problemas con el valor de la verdad}

Que la verdad es un componente necesario del conocimiento ha hecho que muchos epistemólogos centren sus esfuerzos para explicar por qué el conocimiento es valioso porque es verdadero.

Siguiendo esta línea de razonamiento, filósofos como Pritchard \parencite{pritchard2021a} han defendido que la verdad es el valor epistémico fundamental del conocimiento, \textit{i. e.}, un valor final, no instrumental.
Pritchard, por ejemplo, señala que:

\begin{quote}
   It wasn’t all that long ago that the idea that truth is the fundamental epistemic good was orthodoxy in epistemology. Indeed, this was the kind of claim that was so commonplace that it was almost not worth stating, as to do so would be somewhat superfulous. \parencite{pritchard2021a}
\end{quote}

Esta sugerencia es altamente intuitiva.
Pero como también podemos ver en la cita anterior, esto ha ido cambiando a lo largo de la historia de la filosofía.

La verdad como componente del conocimiento no hace una diferencia con las creencias verdaderas.
En ambos casos la verdad estñá involucrada.
Una sugerencia es desechar el concepto de conocimiento y quedarnos con la pura creencia verdadera \parencite{papineau2021}.
Sin embargo, dicha sugerencia es apresurada.

Con esta sugerencia, me parece, estamos olvidando un componente importante de cómo obtenemos verdades, en particular en la justificación de creencias.
La justificación de nuestras creencias no es algo que podamos obtener fácilmente.
Requerimos trabajo para afirmar que las creencias que tenemos son verdaderas.
Recabamos evidencia, escuchamos atentamente, comparamos información y buscamos nueva información para fijar o rechazar nuestras creencias.
Todos estos fenómenos ese encuentran entre los esfurerzos que usamos para justificar lo que creemos.

La justificación juega un papel importante en el conocimiento.
Sugiero que la \emph{justificación} es un componente en el que recae el esfuerzo de nosotros los agentes, y que el pago que obtenemos por dicho esfuerzo es a tener verdades.
Justificar creencias no es tarea fácil y trataré de motivar esta sugerencia en la siguiente sección.
Quiero mencionar que esta tesis marca una diferencia entre la \emph{creencia verdadera} y el \emph{conocimiento}, al mismo tiempo que es compatible con el \emph{veritismo}: porque el valor del conocimiento depende de que sea verdadero.
Esto entra en clara contradicción con la propuesta de Elgin expuesta al principio de este capítulo.

Dada la sugerencia anterior y los componentes que tomo de William: \emph{sinceridad} y \emph{certeza}, puedo señalar por qué maximizar la verdad es uno de los puntos centrales del veritismo.

Pero esta sugerencia rápidamente nos pone a merced de VI.
Si el veritismo depende de maximizar el número de proposiciones verdaderas, entonces es muy sencillo obtener verdades.
Tomemos un número natural cualquiera $n$ y sumemos 1.
Sumemos 1 consecutivamente a cada $n+1$ y tenemos un número potencialmente infinito de creencias verdaderas, que además están justificadas.
Pero claramente hay una diferencia entre este proceso trivial y, digamos, los axiomas de Peano.
Claramente es más valioso el trabajo realizado por Peano en su investigación para axiomatizar la aritmética.
Por ello, maximizar el número de proposiciones verdaderas es una empresa fútil.
Llamemos a esto \emph{la tesis de la maximización}

Sin embargo, un veritista no tiene por qué aceptar la tesis de la maximización. Pritchard nos dice que un veritista no tiene por qué maximizar el \emph{número} de creencias verdaderas.
Esto no es un componente necesario de la tesis veritista.
Pritchard dice que el veritista está interesado en verdades que "tengan un contacto cognitivo con la realidad" \parencite{pritchardEpistemicValueCognitive2021}.
Nos importa la verdad, pero no cualquier verdad. 

En esta sección me dedicaré a presentar los argumentos de Pritchard contra PD y VI. Antes de continuar con dicha exposición quiero motivar por qué es deseable centrarse en la verdad como valor fundamental del conocimiento.

Primero que nada, de alguna manera tenemos que ser capaces de corregir creencias y quizás la forma más intuitiva de hacer esto es que el mundo impone ciertos constreñimientos sobre el conocimiento humano.
Gila Sher usa el término 'fricción epistémica' para describir esta relación.
La preocupación de Sher reside en que el conocimiento debe tener un fundamento.
En particular, Sher sugiere que este fundamento lo encontramos en el mundo "Groundedness in the world is veridicality, i.e., compliance with substantial standards of truth, evidence, and justification." \parencite[][p. 9]{sher2016}.

Una sugerencia similar es la que hace Blåsjö \parencite{blasjo2022}.
El autor señala que los geómetras griegos estaban comprometidos con un programa operacionalista: las construcciones geométricas son construcciones físicas concretas, que buscan representan cómo es el mundo.
Incluso Descartesa en su libro de geometría genera instrumentos físicos para trazar curvas algebráicas \parencite{descartes2018}. 
Resumiendo mucho de su tesis, Blåsjö dice:

\begin{quote}
    Yet operationalism celebrates concrete constructions and embraces their physicality and real-worldness. This is a point that invites confusion, and indeed I shall argue that previous literature has fallen into misinterpretations for this reason. From a modern point of view, it is natural to take for granted that the foundations of mathematics is a matter of pure theory, while constructions with physical tools can only be of practical relevance. This is completely wrong, according to the operationalist perspective. To understand the philosophy of Greek geometry, we must abandon the dogma that to make mathematics rigorous it “should” be separated from any links to physical reality and turned into purely formal and abstract theory. Operationalism, in contrast to this modern dogma,anchors mathematical rigour in the physical realm. Technical mathematical sources detailing constructions with various curve-tracing devices have often been misinterpreted as quasi-practical, whereas the operationalist perspective suggests that they should instead be read as epistemologically motivated foundational investigations. (p. 590)
\end{quote}

Según el autor, esto además servía para evitar contradicciones en los métodos geométricos de la época.
Aún cuando fallamos en representar de forma precisa cómo es el mundo, buscamos que nuestras creencias estén lo suficientemente justificadas. 
Por el momento asumamos que la verdad juega un papel importante en nuestras 'empresas epistémicas' tanto mundanas como teóricas.
Continúo la siguiente sección exponiendo los argumentos de Pritchard.


\subsection{Pritchard contra VI y PD}

Como mencioné antes, a pesar del peso intuitivo que tiene la tesis del valor epistémico de la verdad, varios filósofos han presentado problemas en contra de esta tesis.
Entre los problemas más apremiantes están  PD y VI.

Pritchard explica ambos problemas de la siguiente manera.
PD nos indica que aún cuando el conocimiento es verdadero, lo valioso de saber algo depende de otros factores distintos a la verdad, por ejemplo, que es útil.

Esta sugerencia no distingue entre la mera creencia verdadera y el conocimiento.
Además, esto motiva la sugerencia de Elgin, si el valor del conocimiento depende de otros factores, no hace una diferencia tener creencias falsas, siempre y cuando sean epistémicamente útiles para otros propósitos.

En este sentido, la verdad no es un valor fundamental, sino sólo una manera de obtener otros bienes valiosos: modificar el mundo de alguna manera, obtener beneficios de algún tipo, etc.
Pero si esto es verdad, entonces podemos llevar a cabo todas nuestras actividades con sólo creencias verdaderas, incluso creencias falsas. Si esto es el caso, entonces que valoremos la verdad es parasitario a la utilidad que esta nos brinda.

Por otro lado, VI establece que si la verdad es el valor fundamental, entonces ante dos verdades cualesquiera, no podríamos elegir cuál deberíamos creer.
Si maximizar el número de creencias verdaderas es el objetivo del conocimiento, obtener dicho objetivo es bastante sencillo: podemos simplemente memorizar el contenido de un libro de "fun facts"

Más aún, hay verdades que no son interesantes en absoluto.
Por ejemplo, hay una respuesta verdadera sobre el número total de granos de arena en Cancún y una respuesta verdadera sobre si la luna gira alrededor de la tierra.
Dado que queremos distinguir entre las respuestas que son importantes de las que no, entonces la verdad no puede ser lo único que da valor al conocimiento.

Para dar una respuesta a estos problemas, Pritchard \parencite{pritchard2021, pritchard2021a} nos dice que estos problemas surgen al asumir que el objeto de evaluación epistémica es el \emph{número de proposiciones} verdaderas.
Pritchard señala que si además involucramos la teoría de la virtud epistémica en el veritismo, entonces somos capaces de resolver los problemas antes mencionados, Pritchard apunta que:

\begin{quote}
  A true statement of fundamental science may be expressed as a single proposition, but it ofers us a great deal by way of cognitive contact with reality. In contrast a long list of trivial empirical claims might offer us hardly any cognitive contact with reality at all. In the sense that matters to us, there is more truth in the former than in the latter, even if the latter involves more true propositions. \parencite[][pp. 1353-1354]{pritchard2021}
\end{quote}

Y el paso a las virtudes intelectuales, está en el siguiente párrafo del mismo artículo "In particular, we should understand how to achieve the epistemic good of truth via appeal to what an intellectually virtuous inquiry would involve." (p. 1354).

Antes de continuar, quisiera exponer de qué trata la teoría de las virtudes epistémicas, que es una forma de fiabilismo sobre la justificación \parencite{klein2019}.

\paragraph{Desfase: teoría de las virtudes epistémicas}

A grandes rasgos, los teóricos de las virtudes epistémicas, se dividen en dos grandes grupos: \emph{fiabilistas} y \emph{responsabilistas}.
Ambas facciones describen a las virtudes a la manera en como Aristóteles\footnote{"Con relación a las mismas cosas son, pues, el cobarde, el temerario y el valiente, pero se conducen diferentemente a su respecto. Aquéllos pecan por exceso y por defecto, en tanto que éste guarda el medio y el deber." \parencite{aristoteles2012}, especialmente los capítulos 2-8 del libro III.}
entendía la virtud: acciones deliberadas llegar a un fin.
Por supuesto, esto no incluye cualquier acción posible.
Supongamos, por ejemplo que mi meta es amasar una fortuna.
La forma más complicada para hacerlo es trabajando (e idealmente ganando un sueldo justo) para generar ingresos, diseñando estrategias de inversión y ahorrando; otra manera de hacerlo es explotando trabajadores y robando tanto como pueda; otra manera de lograr este fin es simplemente heredar una gran fortuna.

De las tres estrategias mencionadas anteriormente, en términos evaluativos, la primera estrategia es más virtuosa que la segunda.
Mientras que la última no constituye un ejercicio de mi parte para ganar ingresos.
Es por ello que el ejercicio de estas virtudes, decisiones voluntarias que practicamos para llegar al fin que queremos lograr.\footnote{Es controvertido señalar exactamente cuál es el \emph{telos} de nuestras empresas epistémicas. En particular el fin de la investigación o de la indagación. Varios candidatos entran en esta canasta: la verdad, la comprensión, etc. Por ahora dejaré de lado este problema, pero el trabajo de Friedman puede ofrecer una respuesta para esto.}

Uno de los pioneros de la teoría de la virtud epistémica es Ernest Sosa.
Sosa \parencite{sosa2017} articula una teoría del conocimiento que caracteriza a las virtudes epistémicas como el ejercicio de ciertas habilidades de los agentes, tales que al ejercitarlas, son constitutivas del conocimiento.
Podemos describir este proceso diciendo que dichas habilidades virtuosas aseguran que tengamos más creencias verdaderas que falsas, de ahí la etiqueta \emph{fiabilista}.

Sin embargo la versión de Sosa, no es la única caracterización de las virtudes epistémicas. 
Zagzebski \parencite{zagzebski1996} distingue su teoría de la de Sosa, señalando que la teoría de Sosa está más relacionada con el consecuencialismo que con la teoría de virtudes aritotélicas.
Zagzebski señala que si el objetivo de la teoría de Sosa es obtener más creencias verdaderas que falsas, entonces la teoría ética relevante es la consecuencialista: mientras más creencias verdaderas tengamos, mucho mejor.
Pero una teoría de las virtudes más comprometida con la teoría de las virtudes aristotélicas, como la que propone Zagzebski, puede explicar cómo le damos valor a los productos generados por los agentes epistémicos, aún cuando no produzcan creencias verdaderas.

La literatura ha llamado \emph{responsabilista} a la teoría de Zagzebski, porque no se centra sólo en los productos epistémicos \textit{per se} (habitualmente creencias verdaderas), sino que toma en cuenta otras características que no necesariamente están relacionadas con obtener la verdad: el deseo de obtener creencias verdaderas no constituye conocimiento, aún cuando es una motivación importante para obtener conocimiento.

Hay un debate sobre si ambas aproximaciones son realmente excluyentes.
A primera vista parece que es sólo una cuestión acerca de a qué decidimos llamar 'virtud epistémica'.
Esto incluso se vuelve más complicado debido a que tanto Sosa como Zagzebski, nos dan un conjunto de virtudes y extensionalmente no es claro que sean excluyentes.
No es claro porque cuando tomamos en cuenta qué habilidades son necesarias para obtener conocimiento, ambas teorías (representadas extensionalmente) se traslapan.
Supongamos, por ejemplo, que un investigador virtuoso está recolectando evidencia para cierta hipótesis $h$.
Debido a que es un investigador virtuoso, es capaz de observar con atención la evidencia y seguir buenos patrones de inferencia (virtudes fiabilistas).

Sin embargo, sin darse cuenta, hay un error en los datos recabados, digamos que definió una función en R que devolvía valores muy bajos que apoyan la hipótesis. Uno de sus asistentes de investigación se da cuenta de el error y le señala esto al investigador, el investigador amablemente revisa el código y lo corrige (virtudes responsabilistas).

El ejemplo anterior pretende ilustrar un caso en el que ambos tipos de virtudes contribuyen a evitar errores y obtener una creencia verdadera.
Por lo que no es claro que haya una distinción de los diferentes tipos si para distinguirlos echamos mano de que las virtudes del fiabilista constituyen conocimiento, mientras que las virtudes del responsabilista son auxiliares \parencite[][p. 144]{sosa2017}
Greco \parencite{greco2002} señala que, si bien es verdad que esto sólo parece un debate terminológico, hay casos interesantes en los que ambas teorías difieren.
En particular, los marcos de virtudes responsabilistas han servido para debatir problemas que se alejan de la epistemología clásica, por ejemplo, las injusticias epistémicas.
Pero en principio no son teorías excluyentes.


\paragraph{Solución de Pritchard}

Con este marco explicado, Pritchard resuelve los dos problemas que se le achacan al veritismo.
Recordemos que hay al menos dos problemas que Pritchard discute para señalar defender al veritismo.
Señalé ambos problemas al inicio: el problema de las verdades irrelevantes [VI] y el problema del drenado [PD].

El problema de las verdades irrelevantes nos dice que si lo único que nos importa en nuestras empresas epistémicas es la verdad, entonces antes dos proposiciones verdaderas: una de peso y una irrelevante; deberíamos creer ambas.

Por ejemplo, hay una respuesta correcta sobre el número de hojas que tiene un árbol y una respuesta correcta a si $a² + b² = c²$ cuando a y b son los catetos de un triángulo rectángulo.
Intuitivamente es más valiosa la segunda proposición que la primera.
Pero si el veritismo es correcto, entonces deberíamos creer ambas proposiciones.
Pero esto es claramente absurdo, por tanto, el veritismo es falso.

El problema del drenado [PD], argumentan, muestra que el veritismo no puede ser verdadero.
Los veritistas nos dicen que la verdad es el valor final para diferentes estados epistémicos.
Pero si esto es verdad, entonces el conocimiento no tiene un valor diferente a la creencia verdadera.
Entonces, según el veritismo, no hay una diferencia de valor entre tener conocimiento de $p$ y una creencia verdadera de que $p$.
Si ambos estados son igualmente valiosos por ser verdaderos, entonces el valor reside en otros factores distintos a la verdad.
La utilidad ha drenado todo valor que pudiéramos darle al conocimiento.
Por tanto el veritismo es falso.

Es interesante la analogía que Pritchard nos presenta para ilustrar lo que tiene en mente.
La analogía es la siguiente: que un chef haga comida deliciosa y luego la pruebe para saber si es deliciosa, no significa que su fin era probarla y no hacerla deliciosa.
El probarla sólo es una manera de asegurarse que es deliciosa.
La analogía indica que si obtener verdades es el fin de nuestras empresas epistémicas, eso no implica que sólo con la mera creencia verdadera, hemos llegado a nuestro objetivo, ni que los factores que no están vinculados a la verdad sean en lo que reside el valor del conocimiento.
Cualquier consecuencia práctica sirve sólo para asegurarse de que nuestro conocimiento es verdadero, es decir \emph{certero}. 

Ahora, una parte crucial de la solución a estos problemas, consiste en que el veritismo no implica que hay que maximizar el número de proposiciones verdaderas.
Lo que buscamos es que nuestro conocimiento tenga contacto sustantivo con la realidad.
Además de involucrar el marco que nos ofrece la teoría de las virtudes epistémicas para señalar que la justificación es un esfuerzo de los agentes.

Si es verdad que el valor fundamental de cualquier empresa epistémica es la verdad, eso explicaría por qué nos interesa obtener verdades: si bien esto tiene implicaciones prácticas, cualquier consecuencia depende de que de hecho nuestro conocimiento apunte a la verdad y la consiga.

La mera creencia verdadera no está motivada por las virtudes del agente.
Como bien señala Elgin, tener creencias verdaderas es muy barato.
Pero es falso que el conocimiento deba tener el mismo valor, porque es más complicado obtener conocimiento.
El conocimiento no sólo depende de la verdad, sino de que haya sido producida por un agente virtuoso.
Un agente que se esfuerza en aprender y ejercer sus virtudes.
Si esto es verdad, entonces el veritismo sí puede distinguir entre la mera creencia verdadera y el conocimiento sin depender de factores como la utilidad.
Resolviendo el problema con PD.

Resolver VI sigue una estrategia parecida. 
Una vez que nutrimos el veritismo con la teoría de las virtudes epistémicas, VI deja de ser problemático.
Como agentes virtuosos, queremos no sólo que nuestro conocimiento sea certero, sino además que el proceso para llegar a la verdad sea fiable.
En particular que esté guiado por características virtuosas de un agente.
Un agente virtuoso puede sopesar entre dos verdades: una irrelevante, la otra de más peso.
Recordemos también que Pritchard caracteriza al veritismo, de tal manera que no importa el \emph{número} de proposiciones verdades, sino que nuestras creencias verdaderas tengan contacto sustantivo con la realidad.

Una vez que Pritchard introduce a la teoría de las virtudes, VI no es un problema. 

\begin{quote}
    So once we unpack EVTM properly, in line with the intellectual virtues, then it follows that it isn’t committed to a view according to which any true belief is thereby of final epistemic value; rather, it is only those true beliefs that offer one genuine cognitive contact with reality $\ldots$ \parencite[][p. 11]{pritchard2021}. 
\end{quote}

Si podemos resolver estos problemas para el veritismo, entonces no es claro que debamos abandonar a la verdad como el valor fundamental del conocimiento.

Sin embargo, cuando empatamos el veritismo con la práctica científica, parece que entramos en problemas.
Uno de los casos más claros de logro epistémico son nuestras teorías científicas.
Regularmente evaluamos teorías científicas si nos permiten explicar fenómenos. 
Las explicaciones de los fenómenos tienen que ser correctas para contar como una buena explicación.
Sin embargo, la historia de la ciencia 

Creo con convicción que el objetivo de la indagación, en contextos de investigación, es obtener conocimiento. Esto implica obtener verdades.

\subsection{Veristismo en problemas: dos casos históricos}

Hasta este punto desarrollé los argumentos de Pritchard. 
Las razones expuestas están a favor de que la verdad es lo que hace valioso al conocimiento.
La verdad es un factor que nos asegura que vamos por el camino correcto, cuando nuestras creencias están bien justificadas.
Las explicaciones y demás consecuencias del conocimiento también dependen de la verdad. 
Más aún, las virtudes intelectuales nos ofrecen una manera de evitar el conocimiento espurio: queremos no sólo que nuestras creencias sean verdaderas, sino que además el proceso para llegar a la verdad sea fiable y guiado por características de un agente. 
Un agente virtuoso puede sopesar entre verdades relevantes e irrelevantes.

Estoy de acuerdo con lo que señala Pritchard, que la verdad es de dónde el conocimiento obtiene su valor (dejemos por el momento si la verdad es una motivación para la indagación). 
Este marco, incluso encaja bien con cómo podemos evaluar la investigación científica a la luz de la teoría presentada: evaluamos sus logros cognitivos las virtudes involucradas en el proceso y que sus resultados sean correctos. 

Sin embargo, no es claro que la tesis veritista sea adecuada para evaluar teorías científicas. 
La investigación científica es probablemente la manera más sistematizada que tenemos los seres humanos para producir conocimiento.
Muchas de nuestras explicaciones dependen de saber algunos hechos acerca de las diferentes disciplinas científicas. 
Sabemos, por ejemplo, que para que haya una combustión se necesita combustible y oxígeno.
Si cualquiera de estos factores está ausente, entonces no hay combustión.
Esto es una explicación del fenomeno de la combustión, sabemos que es una buena explicación que involucra muchos casos particulares de el mismo proceso: la combustión interna, que no seamos capaces de encender una fogata bajo el agua; que cuando la combustión agota el oxígeno a su alrededor, se apaga, etc.

Por supuesto, una explicación que utilice información falsa no es una buena explicación\footnote{Por el momento esto es una suposición, hay literatura que afirma la tesis contraria. Diré un poco más sobre esto en la sección final del capítulo y en el capítulo siguiente.}. 
Consideremos, por ejemplo, que tiro mi taza de café al piso.
Y que explico esto con base en señalar (falsamente) que justo antes de que mi brazo golpeara la taza, el viento la empujo y fue esta la razón por la que cayó al piso.
Esto es una mala explicación.
A lo menos es una mentira para evadir culpabilidad, pero es incorrecta en muchos sentidos.
Alguien que haya escuchado lo que dije puede preguntar por el peso de la taza y la velocidad del viento.
Fenómenos que sin las condiciones adecuadas, no pueden tirar una taza.
Luego concluir que mi explicación fue mala.
De manera sucinta, las teorías y explicaciones parecen tener que representar adecuadamente el mundo.
Todo esto es consistente con la tesis veritista.

Supongamos por un momento que las explicaciones verdaderas son las únicas explicaciones que nos interesan. 
La suposición anterior es claramente falsa. 
Podemos incluso citar el uso contemporáneo de teorías, por ejemplo, la mecánica Newtoniana o la teoría de la selección natural de Darwin. 
Usamos cotidianamente ambas teorías para explicar diferentes fenómenos. 
La teoría de la selección natural nos ayuda a explicar fenómenos biológicos como la adaptación y la especiación. 
Con base en esta teoría podemos explicar, por ejemplo, por qué un grupo dentro de una población tiene más descendencia que otro grupo dentro de la misma población; podemos también saber, por ejemplo, cuáles son los ancestros comunes de especies contemporáneas, \textit{e. g.}, las aves de los dinosaurios. 

Por otra parte, utilizamos la teoría newtoniana para explicar el movimiento de los astros y hacer predicciones de qué posición tomarán en un momento dado, podemos explicar la fuerza que se imprime en una superficie cuando es golpeada por una masa con cierta aceleración y nos sirve para explicar el movimiento de objetos apelando a la inercia.

Ambas teorías tienen grandes ventajas: nos permiten explicar un amplio rango de fenómenos.
A pesar de que estas teorías nos permiten explicar una amplia variedad de fenómenos, no es claro que sean \emph{literalmente} verdaderas. 
En lo siguiente presentaré la teoría newtoniana y la teoría de la selección natural de Darwin, luego expondré algunos de los problemas que varios investigadores han detectado en ellas.

\paragraph{Mecánica newtoniana y selección natural darwiniana}

Tanto Darwin como Newton son dos personajes históricos que asociamos con logros científicos.
Para poder explicar el movimiento, Newton desarrolló una teoría que tiene como entidades el \emph{tiempo absoluto} y el \emph{espacio absoluto}. 
Un cuerpo se mueve o permanece en reposo con respecto al espacio y tiempo absolutos. 
Todos los puntos en el espacio absoluto permanecen constantes durante diferentes intervalos temporales. 
Para explicar la llamada "primera ley" de Newton (inercia), es necesario definir qué significa que un cuerpo esté en reposo o en movimiento. 
Siguiendo a Newton, sabemos que un cuerpo está en movimiento, porque ocupa distintos puntos del espacio en diferentes intervalos de tiempo.
Si el movimiento entre los puntos se da en intervalos iguales de tiempo, entonces el movimiento es uniforme.
Como señala Freudenthal "The distinction between 'rest' and 'uniform motion' implies, however, an absolutely resting frame of reference, and this can only be absolute space'' \parencite{freudenthalNewtonJustificationTheory1986}.
El movimiento en el espacio absoluto no puede ser percibido, mientras que las posiciones relativas entre los cuerpos sí.

La mecánica newtoniana explica el movimiento de los cuerpos con masa. 
La selección natural explica los cambios y las diferencias entre los del planeta. 
Darwin fue quien desarrolló la idea de la selección natural. 
Si bien, antes de la teoría de la selección natural, ya había intentos por explicar cómo se modifican los organismo, Darwin presentó un mecanismo mediante el cuál esto sucede.

Darwin no fue el primero en formular la diea de que los organismos se modifican durante el transcurso del tiempo.
Lamarck tenía una teoría de que los organismos se transforman.
La idea de Lamarck era que el "uso y desuso" de ciertos rasgos de los organismos, los hacía adaptarse mejor a su ambiente.
Pero Lamarck creía que las especies se generaban espontáneamente y luego se adaptaban a su ambiente. 
Fue Darwin quien tuvo la idea del mecanismo de la selección natural.

Darwin fue un personaje interesante, hizo un viaje en barco para recabar información acerca de las corrientes, información meteorológica y la profundidad del mar \parencite{allen2014}.
Darwin originalmente estaba convencido del transformacionismo lamarckiano, pero durante el viaje modificó sus creencias al descubrir evidencia fósil y patrones idénticos en diferentes regiones geográficas. 
Al regresar a Inglaterra, comenzó e ditar el libro que después se publicaría bajo el nombre "El origen de las especies".

A grandes rasgos, la teoría de la selección natural de Darwin la podemos formular de la siguiente manera:

\begin{enumerate}
    \item Hay variación entre los organismos de una población.
    \item La variación es heredable, por lo que la descendencia se parece más a los padres que otros organismos dentro de la misma población.
    \item Las variantes mejor adaptadas tienden a tener más descendencia que las otras.\footnote{Estos tres puntos son una ligera modificación de los puntos que presenta Peter Godfrey-Smith en su libro "Philosophy of Biology" \parencite[][p. 30]{godfrey-smith2014}. El lector puede encontrar una formulación parecida en \parencite{lloyd1988}. Originalmente esta formulación se la debemos a Lewontin \parencite[][]{lewontin1970}.}
\end{enumerate}



Esto expica por qué los descendientes se parecen más a los padres que a otros organismos de la población, algo que le preocupaba a Buffon \parencite{buffon1885} y a otros naturalistas.
Además, explica la variación de las especies a partir del mecanismo de la selección natural.
Muchas dudas se sucitaron con respecto a la teoría, en particular, hay personas que dudan de que la teoría sea correcta porque la selección natural no puede "ser vista"

\paragraph{Problemas con las teorías}

Ambas teorías explican un amplia variedad de fenómenos naturales.
Dada nuestra suposición inicial, es de esperarse que sean verdaderas.
Sin embargo, esta última afirmación dista de ser obvia.

Señalé que la teoría newtoniana sugiere la existencia de una entidad de la que no es claro que tengamos certeza de que existe: el espacio absoluto. 
Algunos teóricos como Leibniz sugirieron que algo extraño estaba sucediendo en la física newtoniana: si sólo tenemos evidencia de los movimientos relativos y un cuerpo en el espacio absoluto no se mueve con respecto a nada, entonces no tenemos evidencia de que exista el espacio absoluto. 
Newton en la \emph{principia} presenta dos experimentos mentales para señalar la existencia del espacio absoluto.
Uno de ellos es en el cual señala que si atamos dos globos con un cordón y los hacemos rotar sobre su eje en direcciones contrarias, la cuerda se tensa.
Si aceptamos que es verdad que la cuerda se tensa incluso en el espacio absoluto, entonces tenemos que aceptar que sí hay movimiento en el espacio absoluto, los globos se mueven con respecto a dicho marco, aunque estén en reposo uno respecto del otro \parencite[][pp. 6-12]{newton1966}.

Leibniz no fue el único en sospechar de la teoría newtoniana. Físicos como Ernst Mach (véase especialmente el capítulo 2 \parencite{mach2013} señala que debido a que diferentes marcos de referencia inerciales tienen las mismas consecuencias empíricas que hablar de espacio absoluto, entonces no es necesario apelar a dicha entidad, algo que sabemos gracias a la relatividad galileana\footnote{Por supuesto, Newton no hablaba de un marco privilegiado, sino que se refería al espacio absoluto como una entidad física. Pero lo que es importante notar es que podemos tener las mismas consecuencias sin apelar a dicha entidad.}.
Más aún, la teoría de la relatividad moderna, señala que donde Newton distinguía dos entidades, realmente sólo hay una: espacio-tiempo.
Incluso en la teoría de la relatividad, no es necesario apelar al tiempo absoluto para dar explicaciones. 
En la relatividad no hay un marco privilegiado y como consecuencia de la estructura del espacio-tiempo tenemos la dilatación temporal.

La discusión anterior está relacionada sólo con el problema del tiempo y el espacio absoluto.
No quisiera entrar en demasiado detalles de la teoría newtoniana (como la naturaleza de las fuerzas gravitacionales).
Pero el ejemplo anterior sirve para ilustrar que hubo dudas sobre las entidades de la teoría newtoniana y que, más aun, sabemos que no existen.

La teoría de la selección natural de Darwin tampoco está excenta de problemas.
Primero que todo, sabemos que Darwin no sugirió cuál era el mecanismo segun el cual hay herencia de caracteres.
Darwin creía, como también formuló Buffon, que la herencia de caracteres se daba a través de la sangre. 
En ese momento, la teoría de la herencia mendeleiana no había sido revisada por Darwin. 
Esta mezcla entre herencia medeleiana y selección natural, la hicieron los investigadores de la "síntesis moderna"\footnote{La síntesis moderna es un periodo de la historia de la teoría evolutiva que se desarrolló al principio del siglo XX. Los biólogos de este periodo incorporaron la genética mendeleiana a la teoría de la selección de Darwin.}. \parencite[][p. 50]{dieguez2012}
El anterior no es el único problema de la teoría evolutiva.
Darwin pensaba que la selección natural es gradual y que va a un paso considerablemente lento \parencite{losos2014}.

Hablando de Darwin, ahora sabemos que hay periodos de estásis, esto es, periodos donde no hay cambio evolutivo durante largos periodos de tiempo.
Sabemos, además, que la selección natural puede durar periodos cortos de tiempo, por ejemplo, en poblaciones de bacterias.

Durante la síntesis moderna, tuvimos una mejor formulación de la teoría darwinista.
Sin embargo, dada nueva evidencia, sabemos también que algunas afirmaciones de los investigadores de la síntesis moderna son erróneas. 
El programa adaptacionista de Mayr fue debatido por Lewontin y Gould.
Los adaptacionistas nos dicen que todos y cada uno de los rasgos de un organismo fueron seleccionados naturalmente.
Gould y Lewontin tienen un artículo famoso acerca de por qué este programa es falso \parencite{gould1979}.
Sabemos además que la herencia genética no es la única forma de selección natural.
La plasticidad fenotípica y otras formas de herencia son evidencia en contra de la herencia genética \parencite{uller2019}. 

A partir de lo dicho aquí podemos obtener dos conclusiones.
La primera de ellas es que ambas teorías son muy útiles.
Nos permiten describir una amplia variedad de fenómenos. 
La mecánica newtoniana permite explicar la aceleración en caída libre de cuerpos en la tierra, el movimiento de los cuerpos celestes e incluso nos permitió llegar a la luna \parencite{nasa}.
Con la teoría de la selección natural de Darwin podemos explicar la variación de las especies, los cambios que han sufrido a lo largo del tiempo e incorporar especies bajo el mismo clado de manera que expliquemos las relaciones entre diferentes organismos.

Sin embargo, teorías físicas más recientes niegan que haya \emph{espacio absoluto}, incluso señalan que la distinción entre dos entidades: espacio y tiempo, no es la más adecuada para describir la estructura del mundo.
Nueva evidencia recabada, nos ha mostrado que la teoría darwiniana no es correcta: hay un mecanismo para la herencia, la herencia no es sólo genética y el gradualismo de la teoría de Darwin es incorrecto.
Dado esto, parece que podemos concluir que las teorías, tanto de Newton como de Darwin, son falsas.
Lo que contradice nuestra suposición de que sólo las explicaciones verdaderas son genuinamente explicaciones, nuestra suposición es falsa.

Pero a lo largo de este capítulo he tratado de defender que la verdad es el valor fundamental del conocimiento, aquello por lo que el conocimiento es valioso.
La tesis veritista, entonces, está en serios problemas.

Ambas conclusiones no pueden ser verdaderas: afirmar la tesis veritista (cuya consecuencia es que sólo las explicaciones verdaderas son útiles) y al mismo tiempo afirmar que teorías literalmente falsas son explicativas.
Personalmente, creo que ambas afirmaciones son verdaderas. 
Claramente hay una inconsistencia en afirmar ambas.
Pero sabemos que si sospechamos que hay una inconsistencia: o bien hay que elegir sólo una afirmación, o bien no son afirmaciones exluyentes, o bien no son afirmaciones exhaustivas. 

Creo que hasta ahora he dado razones a favor de ambas afirmaciones y no creo que ninguna de ellas sea falsa.
No estoy seguro de que sean afirmaciones exhaustivas, pero confieso que no me parece que sean afirmaciones excluyentes.
Lo que resta es decir por qué creo que no son afirmaciones excluyentes.
En la siguiente sección trataré de dar razones para esto.

\subsection{Promesas: veritismo, virtudes y explicaciones falsas}

Es ecesario dar razones a favor de cómo ambas conclusiones pueden ser compatibles.
Si tomamos la segunda estrategia, hasta donde veo, tenemos dos opciones: o bien ambas teorías son verdaderas después de todo, o bien la verdad es fudamental para el conocimiento.

Tomemos el segundo disyunto: que la verdad no es fundamental para para el conocimiento.
El argumento que ofrece Pritchard, y que expusimos líneas arriba, señala que la verdad es el valor fundamental.
Pero esto también es problemático para cualquier teoría en filosofía de la ciencia que se tome en serio que somos agentes falibles. 
El argumento de Laudan \parencite{laudan1981} nos enseñó que las buenas explicaciones no están necesariamente conectadas a la verdad.
La teoría de Newton funciona, aunque no es claro que debamos asumir que es verdadera.

Pero el marco que presenta Pritchard tiene la ventaja de resolver los problemas con los que comúnmente se embiste esta tesis.
Pero este marco no tiene por qué restringir las explicaiones con teorías que no son literalmente verdaderas.
Lo únicop necesario es que tengamos un "contacto cognitivo con la realidad".
No es obvio lo que Pritchard quiere decir con esto, pero al menos, creo que podemos entenderlo con el supuesto de que las virtudes intelectuales de los agentes ofrecen dicho contacto.

En general, justificar nuestras creencias es complicado, en particular, la investigación científica no es empresa fácil.
Hay bastantes detalles que debemos en cuenta en el proceso de justificación de hipótesis, en especial cuando usamos modelos, leyes, teorías literalmente falsas para describir fenómenos. 

La verdad sí es el valor fundamental del conocimiento. 
Señalé que el mundo impone cierta fricción en nuestras creencias, el conocimiento es fáctico. 
Es más, no creo que haya duda de que Newton y Darwin son agentes virtuosos, cuyo uso de virtudes intelectuales no los llevó a la verdad.
Pero las virtudes epistémicas juegan un papel fundamental en la justificación de sus hipótesis.
Ambos son personajes que se involucraron con sus teorías y que ofrecieron las mejores razones que tuvieron para justificar sus hipótesis.

El conocimiento tiene como componente a la justificación y justificar hipótesis es un proceso dinámico. 
Pero podemos afirmar que los agentes sabemos que algo es verdadero aún cuando no podamos afirmar con 100\% de seguridad que no habrá nueva evidencia en contra de nuestras creencias.
Lo que importa durante el proceso es el ejercicio de las virtudes intelectuales.

Desde hace algunos años, los epistemólogos han tratado de modificar el objeto de estudio de la epistemología.
En lugar de tratar de explicar el valor del conocimiento proposicional, han sugerido un cambio hacia el entendimiento. E
ste cambio tiene algunas motivaciones: entre ellas está la falla del programa del análisis del conocimiento y la falta de una respuesta contundente para el escéptico. 

El cambio que proponen aquellas filósofas que sugieren el cambio de conocimiento a entendimiento depende de que no juzgamos las proposiciones, sino que tenemos más objetos que somos capaces de juzgar: relaciones entre diferentes fenómenos, estructuras, pedazos de información, relaciones de dependencia, etc.

Las epistemólogas que han transitado de un enfoque proposicional a uno que toma otro tipo de objetos, señalan también que tenemos más formas de juzgar creencias además de ser verdaderas o falsas. 
Entre estas diferentes maneras de juzgar creencias tenemos: creencias justificadas, racionales, fiablemente formadas, virtuosamente formadas, etc. 
Al mismo tiempo este enfoque no renuncia a la verdad como uno de los objetivos de nuestras empresas epistémicas \parencite{grimm2012}.

Si bien las motivaciones para realizar este cambio responden a problemas puramente epistémicos, este cambio de enfoque nos permite incorporar el hecho de que el número de proposiciones no es lo único a lo que los investigadores deberían prestar atención.
Somos agentes falibles que pueden tener entendimiento de un fenómeno, aun cuando no es claro que hemos llegado a proposiciones verdaderas.

Esto indica que si bien nuestro objetivo en la investigación es la verdad, no es algo que podamos obtener tan fácilmente. 
No parece que esto sea una afirmación problemática: somos agentes falibles y las herramientas que tenemos disponibles para justificar hipótesis no son perfectas. 
Por ejemplo, una de las herramientas más utilizadas en investigación empírica es la estadística.
Sabemos que un resultado estadístico no justifica al 100\% una hipótesis. 

Esto parecería ir en contra de nuestra intuición inicial: que el objetivo de la investigación científica es la verdad. 
Pero como he señalado, la verdad es difícil de obtener: nuestras capacidades cognitivas y herramientas para justificar hipótesis no son perfectas.
Aun así, nuestro problema es cómo obtener la verdad a pesar de nuestra falibilidad. 
Que nuestros métodos sean falibles, no implica que no haya manera de obtener conocimiento.
Tenemos maneras para modificar nuestras creencias con base en nueva evidencia.

Si bien es verdad que somos agentes falibles, de esto no se sigue que debamos abandonar por completo nuestros compromisos veritistas. 
En el libro citado anteriormente de Deborah Mayo, hay una frase que encaja muy bien con la teoría falibilista de Peter Klein \parencite{klein2019} que quiero presentar a continuación. 
Mayo nos dice "We set sail with a simple tool: If little or nothing has been done to rule out flaws in inferring a claim, then it has not passed a severe test." \parencite[][p. xii]{mayo2018}.

Lo que señala Mayo es una intuición que sólo se puede descartar con razones de peso. 
Como agentes en investigación, sabemos que somos falibles.
Sabemos que nuestras creencias pueden ser falsas, pero también sabemos que al justificar creencias, deberíamos ser capaces de eliminar alternativas contrarias (derroteros) a nuestras creencias.

La pregunta importante es cómo a pesar de nuestra falibilidad podemos obtener conocimiento. 
Afortunadamente, Peter Klein \parencite{klein2019} ha desarrollado una teoría del conocimiento que encaja muy bien con lo que he señalado hasta ahora.

De acuerdo con lo que Klein llama "infinitismo derrotable" [defeasible infinitism], los seres humanos somos agentes falibles.
Pero valoramos el conocimiento, el "conocimiento de verdad". 
Esto quiere decir: conocimiento para el que tenemos suficientes razones.

El punto de Klein es que, como agentes, justificamos nuestras creencias con base en las mejores razonez disponibles.
Si tenemos a nuestra disposición razones para sostener una creencia cualquiera $x$, y no tenemos a nuestra disposición una creencia $y$ que disminuya nuestra justificación de $x$, tenemos conocimiento certero.
Si obtenemos nueva información $z$ que haga que dudemos de $x$ y nos haga retractarnos de nuestra creencia original, entonces hay que evaluar nuestra creencia a la luz de la nueva evidencia.

Por ejemplo, supongamos que leemos un estudio que señala una fuerte correlación entre el omeparazol y problemas cardiacos.
Luego descubrimos que el estudio está sesgado.
En este caso dos opciones se abren ante nosotros: o bien retractamos nuestra creencia, o bien ofrecemos razones para señalar por qué no hay sesgo después de todo.
Si somos capaces de dar razones de por qué no hay sesgo en el estudio, entonces tenemos conocimiento certero.

Para que esta tesis sea plausible, Klein defiende su teoría en contra de lo que él llama "el riesgo de desconfirmación empírica".
Este problema, señala Klein, embiste a las teorías epistémicas que afirman que una creencia debe tener una cadena causal adecuada.

El argumento de Klein descansa en lo siguiente: supongamos que tenemos una creencia falsa causada por evidencia empírica.
Al momento de corregir nuestra creencia con base en nueva evidencia empírica, no podemos estar seguro de si la nueva creencia modifica la cadena causal o es parte de la misma cadena.

Una sugerencia es señalar que lo que hice fue corregir mi creencia.
Pero la creencia formada por nueva evidencia tiene una historia causal completamente diferente a la anterior.
No sé si ahora tengo dos creencias causalmente formadas, o una creencia causalmente revisada. 

Siempre es difícil saber si hay una relación causal entre eventos (es necesario investigar empíricamente), más difícil aún saber si nuestras creencias fueron causadas por diferentes eventos o hemos modificado la cadena. 
En palabras de Klein:

\begin{quote}
    I take that as a good prima facie reason for thinking that the difference between real knowledge and less paradigmatic forms of knowledge or ignorance depend on the quality of reasons for the belief, not the etiology of the belief. \parencite[][p. 403]{klein2019}
\end{quote}

Esto hace sentido del hecho de que tanto Newton como Darwin \emph{entendían} tenían conocimiento certero. 
Esto es la calidad de sus razones era adecuada. 
Lo cual indica que hay valor en las investigaciones que realizaron, aun cuando nueva información nos dice que estaban equivocados. 

Siempre podemos encontrar derroteros para muchas de nuestras razones para sostener creencias.
Nueva información hace que nos retractemos de nuestras creencias, a menos que podamos derrotar la información.
En la época de Newton hubo investigadores que dudaron de la existencia del espacio absoluto.

Ahora sabemos que podemos hacer mucho trabajo en física sin la necesidad de postular dicha entidad. 
Pero Newton tenía buenas razones para sostener sus creencias, además la investigación racional opera de esa manera: desarrollamos teorías y nueva información es capaz de derrotar nuestras creencias. 
A menos que tengamos razones para desechar esa nueva información, estamos justificados y tenemos conocimiento.

Por supuesto, esto sólo indica que Newton tenía conocimiento, pero nosotros sabemos que la teoría es falsa.
Me parece que esto no es problemático en absoluto.
Como agentes preocupados por tener creencias justificadas, usamos modelos, abstracciones e idealizaciones para justificar dichas creencias.

Estos métodos no reflejan de manera precisa los fenómenos \parencite{bokulich2016}.
Pero son métodos que justifican lo suficiente como para decir que hay conocimiento.
Lo que importa es que estemos conscientes de que no reflejan de manera precisa su objetivo y esto puede ser problemático.
En estos métodos, ampliamente usados en investigación, fácilmente podemos cometer errores.
Errores que pueden llevarnos a conclusiones equivocadas, pero que somos capaces de solventar.
Pensemos por ejemplo en la definición de agente de la economía clásica y las diferentes maneras en la que los economistas han tratado de resolver.
Esto no quiere decir que la economía clásica sea algo que deberíamos abandonar por completo, sólo que hay que ser cuidadosos y tener en cuenta los diferentes factores involucrados \parencite{sen1977}.

El problema es, me parece, como entendemos la naturaleza de la verdad y cómo estos métodos encajan con lo que he dicho hasta ahora.
Quiero discutir estos temas en el siguiente capítulo porque, al final, me interesa discutir cómo la verdad juega un papel en nuestros modelos causales.

\subsection{Conclusiones}

En este capítulo defendí la tesis veritista: que el conocimiento es valioso porque es verdadero.
Presenté dos problemas que algunas filósofas señalan en contra de la tesis.
A estos problemas, expuse las soluciones que ofrece Pritchard a favor del veritismo. 

Sin embargo, no quedaba claro que si la verdad hace que el conocimiento sea valioso, cómo dicha tesis es compatible con casos de teorías que son literalmente falsas.
Presenté dos casos históricos que son casos de éxito epistémico, pero no son teorías verdaderas. 
Para hacer compatible la tesis veritista con estos casos, presenté la teoría de Klein que nos permite rescatar nuestras intuiciones originales: el veritismo y una teoría de la certeza basada en justificación falible.

Quisiera terminar señalando dos cosas que restan por hacer. 
Mi objetivo final es decir cómo los modelos causales sirven en la investigación, incluso cuando son abstracciones de los fenómenos que represenmtan.
Creo que el marco presentado en este capítulo es útil para continuar argumentando que los compromisos que tienen los investigadores cuando utilizan modelos es menos problemático de lo que pudiera parecer a primera vista.

% chapter chapter_4 (end)


%!TEX root = ../main.tex




\chapter{Prácticas biológicas}
\label{ch:practices}

Parece claro que los investigadores en general buscan explicaciones.
Dichas explicaciones lucen diferente en distintas disciplinas.
Por ejemplo, en matemáticas, buscamos una prueba para saber ciertas propiedades de los números.
Esto, sin embargo, es distinto en las ciencias empíricas.

Los físicos tratan de describir fenómenos como el movimiento, velocidad y aceleración de los objetos.
Tratan de descrtibir la fuerza necesaria para mover un objeto con cierto peso, etc.
Ahora bien, la física tiene descripciones muy precisas de estos fenómenos, sin embargo, otras disciplinas no tienen descirpciones tan precisas: hay más variables involucradas.
Comúnmente se acepta que la física usa leyes para describir fenómenos.
En mecánica clásica, dichas leyes están cuantificadas universalmente, son reversibles y deterministas.
De manera tal que sistemas que cumplan las misma propiedades pueden describirse con una misma ecuación.

Pienso, por ejemplo, en el movimiento uniformemente acelerado.
Este tipo de movimiento es tal que, la aceleración de un cuerpo con masa es constante en intervalos iguales de tiempo. 

En biología difícilmente tenemos este tipo de generalizaciones.
Por lo general se asume que en biología no hay leyes. 
En particular en biología evolutiva, sabemos que las condiciones no son lo sufcientementer estables para hacer generalizaciones.
Pensemo, por ejemplo, en las leyes de Mendel. 
Las ley de segreación de Mendel nos dice que los alelos de un gen están segregados entre ambos padres.
Esto significa que si ambos alelos son recesivos no codifican, mientras que si uno de ellos es dominante, entonces codifican para un rasgo fenotípico particular.





% chapter chapter_5 (end)


%%%%%%%%%%%%%%%%%%%%%%%%%%%%%%%%%%%%%%%%%%%%%%%%%%%%%%%%%%%%%%%%%%%%%%%%%%
% The back matter contains unnumbered chapters
% conclusion, french summary, bibliographies, indices, glossaries
%%%%%%%%%%%%%%%%%%%%%%%%%%%%%%%%%%%%%%%%%%%%%%%%%%%%%%%%%%%%%%%%%%%%%%%%%%

\backmatter

%!TEX root = ../main.tex

\chapter{Discussion} % (fold)
\label{ch:conclusion}

% chapter conclusion (end)

\begin{fullwidth}
	\printbibliography[heading=bibintoc]
\end{fullwidth} %04 de junio del 2024. Creo que esta madre no copila porque la base de datos .bib es muy grande.

\begin{fullwidth}
	\printindex
\end{fullwidth}

%\begin{fullwidth}
%   \printglossary[type=\acronymtype]
%   \printglossary
%\end{fullwidth} %movido a posopts. "Movido" es un anglicismo? no tengo una mejor palabra en español para esto.

%\input{input_files/abstract} %Movido a posopts, o sea, posibles opciones

\end{document}
