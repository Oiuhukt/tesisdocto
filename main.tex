% Template inspired by
% https://tufte-latex.github.io/tufte-latex/
% http://c-elvira.github.io/ Ph.D. thesis
% https://jflamant.github.io/ Ph.D. thesis
% https://guilgautier.github.io/ Ph.D. thesis

\documentclass[letterpaper, 13pt, twoside, table, justified, 
nofonts, nobib, nohyper]{tufte-book}

\usepackage{xkeyval}
\usepackage{soul}
\usepackage{dirtytalk}
\usepackage[spanish]{babel}
%\usepackage{hyphenat}
%\hyphenation{ex-ter-na-lis-ta preo-cu-pa-ban in-ves-ti-ga-do-res des-cu-bri-mien-to ac-ti-vi-dad his-tó-ri-cas in-ves-ti-ga-ción des-ha-cer-nos preo-cu-pa-ción ex-ter-nal in-ter-na-lis-tas Dis-covery his-tó-ri-cas un-der-stan-ding preo-cu-pa-ción}

% Book metadata
\title{¿Realmente importa?\\  {\LARGE Semántica de contrafácticos y el valor epistémico de la verdad}}
\author[Oscar Abraham Olivetti Alvarez]{Oscar Abraham Olivetti Alvarez}
\publisher{UNAM}

%%%%%%%%%%%%%%%%%%%%%%%%%%%%%%%%%%%%%%%%%%%%%
% default imports and commands are located in
% tufte-common-local.tex
%%%%%%%%%%%%%%%%%%%%%%%%%%%%%%%%%%%%%%%%%%%%%

%\input{input_files/acronyms_and_glossary}  % must be imported in preamble %movido a posopts

% The bibliography is managed with biblatexmain.tex
\addbibresource{bibliography.bib}

\usepackage{fontawesome}
\newcommand{\conferenceIcon}{\textcolor{myblue}{\faUsers}}
\newcommand{\paperIcon}{\textcolor{burgundy}{\faNewspaperO}}

\usepackage{amsthm}
\makeatletter
\let\c@theorem\relax
\let\c@corollary\relax
\let\c@definition\relax
\let\c@example\relax
\let\c@lemma\relax
\let\c@proposition\relax

\let\corollary\relax
\let\definition\relax
\let\example\relax
\let\lemma\relax
\let\proposition\relax

\newtheorem{theorem}{Theorem}[section]
\newtheorem{assumption}[theorem]{Assumption}
\newtheorem{corollary}[theorem]{Corollary}
\newtheorem{definition}[theorem]{Definition}
\newtheorem{example}[theorem]{Example}
\newtheorem{lemma}[theorem]{Lemma}
\newtheorem{proposition}[theorem]{Proposition}
\newtheorem{remark}[theorem]{Remark}
\makeatother

\usepackage{amsmath}
\counterwithin{equation}{section}

\usepackage{amsfonts, amssymb}
\usepackage{url}
\begin{document}

%%%%%%%%%%%%%%%%%%%%%%%%%%%%%%%%%%%%%%%%%%%%%%%%%%%%%%%%%%%%%%%%%%%%%%%%%%%
% The front matter contains title page, acknowledgements, toc, nomenclature and dedication
%%%%%%%%%%%%%%%%%%%%%%%%%%%%%%%%%%%%%%%%%%%%%%%%%%%%%%%%%%%%%%%%%%%%%%%%%%%

%\frontmatter

\frontmatter
\maketitle

\title{¿Realmente Importa?} % <========================================


\setcounter{chapter}{-1}  % Start at Chapter 0 (Introduction)
\setcounter{secnumdepth}{3}
\setcounter{tocdepth}{3}

%\input{input_files/acknowledgements} %movido a posopts, es decir, posibles opciones

\begin{fullwidth}
	\tableofcontents
\end{fullwidth}

%\listoffigures

%\listoftables

%\input{input_files/nomenclature} %movido a posopts

%\input{input_files/dedication} %movido a posopts

%%%%%%%%%%%%%%%%%%%%%%%%%%%%%%%%%%%%%%%%%%%%%%%%%%%%%%%%%%%%%%%%%%%
% The main matter contains numbered chapter: introduction, chapters
%%%%%%%%%%%%%%%%%%%%%%%%%%%%%%%%%%%%%%%%%%%%%%%%%%%%%%%%%%%%%%%%%%%

\mainmatter

\chapter{Promesas}

\section{Promesas}

\noindent Este apartado es un mensaje para mi comité tutor escrito el 07 de junio del 2024. Sé que la fecha límite de evaluación se acerca y que hay muy poco tiempo para leer lo que les mando hoy.
Quería en este par de semanas escribir la introducción, porque me he dado cuenta de que no he sido muy claro con cuál quiero que sea la estructura de mi trabajo--me disculpo por eso.
Pero creo que ahora puedo hacer más explícita la estructura del proyecto.

Lo que quiero hacer es lo siguiente.
En el primer capítulo quiero argumentar por qué el \emph{contexto de descubrimiento} y el \emph{contexto de justificación} son genuinamente separables.
Con esto quiero decir que, para hablar de racionalidad científica, necesitamos ser capaces de vagamente aislar cuáles son los factores genuinamente epistemológicos y los no epistemológicos.
Y esto, me parece, es fácilmente capturado por la distinción.

Defiendo esto respondiendo a un artículo de \textcite{Yturbe1995}.
El artículo es viejo y creo que claramente es una caricatura de las posturas de \emph{los positivistas}
Pero para esas fechas Mary Hesse ya había publicado \say{Models and Analogies in Science}, la caricatura está injustificada.
Quiero aclarar que  al responder a este artículo tengo la intención de desarrollar las teorías filosóficas de \textcite{Hessen2009, Grossmann2009} tal como las reconstruye \textcite{Freudenthal2009} y las lecciones de \textcite{blasjo2022} sobre historia de la ciencia.
Porque me parece que una distinción importante porque, a diferencia de otras actividades humanas, la investigación tiene un interés por la Verdad.
Por ahora dejemos el término sin analizar.

En palabras muy generales, y poco adecuadas, Blasjo defiende que la historia de la ciencia debería verse como un proceso continuo entre las teorías del pasado y las teorías más recientes.
Esta postura tiene ciertos detalles que la hacen atractiva, por ejemplo, que asume que los investigadores del pasado se comportan más o menos como lo hacen los investigadores actuales--a mí me gustaría pensar eso.
En particular en que realmente buscan justificar sus teorías.
Esta parte es importante.

Sabemos que si bien había representaciones, digamos gráficas, en los primeros años del cálculo; y las representaciones analíticas aparecieron posteriormente, esto no implica que había una "barrera" en lo que podían pensar los investigadores.
No es que no les alcanzara para pensar en las soluciones analíticas: de hecho usaban las representaciones analíticas para sus apuntes, para hacerlos más breves.
Esta \emph{barrera} es algo de lo que hablan \textcite{Carrillo2021-CARAAP-12}  y que les sirve para argumentar que su postura artefactualista explica mejor cómo funcionan y se generan las idealizaciones, a diferencia de las posturas de los \emph{relevantistas causales} y lo que ellos llaman los \emph{desidealizadores}.
Su argumento o explicación, depende de señalar que su postura no parte de la relación modelo-mundo.

Pero esto no es una ventaja de ningún tipo.
Aquí es donde importan los autores raros de los que he estado hablando.
Su ejemplo principal es el del modelo Hodgkin-Huxley.
Su reconstrucción es equivocada.
Además, es falso que los investigadores no sean capaces de \emph{ver más allá} de sus compromisos teóricos.
Hodgkin y Huxley sabían qué posibles hipótesis podrían hacer que su modelo fuera más adecuado, pero siempre tomando en cuenta cuál es el objetivo específico.
Porque el modelo no está \emph{superado}, se sigue usando como metodología.
Pero me queda claro que H\&H no estaban haciendo lo que dice Carrillo y Knutiila.
Para esto estaba usando el libro \citetitle{Piccolino2013} que señalan algo muy parecido a lo que señalan Blasjo, Hessen, Grossman, etc., en corto, que la distinción \emph{factores externos} $|$ \emph{factores internos} es espuria.
Pero aún deberíamos poder explicar cómo diferentes aspectos sociales e históricos influyen en la justificación, pero centrándonos en la justificación.
No soy historiador y creo que esto lo expresan más bellamente Blasjo, Hessen y Grossman.
Pero como tengo a la mano el libro de \emph{shoking frogs} que toma una postura parecida a los autores rusos y de los paises bajos de los que he venido hablando.

Además, quiero decir que tomar el camino que sugieren los artefactualistas es erróneo porque no puede lidiar con casos como el de los estudios de IQ.
Sabemos que el IQ no mide nada y si mide algo, el estudio está completamente sesgado, ¿justificamos esto como lo dice Carrillo?
Por supuesto que no, porque tendríamos que decir que los involucrados en el diseño estaban decidiendo racionalmente con base en las analogías, modelos e idealizaciones, que tenían a la mano.
Esto es claramente falso, el estudio estuvo sesgado desde el inicio.
Revisar los estudios del pasado con las herramientas del presente no es más que hacer investigación.
Nunca me ha quedado claro exactamente como lidian los internalistas ni los externalistas con esto.
Afortunadamente, no tenemos que elegir, sólo justificar.

Para esto está el capítulo que ya les había mandado sobre Pritchard.
Realmente no es mucho más que una explicación de las diferentes posturas.
Como dije casi al inicio, la verdad es lo que le da valor al conocimiento, tal como defiende Pritchard.
Y creo que la noción de verdad es lo que falta por explicar: la metáfora de Pritchard de un \emph{contacto cognitivo profundo con la realidad} es muy vaga.
PEro creo que podemos analizar la naturaleza de la verdad en término distintos al de \say{mind independence}
Por eso quería decir algo al final sobre el artículo de \textcite{klein2019} y el trabajo de \textcite{friedman, friedmana}, en particular con los casos de Newton y Darwin.
Los historiadores mencionados y sus reconstrucciones, empatan muy bien con la teoría coherentista y con el trabajo que ha desarrollado Friedman.
Soy pésimo apostando, así que sólo apuesto si tengo la información completa.
Entonces me gustaría darles al menos la información necesaria para proseguir.
Creo que el conocimiento es al menos una creencia, verdadera y justificada.
Estoy al tanto del trabajo de Gettier, de las respuestas que se le dieron, del artículo de Zagzebski.
Pero no he encontrado una mejor manera de describir el conocimiento.
Estoy también al tanto de los trabajos posteriores acerca de la estructura de la justificación y estoy muy influenciado por el artículo de Sosa \textcite{Sosa1991}, por lo que creo que la estructura de la justificación es coherentista.
Me parece que esto está muy en línea con mis intereses en torno a la historia de la ciencia y que no podemos pedir mucho más para la justificación.
Estoy al tanto, también de los problemas que tienen las teorías coherentistas de la justificación.
Pero no quiero discutir esto porque el trabajo sería interminable: creo que con hacer referencia a los autores pertinentes es suficiente.

¿Qué tiene que ver todo esto con los modelos y la teoría de la verdad?, que debido a que los investigadores en el proceso de creación de modelos, sí parten de la intención de representar el fenómeno de alguna manera.
Además, esto está respaldado por la historia de la ciencia, tal como la describen \textcite{Hessen2009, Grossmann2009, blasjo2022}.
Que además los investigadores sí tiene la intención de \emph{representar adecuadamente} el fenómeno y que mucho de esto está guiado por sus intereses partículares \textcite{friedman}.
Que \emph{intereses particulares} no significa siempre completamente racionales o completamente irracionales, porque no hay distinción entre factores externos y factores externos.
Que los investigadores quieren \emph{justificar} sus creencias y por qué su investigación explica un fenómenode interés; también tienen en cuenta qué puede fallar con su modelo y pueden hipotetizar qué haría que su hipótesis o creencia sea falsa.
De hecho es un ejercicio que hacen \textcite{Hodgkin1951, Hodgkin1952a, Hodgkin1952b}.
El artículo de 1951 es genial porque el mismo Hodgkin reconstruye todo el debate que hay en su tiempo sobre el potencial eléctrico, cómo se le ocurrió la idea de usar los axones del calamar gigante, que muestra su modelo, cómo derivar las ecuaciones (usaron un método de \say{fuerza bruta} y no una solución analítica), etc.
Si les soy sincero, me parece extraño que Carrillo y Knuutiila no citen ese artículo.
Todo esto apunta a que el conocimiento de la ciencia es proposicional y son creencias verdaderas justificadas.
Porque es muy fácil hablar de conjuntos de proposiciones verdaderas o falsas.
Porque si hablamos en estos términos, podemos, por decir algo, aislar un subconjunto de las creencias de dos investigadores sosteniendo teorías distintas.
Y comprobar a la luz de nueva evidencia cómo corregir creencias.
Esto es algo que ya hace la teoría de la argumentació: sea formal o informal.
Y en este espacio no hay un terreno muy claro, y yo no soy un teórico de la argumentación, mis colegas lo hacen mucho mejor sin mi.
Además no puedo hablar de \emph{truthbearers} y \emph{truthmakers}, porque no hago metafísica, ellos ya lo hacen mucho mejor.
Por eso voy a tomar esto sin discutirlo: conjuntos de creencias, que son proposiciones, que son los \emph{truthbearers} de lo que sea que sean los \emph{truthmakers}.
Pero sea lo que sea ese truthmaker, debe estar relacionado de alguna manera con nuestras creencias: ya sea generalizar, hipotetizar, abstraer, idealizar, etc.
Y esta relación está de hecho en la historia de la ciencia y en las prácticas actuales: representar adecuadamente el fenómeno.
O como dice MClaughlin, allá donde nuestras creencias son exitosas, sabemos que estamos obedeciendo las leyes de la naturaleza.
Pero este conocimiento sólo es \say{parcial}, por usar un término poco adecuado.
Y que la única manera de saber cuándo una creencia es verdadera, es explicando mejor.
Y que el conocimiento es falibilista a la \emph{Klein} y que la teoría de virtudes nos ofrece la fuente de la normatividad para evaluar casos de conocimiento.
Y que podemos tener \citetitle{klein2019}.
Y que esto es lo más que le podemos pedir a la verdad con todos los constreñimientos anteriores: históricos, sociales, económicos en los que se desarrollaron las teorías.
Pero siempre buscando la justificación de hipótesis.

Esto es una imagen de lo que quiero decir.
Por supuesto, esta tesis sigue teniendo el en el título \emph{semántica de contrafácticos}
Pero con lo dicho hasta ahora sirve para dejar mis cartas sobre la mesa.
Entonces, asumiendo que todo lo dicho anteriormente es verdad--al menos tengo razonespara creerlo--la teoría más adecuada para repsentar las prácticas científicas bajo estos lentes es el empirismo modal \textcite{russelllogical}.
Que lidia con el problema del acceso: sabemos que los contrafácticos son verdaderos\footnote{en el torpe sentido en el que lo presenté} cuando funcionan\footnote{En el torpe sentido en el que lo presenté}.
Lo que justifica nuestro conocimiento causal basado en intervenciones\footnote{Ya deben estar cansado de esto}.
Y la teoría de Potochnik explica muy bien por qué \parencite{Potochnik2017-POTIAT-3}, y lo hace mucho mejor de lo que yo podría hacerlo.
También creo que es más amable con la postura de \textcite{suarez2004} del inferencialismo.


En corto, estoy tratando de participar en el desafío que presenta suárez.
Su libro no es sobre la verdad y otros conceptos semánticos.
Pero creo que modificando nuestro concepto de verdad\footnote{como la torpe presentación que realicé} de esta manera: tomando en cuenta la práctica y el uso de estas herramientas en la investigación científica.
Señalando cómo decimos que una creencia es verdadera por los constreñimientos a la mano y llenar el hueco semántico que señala \textcite[][p. 16]{suarez2004}

\begin{quote}
	The deflationary approach adopted in this book is aimed instead at scientific representation via modeling.
	It does advance and defend a deflationary account of the use of models in practice.
	It does not, however, advance  a thesis about semantics, truth, or any of the categories typical of analytic  philosophy of language.
\end{quote}


Ahora, si entendemos el debate bajo esta luz, es más claro el problema entre causalistas $|$ estadisticalistas.
Una de las premisas que he visto continuamente, pero que no he visto discutidas, es la de que los modelos estadísticos no pueden ser causales porque pertenecen al reino de lo abstracto.
Pero si nuestros modelos, al contrario, parten de un fenómeno y luego decidimos si es adecuada nuestra representación (decidiendo entonces si darle la etiqueta de verdadero), no hay misterio en fenómenos causando fenómenos.
Hacer cierta distinción entre la selección natural como proceso y la selección natural como causa, que es algo en lo que ha venido insitiendo \textcite{Millstein2006, Millstein2002}.
Pero no sé si quiero entrar a detalle en esto, fue lo que intenté hacer--fracasando abruptamente--en la maestría.
Sólo puedo señalar ciertas consecuencias de la imagen que les presento.

Creo que el proyecto no es muy original y creo que es hasta el día de hoy (08 de junio del 2024) que puedo llamarlo proyecto.
Me disculpo por esto.
Tengo buena parte de lo que aparece en el pdf como el capítulo 2, pero lo tengo en otra carpeta\footnote{está en el git}.
Ese documento necesita mucha edición, entonces no recomiendo que lo lean.
El capítulo está en mi mal inglés porque partió de un abstract que mandé a un evento, luego sólo continué.
Las partes relevantes para la evaluación son la Introducción  y el capítulo 2.
Todavía tengo que editar lo demás del texto, entonces tampoco recomiendo leerlo.
Recomiendo encarecidamente leer esta sección.



% !TEX root = ../main.tex
% LTeX: language=es, en

\chapter{Introducción}\label{ch:introduction}

\section{Una breve anécdota}

\noindent El doctorado ha sido un largo viaje durante el cuál aprendí mucho.
Una parte considerable del aprendizaje está relacionado con la filosofía\footnote{Juro que este breve relato tiene un punto.}
En el posgrado profundicé en los temas que me interesaban, y quiero mencionar que mi interés en estos temas nació durante mi licenciatura.

Durante la Licenciatura tuve profesores muy malos y un puñado de muy buenos profesores.
Pero mi interés en filosofía de la ciencia se debe a dos de ellos; ambos son excelentes maestros y excelentes filósofos.
En las clases que tomé con ellos vimos temas que despertaron mi curiosidad.
Todos los temas fueron sumamente interesantes, pero mi atención la dirigí especialmente a los temas de la \emph{explicación científica} y la \emph{naturaleza de la causalidad}.
Mis intuiciones --todavía queda un resabio de esto-- señalaban que encontrar la causa de un fenómeno es la mejor manera de explicar por qué sucede ese fenómeno.
Pero al finalizar mi licenciaturas tuve teniendo severos problemas con la filosofía.
Sentí que lo que había aprendido era fútil y creo que esto se debió a que los profesores malos fueron más que los buenos.

Tenía entonces esta \say{crisis existencial filosófica} y el hecho de que los dos profesores buenos trabajaran filosofía de la ciencia --al menos lo que en ese momento creía que era la filosofía de la ciencia-- sesgó mis intereses a lo que hasta ahora ha sido mi trabajo.
Esta influencia me hizo tomar la decisión de entrar al posgrado en filosofía de la ciencia, por supuesto, empezando con la maestría.

En las clases de maestría hubo profesores buenos, pero también los hubo excelentes.
Mucho de lo que aprendí en estos cursos me sirvió para enmarcar de diferente manera las preguntas que me preocupaban, lo cuál me permitió entender --mejor, me parece-- la naturaleza de la investigación científica.
En particular quería responder preguntas sobre la epistemología de la ciencia: \say{¿qué estamos justificados a creer?}, \say{¿cuándo podemos afirmar que una hipótesis ha sido corroborada?}, \say{¿es la explicación o compresión más básica que el conocimiento?}, etc.
Preguntas que surgen, por supuesto, a partir de mi interés por la explicación causal.
En la maestría aprendí nuevas metodologías de investigación y maneras diferentes de plantearme y entender las preguntas que me preocupaban.

Pero sin lugar a dudas, lo más valioso que aprendí, fue la importancia que tiene la historia de la ciencia en la filosofía de la ciencia.
Me voy a permitir hacer una breve caracterización, exageradamente general, de dos posturas en historia de la ciencia.


\section{Un breve repaso de dos posturas}

\noindent Este es un breve y burdo repaso de las \say{metodologías} \emph{externalistas} e \emph{internalistas}.
Las llamo \say{metodologías} porque difieren en términos sustanciales --por ejemplo, el de \emph{verdad}-- y es a partir de estas diferencias que ambas metodologías ofrecen una interpretación distinta de la historia de la ciencia.
Al usar el término \emph{metodología}, no quiero comprometerme con si son teorías en contradicción, o teorías complemetarias; esto es algo que no puedo discutir aquí y el nombre \say{metodologías}, me parece, es lo suficientemente neutro.

%La sección \ref{sbc:yturbe} la dedico a analizar una supuesta alternativa para el debate anterior.
%Además, esta alternativa, es más cercana a la filosofía de la ciencia.

\subsection{Dos metodologías: \emph{externalismo} e \emph{internalismo}}

\noindent Digamos que en historia de la ciencia se suelen distinguir dos corrientes: los \emph{internalistas} y los \emph{externalistas}.
De manera procaz, los internalistas argumentan que la historia de la ciencia debería interpretarse como un progreso de ideas; cada uno de los nuevos descubrimientos en la ciencia están motivados por intenciones internas a los investigadores: la búsqueda de la verdad, curiosidad natural, el gusto por explicar fenómenos, etc.

Con mejores métodos de investigación, los investigadores ofrecerán mejores respuestas a sus preguntas.
Tendrán, por decirlo de alguna manera, respuestas más detalladas del fenómeno que les interesa y descubrirán nuevos fenómenos que no contemplaban originalmente.
Todo este proceso guiado siempre por sus intereses.
Al ser guiados por sus intereses, y sólo por sus intereses, los investigadores son inmunes a los fenómenos culturales que les rodean.
Dicho de otra manera: que la ciencia es inmune a la cultura.
Creo que podemos identificar a Weinberg\footnote{
	Esto es tramposo porque no sé si los hitoriadores de la ciencia estiman a Weinberg como uno de sus colegas.
	Aparece en este escrito para fines puramente explicativos.} como un internalista, cuando dice que

\begin{quote}
	The word \say{discovery} in the subtitle is also problematic.
	I had thought of using The Invention of Modern Science as a subtitle.
	After all, science could hardly exist without human beings to practice it.
	I chose \say{Discovery} instead of \say{Invention} to suggest that science is the way it is not so much because of various adventitious historic acts of invention, but \emph{because of the way nature is}. \parencite[Prefacio, énfasis agregado]{Weinberg2015}
\end{quote}

Quiero destacar el énfasis que hice en la cita anterior.
Creo que no hay mucho peligro en asumir que la investigación es la mejor manera que tenemos de producir conocimiento; y en particular, creo que parte de este conocimiento depende de \say{la forma en la que la naturaleza es.}
Creo, además, que la investigación tiene como meta ofrecer explicaciones \emph{verdaderas} de los fenómenos; puedo decir entonces que me alíneo con Weinberg en un sentido restringido de \say{descubrimiento}\footnote{
	El sentido restringido del que hablo es algo que quiero discutir a lo largo de este capítulo introductorio. Al menos es parte de lo que voy a defender en este documento.
	El lector tendrá que leer el trabajo completo y regresar a esto cuando termine.}.
Para regresar a la caracterización del \emph{internalismo}: el internalista estima el papel que la \emph{verdad} juega en la ciencia; la ciencia \emph{descubre} fenómenos, no los \emph{inventa}.

Por supuesto, decirlo de esta manera esconde varios problemas, por ejemplo, esconde que \say{ciencia} no es un término que esté libre de debate; debate que una vez que entramos en los detalles se torna bastante complejo.
Hay por lo menos dos maneras en las que el término es problemático.

En primer lugar, en filosofía de la ciencia, los filósofos debaten si es posible distinguir entre ciencia $/$ no-ciencia.\footnote{El debate es más sofisticado y no podría explicarlo con justicia en este documento.
	El lector interesado puede revisar la entrada \parencite{sep-pseudo-science}.}.
En segundo lugar, los historiadiores de la ciencia debaten si es posible definir en qué periodo comienza la \emph{ciencia} tal como la conocemos en tiempos contemporáneos.

Y si adoptamos una metodología externalista, estas discusiones se vuelven cada vez más claras; y para lograr esta claridad debemos resaltar los \say{actos históricos fortuitos} de los que habla Weinberg; todo ello con la finalidad de elucidar cómo proceden las personas cuando investigan, qué influencias externas afectan el proceso de investigación; y que la historia debe jugar un papel central, si es que queremos responder estas preguntas.

Un externalista explicará los procesos de investigación científica a partir de las condiciones históricas en las que se desarrollaron; periodos de tiempo durante los cuales se desenvolvieron los investigadores y sus investigaciones.
El externalista resalta que la investigación científica no es una actividad que pueda separarse de su contexto \say{cultural}.

Quiero hacer notar que introduje deliberadamente el término \say{verdad.}
La naturaleza de la verdad es una discusión que requiere más detalle; pero este término es central para la discusión que seguiré a lo largo del documento.
Por ello, quiero rápidamente señalar cómo la \say{verdad} difiere en las dos metoddologías que he descrito.
Sobre la \say{verdad,} Shapin se expresa claramente cuando afirma que \say{All claims have to win credibility, and credibility is the outcome of contingent social and cultural practice.} \parencite[Capítulo 2]{shapin2010never}

Adoptando, digamos, una metodología externalista, Shapin se ha dedicado a estudiar detalladamente como los fenómenos sociales irrigan conceptos tan centrales en la investigación científica como es el de \emph{verdad}; y para aclarar el punto, la siguiente cita es ilustrativa

\begin{quote}
	The notion of truthfulness was thus central to the description of gentle qualities.
	Through the Renaissance and into the eighteenth century an honorable man and an honest man were interchangeable designations: \say{honesty} included the notion of truthtelling but was understood far more broadly to include concepts of probity, uprightness, fairdealing, and respectability. \parencite[pp. 70-71]{Shapin1995}
\end{quote}

Es decir que la verdad estaba asociada no sólo a ser honesto, sino con la figura del hombre honorable; la nobleza y la riqueza suelen estar asociadas al hombre honorable, por tanto su \say{palabra} valía más que las de otros.

Si lo que he descrito hasta ahora es ligeramente correcto y asumimos que Weinberg es un \emph{internalista} y que Shapin es un \emph{externalista}, entonces los párrafos de arriba son suficientes para saber cómo la \emph{verdad} juega un papel en ambas metodologías.
Para el internalista el objetivo de la ciencia es la verdad.
Esta verdad, por supuesto, debería estar acorde con la forma en la que la naturaleza es.
Por otro lado, el externalista nos dice que la verdad siempre está atravesada por un contexto histórico y que no siempre depende del modo en que la naturaleza es, sino también por consideraciones sociales como la honorabilidad.

Con esta breve caracterización, parece que me he desviado del tema.
Pero este papel que juega la verdad en ambas metodologías es algo que los filósofos han discutido ampliamente.
Además, esta exposición es importante para lo que quiero discutir a continuación: que hay que considerar seriamente la distinción entre el \emph{contexto de descubrimiento} y el \emph{contexto de justificación.}

Para este propósito, quiero discutir un artículo que presenta un caso más cercano a la filosofía de la ciencia.
Un artículo que relaciona por un lado a las metodologías que he discutido hasta aquí, y por otro cómo éstas juegan un papel en la filosofía de la ciencia.
Me dedico a este caso a continuación.

\subsection{Un caso más cercano a la filosofía de la ciencia}\label{sbc:yturbe}

\noindent En el artículo \citetitle{Yturbe1995}, la autora argumenta que es un falso dilema tener que elegir entre las metodologías internalista y externalista.
Yturbe dice, por ejemplo que

\begin{quote}
	The general tendency in the historiography of the sciences is to consider the social function of science, as weIl as some aspects of the matrix from which the problematic is formed, as external elements, while the conceptual apparatus and the problem field are treated as internal. \emph{But we should not think of science as having two independent histories.} \parencite[p.85, Énfasis agregado]{Yturbe1995}
\end{quote}

La oración en itálicas implica que no deberíamos hacer una distinción entre factores internos y factores externos.
Además, Corina defiende que una teoría a la vez internalista como externalista no es una teoría plausible.
Al menos esto parece sugerir cuando dice que

\begin{quote}
	The search for new explanatory programs is characterized above all by the attempt to reconcile the internalist and externalist approaches.
	But, in view of the fact that both approaches are committed to theses concerning the nature of science that are not only incompatible, but are unsustainable, their union without important changes in their philosophical presuppositions cannot result in a viable program. \parencite[p. 79]{Yturbe1995}
\end{quote}

Su conclusión se basa en que no es posible marginar ambos factores; además, la autora sugiere que esta supuesta distinción tiene sus orígenes en la distinción entre el \emph{contexto de descubrimiento} y el \emph{contexto de justificación} que trazaron los positivistas.
La autora señala, por ejemplo que

\begin{quote}
	External factors are not only found in the context of discovery, they are present also in the development of the concepts, problems, methods, problem fields, etc. of scientific discourses: that is, external factors are present in the context of validation itself.
	There are scientific discourses in which ideological conceptions pass on to form part of the body of the science itself, functioning as principles which define its field of study or guide its research; thus, external factors can become internal. \parencite[p. 85]{Yturbe1995}
\end{quote}

La cita anterior sugiere que el \emph{contexto de justificación} involucra sólo \emph{factores internos,} mientras que el \emph{contexto de descubrimiento} involucra sólo \emph{factores externos.}
La autora nos dice que los factores externos contagian al contexto de justificación, cuando dichos factores se convierten en parte de las teorías --generando conceptos, principios centrales, etc.--
La autora dice que esta ceguera a considerar que los \emph{factores externos} afectan al \emph{contexto de justificación,} surge de la \say{doctrina de los dos contextos.}
Corina nos dice que \say{One of the philosophical theses in favor of the contraposition between the internalist approach and the externalist approach is the so-called doctrine of the two contexts, developed by positivism.} \parencite[p. 75]{Yturbe1995}

Creo que la premisa, en la que la autora nos dice que el contexto de justificación/descubrimiento es equivalente a los factores internos/externos, es falsa y que la equivalencia descansa en una confusión.
Esta confusión se debe, en parte, a la mala caracterización de las tesis del \say{positivismo}; la otra parte, me parece, se debe a una interpretación poco caritativa de la historia de la ciencia y los propósitos de los investigadores.

El argumento de la mala caracterización del positivismo lo doy en este mismo capítulo.
El argumento sobre la interpretación y los propósitos de los investigadores será dado a lo largo del capítulo 2.
Pero, este argumento comienza en este capítulo, en especial la seccion \ref{ssc:aprender}.

Recordemos que el argumento que voy a ofrecer es para concluir que la autora ofrece una mala caracterización del positivismo.
Esta conclusión depende de que el \emph{contexto de descubrimiento} y el \emph{contexto de justificación} no son equivalentes a los \emph{factores externos} y \emph{factores internos} respectivamente; lo que Yturbe llama \say{la doctrina de los dos contextos}.
Para este propósito quiero repasar brevemente en qué consisten el \emph{contexto de descubrimiento} y el \emph{contexto de justificación.}



\newthought{La distinción} entre el \emph{contexto de descubrimiento} y el \emph{contexto de justificación}, la asociamos a Reichenbach\footnote{Stillwell comenta que la distinción puede trazarse hasta Arquímedes.
	Al menos eso entiendo cuando afirma que \say{Archimedes was probably the first mathematician candid enough to explain that there is a difference between the way theorems are discovered and the way they are proved.} \parencite[p. 56]{stillwell1989mathematics}}.
En su libro \citetitle{reichenbach1938experience}, el autor usa esta distinción para \say{excluir} los así llamados factores externos (sean sociales, políticos o económicos)\footnote{
	El pasaje no dice exactamente esto, sino algo más cercano a que tenemos claro cómo hacer un análisis filosófico$/$formal para la justificación de teorías, mientras que no tenemos manera de hacer ese análisis en el contexto de descubrimiento.
	Este lado de la investigación científica, digamos, es muy heterogéneo como para ofrecer un análisis formal.
	}.
Esta distinción, nos dice Reichenbach, sirve para ilustrar el hecho de que no tenemos a la mano un \say{método} para analizar el fenómeno del descubrimiento científico; lo que algunos filósofos \parencite{reichenbach1938experience, Seo2015} llaman el momento \emph{Eureka}.
Pero no tener una herramienta de análisis formal, no implica que lo que sucede en el contexto de descubrimiento sea absolutamente deleznable.

Que los factores externos pueden afectar el contexto de justificación, fue un tema que se discutió ampliamente durante los años del círculo de Viena\footnote{
	O positivismo lógico o empirismo lógico, cada quien elige su etiqueta favorita.
} y no es una distinción que permaneció fija a lo largo de lo que, asumo, Yturbe llama \say{positivismo}.

Como me parece que la distinción juega un papel importante en cómo interpretamos filosóficamente la historia de la ciencia y me parece que juega un papel clave en cómo los investigadores realizan investigación; entonces creo que es una distinción que vale la pena retomar.
Para lograr este objetivo, quiero repasar brevemente cómo algunos de los miembros del círculo de Viena lidiaron con la distinción.


\subsection{El círculo de Viena}

\noindent Quiero comenzar señalando que los positivistas\footnote{
	Me voy a referir con este término a los filósofos que aparecen en la publicación de la \emph{International Encyclopedia of Unified Science} \parencite{Carnap1938-CARFOL-10}.
	Tanto del comité organizador, como el comité asesor.
	}, 
estaban al tanto de cómo los \emph{factores externos} influyen en el \emph{contexto de justificación}.
Dicho de otra manera, que los aspectos sociales, políticos y económicos afectan las prácticas de las personas dedicadas a hacer investigación.

La afirmación de Reichernbach sobre la carencia de un análisis formal del \emph{contexto de justificación} es algo que los miembros del círculo tenían en su agenda de investigación; 
y lograron dicho análisis con mayor o menor éxito.

\newthought{Neurath}, nos recuerda Joseph Bentley, sostenía que

\begin{quote}
	Despite his advocacy for scientific methods, Neurath never takes this method to be set in stone, nor does he attempt to portray science as an enterprise of purely objective methods, completely divorced from social, historical, and material contexts or the personalities of scientific practitioners. \parencite[p.41]{Bentley2023}
\end{quote}

Neurath sabía que los fenómenos sociales y políticos afectan las prácticas científicas; sin embargo, sostenía también, que estudiar esta relación debe llevarse a cabo con otro tipo de herramientas.
Bentley lo expresa mucho mejor cuando señala que \say{As in the case of theory-choice, decision-making is central. But to make metatheoretical decisions, metatheoretical information is needed.} \parencite[p. 62]{Bentley2023}

Neurath llama \say{beahaviouristics of scholars} al conjunto de herramientas y métodos para reunir esta información metateórica; 
la cuál sería la disciplina con las herramientas para analizar aspectos históricos, políticos, económicos, etc;
que es a lo que estamos llamando factores externos.

Y al hablar específicamente de la justificación de creencias, Bentley señala que \say{[science], Neurath maintains, it is still the best we have} \parencite[p. ~41]{Bentley2023}.
Y si somos capaces de reconocer cómo las prácticas juegan un papel, podemos evaluar y mejorar dichas prácticas.

Mejorar las herramientas que usamos en investigación, no es una tarea fácil. 
Porque modificar una herramienta tiene como consecuencia analizar detalladamente los resultados que dependen de dicha herramienta.

Para mejorar nuestras prácticas, no debemos olvidar que Neurath fue un \say{holista} confirmacional; esto significa que Neurath creía que no podemos confirmar o falsificar una oración aislada, sino que todo el conocimiento es juzgado a la vez; cada vez que modifiquemos una hipótesis o alguna de nuestras herramientas, hay que hacer cambios en otras hipótesis y herramientas.\footnote{
	Tanto fue su interés por sistematizar la comunicación entre diferentes comunidades científicas que diseñó el proyecto de la Enciclopedia de la Ciencia Unificada.
	De esta manera el trabajo se distribuye entre distintas comunidades y se coteja con el trabajo de otras comunidades de investigación.
	}

Bentley expresa mejor el objetivo de Neurath diciendo \say{if we recognize the ... creation of the norms, methods and values of science, it can be made better.} \parencite[p. 41]{Bentley2023}

Quiero enfatizar que Neurath no fue el único miembro de los positivistas lógicos que reconoció el papel que juegan las personas que realizan investigación; 
personas que tienen ciertos sesgos políticos y sociales, que tienen que tomar decisiones y que tratan de justificar sus hipótesis cion las herramientas a la mano.
Esto es más claro cuando lo expone Philipp Frank.



\newthought{Philipp Frank} expresa una actitud semejante a la de Neurath.
Ambos filósofos sostenían que los aspectos \say{externos} eran una parte crucial del análisis que los filósofos pueden ofrecer de las ciencias.
En su artículo \citetile{Frank1956}, frank argumenta que los aspectos sociales jamás han estado excluidos del proceso de selección de teorías.
Esto es más o menos claro cuando Frank nos dice que \say{The special mechanism by which social powers bring about a tendency to accept or reject a certain theory depends upon the structure of the society within which the scientist operate.} \parencite[p. 143]{Frank1954}

Para su argumento, Frank comienza enfatizando que a lo largo de la historia, la elección de teorías no es arbitraria;
Supongamos que tenemos dos teorías en competencia $a$ y $b$.
Los investigadores no simplemente deciden entre $a$ o $b$ sopesando cuál de las dos tiene más consecuencias empíricas.

Frank nos recuerda que tomar una decisión entre $a$ y $b$ involucra diferentes aspectos políticos y sociales.
Estos aspectos son parte del proceso de justificación de una teoría, porque no sólo revisamos sus consecuencias empíricas, sino qué tan coherente es con otros dominios (digamos biología, química, etc.)
Estos aspectos están involucrados porque la investigación es un producto realizado por seres humanos;
y como seres humanos, tenemos capacidades cognitivas limitadas \parencite{Potochnik2017-POTIAT-3}, porque tenemos una gran cantidad de sesgos implícitos \parencite{nordell2021end}, etc.
Si para el proceso de elección de teorías sólo evaluamos a la teoría al medir qué tanto está \emph{de acuerrdo con los hechos}, nunca podríamos decir si una teoría es \emph{mejor que} la otra.
Donde estamos midiendo \say{mejor que} en términos de la \emph{utilidad} de la teoría en función de los propósitos que queremos lograr;
y si la teoría va a ser \emph{usada}, entonces debe ser \emph{simple.}
Los diferentes usos que se le pueden dar a una teoría se miden por la utilidad de la teoría cuando es usada para fines \emph{prácticos.}

Frank lo expresa mejor \say{However, the situation becomes much more complex, if we mean by simplicity not only simplicity of the mathematical scheme but also simplicity of the whole discourse by which the theory is formulated.} \parencite[p.~4]{Frank1956}
Más adelante afirma que \say{The final theory has to be in fair agreement with observations and also has to be sufficiently simple to be usable.} \parentcite[p.~14]{Frank1956}

Es en este sentido en el cuál los factores externos son parte del contexto de justificación.
Aceptar o rechazar una teoría es un proceso que involucra \emph{justificar} las teorías y decidir cuál es \emph{mejor} para los propósitos que deseamos.
Es durante la fase de aceptación de la teoría, cuando los factores externos toman un papel central, porque de no ser por la finalidad práctica de una teoría, no podríamos decidir qué teoría es más adecuada; 
haciendo énfasis en que estos factores sociales jamás han desaparecido del \emph{contexto de justificación}.
En particular, ante dos teeorías en competencia, la decisión no puede tomarse sin este tipo de consideraciones prácticas.
Consideraciones prácticas que dependen del contexto histórico y sociocultural\footnote{En algunas ocasiones la utilidad de la teoría es funcionar como propaganda, algo que señalan los autores de \parencite{Lewontin2017}  y también Frank en \parencite{}.}.

Hasta este punto, me parece que la evidencia textual muestra suficiente información para concluir que hay una confusión en la caracterización de la llamada \emph{doctrina de los dos contextos} \parencite{Yturbe1995}.
Se suele caracterizar a los miembros del círculo de Viena defendiendo tesis exageradamente estrictas, Philipp Frank incluso lo menciona en \say{cita de Frank} y más recientemente, Bentley también cuando dice que \say{citga de bentley}.
Y esta no es la única mención a esto \say{cita del número especial}

Suele mencionarse también que los miembros del círculo no prestaron atención a ciertos temas por estar demasiado ocupados con la física [citar al de Sahorta] o que el problema de la reducción de teorías incluía el proyecto de reducción de las ciencias especiales a la física [citar a Raphael van Riel].
Afortunadamente, existen intentos recientes por ofrecer una caracterización justa de las afirmaciones de los miémbros del \emph{círculo de Viena.}

Quiero señalar que esto puede ser una preocupación más filosófica que histórica, y por lo tanto, parecer que como Yturbe \cite{Yturbe1995} estoy confundiendo una tesis filosófica y una tesis sobre historiografía.
No soy historiador, por lo que no puedo más que referirme a dos autores en particular que, me parece, vinculan las preocupaciones de los miembros del círculo con la historiografía de la ciencia.
A esto me deddico en la siguiente sección.


\subsection{Justificar una teoría es el objetivo principal de la investigación}

Llegado a este punto, he señalado que la caracterización que hace la autora de la \emph{doctrina de los dos contextos} es incorrecta; 
más aún, prometí que valía la pena recuperar esta distinción porque me parece que el objetivo principal de la investigación 
científica es el \emph{proceso} de \emph{justificar} teorías.
Quiero llegar a esta conclusión a partir de lo que dije anteriormente: como la caracterización de Yturbe es errónea, tenemos una alternativa; 
nuestra alternativa es que hay una manera de separar el \emph{contexto de justificación} del \emph{contexto de descubrimiento}.




\newthought{Justificar} una teoría, nos dicen ambos filósofos, involucra necesariamente aspectos pragmáticos.
La práctica científica no está separada de su contexto cultural.
Pero cuando lidiamos con procesos de justificación, lo mejor que tenemos son las herramientas que usamos en investigación.
Herramientas que pueden cambiar con el tiempo y que necesitarán una justificación más robusta.
Todo esto con el objetivo de señalar si una oración es verdadera o falsa.
Reconociendo que la verdad y la falsedad de las oraciones no está separada de otras oraciones.

Estos comentarios sobre el \say{materialismo dialéctico} y el \say{pragmatismo americano},
% Mencionar la mención al operacionalismo 
, pueden sonar fuera de lugar.
Pero es importante mencionar esto porque las fuentes históricas que voy a citar son, una de ellas soviéticas, usando la llamada metodología del \say{materialismo histórico}, mientras que la otra fuente ofrece una interpretación operacionalista de la matemática griega.

El pragmatismo americano se vincula al operacionalismo

Pero

Cómo juega el verificacionismo un papel?, esto es, por qué la manera de verificar si una oración es verdadera o falsa depende de cómo se verifica?
Lo único que nos piden los positivistas es que tengamos cuidado al juzgar si la oración es verdadera.
Al juzgar una oración, debemos tener en cuenta qué operaciones serían necesarias para saber si la oración es verdadera.
A los términos singulares de la oración, se les da una definición \emph{operacional}, donde por operacional, los positivistas se refieren a qué experiencias físicas serían necesarias para decir que una oración es verdadera.

Some has to do with it's sustainability.
Sólo para hacer la distinción entre el contexto de descubrimiento y el contexto de justificación.
Pero no hay que confundir a la operación con la cosa.
Hay que distinguir claramente entre la decisión de cómo representar un fenómeno y entre el fenómeno mismo.

Sobre la representación

pues introduciendo
Digamos que el operacion

Frank traza también un puente entre el pragmatismo, el materialismo histórico soviético y la filosofía del positivismo.


\section{citas}

This means there is no possibilityt of isolating a class of priviledged sentences, to act as a fixed foundation Naturalism and the Vienna Circle

Every term  introduced into the theory must be accompanied by  a description of the physical operations by which may  be tested the degree to which the property expressed  by this term may be attributed to a given physical  system. The description of these operations, the  "operational definition" of this term, may be more or  less direct; perhaps only a combination of terms will  correspond to a certain operation. Professor Bridgman's views have frequently been labelled "operationalism," although he himself is not pleased by this name.

But the outlook of the so-called "Vienna Circle"  has been only one particularly coherent doctrine  among the many intellectual fruits which have  emerged from the soil of Central-European positivism.

In particular, Charles  W . Morris of Chicago recognized the connection  with American pragmatism and publicized the idea  of cooperation between the two groups. It was decided, for the purpose of this cooperation, to call a  special congress, for which the name "Congress for  the Unity of Science" was coined by Otto Neurath.

The conception of the relative worthlessness of the theory in  comparison to the phenomenon gives to the theorizing  of such an investigator something especially free and  imaginative.

The known connections among  phenomena form a network; the theory seeks to pass  a continuous surface through the knots and threads  of the net. Naturally, the smaller the meshes, the  more closely is the surface fixed by the net. Hence,  as our experience progresses the surface is permitted  less and less play, without ever being unequivocally  determined by the net.

nce this possibility has been substantiated,  the whole of analysis can proceed to develop as usual.  But now when a theorem about derivatives is set up  and somebody begins to subtilize about it, asking  whether this theorem is really in agreement with the  "nature" of the differential and going into profound  and skeptical deliberations concerning this "nature,"  he can be told quite simply: "I could express this  theorem, if I took enough time, as a theorem about  integers; the nature of this theorem is hence no more  and no less mysterious than that of the natural  numbers."

The  atoms are auxiliary conceptions just like others which  can be employed advantageously in a limited domain.  They are not suitable for an epistemological foundation.

Más aún, Frank veía en la práctica científica una manera de resolver problemas como el de la representación y el de la elección, digamos \say{temporal}, de una teoría.
Esto queda bastante claro cuando Frank afirma

\begin{quote}
	If we look for an answer to the question of wether a certain theory, say the copernican system or the theory of relativity, is preferred or true, we have to ask the preliminary question: what purpose is the theory to serve? \parencite[p. 15]{Frank1954}
\end{quote}

Ambos filósofos estaban al tanto de que los factores externos afectan el contexto de justificación.
Ambos fueron miembros del círculo de Viena y si ambos sostenían que hay una distinción entre el \emph{contexto de descubrimiento} y \emph{el contexto de justificación}; entonces no son equivalentes el \emph{contexto de descubrimiento} y los \emph{factores externos}; y  tampoco son equivalentes el \emph{contexto de justificación} y \emph{los factores internos}.

Me parece que con señalar que los miembros del círculo de Viena que he discutido hasta ahora, no se alineaban con las tesis descritas por \parencite{Yturbe1995}, es suficiente para decir que hay una confusión en la premisa que usa la autora.
Sin embargo, quiero decir un poco más sobre las tesis a las que se adherían Neurath y Frank; además quiero señalar la relación que ve Frank entre la postura del \say{empirismo lógico} y la \say{filosofía de la Unión Soviética}, que es algo que quiero discutir en la siguiente sección.

\newthought{Tanto Neurath como Frank} sostienen una forma de verificacionismo: que el significado de de una oración depende del método de comprobación.
Pero, ambos están de acuerdo en que nunca tenemos una imagen perfecta, con la que podamos saber con completa seguridad, que una oración es verdadera o falsa.



%\section{Citas Amsterdamski}

%The sociology of science studies the evolution of science as a  social institution, the styles of scientific thinking, the reception of scientific  ideas and their social determination. Finally, logic and the methodology  of science (often called the philosophy of science), as a branch of epistemology, study the structure of scientific theories, their development, the  rules of concept and theory formation, the criteria of accepting and  refuting the claims of science, and the relations between theory and  empirical data.

%Anyone who has even a slight interest in the history of science must have  found himself faced by many bewildering problems. How was it possible,  for instance, that the most remarkable minds of their times held opinions  that could today be disproved by any high school student, often on the  basis of empirical facts known even then? It is just this last circumstance  which appears to be most striking, for it is one thing to accept theories  which later may be proved imprecise or simply false, but are not contradicted by known facts, and quite a different matter to maintain opinions regardless of known facts which clearly contradict them. If the first  instance seems quite natural to us, as we believe that it is the discovery of  new facts which usually compel us to revise old theories, then the second  instance appears to be perplexing; in fact, it seems to contradict the very  nature of scientific knowledge.

%the paintings of the Aztecs were more valuable than those of the European  Renaissance, the philosopher or historian of science sees nothing wrong  with comparing the theories of Ptolemy and Copernicus, Newton and  Einstein, or Darwin and Mendel. He is convinced that these theories had  to provide a coherent explanation of the same domains of phenomena (the  movement of celestial bodies, mechanics, the mechanism of heredity),  and that in attempting to formulate explanations they utilized the same  criteria for evaluating the results of their work.

%Although we are prone to evaluate works of art in a relativistic manner,  taking into account the aesthetic criteria and historical circumstances  which are particular to a given period, we tend to evaluate the achievements of scientists (just their achievements, not their merits) on the grounds  of some supra-historical criteria. More precisely, we tend to evaluate  scientific results on the grounds of those criteria which are accepted by  contemporary science and which we take (for better or for worse - we  will come back to this problem) as supra-historical.

%It was not only the Copernican theory that for years had to deal with a  mass of anomalies. The same thing happened with Newton's theory, with  quantum mechanics, and in general with any new theory which required a  radical reconstruction of large areas of acquired knowledge. This is the  fate of any new theory which attempts to introduce a new order into the  domain of phenomena which it describes and which were ordered in a  different way by its predecessors. Since every such theory must take into  account experiments and observations previously carried out, it must  therefore reinterpret them, make them fit into the new theoretical framework. Sometimes this reinterpretation is extremely complex and hardly  discernible to anyone but the specialist, especially in cases when the meaning of those basic concepts which became a part of everyday language  undergo a radical change. Thus after the rise of the theory of relativity, the  meaning of such notions as 'mass', 'velocity' or 'simultaneity' had  changed, and this fact should be taken in account when we investigate its  relation to Newtonian mechanics.

%such as, for instance, the contemporary state of mathematical knowledge,  the theoretical situation in other disciplines, the state of the technical  apparatus available or the accepted epistemological and ontological  beliefs, etc. To quote F. Jacob, there is" ... a domain which thought strives  to explore, where it seeks to establish order and attempts to construct a  world of abstract relationships in harmony not only with observations and  techniques but also with current practices, values and interpretations."15

%It is worth remembering that the empirical confirmation of the Copernican theory, which at the same time disproved the Ptolemaic system,  was only provided many years after the death of Copernicus. Copernicus  himself could offer no decisive fact which would confirm his own theory.  "No fundamental astronomical discovery, no new sort of astronomical  observation, persuaded Copernicus of ancient astronomy's inadequacy or  of the necessity for change."!7 The prognoses of celestial phenomena  provided by either theory proved to be essentially the same.

%The content of Newtonian mechanics not only stepped outside the  boundaries of empirical data and, therefore, could not be deduced from  them, but also was incompatible with observational data which were  known at the time when it was formulated - such as the impossibility of  distinguishing between absolute and relative motion.

%In any case, neither statements of scientists nor  reasonings of philosophers provide sufficient confirmation of the thesis,  that in posing experimental questions about nature, and reading the  answers by means of measuring instruments, scientists were free from  hypotheses and presuppositions and that in declaring the glory of radical  empiricism they were not subject to any philosophy or metaphysics. We  would indeed be quite naive if we took their words in this matter.


%Secondly, what is the true relationship between the empirical basis of  science (i.e., facts), and theory, and what changes of this basis can provide  a satisfactory foundation for the understanding of the process of the  evolution of scientific cognition?



%Los factores sociales y políticos son medios para sleccionar una teoría.
%De acuerdo a los objetivos que buscamos en la investigación.
%Y estos objetivos son influencias que imbuyen el contexto de justificación.
%
%Incluso Carnap, que es al que más le llueven vergazos y del que se sotiene tenía las ideas máws radicales deel círculo, llegó a aceptar que las teorías tienen sin duda un aspecto pragmático. 
%Neurath y Philipp Frank exponen este tipo de preocupaciones.


%Neurath fue falibilista, coherentista (sobre la justificación) y holista.
%Por supuesto que sostenía una forma de \emph{verificacionismo}, Bentley lo expresa más claramente \say{Theories must be sensitive to, grounded in and testible by experience} \parencite[p. 38][]{Bentley2023}.
%Este criterio es bastante vago, pero Bentley lo describe no como una tesis, sino como una postura; una postura que es difícil capturar con una sola afirmación \parencite[p.38][Cfr.]{Bentley2023} 


%\subsection{Citas Phillip Frank}

%The misinterpretation of scientific principles, as will be shown, can be avoided if, in every statement found in books on physics or chemistry, one is careful to distinguish an experimentally testable assertion  about observable facts from a proposal to represent the facts in a certain way by word or diagram.
%If this distinction is sharply drawn, there will no longer be any room for the interpretation of physics in favor of a spiritualistic or a materialistic metaphysics. [p. 4-5]

%The views represented in the present book are closely associated with the movement now generally  called "logical empiricism" or "logical positivism."
%I must confess that I do not like these words either.
%But a long life among views and theories has shown me that if we want a view to be regarded as a respectable tree in the garden of opinions it must have a label just as much as the elms and oaks in our public gardens. [p. 5]

%* recordar que Philipp Frank fue fundador del grupo original, que 20 años después se convirtió en el círculo de Vienna.

%* Cualquier reconstrucción se le va a quedar corta a este libro.

%But the outlook of the so-called "Vienna Circle"  has been only one particularly coherent doctrine  among the many intellectual fruits which have  emerged from the soil of Central-European positivism. Mathematicians such as R. von Mises, Κ. Menger, and Κ. Godei, physicists such as E. Schroedinger,  economists such as J. Schumpeter, lawyers such as  H. Kelsen, and sociologists such as E. Zilsel had their  roots in this environment. The whole intellectual  background of this general movement can be understood best through R. von Mises' textbook of positivism. [p. 9]

%Several young American philosophers  traveled to Vienna and Prague in order to come into  scientific contact with Schlick and Carnap. Among  them were W . V . Quine (now at Harvard) and  E. Nagel (now at Columbia). In particular, Charles  W . Morris of Chicago recognized the connection  with American pragmatism and publicized the idea  of cooperation between the two groups. []

%But what does it mean when we say that a question  is insoluble? Let us suppose, for example, that someone has asserted that the problem of a regular airplane  route to the planet Neptune is insoluble, or that the  production of a living organism from lifeless matter  is insoluble. 
%Despite this assertion, the person making  it can describe quite accurately the concrete experience we should have if the problem were solved.

\section{¿Qué podemos aprender de la historia de la ciencia?}\label{sub:aprender}


\noindent Quiero comenzar esta sección confesando lo siguiente: estoy de acuerdo con la autora, en particular con el carácter de la conclusión anterior, esto es, que es complicado--si no es que imposible--, separar entre los procesos externos y los procesos internos que influyen en la investigación científica.
Pero aceptar esto --eso quiero argumentar-- no implica que debamos deshacernos de la distinción entre el \emph{contexto de investigación} y el \emph{contexto de descubrimiento}.
En lo que resta de la sección quiero hacer un breve repaso de los argumentos de la autora.

Yturbe nos dice que el contexto de descubrimiento, se dedica a analizar factores externos que fueron influyendo en los factores internos; mientras que el \emph{contexto de justificación} sólo se dedica a analizar los factores internos al desarrollo teórico.
Lo que ella llama \say{the doctrine of the two contexts}\footnote{
	\say{one of the philosophical theses in favor of the contraposition between the internalist approach and the externalist approach is the so-called doctrine of the two contexts, developed by positivism.} \cite[p. 75]{Yturbe1995}
}
La autora nos dice que la doctrina de los dos contextos hace una distinción que no se puede hacer.
El contexto de justificación es siempre influenciado por el contexto de descubrimiento porque la ideología, la economía, las relaciones de poder, etc. influyen en la toma de decisiones\footnote{Por usar una caricatura: a quién se le asigna presupuesto}, por lo tanto la doctrina es incorrecta.

Como dije al principio de esta sección, estoy de acuerdo con que es casi imposible hacer la distinción entre \emph{historia interna} e \emph{historia externa}, pero aceptar esto, no implica deshacernos de la distinsión \emph{contexto de descubrimiento} y \emph{contexto de justificación}.
Para sustentar esto, dependo de la caracterización que hace la autora sobre \say{la doctrina de los dos contextos}, la caracterización de la doctrina está ligeramente sesgada, si no es que completamente inadecuada.
En la siguiente sección quiero ofrecer mis razones para esta conclusión.

\subsection{La doctrina}

\noindent En el texto de \textcite[p.75]{Yturbe1995}, la autora ofrece una descripción de la doctrina, ella señala que los filósofos doctrinarios afirman que \say{according to this position, [factores externos], not only fails to increase our understanding of scientific development, but even obscures the fundamental question of the rational validation of scientific arguments.}
Además que \say{This conception constitutes the dominant framework in which the philosophy of science has been developed, and consists in drawing a radlcal distinction between the context of discovery and the context of validation.}


Su argumento para decir que hay un colapso entre factores externos (contexto de descubrimiento) y factores internos (contexto de justificación) se encuentra en la página 85

\begin{quote}

	External factors are not only found in the context of discovery, they are present also in the development of the concepts, problems, methods, problem fields, etc. of scientific discourses: that is, external factors are present in the context of validation itself.
	There are scientific discourses in which ideological conceptions pass on to form part of the body of the science itself, functioning as principles which define its field of study or guide its research; thus, external factors can become internal.

\end{quote}

La doctrina de los dos contextos implica que podemos claramente separar entre el \emph{contexto de descubrimiento} y el \emph{contexto de justificación}.
Pero los factores externos están presentes en el contexto de justificación, tanto así que pueden convertirse en factores internos.
Esto implica que los contextos no son claramente separables, y que, por tanto, la doctrina es incorrecta.

Contrario a lo que dice la autora

\begin{quote}
	The search for new explanatory programs is characterized above all by the attempt to reconcile the internalist and externalist approaches. But, in view of the fact that both approaches are committed to theses concerning the nature of science that are not only incompatible, but are unsustainable, their union without important changes in their philosophical presuppositions cannot result in a viable program. \parencite[p. 79]{Yturbe1995}
\end{quote}

Creo que es verdad que no podemos distinguir entre factores externos y factores internos en la ciencia.
Sin embargo, me parece que lo que expresa la cita anterior es claramente erróneo: no se siguie la conclusión \say{que no pueden resultar en un programa viable}, porque depende de señalar que la distinción entre  \emph{contexto de descubrimiento} $|$ \emph{contexto de justificación} es equivalente a la distinción \emph{factores externos} $|$ \emph{factores internos} y estas distinciones no son equivalentes.



\subsection{En defensa de la doctrina}

\noindent En años recientes ha habido un creciente interés en discutir las publicaciones del Círculo de Vienna \parencite{Bentley2023, Richardson2023, Suarez2024, Riel2014}.
Si hago una tosca generalización, diría que literatura reciente se ha dedicado a señalar que los miembros del Círculo de Vienna, no eran tan herméticos como se pensaba.
Además, la cantidad de temas que discutieron es más vasta de lo que se creía.
Particularmente pensando en que cada uno de los miembros difería de los otros en tesis centrales.

\textcite{Suarez2024}, por ejemplo, discute que una preocupación genuina sobre los modelos en la ciencia, precede, data y prosigue a los años del Círculo de Vienna, el autor nos dice \say{I focus particularly on the nineteenth-century modelers and summarize their insights and contributions, which, I claim, remain essentially unsurpassed.} (p. 20)

Además, los miembros del Círculo de Vienna tuvieron una preocupación genuina por las prácticas científicas--Neurath, por ejemplo--
Tenían presente que ciertos factores externos pueden influenciar a la investigación científica.
\textcite[p. 24]{Bentley2023} lo expresa mejor: \say{Frank's historical work, which frequently anticipates Kuhnian or post-Kuhnian themes, consistently emphasizes the significance of non-scientific, external factors on the decision making of scientists.}
Los miembros del Círculo de Vienna tenían claro que los factores externos influyen en el proceso de la investigación científica, presentes incluso en el \emph{contexto de justificación}.

Hablando en particular del trabajo de Neurath y las tesis que sostenía, Otto defendía una teoría coherentista de la justificación.
Neurath argumentaba que nuestras teorías científicas están subdeterminadas empíricamente.
En cualquier momento del tiempo hay hipótesis en pugna.
Pero si las teorías están empíricamente subdeterminadas, es posible que dos teorías internamente coherentes y externamente inconsistentes entre sí, convivan al mismo tiempo.

Pero si el escenario anterior es plausible, entonces es patente que aquél que defienda una teoría coherentista de la confirmación, no puede resolver el problema de la elección racional de teorías.
Debido a que todo el tiempo hay teorías en pugna, como filósofos, deberíamos poder explicar por qué es racional elegir entre una teoría y sus rivales. Para poder resolver este problema, Neurath defendía que los factores externos son información clave para explicar la racionalidad de la elección de los investigadores.
Factores externos como los eventos históricos fortuitos.

\textcite{Bentley2023} nos recuerda que Neurath fue falibilista sobre el conocimiento, es decir,  sostenía que cada una de nuestras creencias puede estar equivocada.
Sumado a esto, sostenía que la confirmación es holista: no son oraciones particulares, sino grupos de oraciones, los que contrastamos con el mundo.
Bentley lo expresa mejor \say{$\ldots$ it is always possible for a system of beliefs to be altered to accommodate a statement or for the statement to be rejected $\ldots$} (p. 21).



Esto último inició como un abstract que mandé a un taller y se fue volviendo más largo mientras leía.

Hablando de modelos en la filosofía de la ciencia durante el periodo del Círculo de Vienna, una figura importante que mencionar es Mary Hesse.
Mary Hesse trabajó con especial atención el uso de modelos en la ciencia.

``This historical chapter introduces the emergence in the nineteenth century of  what I call the modeling attitude.
This is a stance toward scientific work and  discovery, and it continues to this day." (p. 43)






  % Chapter 0

\input{input_files/chapter_1}

% !TEX root = ../main.tex

\chapter{La naturaleza de la \emph{verdad}}
\label{ch:natruth}
 



\section{Naturaleza}

La naturaleza de la verdad es un tema que sigue siendo motivo de debate entre filósofos.
Hay al menos tres posturas clásicas sobre la verdad: la pragmática, correspondentista y coherentista.
Por supuesto hay más detalles y más teorías sobre la naturaleza de la verdad, pero, por motivos expositivos, concentrémonos en las teorías clásicas.
Además, quiero hacer claros un par de supuestos.
En primer lugar, el mundo y los fenómenos que queremos estudiar son complejos.
La segunda y última suposición, los seres humanos tenemos capacidades cognitivas limitadas, muchas de las veces en las que queremos estudiar un fenómeno, limitamos las diferentes variables que pueden afectar el fenómeno que queremos estudiar 
%<--- Poner ejemplos y citar el trabajo de Cartwright y Potochnik ---!>

"If the world is highly complex relative to our cognitive capacities and we nevertheless seek to know it in its full complexity, this requires stretching our cognitive endowments, devising multiple means for reaching its less accessible regions, improvising, experimenting, tinkering, exercising our imagination, etc" Gila sher

It requires a balance between unity and diversity, between observing and proposing, between describing and constructing, between being critical and understanding. I will call a theory of truth that requires a substantial correspondence (of one kind or another) between true cognition and reality, allows multiple—including intricate—routes of correspondence from language to reality, yet seeks maximal unity and systematicity, a “composite correspondence” theory.

\subsection{Teoría pragmática de la verdad}


\section{Veritismo: limitaciones y alcances}

\subsection{Introducción}

En el capítulo anterior, prometí señalar cómo nuestro compromiso, el veritismo, es capaz de lidiar con los problemas de idealización y modelos en la ciencia.
El problema, a grandes rasgos, consiste en que en la investigación científica  se usan idealizaciones, abstracciones y modelos.
Los métodos anteriores difieren del fenómeno que pretenden representar.
Por decirlo de manera sucinta: idealizaciones, modelos y abstracciones son literalmente falsos (se desvían de la realidad).


Muchas de las teorías científica hacen uso de estos métodos.
El equilibrio de Hardy-Weinberg asume que una población de organismos es infinitamente grande y que la frequencia alélica permanece constante.

Para la caída libre, Galileo generalizó sus experimentos, asumiendo que los planos no tienen fricción (usó esferas y planos lo más pulidos posibles, pero 
esto no significa que haya ausencia de fricción).

%<!--- Más ejemplos aquí --->

Estos métodos, ampliamente usados, son desviaciones del fenómeno y muchas filósofas han dicho que esto es un problema para las aproximaciones veritistas en filosofía de la ciencia \parencite{elgin2004, Potochnik2017-POTIAT-3, bokulich2016}. 
%<!--- Agregar las citas --->
Un veritista como Strevens y compañía, dirán que el uso de estos métodos es temporal.
Esto es, que mientras más avance la investigación , vamos a deshacernos de tales idealizaciones, modelos y abstracciones.
Sin embargo, una amplia literatura, por ejemplo, las autoras anteriores, señalan que algunos de los casos son imprescindibles y que, de hecho, son explicativos porque son falsos.
La conclusión que hay que extraer de está literatura es que debemos relajar nuestros compromisos veritistas.

En el capítulo anterior, discutí por qué la verdad es un valor necesario en la investigación. 
Hay un par de teorías epistemológicas acerca de cómo se entrelazan la teoría de virtudes y la investigación. 
Entre estas, una desarrollada por Haomiao Yu (\citeyear{yu2021}).
Quisiera hacer una comparación entre lo que señalé en el capítulo anterior y las teorías mencionadas.

Haomiao Yu desarrolla una teoría parecida a lo que presenté en el capítulo anterior. Uno de los puntos cruciales de su teoría es que trata de hacer explícito qué es el "entendimiento". Su trabajo trata de hacer compatible una teoría epistémica con las teorías de la explicación científica. Yu detecta un problema en la literatura sobre explicación científica: muchos asumen que su modelo de explicación se conecta con cómo comprendemos [entendemos] el mundo. Sin embargo, sólo asumen tal conexión sin dar razones a favor de su afirmación.

Pera solventar esta brecha, Yu apuesta a favor de una teoría de las virtudes. Para esto es pertinente indtriducir un par de disttinviones. La primera distinción que hace Yu es entre teorías responsabilistas de las virtudes y teorías fiabilistas de las virtudes. Esto es algo que no señalé explícitamente en el capítulo anterior y vale la pena aclararlo aquí: las teorías fiabilistas de las virtudes se comprometen con que las virtudes de los agentes son constitutivas del conoicimiento. Por su parte, las teorías responsabilistas no se comprometen con esto. Lo único que señalan las teorías responsabilistas es que las virtudes son rasgos que tiene un buen conocedor. Yu nos dice que en su teoría el fiabilismo de virtudes es crucial; el responsabilismo de virtudes es sólo auxiliar.

Lsa virtudes espisémicas identifican diferentes habilidades de los agentes involucrados en la investigación. Yu señala que muchas de las capacidades cognitivas juegan un papel que cierra la brecha entre explicación y entendimiento. El autor señala que una de las teorías del entendimiento basadas en habilidades de los agentes es la de Khalifa. Sin embargo, Yu argumenta que la teoría de Khalifa sólo nos permite hacer distinciones de grado, mientras que hay claros casos de diferentes tipos de entendimiento. Hablidades comunes que señala Yu son: razonamiento deductivo/inductivo, razonamiento causal/mecánico, razonamiento contrafáctico, generalización/categorización y abtrsacción. 

Para aclarar su teoría, Yu nos presenta un caso de estudio: Galileo y su prueba de que el péndulo es isocrónico. Primero que nada, la recosntrucción del caso describe cómo Galileo hace uso de diferentes habilidades. Para su investigación, Galileo hizo uso de sus habilidades matemáticas. Sin embargo, Yu señala que el entendimiento de Galileo sólo es correcto hasta cierto grado. Esto se debe a que Galileo usó las leyes de Kepler para derivar su modelo, entonces sólo tenía disponible un mapeo de la estructura matemática del péndulo. Para un entendimiento completo del fenómeno, es necesario usar las leyes de Newton.

La primer diferencia entre lo que presenta Yu y lo que señalé en el capítulo anterior es que estoy en desacuerdo con su conclusión. Si bien estoy de acuerdo con su teoría de las habilidades/virtudes de los agentes, me parece que imputarle a Galileo que su entendimiento de un fenómeno es de un grado menor que quien tenga a su disposición las leyes de Newton. Esto puede parecer ser muy obvio, pero no estoy de acuerdo. No lo estoy porque es un caso de anacronismo.

En primer lugar Yu señala que lo que constituye el entendimiento es el uso de las habilidades de los agentes. Si esto es verdad, no es claro cómo una situación externa al agente (el hecho de que vivió antes del desarrollo de la física newtoniana), tiene cualqueir cosa que ver con el entendimiento que tuvo Galileo. Más aún, recordemos que, si un sistema es axiomático, una vez que aceptamos las premisas estamos condenados a aceptar la conclusión (suponiendo que el argumento es válido). Siempre podemos renegar de los axiomas: desecharlos por alguna razón, lo que implica que el argumento deja de ser sólido. Dado esto que señalo, si Galileo derivó la fórmula del péndulo, entonces no es claro cómo podía estar equivocado (suponiendo que las premisas son verdaderas). Más aún, debido al marco presentado en la sección anterior, no podemos imputarle errores a Galileo debido a que no tenía información disponible acerca de las leyes de Newton. Hasta el punto de su conocimiento, podemos afirmar que tenía conocimiento certero ``real knowledge''.

El caso del péndulo en Galileo es más intrincado. Para exponer este caso, usaré la reconstrucción que hace Ariotti en \citeyear{Ariotti1968}. Galileo usa un par de experimentos para motivar la intuición de que el periodo del péndulo es proporcional a la raíz cuadrada de la longitud de la cuerda. Galileo afirma que el péndulo es \emph{isocrónico}, esto quiere decir que el tiempo que tarda el péndulo en llegar al punto más bajo del círculo descrito es independiente al ángulo. Sin embargo, las pruebas de Galileo no son suficientes para mostrar la isocronía del péndulo, sólo su \emph{sincronía}. Es decir que el tiempo que tardan dos cuerpos de diferentes pesos en llegar al punto más bajo del círculo descrito es el mismo. El experimento es sencillo: tomemos dos cuerpos con diferentes pesos atados a cuerdas con la misma longitud. Si dejamos caer ambos cuerpos, llegarán al mismo tiempo al punto más bajo y su periodo será el mismo.

Lo anterior sólo muestra la sincronía del péndulo, no la isocronía. Para apoyar la hipótesis de que el péndulo es isocrónico, Galileo recurre a un modelo rígido del péndulo. En lugar de usar cuerdas (que pueden doblarse por no ser rígidas), Galileo utiliza una construcción de madera lo suficientemente pulida para que la fricción no afecte el tiempo de traslado. Para medir los tiempos en este experimento, Galileo usó agua. Dejaba caer agua en desde un jarrón con un orificio hacia un bote. Luego pesaba el agua para saber si la cantidad era la misma, por tanto los tiempos iguales en diferentes ángulos. Debido a la poca discrepancia entre los pesos del agua recolectada, Galileo concluye que el péndulo es isocrónico.

Pero este modelo rígido no se comporta igual que un péndulo de cuerda. En la cuerda hay discrepancias debido a que la cuerda hace una curva cuando dejamos caer un cuerpo atado a ella. Por tanto hay discrepancias entre los tiempos. El modelo rígido no es completamente exacto con el fenómeno a estudiar. Más aún, los resultados de Huygens muestran que para que un péndulo sea isocrónico, la curva no debe ser un círculo, sino un cicloide \cite{Ramond2023} pero las investigaciones de Galileo fueron un paso hacia el teorema de Huygens de que el péndulo es isocrónico. 

Cabe notar que en esta descripción del caso del péndulo en Galileo, es claro el uso de las habilidades que Yu señala. Habilidades para modelar, habilidades matemáticas y habilidades de deducción. También hay que destacar que son estas habilidades lo que lleva a Galileo a afirmar que el péndulo es sincrónico e isocrónico (casi isocrónico: para distancias de cuerda mayores a la distancia de donde liberamos la masa). 

Sin embargo, toda esta historia nos muestra más que un fallo, un éxito. Vaya, la incvestigación en general depende no sólo de un experimento, sino de diferentes personas trabajando en un mismo tema. Galileo fue un paso para el teorema de Huygens. No podía afirmar la isocronía del péndulo debido a que el experimento no es exacto: el modelo rígido es diferente al modelo con cuerdas. El método de medición por supuesto no es perfecto. Además el círculo no es la curva para que un péndulo sea isocrónico, la curva es un ciloide, como muestra Huygens. Cabe notar además que la solución de Huygens vino antes de la teoría de Newton, por tanto, tampoco son necesarias las leyes de Newton para mostrar la isocronía del péndulo. Los fallos de Galileo son fallos de los materiales a su disposición. Por supuesto, en estos materiales también incluyo la teoría newtoniana.

Pero no se sigue que cualquiera que tenga a su disposición mejores materiales, incluyendo la teoría de Newton, podría haber entendido completamente el fenómeno. Las virtudes y habilidades de Galileo deberían ser lo único relevante para decir si hizo uso de sus habilidades (aunque suene trivial), que es lo que señala Yu. Aunque siendo sinceros, no sé si Yu esté dispuesto a aceptar este argumento, quizás haya algo más en la teoría que diseña, de manera que, la comprensión completa del fenómeno sea un factor relevante.

%No son del todo precisas, claro qwue tenía entendimiento. Dos preguntas Juan. 

\section{Modelos}

Lo dicho anteriormente, por supuesto, se relaciona con el problema del realismo científico. De manera muy sucinta y poco precisa, los realistas científicos defienden la tesis de que las teorías científicas son literalmente verdaderas. Esto implica que las entidades que aparecen en la teoría de hecho existen y que las relaciones entre objetos que señala la teoría reflejan cómo de hecho es el mundo.

Esta tesis tiene muchas aristas. En primer lugar hay un aspecto epistémico involucrado. Si saber implica verdad, entonces el hecho de que alguien señale que Newton sabía que el espacio absoluto existe, el espacio absoluto de hecho existe. Otro aspecto del realismo científico es semántico. Saber si efectivamente los términos individuales que son parte de las oraciones de una teoría, de hecho refieren a un objeto. También hay un aspecto axiológico involucrado: que el objetivo de la ciencia es la verdad\footnote{Estas aristas corresponden más o menos a cómo Khalifa describe la tesis del realismo científico. Véase (\citeyear{khalifa2010})}.

Si bien todos estos problemas están relacionados (por ejemplo determinar que el espacio absoluto de hecho existe, haría que las oraciones donde aparece dicho término individual refiriera y que alguien que sepa la teoría sabe que el tiempo absoluto existe), vale la pena tener estos aspectos separados. 

Lo que he señalado hasta ahora está relacionado sólo con el problema epistémico. Nuestro problema es que no es claro que la teoría de Newton sea verdadera, ya que nuevas teorías han mostrado ser mejores descripciones del mundo que la teoría newtoniana, por tanto, Newton no sabía que el espacio absoluto existiese. 
La justificación de Newton no es suficiente. 
Esto es sólo una manera de exponer algo que Laudan ya había señalado: la historia de la ciencia ha mostrado que los términos individuales de teorías exitosas no siempre refieren, por lo que dichas teorías son falsas \cite{laudan1981}. 
Creo que la moraleja que nos da Laudan es que debemos ser cautelosos al formular una tesis realista sobre la ciencia\footnote{Más aún, el punto de Laudan es más débil que señalar que el realismo científico es falso. El punto de Laudan es señalar que no hay una relación tan fuerte entre verdad y poder describir y predecir correctamente. Es por eso que el argumento de los "no-milagros" falla, ya que supone que dicha conexión es más fuerte.}.

Pero hemos llegado de nuevo al punto inicial. Si la hipótesis del espacio absoluto es falsa, y la teoría newtoniana depende de dicha entidad y la estructura es de derivación, entonces los teoremas extraídos de dichas hipótesis son falsas. Por lo que nadie sabría la teoría newtoniana. Más aún, si suponemos que las buenas explicaciones son explicaciones verdaderas, entonces no podríamos explicar nada con la teoría newtoniana. Hay que desecharla por alguna teoría física más moderna.

Pero lo anterior claramente es falso, la teoría de Newton es explicativa. Tal como señala Laudan, hay teorías explicativas que postulan entidades falsas: aún cuando la teoría depende de dichas entidades, sigue siendo explicativa. El problema ahora está a nivel ontológico y relacionado con el argumento a la mejor explicación: del hecho de que una teoría sea explicativa, no podemos pasar a que las entidades postuladas de hecho existan. Recordemos que uno de los argumentos más usados para defender el realismo el así llamado ``argumento del no-milagro'' depende de la inferencia a la mejor explicación\footnote{No todos los argumentos a favor del realismo dependen de dicho tipo de inferencia. Si bien el argumento del ``no-milagro'' depende de aceptar que esta inferencia es válida, no quiere decir que este sea el único argumento a favor del realismo científico. Por otro lado, no es necesario negar que hay que desechar este tipo de inferencia. Cabe destacar que este tipo de inferencia no asegura que de premisas verdaderas pasemos a una conclusión verdadera, por lo que una manera de defender el realismo científico es hacer que las inferencias de este tipo sean más ``robustas'', es decir, que podamos asegurar la existencia de los objetos de acuerdo a una inferencia no-deductiva. Por supuesto, aún cuando podamos hacer esto todavía queda margen de error ya que estas inferencias son falibles. Para una exposición más detallada véase \cite{Saatsi2010-SAAFVC-2}.}. Este argumento no es deductivamente válido, por lo que siempre hay lugar para el error. Si esto es verdad, entonces el realista que dependa del argumento de los no-milagros está en problemas para justificar su tesis.

Parece entonces que el problema no está en cómo justificamos creencias, sino en el hecho de que esperamos demasiado de la justificación que podemos ofrecer y de cómo se conecta con la verdad. Incluso un argumento deductivo perfectamente válido puede no tener conclusión verdadera debido a que una de las premisas es falsa. Difícilmente en ciencias empíricas podemos ofrecer un grado de certeza del 100\%. Los métodos más utilizados, entre ellos las herramientas estadísticas, no nos dan ese grado de certeza. Aún cuando deseemos que toda la información que usemos en una inferencia sea verdadera y, por tanto, nuestras conclusiones sean verdaderas (utilizando métodos deductivos) si nuestras premisas son conclusiones de un argumento no-deductivo, entonces no hay manera de asegurar tal certeza.

%Alguien podría señalar que si bien la verdad no es algo que podamos obtener (en ciencias empíricas), sin duda es una motivación para los investigadores. Pero esto también es falso, véase por ejemplo el artículo de Boris Hessen (al menos creo que es una forma de obtener la negación de la hipótesis).

Pero entonces qué teoría de la verdad podríamos adoptar tal que no incluyamoLa teoría correspondentista de la verdad tiene una carga intuitiva muy fuerte, pero difícilmente da lugar a los errores que hay en la historia de la ciencia. Por otro lado, no es claro cómo podemos incluir proposiciones que sólo son plausibles en los razonamientos (como aquellas conclusiones de argumentos no-deductivos). Entonces, ¿deberíamos deshacernos de la verdad como una condición del conocimiento?

Si defendemos algo como lo qeu sugiere esta última pregunta

%La verdad juega un papel mediador. Si la verdad es un concepto que no involucra propiedades epistémicas, entonces tiene sentido que podamos corregir nuestras creencias con base en nuevos descubrimientos. Si tiene propiedades epistémicas, entonces la verdad cambia  en tanto nuestras creencias cambian. Esto último no sucede, de ser así, podríamos modificar a conveniencia el color, tamaño y forma de los objetos. Entonces la verdad no involucra propiedades epistémicas. 

% Habrá qué decir a este respecto de cómo la verdad por correspondencia es sumamente intuitiva cuando hablamos de conocimiento cotidiano, mientras que en ámbitos donde tenemos que evaluar entidades no-observables es más difuso cómo juega el mismo papel. PAra esto leer el artículo "A note on truth and reference" de Penelope Maddy. Aunque MAddy señala que podemos decir que no hay un concepto homogéneo de verdad, sino que en contextos cotidianos una teoría correspondentista de la verdad podría ser útil, mientras que en un contexto de investigación no parece muy útil, no me queda claro cómo defender esto. No me queda claro porque muchas de nuestras explicaciones cotidianas dependen de nuestras "empresas epistémicas". Por ejemplo explicar por qué el cielo es azul, o por qué mi cultivo murió ¿es conocimiento cotidiano o de investigación? ¿Cómo hacemos ese corte? Los ejemplos pueden ser ilustrativos "Consider, for example, Hartry Field’s story of the  ancient Greek whose directivities for ‘Zeus is throwing thunderbolts’ correlate neatly with the worldly support of lightning; his utetrance would serve  as a warning to others." MAddy dice que para decir que estos casos son claramente falsos debemos hablar de qué correlaciones vamos a tomar como genuinamente referenciales.  Aunque también menciona que "Here I’m not  out to define ‘truth’ or to provide any theoretical account at all; the whole  point of classifying ‘true’ and ‘refers’ with ‘excuses’ and ‘know’ rather than  with ‘time’ and ‘hardness’ is to deny that they’re things to be theorized about  rather than words whose complex usage is to be studied." Ahora creo que lo qeu defiende MAddy no es un pluralismo alético, sino una nueva manera de investigar el uso de la palabra verdad, que no es homogénea en todo contexto, ya que otras palabras pueden describir mejor la situación que sólo ser verdaderas. 



Otra postura que evita los problemas relacionados con el conocimiento (y con ello la verdad) involucrados en la investigación científica, es el antirrealismo. El argumento de Laudan me parece contundente en el debate entre realistas y antirrealistas. Dado lo expuesto hasta ahora, vale la pena tomar esta postura en serio.

Hasta ahora he descrito una gran variedad de tesis. 


\section{Citas}

But my aim will be to vindicate the possible-worlds theory while making minimal commitments about substantive metaphysical questions, for example, about whether there are things, or properties, that exist only contingently, whether there are individual essences that are irreducible to qualitative properties, whether there could be distinct but qualitatively indiscernible worlds. In the balance of costs and benefits, I give positive weight to this kind of neutrality. \cite{stalnaker2012}

But this would still be an insufficient criterion for we cannot
exclude that a breakthrough in a certain field makes more realistic
explanatory models available. Second, if we do not have a definitive
criterion of indispensability, some idealized models that we currently
deem necessary may turn out to be incompatible with realism.

Although the rule itself is not an explanation, it imposes on cognizers
epistemic constraints on explanations through two steps. First, it gives a
set of phenomena a common referent point or starting point for explanation. This is the constructive function of idealization. Employing HWE
is the starting point for all population-genetical investigation. In Kantian
terms, it gives ‘systematic unity’ to our particular cognitions as a "focus
imaginarius". Second, idealizations are not explanations of phenomena,
but rather normative standards for the assessment of explanations.
It would be misleading to say that HWE as such ‘explains’ anything. 
It furnishes instead a standard to which we can compare collected data. 
By doing so we further the unity and accuracy of our explanations. 
More specifically, we identify cases where deviations are small and thereby do
not reject HWE, and cases that deeply challenge HWE and thus lead us to
new findings.
Ojalá no sea el caso.

% chapter chapter_2 (end)


\input{input_files/chapter_3}

\input{input_files/chapter_4}

%!TEX root = ../main.tex




\chapter{Prácticas biológicas}
\label{ch:practices}

Parece claro que los investigadores en general buscan explicaciones.
Dichas explicaciones lucen diferente en distintas disciplinas.
Por ejemplo, en matemáticas, buscamos una prueba para saber ciertas propiedades de los números.
Esto, sin embargo, es distinto en las ciencias empíricas.

Los físicos tratan de describir fenómenos como el movimiento, velocidad y aceleración de los objetos.
Tratan de descrtibir la fuerza necesaria para mover un objeto con cierto peso, etc.
Ahora bien, la física tiene descripciones muy precisas de estos fenómenos, sin embargo, otras disciplinas no tienen descirpciones tan precisas: hay más variables involucradas.
Comúnmente se acepta que la física usa leyes para describir fenómenos.
En mecánica clásica, dichas leyes están cuantificadas universalmente, son reversibles y deterministas.
De manera tal que sistemas que cumplan las misma propiedades pueden describirse con una misma ecuación.

Pienso, por ejemplo, en el movimiento uniformemente acelerado.
Este tipo de movimiento es tal que, la aceleración de un cuerpo con masa es constante en intervalos iguales de tiempo. 

En biología difícilmente tenemos este tipo de generalizaciones.
Por lo general se asume que en biología no hay leyes. 
En particular en biología evolutiva, sabemos que las condiciones no son lo sufcientementer estables para hacer generalizaciones.
Pensemo, por ejemplo, en las leyes de Mendel. 
Las ley de segreación de Mendel nos dice que los alelos de un gen están segregados entre ambos padres.
Esto significa que si ambos alelos son recesivos no codifican, mientras que si uno de ellos es dominante, entonces codifican para un rasgo fenotípico particular.





% chapter chapter_5 (end)


%%%%%%%%%%%%%%%%%%%%%%%%%%%%%%%%%%%%%%%%%%%%%%%%%%%%%%%%%%%%%%%%%%%%%%%%%%
% The back matter contains unnumbered chapters
% conclusion, french summary, bibliographies, indices, glossaries
%%%%%%%%%%%%%%%%%%%%%%%%%%%%%%%%%%%%%%%%%%%%%%%%%%%%%%%%%%%%%%%%%%%%%%%%%%

\backmatter

\input{input_files/conclusion}

\begin{fullwidth}
	\printbibliography[heading=bibintoc]
\end{fullwidth} %04 de junio del 2024. Creo que esta madre no copila porque la base de datos .bib es muy grande.

\begin{fullwidth}
	\printindex
\end{fullwidth}

%\begin{fullwidth}
%   \printglossary[type=\acronymtype]
%   \printglossary
%\end{fullwidth} %movido a posopts. "Movido" es un anglicismo? no tengo una mejor palabra en español para esto.

%\input{input_files/abstract} %Movido a posopts, o sea, posibles opciones

\end{document}
