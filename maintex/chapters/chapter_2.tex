% !TEX root = ../main.tex

\chapter{La naturaleza de la \emph{verdad}}
\label{ch:natruth}
 



\section{Naturaleza}

La naturaleza de la verdad es un tema que sigue siendo motivo de debate entre filósofos.
Hay al menos tres posturas clásicas sobre la verdad: la pragmática, correspondentista y coherentista.
Por supuesto hay más detalles y más teorías sobre la naturaleza de la verdad, pero, por motivos expositivos, concentrémonos en las teorías clásicas.
Además, quiero hacer claros un par de supuestos.
En primer lugar, el mundo y los fenómenos que queremos estudiar son complejos.
La segunda y última suposición, los seres humanos tenemos capacidades cognitivas limitadas, muchas de las veces en las que queremos estudiar un fenómeno, limitamos las diferentes variables que pueden afectar el fenómeno que queremos estudiar 
%<--- Poner ejemplos y citar el trabajo de Cartwright y Potochnik ---!>

"If the world is highly complex relative to our cognitive capacities and we nevertheless seek to know it in its full complexity, this requires stretching our cognitive endowments, devising multiple means for reaching its less accessible regions, improvising, experimenting, tinkering, exercising our imagination, etc" Gila sher

It requires a balance between unity and diversity, between observing and proposing, between describing and constructing, between being critical and understanding. I will call a theory of truth that requires a substantial correspondence (of one kind or another) between true cognition and reality, allows multiple—including intricate—routes of correspondence from language to reality, yet seeks maximal unity and systematicity, a “composite correspondence” theory.

\subsection{Teoría pragmática de la verdad}


\section{Veritismo: limitaciones y alcances}

\subsection{Introducción}

En el capítulo anterior, prometí señalar cómo nuestro compromiso, el veritismo, es capaz de lidiar con los problemas de idealización y modelos en la ciencia.
El problema, a grandes rasgos, consiste en que en la investigación científica  se usan idealizaciones, abstracciones y modelos.
Los métodos anteriores difieren del fenómeno que pretenden representar.
Por decirlo de manera sucinta: idealizaciones, modelos y abstracciones son literalmente falsos (se desvían de la realidad).


Muchas de las teorías científica hacen uso de estos métodos.
El equilibrio de Hardy-Weinberg asume que una población de organismos es infinitamente grande y que la frequencia alélica permanece constante.

Para la caída libre, Galileo generalizó sus experimentos, asumiendo que los planos no tienen fricción (usó esferas y planos lo más pulidos posibles, pero 
esto no significa que haya ausencia de fricción).

%<!--- Más ejemplos aquí --->

Estos métodos, ampliamente usados, son desviaciones del fenómeno y muchas filósofas han dicho que esto es un problema para las aproximaciones veritistas en filosofía de la ciencia \parencite{elgin2004, Potochnik2017-POTIAT-3, bokulich2016}. 
%<!--- Agregar las citas --->
Un veritista como Strevens y compañía, dirán que el uso de estos métodos es temporal.
Esto es, que mientras más avance la investigación , vamos a deshacernos de tales idealizaciones, modelos y abstracciones.
Sin embargo, una amplia literatura, por ejemplo, las autoras anteriores, señalan que algunos de los casos son imprescindibles y que, de hecho, son explicativos porque son falsos.
La conclusión que hay que extraer de está literatura es que debemos relajar nuestros compromisos veritistas.

En el capítulo anterior, discutí por qué la verdad es un valor necesario en la investigación. 
Hay un par de teorías epistemológicas acerca de cómo se entrelazan la teoría de virtudes y la investigación. 
Entre estas, una desarrollada por Haomiao Yu (\citeyear{yu2021}).
Quisiera hacer una comparación entre lo que señalé en el capítulo anterior y las teorías mencionadas.

Haomiao Yu desarrolla una teoría parecida a lo que presenté en el capítulo anterior. Uno de los puntos cruciales de su teoría es que trata de hacer explícito qué es el "entendimiento". Su trabajo trata de hacer compatible una teoría epistémica con las teorías de la explicación científica. Yu detecta un problema en la literatura sobre explicación científica: muchos asumen que su modelo de explicación se conecta con cómo comprendemos [entendemos] el mundo. Sin embargo, sólo asumen tal conexión sin dar razones a favor de su afirmación.

Pera solventar esta brecha, Yu apuesta a favor de una teoría de las virtudes. Para esto es pertinente indtriducir un par de disttinviones. La primera distinción que hace Yu es entre teorías responsabilistas de las virtudes y teorías fiabilistas de las virtudes. Esto es algo que no señalé explícitamente en el capítulo anterior y vale la pena aclararlo aquí: las teorías fiabilistas de las virtudes se comprometen con que las virtudes de los agentes son constitutivas del conoicimiento. Por su parte, las teorías responsabilistas no se comprometen con esto. Lo único que señalan las teorías responsabilistas es que las virtudes son rasgos que tiene un buen conocedor. Yu nos dice que en su teoría el fiabilismo de virtudes es crucial; el responsabilismo de virtudes es sólo auxiliar.

Lsa virtudes espisémicas identifican diferentes habilidades de los agentes involucrados en la investigación. Yu señala que muchas de las capacidades cognitivas juegan un papel que cierra la brecha entre explicación y entendimiento. El autor señala que una de las teorías del entendimiento basadas en habilidades de los agentes es la de Khalifa. Sin embargo, Yu argumenta que la teoría de Khalifa sólo nos permite hacer distinciones de grado, mientras que hay claros casos de diferentes tipos de entendimiento. Hablidades comunes que señala Yu son: razonamiento deductivo/inductivo, razonamiento causal/mecánico, razonamiento contrafáctico, generalización/categorización y abtrsacción. 

Para aclarar su teoría, Yu nos presenta un caso de estudio: Galileo y su prueba de que el péndulo es isocrónico. Primero que nada, la recosntrucción del caso describe cómo Galileo hace uso de diferentes habilidades. Para su investigación, Galileo hizo uso de sus habilidades matemáticas. Sin embargo, Yu señala que el entendimiento de Galileo sólo es correcto hasta cierto grado. Esto se debe a que Galileo usó las leyes de Kepler para derivar su modelo, entonces sólo tenía disponible un mapeo de la estructura matemática del péndulo. Para un entendimiento completo del fenómeno, es necesario usar las leyes de Newton.

La primer diferencia entre lo que presenta Yu y lo que señalé en el capítulo anterior es que estoy en desacuerdo con su conclusión. Si bien estoy de acuerdo con su teoría de las habilidades/virtudes de los agentes, me parece que imputarle a Galileo que su entendimiento de un fenómeno es de un grado menor que quien tenga a su disposición las leyes de Newton. Esto puede parecer ser muy obvio, pero no estoy de acuerdo. No lo estoy porque es un caso de anacronismo.

En primer lugar Yu señala que lo que constituye el entendimiento es el uso de las habilidades de los agentes. Si esto es verdad, no es claro cómo una situación externa al agente (el hecho de que vivió antes del desarrollo de la física newtoniana), tiene cualqueir cosa que ver con el entendimiento que tuvo Galileo. Más aún, recordemos que, si un sistema es axiomático, una vez que aceptamos las premisas estamos condenados a aceptar la conclusión (suponiendo que el argumento es válido). Siempre podemos renegar de los axiomas: desecharlos por alguna razón, lo que implica que el argumento deja de ser sólido. Dado esto que señalo, si Galileo derivó la fórmula del péndulo, entonces no es claro cómo podía estar equivocado (suponiendo que las premisas son verdaderas). Más aún, debido al marco presentado en la sección anterior, no podemos imputarle errores a Galileo debido a que no tenía información disponible acerca de las leyes de Newton. Hasta el punto de su conocimiento, podemos afirmar que tenía conocimiento certero ``real knowledge''.

El caso del péndulo en Galileo es más intrincado. Para exponer este caso, usaré la reconstrucción que hace Ariotti en \citeyear{Ariotti1968}. Galileo usa un par de experimentos para motivar la intuición de que el periodo del péndulo es proporcional a la raíz cuadrada de la longitud de la cuerda. Galileo afirma que el péndulo es \emph{isocrónico}, esto quiere decir que el tiempo que tarda el péndulo en llegar al punto más bajo del círculo descrito es independiente al ángulo. Sin embargo, las pruebas de Galileo no son suficientes para mostrar la isocronía del péndulo, sólo su \emph{sincronía}. Es decir que el tiempo que tardan dos cuerpos de diferentes pesos en llegar al punto más bajo del círculo descrito es el mismo. El experimento es sencillo: tomemos dos cuerpos con diferentes pesos atados a cuerdas con la misma longitud. Si dejamos caer ambos cuerpos, llegarán al mismo tiempo al punto más bajo y su periodo será el mismo.

Lo anterior sólo muestra la sincronía del péndulo, no la isocronía. Para apoyar la hipótesis de que el péndulo es isocrónico, Galileo recurre a un modelo rígido del péndulo. En lugar de usar cuerdas (que pueden doblarse por no ser rígidas), Galileo utiliza una construcción de madera lo suficientemente pulida para que la fricción no afecte el tiempo de traslado. Para medir los tiempos en este experimento, Galileo usó agua. Dejaba caer agua en desde un jarrón con un orificio hacia un bote. Luego pesaba el agua para saber si la cantidad era la misma, por tanto los tiempos iguales en diferentes ángulos. Debido a la poca discrepancia entre los pesos del agua recolectada, Galileo concluye que el péndulo es isocrónico.

Pero este modelo rígido no se comporta igual que un péndulo de cuerda. En la cuerda hay discrepancias debido a que la cuerda hace una curva cuando dejamos caer un cuerpo atado a ella. Por tanto hay discrepancias entre los tiempos. El modelo rígido no es completamente exacto con el fenómeno a estudiar. Más aún, los resultados de Huygens muestran que para que un péndulo sea isocrónico, la curva no debe ser un círculo, sino un cicloide \cite{Ramond2023} pero las investigaciones de Galileo fueron un paso hacia el teorema de Huygens de que el péndulo es isocrónico. 

Cabe notar que en esta descripción del caso del péndulo en Galileo, es claro el uso de las habilidades que Yu señala. Habilidades para modelar, habilidades matemáticas y habilidades de deducción. También hay que destacar que son estas habilidades lo que lleva a Galileo a afirmar que el péndulo es sincrónico e isocrónico (casi isocrónico: para distancias de cuerda mayores a la distancia de donde liberamos la masa). 

Sin embargo, toda esta historia nos muestra más que un fallo, un éxito. Vaya, la incvestigación en general depende no sólo de un experimento, sino de diferentes personas trabajando en un mismo tema. Galileo fue un paso para el teorema de Huygens. No podía afirmar la isocronía del péndulo debido a que el experimento no es exacto: el modelo rígido es diferente al modelo con cuerdas. El método de medición por supuesto no es perfecto. Además el círculo no es la curva para que un péndulo sea isocrónico, la curva es un ciloide, como muestra Huygens. Cabe notar además que la solución de Huygens vino antes de la teoría de Newton, por tanto, tampoco son necesarias las leyes de Newton para mostrar la isocronía del péndulo. Los fallos de Galileo son fallos de los materiales a su disposición. Por supuesto, en estos materiales también incluyo la teoría newtoniana.

Pero no se sigue que cualquiera que tenga a su disposición mejores materiales, incluyendo la teoría de Newton, podría haber entendido completamente el fenómeno. Las virtudes y habilidades de Galileo deberían ser lo único relevante para decir si hizo uso de sus habilidades (aunque suene trivial), que es lo que señala Yu. Aunque siendo sinceros, no sé si Yu esté dispuesto a aceptar este argumento, quizás haya algo más en la teoría que diseña, de manera que, la comprensión completa del fenómeno sea un factor relevante.

%No son del todo precisas, claro qwue tenía entendimiento. Dos preguntas Juan. 

\section{Modelos}

Lo dicho anteriormente, por supuesto, se relaciona con el problema del realismo científico. De manera muy sucinta y poco precisa, los realistas científicos defienden la tesis de que las teorías científicas son literalmente verdaderas. Esto implica que las entidades que aparecen en la teoría de hecho existen y que las relaciones entre objetos que señala la teoría reflejan cómo de hecho es el mundo.

Esta tesis tiene muchas aristas. En primer lugar hay un aspecto epistémico involucrado. Si saber implica verdad, entonces el hecho de que alguien señale que Newton sabía que el espacio absoluto existe, el espacio absoluto de hecho existe. Otro aspecto del realismo científico es semántico. Saber si efectivamente los términos individuales que son parte de las oraciones de una teoría, de hecho refieren a un objeto. También hay un aspecto axiológico involucrado: que el objetivo de la ciencia es la verdad\footnote{Estas aristas corresponden más o menos a cómo Khalifa describe la tesis del realismo científico. Véase (\citeyear{khalifa2010})}.

Si bien todos estos problemas están relacionados (por ejemplo determinar que el espacio absoluto de hecho existe, haría que las oraciones donde aparece dicho término individual refiriera y que alguien que sepa la teoría sabe que el tiempo absoluto existe), vale la pena tener estos aspectos separados. 

Lo que he señalado hasta ahora está relacionado sólo con el problema epistémico. Nuestro problema es que no es claro que la teoría de Newton sea verdadera, ya que nuevas teorías han mostrado ser mejores descripciones del mundo que la teoría newtoniana, por tanto, Newton no sabía que el espacio absoluto existiese. 
La justificación de Newton no es suficiente. 
Esto es sólo una manera de exponer algo que Laudan ya había señalado: la historia de la ciencia ha mostrado que los términos individuales de teorías exitosas no siempre refieren, por lo que dichas teorías son falsas \cite{laudan1981}. 
Creo que la moraleja que nos da Laudan es que debemos ser cautelosos al formular una tesis realista sobre la ciencia\footnote{Más aún, el punto de Laudan es más débil que señalar que el realismo científico es falso. El punto de Laudan es señalar que no hay una relación tan fuerte entre verdad y poder describir y predecir correctamente. Es por eso que el argumento de los "no-milagros" falla, ya que supone que dicha conexión es más fuerte.}.

Pero hemos llegado de nuevo al punto inicial. Si la hipótesis del espacio absoluto es falsa, y la teoría newtoniana depende de dicha entidad y la estructura es de derivación, entonces los teoremas extraídos de dichas hipótesis son falsas. Por lo que nadie sabría la teoría newtoniana. Más aún, si suponemos que las buenas explicaciones son explicaciones verdaderas, entonces no podríamos explicar nada con la teoría newtoniana. Hay que desecharla por alguna teoría física más moderna.

Pero lo anterior claramente es falso, la teoría de Newton es explicativa. Tal como señala Laudan, hay teorías explicativas que postulan entidades falsas: aún cuando la teoría depende de dichas entidades, sigue siendo explicativa. El problema ahora está a nivel ontológico y relacionado con el argumento a la mejor explicación: del hecho de que una teoría sea explicativa, no podemos pasar a que las entidades postuladas de hecho existan. Recordemos que uno de los argumentos más usados para defender el realismo el así llamado ``argumento del no-milagro'' depende de la inferencia a la mejor explicación\footnote{No todos los argumentos a favor del realismo dependen de dicho tipo de inferencia. Si bien el argumento del ``no-milagro'' depende de aceptar que esta inferencia es válida, no quiere decir que este sea el único argumento a favor del realismo científico. Por otro lado, no es necesario negar que hay que desechar este tipo de inferencia. Cabe destacar que este tipo de inferencia no asegura que de premisas verdaderas pasemos a una conclusión verdadera, por lo que una manera de defender el realismo científico es hacer que las inferencias de este tipo sean más ``robustas'', es decir, que podamos asegurar la existencia de los objetos de acuerdo a una inferencia no-deductiva. Por supuesto, aún cuando podamos hacer esto todavía queda margen de error ya que estas inferencias son falibles. Para una exposición más detallada véase \cite{Saatsi2010-SAAFVC-2}.}. Este argumento no es deductivamente válido, por lo que siempre hay lugar para el error. Si esto es verdad, entonces el realista que dependa del argumento de los no-milagros está en problemas para justificar su tesis.

Parece entonces que el problema no está en cómo justificamos creencias, sino en el hecho de que esperamos demasiado de la justificación que podemos ofrecer y de cómo se conecta con la verdad. Incluso un argumento deductivo perfectamente válido puede no tener conclusión verdadera debido a que una de las premisas es falsa. Difícilmente en ciencias empíricas podemos ofrecer un grado de certeza del 100\%. Los métodos más utilizados, entre ellos las herramientas estadísticas, no nos dan ese grado de certeza. Aún cuando deseemos que toda la información que usemos en una inferencia sea verdadera y, por tanto, nuestras conclusiones sean verdaderas (utilizando métodos deductivos) si nuestras premisas son conclusiones de un argumento no-deductivo, entonces no hay manera de asegurar tal certeza.

%Alguien podría señalar que si bien la verdad no es algo que podamos obtener (en ciencias empíricas), sin duda es una motivación para los investigadores. Pero esto también es falso, véase por ejemplo el artículo de Boris Hessen (al menos creo que es una forma de obtener la negación de la hipótesis).

Pero entonces qué teoría de la verdad podríamos adoptar tal que no incluyamoLa teoría correspondentista de la verdad tiene una carga intuitiva muy fuerte, pero difícilmente da lugar a los errores que hay en la historia de la ciencia. Por otro lado, no es claro cómo podemos incluir proposiciones que sólo son plausibles en los razonamientos (como aquellas conclusiones de argumentos no-deductivos). Entonces, ¿deberíamos deshacernos de la verdad como una condición del conocimiento?

Si defendemos algo como lo qeu sugiere esta última pregunta

%La verdad juega un papel mediador. Si la verdad es un concepto que no involucra propiedades epistémicas, entonces tiene sentido que podamos corregir nuestras creencias con base en nuevos descubrimientos. Si tiene propiedades epistémicas, entonces la verdad cambia  en tanto nuestras creencias cambian. Esto último no sucede, de ser así, podríamos modificar a conveniencia el color, tamaño y forma de los objetos. Entonces la verdad no involucra propiedades epistémicas. 

% Habrá qué decir a este respecto de cómo la verdad por correspondencia es sumamente intuitiva cuando hablamos de conocimiento cotidiano, mientras que en ámbitos donde tenemos que evaluar entidades no-observables es más difuso cómo juega el mismo papel. PAra esto leer el artículo "A note on truth and reference" de Penelope Maddy. Aunque MAddy señala que podemos decir que no hay un concepto homogéneo de verdad, sino que en contextos cotidianos una teoría correspondentista de la verdad podría ser útil, mientras que en un contexto de investigación no parece muy útil, no me queda claro cómo defender esto. No me queda claro porque muchas de nuestras explicaciones cotidianas dependen de nuestras "empresas epistémicas". Por ejemplo explicar por qué el cielo es azul, o por qué mi cultivo murió ¿es conocimiento cotidiano o de investigación? ¿Cómo hacemos ese corte? Los ejemplos pueden ser ilustrativos "Consider, for example, Hartry Field’s story of the  ancient Greek whose directivities for ‘Zeus is throwing thunderbolts’ correlate neatly with the worldly support of lightning; his utetrance would serve  as a warning to others." MAddy dice que para decir que estos casos son claramente falsos debemos hablar de qué correlaciones vamos a tomar como genuinamente referenciales.  Aunque también menciona que "Here I’m not  out to define ‘truth’ or to provide any theoretical account at all; the whole  point of classifying ‘true’ and ‘refers’ with ‘excuses’ and ‘know’ rather than  with ‘time’ and ‘hardness’ is to deny that they’re things to be theorized about  rather than words whose complex usage is to be studied." Ahora creo que lo qeu defiende MAddy no es un pluralismo alético, sino una nueva manera de investigar el uso de la palabra verdad, que no es homogénea en todo contexto, ya que otras palabras pueden describir mejor la situación que sólo ser verdaderas. 



Otra postura que evita los problemas relacionados con el conocimiento (y con ello la verdad) involucrados en la investigación científica, es el antirrealismo. El argumento de Laudan me parece contundente en el debate entre realistas y antirrealistas. Dado lo expuesto hasta ahora, vale la pena tomar esta postura en serio.

Hasta ahora he descrito una gran variedad de tesis. 


\section{Citas}

But my aim will be to vindicate the possible-worlds theory while making minimal commitments about substantive metaphysical questions, for example, about whether there are things, or properties, that exist only contingently, whether there are individual essences that are irreducible to qualitative properties, whether there could be distinct but qualitatively indiscernible worlds. In the balance of costs and benefits, I give positive weight to this kind of neutrality. \cite{stalnaker2012}

But this would still be an insufficient criterion for we cannot
exclude that a breakthrough in a certain field makes more realistic
explanatory models available. Second, if we do not have a definitive
criterion of indispensability, some idealized models that we currently
deem necessary may turn out to be incompatible with realism.

Although the rule itself is not an explanation, it imposes on cognizers
epistemic constraints on explanations through two steps. First, it gives a
set of phenomena a common referent point or starting point for explanation. This is the constructive function of idealization. Employing HWE
is the starting point for all population-genetical investigation. In Kantian
terms, it gives ‘systematic unity’ to our particular cognitions as a "focus
imaginarius". Second, idealizations are not explanations of phenomena,
but rather normative standards for the assessment of explanations.
It would be misleading to say that HWE as such ‘explains’ anything. 
It furnishes instead a standard to which we can compare collected data. 
By doing so we further the unity and accuracy of our explanations. 
More specifically, we identify cases where deviations are small and thereby do
not reject HWE, and cases that deeply challenge HWE and thus lead us to
new findings.
Ojalá no sea el caso.

% chapter chapter_2 (end)
